\chapter{Security Protocols}

\section{PAM service}
\label{sec:PAM}

In early versions of Unix O/S, many programs need to read \verb!/etc/passwd!
file to authenticate users. However, if the format of this file changed, many
programs that read the file need to be updated, and recompiled. When single
(remote) user database, using technology like NIS, became popular (in large
organizations), then many NIS commands (yp*) need to be updated too. 

To avoid any future change that can affect existing programs, a standard library
was developed, which can be configured to use different database and/or new
algorithms just by adding new shared library in Linux (DLL in Windows). This
system became known as {\bf PAM} (Pluggable Authentication Modules). A text
configuration file can be updated to use the new authentication shared library. 

\begin{mdframed}
Other than PAM, there are other approaches: GSAPI, SSSD. But PAM is the most
widely used.
\end{mdframed}

A program then need to be recompiled only one time to use PAM, check 
\begin{verbatim}
ldd <program> | grep libpam.so
\end{verbatim} 
\url{http://wpollock.com/AUnix2/PAM-Help.htm}

There is not a single PAM modules, but many, each supporting a different
authentication method. It allows different programs, on the same system, to use
different authentication method. \textcolor{red}{Other than for authentication,
PAM modules can also be used for session setup and tear-down, logging, \ldots}.
Each program must have its own configuration file, so that the using of a PAM
module can be changed easily.

The exact set of modules is controlled by the set of {\it policy} files in
\verb!/etc/pam.d! directory (older versions of PAM, just put in a single file
\verb!/etc/pam.conf!). It is common, but not required, to use the application
name as the name of the configuration file. This is the example for
\verb!hwbrowser! application that use PAM, and the policy file is
\verb!/etc/pam.d/hwbrowser!
\begin{verbatim}
/etc/pam.d# cat hwbrowser
#%PAM-1.0
auth       sufficient   pam_rootok.so
auth       sufficient   pam_timestamp.so
auth       required     pam_stack.so service=system-auth
session    required     pam_permit.so
session    optional     pam_xauth.so
account    required     pam_permit.so
\end{verbatim}
If an application can file the policy file for itself, it uses \verb!other!
filename (which deny all access for security by default). The format of the file
above, for non-comment line (a comment line starts with \verb!#!)
\begin{itemize}
  \item module context (or type): authorization (\verb!auth! - who the user is,
  \verb!account! - check if the validated user is allowed to access),
  initializing the environment (\verb!session!), approving new password (\verb!password!)
  
  \item control flag: \verb!sufficient! (if one PAM module passed, then there is
  no need to check others), \verb!required!, \verb!optional!
  
  \item module: just the library name (if it's in \verb!/lib/security!), or the
  full pathname (if it's elsewhere)

By default, all PAM modules are usually stored in \verb!/lib/security! folder.
If we put in a different locations, then we need to list the full pathname of
the module in the policy file for the program that use PAM module. 
  
  \item module options (if any): tells how to react to the module's result
  (pass or fail)
\end{itemize}
The order is \verb!auth! run first to validate user, then \verb!account! to
check if the validated user is allowed to access, then \verb!session! run next,
if a password changed is request, then \verb!password! module is run. 

These are some PAM modules (shared library)
\begin{enumerate}
  \item \verb!pam_limits.so! : Sect.\ref{sec:pam_limit}
  \item \verb!pam_rootok.so! : check if the user is root, then can run the
  program
  \item \verb!pam_timestamp.so! : it caches successful authentication attempts,
  and you can use it as the basis for authentication
\end{enumerate}

To check library version
\begin{verbatim}
 nm --dynamic --defined-only pam_module.so
\end{verbatim}

\subsection{User limits}
\label{sec:pam_limit}

The command \verb!limit! display current limit setting, e.g. the number of
openfiles is 1024.
\begin{verbatim}
% limit
cputime         unlimited
filesize        unlimited
datasize        unlimited
stacksize       8192 kbytes
coredumpsize    0 kbytes
memoryuse       unlimited
descriptors     1024 
memorylocked    unlimited
maxproc         8146 
openfiles       1024 
\end{verbatim}
\url{https://cs.uwaterloo.ca/~brecht/servers/openfiles.html}

\verb!pam_limits.so! set parameters (limit) on the system per user or per group
basis. We can virtually limit everything. First, specify the users (groups)
limits you want to use in the file \verb!/etc/security/limits.conf! which can
configure many other things which are applied as part of the login process
(\verb!maxlogins!, \verb!maxsyslogins! and \verb!chroot!).

\begin{verbatim}
*               soft    core            0
root            hard    core            100000
@student        hard    nproc           20
@faculty        soft    nproc           20
@faculty        hard    nproc           50
ftp             hard    nproc           0
ftp             -       chroot          /ftp
@student        -       maxlogins       4
*                soft    nofile          65535
*                hard    nofile          65535
\end{verbatim}
Here \verb!nofile! refers to the maximum number of concurrently open file.

Then,  enable the module for the software you want to be affected. In this
case, we want it affect at each shell login. There are three scenarios,
depending on the systems
\begin{enumerate}
  \item \verb!/etc/pam.d/login! (Sun OS)
  
\url{http://docs.oracle.com/cd/E19450-01/820-6168/file-descriptor-requirements.html}

  \item \verb!/etc/pam.d/ssh! (Debian)
  \item \verb!/etc/pam.d/sshd! (RedHat)

\verb!sshd! program, then modify the associated policy file (configuration file)
\verb!/etc/pam.d/sshd! to enable the module
\begin{verbatim}
session required pam_limits.so
\end{verbatim}
\url{http://www.serverwatch.com/server-tutorials/set-user-limits-with-pamlimits-and-limits.conf.html}
\end{enumerate}
The login shell is the parent process of all running program launched from the
shell.



\section{NTLM (NT Lan Manager)}


\section{Kerberos}
\label{sec:Kerberos}

Kerberos V5 is an authentication system developed at MIT.
Kerberos dominates the enterprise universe. This security protocol first
appeared in 2000, when Windows 2000 Server and its Active Directory (AD)
security component were released, and as AD and AD-enabled applications grew in
market share, so did the use of Kerberos. It's been the primary security
protocol for the Microsoft-centric world ever since; it handles authentication
and passes authorization data to domain resources.


Under Kerberos, 
\begin{itemize}
  \item a client (generally either a user or a service) sends a request
for a ticket to the Key Distribution Center (KDC)

  \item  The KDC creates a ticket-granting ticket (TGT) for the client, encrypts
  it using the client's password as the key, and sends the encrypted TGT back to the client
  
  \item  The client then attempts to decrypt the TGT, using its password. If the
  client successfully decrypts the TGT (i.e., if the client gave the correct
  password), it keeps the decrypted TGT, which indicates proof of the client's
  identity.
  
\end{itemize}
The TGT, which expires at a specified time, permits the client to obtain
additional tickets, which give permission for specific service.
\url{http://web.mit.edu/kerberos/krb5-1.5/krb5-1.5.4/doc/krb5-user/Introduction.html#Introduction}


\section{SAML [Security Assertion Markup Language]}
\label{sec:SAML}

SAML was the most broadly adopted authentication standard in the web services
world. However, using SAML was slowed and almost every one switched to using
other standards such as OAuth 2.0, OpenID Connect, and SCIM.


\url{http://windowsitpro.com/identity-management/kerberos-might-not-be-dead-its-not-feeling-well}


\section{Fixes}

\subsection{POODLE}

POODLE refers to the bug in SSLv3 protocol, not the implementation.

\begin{enumerate}
  \item Firefox 34+ fix this
  
  \item Chrome in Linux, we need to open
\begin{verbatim}
/usr/share/applications/google-chrome.desktop
\end{verbatim}
and update all lines \verb!Exec=! to add
\begin{verbatim}
--ssl-version-min=tls1

E.g:
Exec=/usr/bin/google-chrome-stable --ssl-version-min=tls1 %U
\end{verbatim}
then make
sure to fully close the browser (Chrome apps may be keeping your browser active
in the background!).

\end{enumerate}


\subsection{SSH: Cipher Block Chaining (CBC) Mode Ciphers }
\label{sec:SSH-patches}

Check Sect.\ref{sec:SSH}.

Disable Cipher Block Chaining (CBC) Mode Ciphers and Weak MAC Algorithms in SSH
by modifying \verb!/etc/sh/sshd_config! file

\begin{verbatim}
Ciphers aes128-ctr,aes192-ctr,aes256-ctr,arcfour256,arcfour128
MACs hmac-sha1,umac-64@openssh.com,hmac-ripemd160        
\end{verbatim}

\url{https://developer.ibm.com/answers/questions/187318/faq-how-do-i-disable-cipher-block-chaining-cbc-mod.html}

