\chapter{Tcl scripting language (Tool Command Language)}
\label{chap:Tcl}

\begin{verbatim}
cd tcl8.6.6
export SRCDIR=`pwd`
cd unix
./configure --prefix=/modules/tcl/8.6.6 --enable-64bit
\end{verbatim}

\url{http://www.linuxfromscratch.org/blfs/view/svn/general/tcl.html}

\section{Introduction}
\label{sec:Tcl}


Tcl stands for {\bf Tool Command Language}. It is the most popular embedded
language. A script file is a series of commands, each on its own line. The first
word is the command name, and nexts are its arguments.

Similar to DOS, or bash scripting language, we can easily extend the list of
commands.

\section{Commands}
\label{sec:Tcl-commands}

\textcolor{red}{In Tcl, every value is a string}. As white spaces are used to
separate words, if we want to say multiple-word string as a single value, we
need to use backslash, braces, or quotes


IMPORTANT: When using braces, each has to be at the beginning or end of the word
to combine words into single string value; otherwise, it is a regular character
as part of the word.

\subsection{File/folder}
\label{sec:tcl-file-folder}

\begin{verbatim}
//  ~ dir/ls
glob /path/to/folder/*

\end{verbatim}

\subsection{I/O strings}
\label{sec:tcl-IO}

Tcl accepts no quotes because all words, including names of commands, are just
strings.

\begin{verbatim}
//  ~ echo   (puts = put string)
puts Hello

puts "Hello World"
puts {Hello World}

\end{verbatim}

\subsection{Variable: assignment}
\label{sec:tcl-assignment}

We use \verb!set! command with 2 arguments

\begin{verbatim}

set foo good\ afternoon
set foo {good afternoon}
set foo "good afternoon"
\end{verbatim}

IMPORTANT: When \verb!set! command has only 1 argument; it returns the
argument's value (if present; otherwise return an error). 

\subsection{Get variable's value or command's return value}

We use \verb!$! sign before variable's name for variable substitution
(replaced by its value). We can also use {\bf command substitution} using square
bracket (replaced by its return value).

\begin{verbatim}
puts $foo

puts [set $foo]
\end{verbatim}

IMPORTANT: When \verb!set! command has only 1 argument; it returns the
argument's value (if present; otherwise return an error). 


\section{Define new commands}
\label{sec:tcl-define-new-commands}

Similar to \verb!function! keyword in Bash, in Tcl, we ue \verb!proc! keyword

\begin{verbatim}
//no argument
proc greet{}
{
  puts Hello!
}

// with single argument
proc greet name
{
  puts "Hello, $name!"
}


// with multiple arguments
proc greet {name, salute}
{
  puts "$salute, $name!"
}


// with variable number of arguments
// the last one must name 'args
// to traverse the list, use 'foreach' loop
proc greet {greeting args} {
    foreach name $args {
        puts "$greeting, $name!"
    }
}

// with returned value
// NOTE The last command is returnd as the value of a procedure (IMPLICIT)
// use 'return' statement  (EXPLICIT)
proc commbine {part1 part2}
{
  set result {$part1$part2} 
}
proc commbine {part1 part2}
{
  return {$part1$part2} 
}
\end{verbatim}

\section{loops: foreach}
\label{sec:tcl-foreach}
\label{sec:tcl-loop-foreach}

\begin{verbatim}
proc greet {greeting args} {
    foreach name $args {
        puts "$greeting, $name!"
    }
}
\end{verbatim}



\section{Helps}


\url{http://wiki.tcl.tk/15009}


\section{tclx}
\label{sec:tclx}

\begin{framed}
TclX provides new operating system interface commands, extended file control,
scanning and status commands and many others. Considered by many to be a
must-have for large Tcl apps. 

TclX 8.4 is installed into /usr/lib/tclx8.4.
\end{framed}

\begin{verbatim}
/usr/lib/tclx8.4/tcllib.tcl
\end{verbatim}

\section{tcllib}
\label{sec:tcllib}

\begin{verbatim}
http://wiki.tcl.tk/12098
\end{verbatim}

