\chapter{Android ROM}


This chapter we explains the difference between different ROMs for Android
phone, how to build it, and how to load onto the phone.

Unlocking your bootloader (Sect.\ref{sec:Unlock_Hboot}) doesn't root your phone
directly, but it does allow you to root (Sect.\ref{sec:Rooting}), then flash
custom ROMs (Sect.\ref{sec:flash-custom-ROM}) if you so desire.

There are three important things to know
\begin{itemize}
  \item Bootloader: On an Android phone this is the so called HBoot
  
  From here you can go to the recovery, system or data partitionn.  By pressing
  the power button on your phone, HBoot will load the OS into RAM. By pressing
  the power and volume down buttons, you'll bring up the HBoot menu.

  \item Android ROM: specific for each device
  
  \item Radio image: very critical, and recommended not to touch it
\end{itemize}


The kernel for the Android ROM is a {\bf hybrid kernel}, it is based on the
Linux kernel. Devices can differ in RAM memory, ROM memory, hardware parts and so on.
So it's really important you have a kernel for your type of device, an HTC
Wildfire kernel won't work on a Nexus One for example.

It is possible to overclock the kernel. When you overclock the kernel, the CPU
will be instructed to do more calculations per second; so, it will increase
performance.
However, keep in mind that this will degenerate your CPU much faster than if it
was at stock. To overclock an Android phone you must root it and install SetCPU
or another overclocking app from the Play Store. Then you will have to flash a
kernel that supports overclocking. I prefer the OC Kernel of HCDR.Jacob, at
XDA-forums (see links right). If you have done these 3 things you're ready to
OC!

Some modded recovery software (e.g. ClockworkMod) allows you to flash a new
kernel, radio image, custom ROM, Nandroid backup, install apps.

Radio is the lowest part of software layer: this is the very first thing that
runs, just before the bootloader. This handles the GPS antenna, GSM antenna and
fires up the CPU; everything that HBoot needs to load the OS.
This can also be upgraded by flashing a new radio image through your recovery.
However, this is not recommended: flashing a new ROM can't brick your phone
(make it unusable), but if anything goes wrong when you flash a ROM with a new
radio image, this can brick your phone. So unless you experience bad reception
or battery drainage, don't touch the radio!



