\chapter{LLVM}
\label{chap:LLVM}

\url{http://gnuu.org/2009/09/18/writing-your-own-toy-compiler/}

When writing a compiler, you need to know:
BNF grammars, AST data structures and the basic compiler pipeline.

\section{BNF grammar}
\label{sec:BNF_grammar}

\section{AST data}
\label{sec:AST-data-structure}

Abstract syntax trees (AST) are data structures widely used in compilers, due to
their property of representing the structure of program code. 

\section{basic compiler pipeline}


Assembly language produces more abstraction than machine code on the same
architecture. However, as assembly code are also machine-dependent, 
it has to be modified or rewritten if the program is to be executed on different
computer hardware architecture.

The first machine-independent high-level programming language is FORTRAN.
A {\bf compiler} is thus needed to translate the high-level source programs
into target programs in machine languages for the specific hardware. 

A compiler needs to do a lot of things:
\begin{enumerate}
  \item verify code syntax
  
  \item generate efficient object code (*.obj, *.o), depending on optimization
  level (-O0, -O1, -O2, \ldots)
  
  \item performs run-time organization
  
  \item format the output according to assembler and linker conventions.
\end{enumerate}
The compiler is thus divided into 3 parts:
\begin{itemize}
  \item {\bf front-end}: lexical analysis (break the code into sequence of
  tokens, i.e. meaningful character of strings), syntax analysis (the sequence of
  tokens is then matched with the language grammar - in BNF
  (Sect.\ref{sec:BNF_grammar}) - this is done by the parser), and semantic
  analysis (check for correct usage of elements in the programs and the language).
  
  lexical analysis is done by a {\bf lexical analyzer} (scanner, lexer,
  tokenizer) with the outputs are sequence of tokens. A {\bf parser} is then use
  the result of lexical analysis to verifies syntax (type checking) and
  semantics (Generates errors and warning, if any, in a useful way), the output is usually in AST data structure
  (Sect.\ref{sec:AST-data-structure}).
  
%   , Performs type
%   checking by collecting type information (for static-typed languages). 
%   AST data is used intensively during semantic analysis
  
  Output: AST is used to generate intermediate representation (IR) for
  processing by the middle-end.
  
  
  
  \item {\bf middle-end}
  
  \item {\bf back-end}: IR analysis (at basic block (function level) or whole
  program (interprocedural level), optimization (IR is transformed into functionally
  equivalent but faster (or smaller) form), and code generation.
  
  The two steps IR analysis and IR optimization are often skipped by some
  compilers by default. This leads to the confusing that back-end only does
  machine-dependent code generation.
  
  The open-source GCC was critized a long time for lacking the first two phases,
  yet this is changing. Another powerful open-source with full analysis and
  optimization infrastructure is {\bf Open64}; however it was discontinued since
  2011. \url{http://en.wikipedia.org/wiki/Open64}
  
  Commercial-compilers do a better job in interprocedural analysis and
  optimization such as HP, IBM, SGI, Intel, Microsoft and Sun compilers.
  
  Code generation: Generates the assembly code, performing register allocation
  in process; Optimizes target code utilization of the hardware by figuring out
  how to keep parallel execution units busy, filling delay slots
  
  NOTE: Most algorithms for optimization are in NP, heuristic techniques are
  widely used to improve the optimization.
\end{itemize}


\section{LLVM}
\label{sec:LLVM}

The Low Level Virtual Machine (LLVM) is a compiler infrastructure, written in
C++, which is designed for compile-time, link-time, run-time, and "idle-time"
optimization of programs written in arbitrary programming languages.

Originally implemented for C/C++, the language-independent design (and the
success) of LLVM has since spawned a wide variety of front-ends, including Objective C, Fortran, Ada, Haskell, Java bytecode, Python, Ruby, ActionScript, GLSL, and others.
