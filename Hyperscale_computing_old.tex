\chapter{Hyperscale computing}

\section{Traditional data center}
\label{sec:traditional-datacenter}

A traditional data center has its root using huge computer rooms. With the
advances of personal computer's computing and low-cost high-speed networking
equipments, it's now possible to setup servers in every company. 

There are 4 levels (tiers) of data centers, defined by  Telecommunications
Industry Association (ANSI/TIA-942, first published in 2005, then amended in
2008 and 2010)
\begin{itemize}
  \item Tier 1: simplest a server room
  \item 
  \item 
  \item Tier 4: the most stringent, designed to host mission critical computer
  systems, with fully redundant subsystems and compartmentalized security zones
  controlled by biometric access controls methods.
\end{itemize}

These data centers employ purpose-built name-brand servers from tier-one vendors
with storage area networks  running Windows or Linux. A data center can occupy
one room of a building, one or more floors, or an entire building. Most of the
equipment is often in the form of servers mounted in 19 inch rack cabinets. 
Servers differ greatly in size from 1U servers to large freestanding storage
silos which occupy many square feet of floor space. 

\textcolor{red}{Communications in data centers today are most often based on
networks running the IP protocol suite}. Data centers contain a set of routers
and switches that transport traffic between the servers and to the outside
world. Redundancy of the Internet connection is often provided by using two or
more upstream service providers. Some of the servers at the data center are used
for running the basic Internet and intranet services needed by internal users in
the organization, e.g. mail server, proxy server, DNS server.

Network security elements are also usually deployed: firewalls, VPN gateways,
intrusion detection systems, etc. \url{http://en.wikipedia.org/wiki/Data_center}

The key advantage of such an architecture is that it's well known and IT
staffers have extensive experience with it. In terms of networking, traditional
servers are connected via commodity network switches running 10 Gigabit Ethernet
(GbE) or InfiniBand. Servers and switches use proprietary management software
but can be easily upgraded or swapped out for other vendors' gear. Each
equipment rack has its own network switch, and these switches are connected to
an overall backbone core switch.  


\section{Hyperscale data centers}

% As mentioned above (Sect.\ref{sec:traditional-datacenter}), a traditional data
% centers has racks, data centers had separate racks, staffs and management tools
% for servers, storage, routers and other networking infrastructure? 

A hyperscale server is a new kind of servers that are customized for particular
data center needs.
The idea behind building a hyperscale architectures is to start small in order
to keep upfront investments as low as possible. Then as demand grows, the
infrastructure should be able to expand simply by adding nodes to the cluster.  

They are also assembled from common components that can be
easily swapped out when failures occur. This is the architecture of choice in
100\% cloud-based businesses such as Facebook, Amazon and Google, and also for
building a new breed of supercomputers.

\begin{itemize}
  \item automated load balancing: When nodes are added, the hyperscale software, in most cases, will then
re-position or auto-balance workloads to the newly added nodes. 
  
  \item hyperscale storage: choosing between using SATA, SSD, PCIe SSD or
  combined to balance the cost, speed, and effectiveness.
  
  \url{http://www.storage-switzerland.com/Articles/Entries/2013/5/20_What_Is_A_Hyperscale_Data_Center.html}
\end{itemize}

A single Hyperscale server node will be formed from at most three pieces:
\begin{itemize}
  \item SoC (server-on-chip): CPU, GPU, I/O, fabric interconnect, management
  controller
   
  \item RAM: 
  
  \item Storage:  individual flash will combine with virtualized
  fabric-distributed I/O
  \url{http://hyperscalecomputing.org/2012/06/30/introducing-hyperscale-computing/}
\end{itemize}
Typically, these servers run some form of Linux and are now sold by both
traditional server vendors such as Hewlett-Packard, with its ProLiant DL2000
Multi Node Server, along with components available from several suppliers. 

Open Compute Project,  which was founded by Facebook to promote standardized
hardware for web-scale data centers, has led to rapid innovation in the server
market and has also developed a storage offering. 
The goal is to help  software-defined networking continue to evolve and
flourish. The advantage of these new server and network designs is that you can
reduce power losses with direct current (DC)-powered drives and save time on
troubleshooting failed components, since every server is uniform. You can also
scale up capacity in smaller increments.  


\url{http://searchdatacenter.techtarget.com/feature/Hyperscale-data-center-means-different-hardware-needs-roles-for-IT}


\section{Metal as a Service (MAAS)}

As we move from "tens" to "hundreds" to "thousands" of nodes in a typical data
centre we need new tools and practices.  AAS - Metal as a Service - is bringing
that agility back to the physical world for hyperscale deployments.

Ubuntu 14.04 has beta version of MAAS 1.7
