\chapter{IRC networks}


{\bf Internet Relay Chat} (IRC) is an application layer protocol that
facilitates the transfer of messages in the form of text.

There are thousands of different IRC networks, covering all kinds of topics,
both private and public, that with an IRC client application, you can joint.
Each IRC network has thousands of channels, each channel covers one topic.

Current stats:
\begin{itemize}
  \item 400,000 in 2014, dropped 60\% from 1.million in 2003.
  
  \item 250,000 channels, dropped 50\% from 0.5million channels in 2003.
\end{itemize}
\url{http://royal.pingdom.com/2012/04/24/irc-is-dead-long-live-irc/}
  
Users access IRC networks by connecting a client (Sect.\ref{sec:IRC-clients}) to
a server (Sect.\ref{sec:IRC-servers}).

\section{IRC clients}
\label{sec:IRC-clients}


There are many client implementations, such as mIRC, HexChat and irssi.

Most IRC servers do not require users to register an account but a nick is
required before being connected. A user need to establish an IRC session to a
channel.
 
Clients send single-line messages to the server, receive replies to those
messages  and receive copies of some messages sent by other clients.

A user can enter a command by prefixing them with a '/'. 
\begin{itemize}
  \item LIST command: list all available channels that do not have  
  modes +s or +p set on that IRC network
  
  A user or a channel can have a mode,
  represented by a single case-sensitive character, and can be set using MODE
  command
  
  \item MODE command: 
  
  \item JOIN command: to join a channel
\begin{verbatim}
/join #channelname
\end{verbatim}  
Channels that are available across an entire IRC network are prefixed with a
\verb!'#'!, while those local to a server use \verb!'&'!. Other types of
channels: \verb!'+'! (modeless channel without an operator), \verb!'!'! channel
(timestamped channel on a normally non-timestamped IRC network)
  
\end{itemize}

\section{IRC servers}
\label{sec:IRC-servers}

Server implementations:
 IRCd.
 
\section{Future of IRC}

\subsection{KVIrc}

KVIrc, which brought video chat to traditional IRC, 

\subsection{Konversation}

Konversation, with which several IRC users can share a virtual whiteboard.
