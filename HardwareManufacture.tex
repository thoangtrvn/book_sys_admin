\chapter{Hardware Manufacture}

\section{Microfabrication}

Microfabrication is the process of fabricating miniature structures of
micrometre scales and smaller. 
\begin{enumerate}
  \item IC (integrated circuit) (semiconductor device fabrication)
\end{enumerate}
\url{https://en.wikipedia.org/wiki/Microfabrication}


\subsection{Photolithography (UV lithography)}
\label{sec:photolithography}

Photolithography, also termed optical lithography or UV lithography, is a
process used in microfabrication. It uses light to transfer a geometric pattern
from a photomask to a surface holding light-sensitive substrate.

A series of chemical treatments then either engraves the exposure pattern into,
or enables deposition of a new material in the desired pattern upon, the
material underneath the photo resist.

When shootting light, the light intensity is amplified via a number of lens, at
going through the tiny air-gap between the last lense before reaching the wafer
surface (in the case of IC manufacturing).


\subsection{Next-generation lithography (NGL)}

Next-generation lithography (NGL) is supposed to replace photolithography
(Sect.\ref{sec:photolithography}).

\label{sec:immersion-lithography}
\begin{enumerate}
  \item  immersion lithography (most advanced form in 2016)	: the air-gap is
  replaced by a liquid medium, i.e. refactory index > 1.0, which means higher
  resolution. The resolution is increased by a factor equal to the refractive
  index of the liquid
  
 ASML, Canon, and Nikon are currently the only manufacturers of immersion
 lithography system.

  \item  NOTE: highly purified water gives feature sizes below 45nm.
 
    
  \item Currently, the most promising high-index lens material is lutetium aluminum
  garnet, with a refractive index of 2.14.

\end{enumerate}


\subsection{EUV (Extreme ultraviolet lithography)}

Beside using liquid to replace air in the gap between wafer surface and last
lense, they also try to increase the intensity of the light using extreme
ultravilet (13.5nm). 

EUVL is a significant departure from the deep ultraviolet lithography standard.
As all matter absorbs EUV radiation. Hence, EUV lithography requires a vacuum.
Since EUV is highly absorbed by all materials, even EUV optical components
inside the lithography tool are susceptible to damage.

An EUV dose of 1 mJ/cm$^2$ generates an equivalent photoelectron dose of 10.9
$\mu$C/cm$^2$.
\url{https://en.wikipedia.org/wiki/Extreme_ultraviolet_lithography}


All optical elements, including the photomask, must use
defect-free Mo/Si multilayers.

The major challenge is the high power required to generate EUV, on the order of
$10^{11}$ W/cm$^2$, compared to the  state-of-the-art 193 nm ArF excimer lasers
of 200 W/cm$^2$.

The primary EUV tool maker, ASML.

Immersion lithography (Sect.\ref{sec:immersion-lithography}) is still roughly 3
times faster than EUV, due to source power limitations.


\section{Wafer size (diameter)}

The wafer serves as the substrate for microelectronic devices built in.
Finally the individual microcircuits are separated (dicing) and packaged.


Wafer are formed of nearly defect-free single crystalline material

Wafer comes with different size
\begin{enumerate}
  \item 25.4 mm (1 inch)
  
  \item 300 mm (11.8 inches): current state-of-the-art fab (thickness 775$\mu$m) 

Adopted in 2000, and reduced price per die about 30-40\%; though it came with
many problems.
  
  \item 450mm (17.7 inch): at prototype with thickness 925$\mu$m.
  
The purpose of larger wafer is that it can  produce more chips proportional to
the increase in wafer area, i.e. to support bigger chip size, more chip per
wafer (higher chip throughput, i.e. lower chip cost)

\url{http://www.lithoguru.com/scientist/essays/why450.html}
\url{http://anysilicon.com/does-size-matter-understanding-wafer-size/}

Higher cost semiconductor fabrication equipment for larger wafers increases the
cost of 450 mm fabs.
\end{enumerate}

NOTE: Wafers grown using materials other than silicon will have different
thicknesses than a silicon wafer of the same diameter. The wafer must be thick
enough to support its own weight without cracking during handling.


Wafers are cleaned with weak acids to remove unwanted particles, or repair
damage caused during the sawing process. 
