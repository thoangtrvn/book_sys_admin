\chapter{Build a HPC cluster}


You need
\begin{enumerate}
  \item a server node (head node - Sect.\ref{sec:node-master}): which serves as
  (1) login node; (2) job management (launch job)
  
  
For small cluster, it can be used as: (3) file server; (4) computation node
  
  \item many computation nodes:
  
  
  \item data server (optional): don't run applications; rather, they store and
  serve data to the rest of the cluster. 
  
For small cluster, it is part of the server node.
  
  \item visualization node: provides data visualization capabilities within the
  cluster (usually remote visualization), 
  
  \item monitoring node: (very large clusters might need) nodes dedicated to
  monitoring the cluster or to logging in users to the cluster and running
  applications.
   
  
\end{enumerate}

\section{Power 8}

Power 8 machines needs Hardware Management Console (HMC) version 8 release
8.1.0 (Sect.\ref{sec:HMC}).  


\subsection{Hardware Management Console (HMC)}
\label{sec:Hardware-Management-Console}
\label{sec:HMC}


An HMC is necessary only to manage Power Systems servers.

An HMC appliance runs a heavily customized version of Linux provided by IBM.
Access to the \verb!root! userid is disabled by default. The \verb!hscroot!
userid has all the necessary privileges to manage and use the HMC.





\url{https://www.ibm.com/developerworks/community/wikis/home?lang=en#!/wiki/Power Systems/page/Hardware Management Console}

\section{Hardware connection}

This steps discuss how to put all nodes into a full rack of equipment, neatly
wired and plugged into power.

\begin{enumerate}
  \item Power 8:
  \url{https://www.ibm.com/support/knowledgecenter/POWER8/p8hdx/p8_cif_8247_21l.htm}
  
  
  \item Intel x86: 
\end{enumerate}
\section{IP configuration (static or dynamic)}


\begin{enumerate}
  \item Microsoft Windows server cluster: static IP is recommended as best
  practice
  
  \url{https://technet.microsoft.com/en-us/library/aa997724(v=exchg.80).aspx}
  
  
  \item 
\end{enumerate}

\section{Storages}


\begin{itemize}
  \item shared-disk cluster filesystem:
  many instances of one filesystem, each instance is distributed on one compute
  node
  
  
  \item parallel filesystem:  only one instance of the filesystem, and site on
  the same node of storage devices. Only thin client of the filesystem 
  
  
  \item
\end{itemize}


\subsection{shared-disk cluster filesystem}
\label{sec:shared-disk-cluster-filesystem}

There is a common pool of storage, in a dedicated node, that is shared among the
servers hosting the distributed application with an instance of the file
system running on each processor and a side band communication channel between
file system instances used for distributed lock management.

By creating a series of independent channels, each instance of the distributed
file system is given direct access to the shared disk storage system.

The file system places a certain load on the storage system which is equal to
the sum of the loads placed by each instance of the file system. This load  is
equally distributed through the independent channels.
At one time, one instance of the file system get the lock, to do write to the
storage.
The performance of a shared disk file system and hence the overall system is
dependent on the speed with which locks can be exchanged between file system
instances.



\subsection{parallel filesystem (e.g. Lustre, pfgs)}
\label{sec:parallel-filesystem}

A parallel file system such as the Lustre file system is similar in some ways 
to a shared disk file system in that it is designed to satisfy storage demand 
from a number of clients in parallel, but with several distinct differences. 

Within the storage system, the control plane ({\bf file system metadata}), is
often separated from the data plane ({\bf object servers}) containing user data.
This promotes parallelism by reducing bottlenecks associated with accessing file
system metadata, and it permits user data to be distributed across a number of
storage servers which are devoted to strictly serving user data.

The file system however, is sited together with the storage 
devices rather than being distributed; and thin filesystem-clients 
co-located with the distributed applications will create the connection to the
metadata server(s) and the object servers.

Each thin filesystem-client creates a unique channel to the file system
metadata. This accelerates the metadata look-up process and allows the metadata
server to service a fairly large number of accesses in parallel, which is
better than the distributed lock system in shared-disk cluster filesystem
(Sect.\ref{sec:shared-disk-cluster-filesystem}).


\subsection{network-attached storage (NAS) filesystem}

\section{Server Virtualization}

It is not uncommon for data center servers to consume less than 15-20 percent of
the available CPU cycles. As many data center servers today operate at very low
levels of CPU utilization, it is a better strategy to host multiple virtual
machines (VM) on the same physical machine (server).

\section{Install master node or admin node or log-in node (on-site)}
\label{sec:node-master}

The admin node will also act as a storage and Kickstart installation server
for the cluster. 

Before you go to the server room, the first step is to download and burn the
\begin{itemize}
  \item CentOS install DVD media for 64-bit
  
  
  \item Ubuntu server install DVD 64-bit 
\end{itemize}
This acts as a kickstart installation server to automate the install of all the
compute nodes that fill out the rest of the cluster.

The admin/storage node and the head/login node both have multiple Ethernet
interfaces, e.g. 2 NIC cards:
\begin{itemize}
  \item one (eth0) to private cluster: to private, internal gigabit switch of
  192.168.1.0 network.

NOTE:  baseboard management controller (BMC/DRAC) boards, which are controllable
via IPMI [7], are configured to share the first Ethernet port. This means you
might see two different MAC addresses and two different IPs on the eth0 port;
you want both facing internally.


  \item one (eth1) connects to public network (remote user and admin login)

Only the master node should have the public IP address, because there is no
reason for compute nodes to have two addresses.
  
\end{itemize}
public IPs: 72.x.x.x and the private address something like 10.x.x.x. 

\subsection{O/S install}

Hook keyboard-video-mouse (KVM) switch to the admin node and powering on.
\begin{itemize}
  \item  Check BIOS settings, set a password, and choose the boot order. 
  
  \item Configure the BMC, setting an admin username and password and
  configuring an internal IP address for the device.
\end{itemize} 


After full RAID initialization, power off the node and unplug the external RAID
enclosure before the OS install because the OS install partitioner can have
difficulty with very big disks.


Insert the CentOS installation DVD into the system that will act as the admin
node. Boot the installer and select mostly all defaults. The only steps that
differ from the defaults are setting a bootloader password, configuring the two
network interfaces, and switching SELinux to permissive mode.



Shared directory across the nodes: NFS (Sect.\ref{sec:NFS}).
\begin{itemize}
  \item classic approach: to export a directory from the master node to the
  compute nodes
  
E.g.: mount '/shared' as '/home' on all machines, including master node.

  \item
\end{itemize}

Shared name information, e.g. NIS (Sect.\ref{sec:NIS}):  you need the same users
and groups on the nodes


Configure libraries on shared too, i.e. make a partition as /libraries, and
mounted as '/devops' on on machines
\begin{itemize}
  \item MPI libraries: openMPI, MPICH
  
  \item SSH
\end{itemize}


\section{Install computation node (remotely)}

A computation node runs the bare minimum OS - meaning that unneeded daemons are
turned off and unneeded packages are not installed - and have the bare minimum
hardware.

The servers must be capable of booting from the network with PXE [6]. PXE (which
stands for Preboot eXecution Environment") lets you install and configure the
compute nodes without having to stop and boot each one from a CD.

\section{Storage node}

For a small to medium-sized cluster, it is okay to combine what is often called
the storage node with the admin node.


\section{N-rack system}

The process needs to be replicated N times for a larger cluster of $N$ racks.


