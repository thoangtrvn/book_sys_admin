\chapter{MPL MIP405 + MIP405T}
\label{chap:embedded_MPL_MIP405-MIP405T}

MPL MIP405 and MIP405T are both embedded systems
(Sect.\ref{sec:embedded-system}) which provides the hardware needed to run as a
stand-alone system such as RAM, CPU, I/O ports, Ethernet port, a pre-installed
boot-loader (Sect.\ref{sec:MIP405_MIP405T_specs}).


The required setting to be able to 

\section{MIP405 vs. MIP405T}
\label{sec:MIP405_MIP405T}

\subsection{Specifications}
\label{sec:MIP405_MIP405T_specs}

MIP405/MIP405T are highly integrated and robust
PC/104-Plus Single Board Computers, which use/have
\begin{enumerate}
  \item Processor: extending x86 MIP family with either
  IBM PowerPC 405GPr 608 DMIPS upto 400 MHz or 
  PPC405GP with 375 DMIPS at 266MHz 
  
  IBM PowerPC 405GPr 608 DMIPS: 405GP processor maintains code compatibility
  with other PowerPC processors.
  \url{https://linuxlink.timesys.com/products/amcc_powerpc_405gp}
  \url{http://archive.linuxgizmos.com/powerpc-405-evaluation-kit-available/}
  
  \item RAM: upto 128MB with ECC (soldered onboard)
  
  \item up to 4MB on-board soldered Flash
  
  \item multi purpose socket SRAM, EPROM, and Flash modules.
  
  \item O/S: various boot source (USB, IDE, Flash\ldots) and the O/S can be real
  time Linux, various Linux distribution, open source U-boot loader for other
  O/S (Sect.\ref{sec:U-boot_loader}).
  
  \item USB: 2
  
  \item 10/100 Ethernet port: 1
   
\end{enumerate}
\url{http://www.mpl.ch/DOCs/MPLdoc_00000148.pdf}

Application: hosting applications for medical, transportation systems, telecom,
airborne systems or any other rugged and/or industrial application.


MIP405 has 5 variants:
\begin{verbatim}
MIP405-1: 266 MHz CPU, 4MB Flash, 2x USB + 4xRS232 on header, 10/100 Ethernet on RJ45,
MIP405-3: 400 MHz CPU, 8MB Flash, 2x USB + 4xRS232 + 10/100 Ethernet on header
MIP405-4: 266 MHz CPU, 4MB Flash, 2x USB + 4xRS232 + 10/100 Ethernet on header
MIP405-x
XTEST-1 : extended operating temperature (-40 to +85 degree celcius)
\end{verbatim}
\url{http://www.mpl.ch/t2720.html}

MIP405T is a further development of the MIP405, even more focussed on special
requirements like small size, wide temperature range and ruggedness. MIP405T has
4 variants
\begin{verbatim}
MIP405T-1 : ... 32MB SDRAM and 4MB Flash
MIP405T-2 : ... 128MB SDRAM and 8MB Flash
MIP405T-x
XTEST-1
\end{verbatim}
\url{http://www.mpl.ch/t2740.html}


\subsection{Setting up MIP405/MIP405T}

It is important that you understand the Linux booting process
(Sect.\ref{sec:boot-process-Linux-kernel}). 
In order to work with
MIP405/MIP405T, we need 3 things:
\begin{enumerate}  
  \item Linux kernel image: we need to build an image of the suggested kernel
  version (Sect.\ref{sec:MIP405T_Linux2.6.x}) using the proper cross-compile tool-chain

Here, we use MIP405T with Linux kernel 2.6.12
(Sect.\ref{sec:MIP405T_Linux2.6.x}).
Output of building Linux kernel: build the \verb!uImage!.

Example: This is the Linux kernel 2.6.12 image built using the crosstool-ng
1.9.3 (Sec.\ref{sec:compressed-kernel-file-format})
\begin{verbatim}
ls ~/Tuan/linux-2.6.12_crosstool_1.9.3./arch/ppc/boot/images/
   uImage
   vmlinux.bin
   vmlinux.gz
   zImage.elf
   
devel@pinky:~/Tuan/linux-2.6.12_crosstool_1.9.3$ ls arch/ppc/boot/images/ -lh
total 6.6M
-rw-r--r-- 1 devel devel  835 2015-03-05 12:09 Makefile
-rw-r--r-- 1 devel devel 1.3M 2015-03-05 12:13 uImage
-rwxr-xr-x 1 devel devel 2.8M 2015-03-05 12:13 vmlinux.bin
-rw-r--r-- 1 devel devel 1.3M 2015-03-05 12:13 vmlinux.gz
-rwxr-xr-x 1 devel devel 1.4M 2015-03-05 12:13 zImage.elf   
\end{verbatim}

At this step, we have \verb!arch/ppc/boot/images/uImage! is the compressed and
U-bootified Linux kernel ready to use.
  
  \item (optional): build the loadable kernel modules (LKMs), and some other
  modules to run on the target machine using the same tool-chain
  
  \item create a root filesystem(Sect.\ref{sec:root-file-system}): the
  complexity of a root filesystem varies, and put the built Linux kernel into
  this RFS, along with some other kernel modules, extra modules, \ldots

Make sure the RFS contains the BusyBox (Sect.\ref{sec:create-RFS-BusyBox})
  
  \item a boot loader (Sect.\ref{sec:bootloader}): 
  MIP405/MIP405T comes with a U-boot bootloader (Sect.\ref{sec:U-boot_loader}.
  
%   For embedded typically that's
%   U-Boot, although not a hard requirement.
  You can update a newer version of U-boot loader, or use the existing one on
  the device.
\end{enumerate}
\url{http://unix.stackexchange.com/questions/136278/what-are-the-minimum-root-filesystem-applications-that-are-required-to-fully-boo}

To configure MIP405T to use the new Linux kernel and/or root filesystem,
connect MIP405T to your  host machine using the proper tool such as minicom
(Sect.\ref{sec:minicom}); then configure it to find the root file system
(which is located on your host machine). NOTE: during testing you want to put
it on your host system for quickly change; and once it is stable, you can
decide to move the whole RFS to the embedded system.

You have a few options:
\begin{enumerate}
  \item Connect to the MIP405T with the host machine using COM port, and run
  minicom (Sect.\ref{sec:minicom}) to communicate

NOTE: MIP405T is a headless device, so there is no graphics port.  
  
  Once the device is one, it shows a command-prompt, where we can 
in teract with the device using U-boot commands
(Sect.\ref{sec:Uboot-commands}), which we use to configure (1)
host name, IP, subnet mask (make sure in the same network with the host
machine so that it can read the RFS on the host machine), (2) where to look for
root filesystem (which can be networked mount).
  
  \item For final product: put the U-bootable (RFS or RFS+kernel) image file 
  onto the MIP405 FLASH ROM.
  
You can download the initial ramdisk image to MPL MIP405T device using one of
the way support by U-boot, e.g. via Ethernet connection
(Sect.\ref{sec:Uboot_MIP405/MIP405T}).
  
OK, now you have two files
\begin{verbatim}
<Linux-kernel-2.6-source-folder>/arch/ppc/boot/images/vmlinux.gz
/home/devel/mip405t_rfs.gz
\end{verbatim}
As the target device comes with the open-source U-boot loader
(Sect.\ref{sec:U-boot_loader}), we may need to create the U-boot bootable image
file. There are 2 supported formats: {\it legacy format} and {\it FIT format}
(Flattened Image Tree). Here, we use legacy format.
We can choose to build rom the RFS file only, or RFS + Kernel files.
Both options need to use \verb!mkimage! command from U-boot
(Sect.\ref{sec:mkimage}) to create a U-bootable image.
\begin{itemize}
  \item RFS image only

\begin{verbatim}
$> cd <Linux-kernel-2.6-source-folder>
$> mkimage -A ppc -O linux -T ramdisk -C gzip -n "MIP405T RFS"  \ 
          -d /home/devel/mip405t_rfs.gz mip405t_rfs.img
Image Name:   MIP405T RFS
Created:      Fri Mar  6 14:01:12 2015
Image Type:   PowerPC Linux RAMDisk Image (gzip compressed)
Data Size:    1031501 Bytes = 1007.33 kB = 0.98 MB
Load Address: 0x00000000
Entry Point:  0x00000000

\end{verbatim}

  \item RFS + kernel image: NOTICE : we need to use the non U-bootified compressed 
  kernel image \verb!./arch/ppc/boot/image/vmlinuz.gz! (not the uImage file)
  
\begin{verbatim}
$> cd <Linux-kernel-2.6-source-folder>
mkimage -T multi -n "MIP405T (Kernel + RFS)" \
    -d arch/ppc/boot/images/vmlinux.gz:/home/devel/mip405t_rfs.gz \ 
    mip405t_multi.img
\end{verbatim}  
\end{itemize}
The command creates an ramdisk image that can be copied to the SDRAM in the
embedded system.

  
  
  
  
\end{enumerate}  
% 
% Here, we explain how to build an image that can be loaded by U-boot.
% Before you can do that, you need to build the Linux kernel
% (Sect.\ref{sec:Linux_kernel_PowerPC}) and then build the RFS
% (Sect.\ref{sec:create-RFS-BusyBox}).


% \textcolor{red}{NOTE:} \verb!uImage! is the compressed and U-Bootified Linux kernel
% and are ready to use. 
%The output built kernel image is in




\subsection{Boot MIP405/MIP405T with U-boot}
\label{sec:Uboot_MIP405/MIP405T}
%\subsection{U-boot command on MIP405/MIP405T}

MPL MIP405/MIP405T devices comes with a pre-installed U-boot boot loader
(Sect.\ref{sec:U-boot_loader}).

U-boot is pre-installed on the . So
from that device, we can download the kernel image via Ethernet connection.

\textcolor{red}{NOTE}: As MIP405/MIP405T is a headless system
(Sect.\ref{sec:headless_system}), you need to connect the system with the laptop
that has a display via COM connection. The laptop should runs a Linux-based
O/S. The program from the laptop that can display the output from the
MIP405/MIP405T is minicom (Sect.\ref{sec:minicom}).

The configuration
\begin{verbatim}


MIP405T  --------[Ethernet]---> router
  |                               /
  |  (COM)                       /
  |                            /
Laptop  --------[Ethernet]---/ 
\end{verbatim}

Once the system boot-up you need to configure the IP address, and other settings
so that it knows how to load the Linux kernel + root filesystem (RFS).
Suppose we use the Kernel+RFS U-bootable ramdisk image file
(\verb!mip405t_kernel-rfs.img!). 

\textcolor{red}{First, configure the host machine (laptop):}
\begin{enumerate}
  \item copy \verb!mip405t_kernel-rfs.img! to \verb!/tftpboot! directory

\begin{verbatim}
# either copy uImage
host$ cp <path to uImage file> /tftpboot

# or copy
host$ cp <path to ramdisk file> /tftpboot
\end{verbatim}  
  \item 
\end{enumerate}

\begin{mdframed}
U-Boot implements \verb!tftp! command to download the kernel and filesystem (in
case of ramdisk) images to SDRAM.
\end{mdframed}

\textcolor{red}{Now, you configure the embedded system}: (for list of commands
check Sect.\ref{sec:Uboot-commands}) Example: from the U-boot command line of
the target device, type the commands
\begin{verbatim}
 # tell the embedded system know where the IP of the host machine
setenv serverip <IP-server-containing-kernel-image>

 # choose option 1: DHCP
 setenv autoload no
 dhcp 
 # option 2: static IP for embedded system
 setenv ipaddr <ip-of-this-MIP405T-device>
 setenv netmask xxx.xxx.xxx.xxx
 saveenv
  
 # (optional) set MAC address
#setenv ethaddr C0:B1:3D:88:88:89 
\end{verbatim}

You can choose to load the ramdisk image from the host (laptop) via Ethernet, 
using \verb!tfpboot! command (which alreadys implement a downlaod capability
over Ethernet using TFTP protocol)

\textcolor{red}{OPTION 1:} Load from the host machine
\begin{verbatim}
set serverip 10.0.0.20
set path2rfs /srv/nfs/test_fileystem
set console ttyS0,115200n8
setenv bootargs console=${console} root=/dev/nfs rootfstype=nfs
ip=dhcp nfsroot=${serverip}:${path2rfs} # The root=/dev/nfs directive tells Linux to instantiate with the virtual
 # device, /dev/nfs, as the root filesystem
 # The rootfstype=nfs directive tells Linux that the root filesystem is of the
 # NFS variety

set uImage mip405/cdsstack_select
set loadaddr 400000
 
  # Now we can load the ramdisk image
  # This line tells U-Boot to load the kernel from the TFTP server, whose IP
  #address was set earlier with the set serverip 10.0.0.20 command.
  #  it stores it in the memory location of the embedded system 
  #  pointed to by  loadaddr 
tftp ${loadaddr} uImage


 # synax:
 #     bootp [loadAddress] [[hostIPaddr:]bootfilename]
 # NOTE: If you do not specify a load address, then the value will be taken from
 # the loadaddr environment variable
 # 0x40000 is the address for the PowerPC architecture
 # here we download over the network
 # the name can be 'bootp' or 'tfpboot'
 #   bootp = boot image viaa network using BOOTP/TFTP protocol
 #   tfpboot = boot image viaa network using TFTP protocol
 
 # Once the transmission using tftpboot finishes, the file will be in memory at
 # the specified load address
 # the two commands are equivalent, as by default it wills look into
 #                ${serverip}/tftpboot/
tfpboot 40000 mip405t_kernel-rfs.img
  # or
tfpboot 40000 ${serverip}/tftpboot/mip405t_kernel-rfs.img
  # The loadaddr environment variable will automatically be set to the address
  # the tftpboot command used. The filesize environment variable will
  # automatically be set to the number of bytes transferred during the load
  # operation.
  
  
   # configure U-boot variables
 # e.g. 'bootargs'
 # to configure the network interface automatically at boot-time
 # replace 'ip=off' by 'ip=dhcp'
setenv bootargs root=/dev/ram0  console=ttyS0,9600 ip=off rw

// (check nfsroot.txt file in the kernel documentation

\end{verbatim}

\textcolor{red}{The BusyBox version of init can run quite happily without an
/etc/inittab, but again this may not work if the shell is running on
/dev/console.}


\textcolor{red}{Make sure you save the changes}
\begin{verbatim}
saveenv
\end{verbatim}

Now you are free to do whatever you like with the loaded image
\begin{verbatim}
# Finally we tell U-boot to boot from RAM with the address given above
# this also pass the command lines defined in 'bootargs' to the kernel
bootm ${loadaddr}
 
// now we can boot the downloaded kernel
bootm 400000  800000

 ## Booting image at 00400000 ...
Image Name: kernel for MIP405
 ## Loading RAMDisk Image at 00800000 ...
Image Name: initrd for MIP405
Created: 2002-04-16 13:17:42 UTC
\end{verbatim}
and finally a login prompt appears
\begin{verbatim}
(none) login:
\end{verbatim}
The default login is 'root' username and 'MIP405' password.
\url{http://www.mpi.ch/files/File/MPL/MIP405_Mini_Linux_RAMdisk.pdf}

Settings depending on the types of memory on embedded system (RAM, NAND Flash,
\ldots)
\url{http://processors.wiki.ti.com/index.php/Booting_Linux_kernel_using_U-Boot}

\textcolor{red}{OPTION 2:} Save the ramdisk image file permanently to the
embedded system
\begin{verbatim}
  # erase the flash
erase FFC00000   FFF7FFFF
#.....
# (erased 56 sectors)

  # now copy the image
cp.b 400000  FFC00000  37FFF0

  # now check the bootargs one-more-time
printenv bootargs
  # bootargs=console=ttyS0,9600 root=/dev/hda5
  # make sure change 'root=' to /dev/ram0
  #bootargs=console=ttyS0,9600 root=/dev/ram0
setenv bootargs console=ttyS0,9600 root=/dev/ram0

saveenv  
\end{verbatim}

Now, we can boot
\begin{verbatim}
bootm 400000
\end{verbatim}
\url{http://www.mpi.ch/files/File/MPL/MIP405_Mini_Linux_RAMdisk.pdf}

\subsection{Testing}


\url{http://wiki.emacinc.com/wiki/Booting_with_an_NFS_Root_Filesystem}
\url{http://raspberrypi.stackexchange.com/questions/628/how-do-i-configure-the-raspberry-pi-to-boot-with-an-nfs-root}

\url{https://www.redhat.com/archives/rhl-list/2005-March/msg04089.html}

\url{http://unix.stackexchange.com/questions/136278/what-are-the-minimum-root-filesystem-applications-that-are-required-to-fully-boo}

\section{Compile Linux kernel for MIP405/MIP405T}
%\subsection{Linux kernel for PowerPC}
\label{sec:Linux_kernel_PowerPC}

The general guide to build a Linux kernel is given in
Sect.\ref{sec:build_Linux-kernel}.
In this section, we focus on the Linux kernel target to MIP405 and MIP405T
embedded system.

\subsection{MIP405 + Linux kernel 2.4.x}

Requirements
\begin{itemize}
  \item PowerPC (cross)-compiler: recommended version 2.95.3
  
  Disable floating-point instruction (as they are not used in MIP405/MIP405T and is not used in the kernel).
  \verb!glibc! (Sect.\ref{sec:GLIBC}) is not used in the kernel too.
  
  \item Choose the appropriate Linux kernel version:  Sect.\ref{sec:Linux_kernel_PowerPC}
  
\end{itemize}


\subsection{MIP405T + Linux kernel 2.6.x}
\label{sec:MIP405T_Linux2.6.x}

MPL AG maintains a patch against the official Linux 2.6.12 kernel that contains
all MIP405T specific changes and drivers.

\begin{verbatim}
CONFIG_IBM_EMAC=y   # Ethernet 0
CONFIG_SERIAL_8250=y           # Both for Serial 0
CONFIG_SERIAL_8250_CONSOLE=y   
\end{verbatim}
\url{https://linuxlink.timesys.com/products/amcc_powerpc_405gp}

The following steps just copy what show in the earlier of
Sect.\ref{sec:Linux_kernel_PowerPC}, with a specific version information
\begin{itemize}
  \item Download the Linux kernel 2.6.x source: 

Fix the folder missing issue 
\begin{verbatim}
cd <linux-kernel-2.6-source-folder>

// then do one of this
// RECOMMENDED
cp -rf <full-path-linux-kernel>/include/asm-ppc \ 
       <full-pathlinux-kernel>/include/asm

// may not work
ln -s <full-path-linux-kernel>/include/asm-ppc \
      <full-path-linux-kernel>/include/asm
\end{verbatim}   
\url{http://forums.bodhilinux.com/index.php?/topic/3723-solved-kernel-header-files-missing/}

  \item PowerPC (cross)-compiler: the distributor website recommended version
  3.2 (Sect.\ref{sec:cross_compiler}),
  but you still can use a different version, e.g. crosstool-ng 1.9.3
  (Sect.\ref{sec:crosstool-ng}).
  
Suppose we use the cross-compiler compiled by crosstool-ng, which is built and
saved into \verb!~/x-tools/! (by default).
You need to configure
\begin{verbatim}
cd <linux-kernel-2.6-source-folder>
export PATH=$PATH:~/x-tools/
export CROSS_COMPILE=powerpc-405-linux-gnu-
make uImage modules 
make modules_install
\end{verbatim}

Finally, you get the kernel image in 
\verb!./arch/ppc/boot/images/! folder.
    
\end{itemize}

Check the link for howto:
\url{http://getglitched.com/?page_id=253}


\textcolor{red}{\bf EXAMPLE}:
Once you have the installed crosstool-NG with the right versions of 
\begin{itemize}
  \item linux kernel
  \item binutils
  \item gcc
\end{itemize}
jump to the Linux kernel source directory, apply the patches if needed
and then copy the \verb!mip405t_config! file to 
\verb!.config! file in the root of the Linux kernel source folder.
If you do not have this file, you can run
\begin{verbatim}
//  either (if use text-based color menu, radiologist and dialogs
make menuconfig

// or (if use X11-based GUI tool)
make xconfig
\end{verbatim}
and adjust the appropriate parameters before kernel build.

and then setup the environment
\begin{verbatim}
cd ~/Tuan/linux-2.6.12_crosstool_1.9.3_kernel_2.6.12
export PATH=$PATH:~/x-tools_1.9.3_kernel_2.6.12/powerpc-405-linux-gnu/bin/
export ARCH=ppc
export CROSS_COMPILE=powerpc-405-linux-gnu-
\end{verbatim}


\begin{verbatim}
make uImage modules

// make -j4 2>&1 | tee m.txt
\end{verbatim}
Output
\begin{verbatim}
$> make uImage modules
  ....
  UIMAGE  arch/ppc/boot/images/uImage
Image Name:   Linux-2.6.12-mip405t
Created:      Fri Mar  6 16:12:16 2015
Image Type:   PowerPC Linux Kernel Image (gzip compressed)
Data Size:    1336057 Bytes = 1304.74 kB = 1.27 MB
Load Address: 00000000
Entry Point:  00000000
  Image: arch/ppc/boot/images/uImage is ready
  Building modules, stage 2.
  MODPOST
\end{verbatim}


\section{SDK image}

SDK image is a collection of tools that help you to do development.

\section{ISA interrupts}

MIP405/MIP405T supports more than 16 interrupt sources, i.e. the ISA interrupts are
remapped to a non PC-like position 
\begin{verbatim}
Interrupt         Usage
0-24          used by PPC405 internal device
25            interrupt from PIIX4 (Master PIC)
26            interrupt from DUART for MIP405
              unused in MIP405T
27            external IRQ2
28-31         PCI INTA, INTB, INTC, INTD
32-47         standard PC ISA interrupts
\end{verbatim}

So, ISA IRQ5 is mappped to interrupt (32+5) or 37 in your driver.

\url{http://www.mpl.ch/t2721.html}

\section{I/O board}

\url{http://www.comedi.org/hardware.html}


\section{Communicate with the board: minicom (Linux) and TELIX (Windows/DOS)}
\label{sec:minicom}
\label{sec:TELEX}

You connect the embedded Linux/BSD device with the PC (laptop) via a null modem cable using a serial connection;
then boot the device and run
\begin{verbatim}
minicom
\end{verbatim}
It will connect to the device and show what is going in the headless device.

First you need to install, 
\begin{verbatim}
sudo apt-get install minicom
\end{verbatim}
and setup minicom, run
\begin{verbatim}
minicom -s

  # with color on
minicom -s -c on
\end{verbatim}
Then, configure \verb!MINICOM! environment (put to \verb!~/.bashrc! file)
\begin{verbatim}
 export MINICOM="-m -c on"
\end{verbatim}
which ensure your terminal has a Meta or key and that color is supported.

\url{http://www.cyberciti.biz/tips/connect-soekris-single-board-computer-using-minicom.html}


\section{Testing with QEMU}

Command-line options for QEMU: Sect.\ref{sec:QEMU}; and the information on how
to emulate a PowerPC machine is given in Sect.\ref{sec:QEMU-emulate-PowerPC}.


% Currently, PowerPC 405 does not work with QEMU
% \url{https://lists.gnu.org/archive/html/qemu-devel/2007-08/msg00421.html}

PowerPC 405 CPU known to work (ie booting at least Linux 2.4):
\url{https://github.com/hackndev/qemu/blob/master/target-ppc/STATUS}

Try to use this instead: 
\url{https://code.google.com/p/cellos/wiki/HowToRunCellOS}

\textcolor{red}{OPTION 1}: You need to have (1) the kernel image, (2) initramfs
(Sect.\ref{sec:initramfs}) (Linux 2.6+) or initrd image (Linux 2.5-)
\begin{verbatim}
qemu-system-ppc -m 128M -kernel uImage -initrd initramfs.cpio 
             #-cpu 405GPd
             -cpu 0x401100c4 
                -M ref405ep -bios /location/to/ppc405_rom.bin 
                -append="console=ttyS0 root=/dev/sda1"
\end{verbatim}

You will see
\begin{verbatim}
>> =============================================================
>> OpenBIOS 1.1 [Mar 12 2015 08:09]
>> Configuration device id QEMU version 1 machine id 2
>> CPUs: 1
>> Memory: 128M
>> UUID: 00000000-0000-0000-0000-000000000000
>> CPU type PowerPC,750
milliseconds isn't unique.
Welcome to OpenBIOS v1.1 built on Mar 12 2015 08:09
>> [ppc] Kernel already loaded (0x01000000 + 0x00146339) (initrd 0x01247000 + 0x01068e00)
>> [ppc] Kernel command line: 

\end{verbatim}


%Once you build the kernel, you copy the image file \verb!uImage!, 
\textcolor{red}{OPTION 2}: You need to create a U-boot Flash image file
(boot.bin) - Sect.\ref{sec:Create-Flash-image} and you can run
\begin{verbatim}
qemu-system-ppc -m 128M -kernel flash.bin -nographic
\end{verbatim}

\begin{verbatim}

\end{verbatim}
\url{https://balau82.wordpress.com/2010/04/12/booting-linux-with-u-boot-on-qemu-arm/}

QEMU can load a Linux kernel using the -kernel and -initrd options; at a low
level, these options have the effect of loading two binary files into the
emulated memory: 
\begin{itemize}
  \item the kernel binary at address 0x10000 (64KiB) and 
  
  \item the ramdisk binary at address 0x800000 (8MiB). 
\end{itemize}

Then QEMU prepares the kernel arguments and jumps at 0x10000 (64KiB) to execute
Linux. I wanted to recreate this same situation using U-Boot, and to keep the
situation similar to a real one I wanted to create a single binary image
containing the whole system, just like having a Flash on board. The -kernel
option in QEMU will be used to load the Flash binary into the emulated memory,
and this means the starting address of the binary image will be 0x10000 (64KiB).

\subsection{Fix bugs}

Use \verb!ppc_booke_timers_init!
\url{https://lists.gnu.org/archive/html/qemu-ppc/2012-03/msg00082.html}