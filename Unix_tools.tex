\chapter{UNIX/Linux tools}
\label{sec:Unix_tools}

\section{program, process and process scheduling algorithm}
\label{sec:shell-process}
\label{sec:process}

In Linux, we have the concept of processes and programs. 
\begin{itemize}
  \item a program is a executable file (holding a sequence of instructions)
  residing on a disk
  
  \item a process (or a task) is an executing instance of a program
  
  Some operating systems use the term 'task' to refer to a program that is
  being executed.
  
  A process is termed as an '{\bf active entity}' since it is always stored in
  the main memory and disappears if the machine is power cycled. Several process
  may be associated with a same program.
  
  On a multiprocessor system, multiple processes can be executed in parallel. On
  a uni-processor system, though true parallelism is not achieved, a process
  scheduling algorithm is applied and the processor is scheduled to execute each
  process one at a time producing an illusion of concurrency.
  The scheduling algorithm is implemented inside the kernel, and the kernel
  control which process uses or release the CPU by sending {\bf signals}
  (Sect.\ref{sec:signals}).
  
\end{itemize}

When a user login, a new process for the shell (e.g. bash) is created
(Sect.\ref{sec:session_job-control}). 

\subsection{(bash) shell and session: job control}
\label{sec:session_job-control}

When a user login a machine, a process is created for the login shell program,
e.g. bash. The interactive shell provides an interface for user to type in
commands. Each time an ENTER is pressed, the interactive shell reads commands
from the user's terminal and create a process to execute the program specified by the
commands.

The shell use \verb!fork()! (to create a new process) and \verb!exec()!
functions (to execute a file). 

\begin{enumerate}
  \item \verb!fork()!: create a bran new process, containing a
  duplicate of the current program. There is a mechanism that allows the process to know which is
  the parent process and which is the child.

  \item \verb!exec()!: replace the program in the current process with the brand
  new program
\end{enumerate}

REMEMBER: a shell login is the parent process, and all programs you
run from that interactive shell is the child processes. If, the shell login session is
terminated, then all the child processes are died as well. How to avoid this?
we can try \verb!screen! or \verb!tmux! or \verb!byobu!
(Sect.\ref{sec:remote_shell-login}) or \verb!nohup! (Sect.\ref{sec:nohup}).


A single command may need one or more processes to be created, especially when
\verb!|! operator is used in the shell-command. Even a single program can also
create multiple processes internally, e.g. MPI or OpenMP. A process can have
many threads (Sect.\ref{sec:thread}).
\textcolor{red}{The processes belong to a single command is called a } {\bf
process group} or {\bf job}.


\subsection{thread}
\label{sec:thread}

A thread is called a 'lightweight process'. It is similar to a real process but
executes within the context of a process and shares the same resources allotted
to the process by the kernel. All the threads running within a process share the
same address space, file descriptor, stack and other process related attributes. 
Since the threads of a process share the same memory, synchronizing the access
to the shared data within the process gains unprecedented importance.

Linux kernel provides the library \verb!pthread! to create a new thread from a
given process (Sect.\ref{sec:pthread}). As two threads in the same thread may
write the same location in memory; it is important that the function being used
is designed with thread-safe in mind.


IMPORTANT: At one time, all threads in the same thread has to either idle, or
executing the same instruction.
Multiple threads of control can exploit the true parallelism possible on
multiprocessor systems. On a uni-processor system, a thread scheduling algorithm
is applied and the processor is scheduled to run each thread one at a time.




\subsection{signals}
\label{sec:signals}

Using {\bf signals} is a way to deliver asynchronous events to a program, i.e.
the kernel controls which process can use the CPU or has to release it.
This is a limited form of inter-process communication, and it needs to use
the kernel's process as the intermediate.
{\tiny
\begin{verbatim}
process-A send signal to process-B
  kernel-process receive signal; determine when to send such signal to process-B
  
kernel-process send signal to process-B
  a scheduling algorithm in the kernel-process initiate the signal
\end{verbatim}
}

Here, process-B is the active process. When a signal is sent to an active
process by the kernel's process, the kernel interupts the normal flow of
execution of the active process so that the process can receive the signal; and
decide what to do with it by the so-called signal handler
(Sect.\ref{sec:signal_handler}).

When a process is started, the status of
all the signals is either {\bf ignore} or {\bf default}, whichever
option is dependent upon the process that calls \verb!exec!. Normally, 'default'
is the case, unless the process that calls \verb!exec! is ignoring the signals.

There are different signals in Linux, starting with 'SIG' characters. All
signals have the default handlers, yet some cannot be changed, e.g. SIGKILL and
SIGSTOP.

\begin{enumerate}
  \item SIGNINT : signal generated when a user press Ctrl-C (to terminate a
  running program)
  \item SIGALRM: signal generated when a timer set by \verb!alarm()! function
  goes off
  \item SIGABRT: signal generated when a process call the \verb!abort()!
  function
  \item SIGKILL :
  \item SIGSTOP :
  \item SIGQUIT: the default handler terminates and dumps core the
  current process.
  \item SIGINFO: the default handler show information about the running command.
  \item \ldots
\end{enumerate}

\url{http://www.thegeekstuff.com/2012/03/linux-signals-fundamentals/}

\subsection{send a signal}

Besides the signal regularly sent by the kernel; a user can also send the signal
from the parent process to its child process. Example: the interactive shell is
also the parent process of all running (foreground) processes, which is the
child process of the 'shell' process. So, we can ask the kernel to send an
appropriate signal by pressing some key combination
\begin{enumerate}
  \item pressing Ctrl-C sends a SIGINT signal to kill one process running on the
  foreground.
 
  \item pressing Ctrl-Z sends a SIGTSTP signal to suspend the execution of the
  running process, and return the control to the parent process.
  
  \item pressing Ctrl-\verb!\! sends SIGQUIT signal: 
  
  \item pressing Ctrl-T (supported in some UNIXes) sends SIGINFO signal: 
\end{enumerate} 
The key combination can be changed with the \verb!stty! command.

\subsection{Signal handler}
\label{sec:signal_handler}

A signal handler is a special function - known as {\bf callbacks} - that binds
to a particular signal (or event) - Sect.\ref{sec:signals}), that is written it
in a given program; and will be evoked when the process of such program receives
the associated signal.
If the program has no registered signal handler for the received event; the {\it
default signal handler} is executed. The default signal handler is the one
provided by the kernel.

% When a signal occurs inside the program, you can choose what to do with it by
% the kernel, or just ignore it.
As a programmer, you can write a callback function, and register it with the
kernel for a given signal, so that when the signal occur, the kernel will call
that function to handle the signal, if the signal is not fatal to the program or
system.

REMEMBER: Some signals (e.g. SIGKILL, SIGSTOP) cannot be caught or ignored, i.e.
it has to be handled directly by the kernel. This allows a root or the kernel to
terminate the process, e.g. SIGKILL and SIGSTOP.

\subsection{ps command}

\begin{verbatim}
ps -aux
USER       PID  %CPU %MEM  VSZ RSS     TTY   STAT START   TIME COMMAND
timothy  29217  0.0  0.0 11916 4560 pts/21   S+   08:15   0:00 pine  
root     29505  0.0  0.0 38196 2728 ?        Ss   Mar07   0:00 sshd: can [priv]   
can      29529  0.0  0.0 38332 1904 ?        S    Mar07   0:00 sshd: can@notty  
\end{verbatim}
NOTE: x (in 'aux' means all user), for a specific user, we use (au\$USERNAME,
e.g. auroot, autmhoangt).

{\tiny
\begin{verbatim}
USER = user owning the process
PID = process ID of the process
%CPU = It is the CPU time used divided by the time the process has been running.
%MEM = ratio of the process's resident set size to the physical memory on the machine
VSZ = virtual memory usage of entire process (in KiB)
RSS = resident set size, the non-swapped physical memory that a task has used (in KiB)
TTY = controlling tty (terminal)
STAT = multi-character process state
START = starting time or date of the process
TIME = cumulative CPU time
COMMAND = command with all its arguments
\end{verbatim}
}

\section{run commands concurrently on multiple hosts}

Suppose you want to run 1 command on multiple machines. The naive approach is to
login each machine, and run the command. The faster approach is to use some tool
\begin{enumerate}
  \item pdsh - Sect.\ref{sec:pdsh}
  
  \item Sect.\ref{sec:cluster-management-tool} for many other options, e.g. Ansible, xCAT
\end{enumerate}

\subsection{pdsh}
\label{sec:pdsh}

PDSH = Parallel Distributed SHell \url{http://sourceforge.net/projects/pdsh/}

\url{http://code.google.com/p/pdsh/wiki/UsingPDSH}
\begin{verbatim}
pdsh [OPTION] ... COMMAND
\end{verbatim}
e.g.
\begin{verbatim}
pdsh -R ssh -w host0,host1,host2  ls
pdsh -w cea[0-4]   mkdir test
\end{verbatim}

\verb!pdsh! will try to login on each specified machine, via ssh or rsh, and
then execute the given COMMAND as the current user or a specified user account. 
To know which method (rsh, ssh) is being used by default, type
\begin{verbatim}
pdsh -V
\end{verbatim}
We may get the error when rsh is being used
\begin{verbatim}
 rcmd: socket: Permission denied
\end{verbatim}
If we use ssh, remember to add \verb!-R ssh! option. We can change the default
by using the environment variable
\begin{verbatim}
PDSH_RCMD_TYPE=ssh
\end{verbatim}
When invoked by pdsh, it is not possible for ssh to prompt for passwords. The
reason is that  ssh processes invoked from pdsh are
connected to the terminal, so they cannot prompt for passwords or
passphrases on the command line

The argument \verb!-w! indicates which machines the COMMAND should be run.
The list of machines can also be read host list from file (one per line), using
a single \verb!^! character in the front of the filename or using \verb!-h!
option
\begin{verbatim}
pdsh -w ^/tmp/hosts
pdsh -h /tmp/hosts
\end{verbatim}
To exclude some hosts (in the file), we use \verb!-x TARGETS! option.  The
number of nodes that pdsh can simultaneously execute remote jobs on  is limited
by the  maximum  number  of  threads  that  can  be  created  concurrently,  as well as the availability of reserved ports in the rsh
       and qshell rcmd modules. In POSIX system, it is given in
       \verb!PTHREADS_THREADS_MAX!

By default, the current user is used as the username to login the remote hosts.
To specify different users on different hosts, we add username and \verb!@! in
the front
\begin{verbatim}
pdsh -w hosts[0-2] -l user1

pdsh -w user1@host[0-2],user2@host[3-5]
\end{verbatim}

To use multiple commands, put them between double quotes, and separated by
semi-colons.
{\tiny
\begin{verbatim}
pdsh -R ssh -w leak,cardiac,cea[1-4] "cd
       /home/minhtuan/install_download/HDF5/hdf5-1.8.5-patch1/; make install"
\end{verbatim}
}

\section{Copy}
\label{sec:copy}

\subsection{Copy whole-file: cp}
\label{sec:copy_cp}

\begin{verbatim}
cp <source> <destination> [options]
\end{verbatim}
with [options] can be 
\verb!-r! (recursive), \verb!-f! (force copy, no question).

\subsection{Copy a few bytes}

To copy a few bytes from the beginning of the file we can use
\begin{verbatim}
head -C N
\end{verbatim}
which 
\begin{itemize}
  \item In GNU coreutils: use  \verb!read()! to do the job and thus never go
  beyond $N$ bytes.
  \item In NetBSD 5.1: use \verb!getc()! to do the job
  \item In FreeBSD 7.4: use \verb!fread()! to do the job
\end{itemize}



\section{Backup}
\label{sec:backup}

A backup is simply a copy of data from one place to another
(Sect.\ref{sec:copy}).

However, backup tools can provide more features, e.g. incremental backup, auto
rename of backup files. A copy can be from one location to another on the same
disk, on a different partitions on the same disk, to a different disk on the
same machine, or to another disk on a different machine via the networking.

RAID only keeps the latest state of the system. Also, any RAID solution won't
protect us from multiple disk failures. A good solution is to backup data using
{\bf rsync} (Sect.\ref{sec:rsync}).

\textcolor{red}{IMPORTANT:} Unless you are transfering the data over the
Internet, it's best to avoid compressing, as it just slow downs the transfer
rate. E.g.: on LAN, \verb!-z! is at least twice as slow.

\textcolor{red}{IMPORTANT:} Using \verb!tar! and pipe is perfect for large tree
files, it just streaming the data from host to destination, without having to do
intense operation with file trees. For small number of files, do NOT use
\verb!tar! as its syntax is harder to remember. If the data exist on the
destination, then \verb!rsync! is faster.

\url{https://www.crucialp.com/resources/tutorials/server-administration/how-to-copy-files-across-a-network-internet-in-unix-linux-redhat-debian-freebsd-scp-tar-rsync-secure-network-copy.php}

\subsection{Backup with cp}

NOTE: For simple backup of a file, we can do
\begin{verbatim}
cp /etc/network/interfaces{,.bak}
\end{verbatim}
to create \verb!interfaces.bak! from \verb!interfaces! file
(Sect.\ref{sec:copy_cp}).


\subsection{scp}

\verb!scp! (use ssh) is more reliable than \verb!rcp! (no encryption as it uses
rsh). To recursively copy all subfolders, use \verb!-r! option
\begin{verbatim}
scp -Cr <source> <dest>
\end{verbatim}
Use \verb!-C! if we want to compress before sending the data.

\begin{verbatim}
scp -Cr my_folder host.name:/destination/path
\end{verbatim}

\verb!scp! encrypts the data using 3DES algorithm (by default). We can use
\verb!-c blowfish! to speed things up. 

\verb!scp! spawn a new process for each file it copies. So, \verb!scp! should
NOT be used in cases
\begin{enumerate}
  \item copy large number of files:
  \item it doesn't know about symbolic link, so if \verb!-r! is used, it will
  follows the link to the file to copy the file (even if it has already make
  the copy of the file), and lead to copying an infinite amount of data (BE
  CAREFUL).
\end{enumerate}

\subsection{tar | tar}

\verb!tar! is an utility for compression; yet it can be used to copy file from
disk to disk. \verb!tar! handles large file trees quite well and preserves all
file permissions.

To copy a directory, first write them to a tape archive with
relative pathnames, and then read back the tape and write it to the destination.
Another way, rather than using tape archive is to write to UNIX pipe, and then
read from the pipe.

Example: copy multiple folders to /dev/hdd

{\tiny
\begin{verbatim}
tar -cvzf /dev/hdd /bin /boot /dev /etc /home /lib /sbin /usr /var

  // on tape driver (do not compress as a single error can corrupt all data)
  // here rewind the SCSI tape, and archive /home folder to it.
  // command 'mt' (magnetic tape) control generic SCSI tape devices
mt -f /dev/st0 rewind
tar -cvf /dev/st0 /home  
\end{verbatim}
}

Example: backup across multiple floppy disk
\begin{verbatim}
tar -cvMf /dev/fd0 /home/simon
\end{verbatim}

Use this if \verb!tar! supports B (reblocking) option (to ensure the copy is
made correctly). IMPORTANT: Do not use \verb!-v! (verbose) on the first
\verb!tar! command (otherwise double data is saved), and only use it on the
second command.
{\tiny
\begin{verbatim}
% mkdir /work/bkup/jane
% cd /home/jane
% tar -cf - . | (cd /work/bkup/jane && tar xBf -)
\end{verbatim}
}

Here, the \verb!&&! operator tells the shell that it only start \verb!tar xBf -!
only if the previous command, \verb!cd /work/bkup/jane/! succeeded. 

If \verb!tar! doesn't support B (reblocking)
{\tiny
\begin{verbatim}
% tar cf - . | cat | (cd /work/backup/jane && tar xbf 1 -)
\end{verbatim}
}

To copy over the network, we can use \verb!rsh! (with no compression)
{\tiny
\begin{verbatim}
% rsh kumquat mkdir /work/bkup/jane
% tar cf - . | rsh kumquat 'cd /work/bkup/jane && tar xBf -'
\end{verbatim}
}

or we can do
{\tiny
\begin{verbatim}
tar zcvf -  MyBackups | ssh user@server "cat > /path/to/backup/foo.tgz"

tar -cf - /some/file | ssh host.name tar -xf - -C /destination
tar -czf - /some/file | ssh host.name tar -xzf - -C /destination
\end{verbatim}
}

\textcolor{red}{NOTE: -C option on the second tar to change
directory on the target host, i.e. uncompress to a specified folder}. Here, the
input (data) for the second command is taken from the stdin

NOTE: Different compression method can be used (z = gzip, Z = compress, j =
bzip). For 7zip compression
{\tiny
\begin{verbatim}
tar cf - MyBackups | 7za a -si -mx=9 -ms=on MyBackups.tar.7z | 
   ssh user@server "cat > /path/to/backup/foo.7z"
\end{verbatim}
}

\textcolor{red}{IMPORTANT}: Do not use 7zip for backup on Linux/Unix, as it
doesnot store owner/group of the file.

\subsection{rsync}
\label{sec:rsync}

\verb!rsync! is famous for its delta-transfer algorithm, to reduce the data
sent over the network by sending only the difference between the source files
and the existing files on the destination. IMPORTANT: \verb!rsync! is NOT
installed by default.

\begin{verbatim}
rsync <options> <source> <dest>
\end{verbatim}

Typically, we need these options \verb!-rvz!
\begin{enumerate} 
  \item -r = recursive
  \item -v = verbose (print out progress)
  \item -z = enable compression
  \item -u = update only (skip files from sources that are older files)
  \item -t = preserve timestamp
  \item -l = preserve symbolic link (i.e. do NOT follow symbolic link)
  \item -p = preserve permissions
  \item -o = preserve ownership and group
  \item -d = copy only directory structures (not the files in the directories)
  \item \textcolor{red}{-W = copy whole file (do NOT use delta-transfer
  algorithm)}
  \item --progress = display detailed progress
  \item --delete = delete files that are at <dest>, but not found in <source>
  \item --existing = copy files from <source> that haves a corresponding files
  in <dest>
  \item --max-size='100K' = set maximum filesize to be considered
  \item -i = display the difference between the <source> and <dest>
  \item -e ``protocol information'' = To copy over the network

{\tiny
  \begin{verbatim}
  rsync -e ssh
  rsync -e "ssh -o StrictHostKeyChecking=no -o  UserKnownHostsFile=/dev/null"
  \end{verbatim}
}
\end{enumerate}

Some options that do multiple things
\begin{enumerate}
  \item -a = archive mode (-rlptgoD)
\end{enumerate}


For advanced goals, we use
\begin{enumerate}
  \item -f''RULE HERE'' = file-filter
  \begin{verbatim}
  # copy directory structures (not files)
  rsync -av -f"+ */" -f"- *" <source> <dest>
  
  rsync -d <source> <dest>
  \end{verbatim}
  NOTE: This is newer feature, and is supposed to replace \verb!--include!,
  \verb!--exclude! in older versions of rsync (See examples below).
  
  \item --exclude=PATTERN : exclude files matching patterns

IMPORTANT: ALL exclude path are relative, i.e. these two are the same
\begin{verbatim}
--exclude '/dir1/name1/file.txt'

--exclude 'dir1/name1/file.txt'
\end{verbatim}

We can combine multiple exclude
\begin{verbatim}
--exclude 'dir1' --exclude 'dir2/filename.txt'
\end{verbatim}
If we have multiple files, we can put them in a file, and use
\verb!--exclude-from! option
\begin{verbatim}
--exclude-from 'exclude-list.txt' <source> <dest>
\end{verbatim}
with
\begin{verbatim}
# content of exclude-list.txt

dir1
dir2/filename.txt
\end{verbatim}
  
  \item --include=PATTERN : copy only files matching patterns  

{\tiny
  \begin{verbatim}
  # exclude a specific directory and all of its subdirectories
   --exclude 'folder_name_here' <source> <dest>
   
#exclude multiple directories match the pattern
--exclude 'folder-*'

#exclude a specific file
--exclude 'folder-name/filename.txt'
  
  # exclude all files (--exclude="*")
  # and include only files/directories starting with 'P*'
  rsync -avz --include 'P*' --exclude '*'
  \end{verbatim}
}
\end{enumerate}


If we want to
\footnote{\url{psung.blogspot.com/2008/05/copying-directory-trees-with-rsync.html}}
\begin{enumerate}
  \item Copy only directory structures, but not files
  \begin{verbatim}
  rsync -a -f"+ */" -f"- *" source/ destination/
  \end{verbatim}
  The two '-f' arguments mean copy directories, except other things
  
  IMPORTANT: The order of the two '-f' arguments are important 
  
  \item Copy all directories, and some files ending with '.py'
  \begin{verbatim}
  rsync -a -f"+ */" -f"+ *.py" -f"- *" source/ destination/
  \end{verbatim}
  
  \item Copy everything, except '*.git' directories
  \begin{verbatim}
  rsync -a -f"- .git/" -f"+ *" source/ destination/
  \end{verbatim}
\end{enumerate}

\subsection{Backup LAN: amanda}

The admin can set up a single master backup server to back up multiple hosts to
a single large capacity tape or disk drive.


\url{http://www.amanda.org/download.php}

\subsection{drsbackup}

DrsBackup is developed by MIT. Currently, it preserves UID's and GID's
numerically, instead by name. While rsync and GNU tar preserve by name by
default. 

\url{http://stromberg.dnsalias.org/~strombrg/drsbackup.htmldoc}


\subsection{rdist - Copy data to multiple hosts}

At first, you need to install 'rdist' on multiple hosts using pdsh command
(Sect.\ref{sec:pdsh})
\begin{verbatim}
pdsh -R ssh -w cea[5-11] -l root "apt-get install rdist -y"
\end{verbatim}
Then we create the input file, default name is 'distfile'
{\tiny
\begin{verbatim}
#file being used by rdist
# command :   rdist -f PATH_TO/distfile

SLAVES = ( cea5 cea6 cea7 cea8 cea9 cea10 cea11)

# Lets push our NVIDIR driver files for all slave hosts to see!
season1: (
    /home/minhtuan/install_download/NVIDIA-Linux-x86_64-295.20.run
) -> ( ${SLAVES} )
#except /home/lost/season_1/deadguys;


# we can also do multiple copies at the same times by adding another rule
# for this case, we want to exclude cea5 or add some new target
#season2: (
#    /home/lost/season_2
#) -> ( ${SLAVES} ) - ( cea5 ) + ( friend.lost.com )
\end{verbatim}
}

When you run, you jump to the folder which has distfile and then just type
\begin{verbatim}
rdist -f distfile
\end{verbatim}


\section{Devices/Partition}

\subsection{mount}

A {\bf block device} (block special file) represents a hardware device that data
are moved in the form of blocks using functions like \verb!fread()!. These
devices can be: hard-drive, CD-ROM, flash drives or some addressable regions of
memory. Block devices are designed to support random seeking and access, and
generally use buffered input/output routines. In other words, the operating
system will allocate a buffer memory to hold a single block of data, one the
buffer is full, the data is flushed to the device and the buffer is clear for
other I/O operations. This can avoid direct access to the block devices as it is
considered dangerous. In opposite to block device is {\bf character device},
where data is input/output in one character at a time, using functions like
\verb!getchar()!. Examples of character devices are keyboard, mouse, etc. 


\begin{framed}
To display information about block devices, we use \verb!blkid! command
\begin{verbatim}
blkid -c /dev/null

blkid -po udev /dev/sda1
\end{verbatim}
\end{framed}

Block devices and character devices map to physical devices. However, in Linux,
there are pseudo-device, i.e. those do not necessarilty have a corresponding
physical devices. They are
\begin{itemize}
  \item /dev/null
  \item /dev/zero
  \item /dev/full
  \item /dev/random
  \item /dev/urandom
\end{itemize}

Regardless of the device types, they are represented by a file in the
\verb!/dev/! folder (Sect.\ref{sec:udev}). To see which which one is the USB, we
can do
{\tiny
\begin{verbatim}
> blkid

/dev/sdc1: LABEL="Transcend" UUID="AE745F8E745F5863" TYPE="ntfs"
\end{verbatim}
}
Then, we can mount it now
\begin{verbatim}
mkdir /media/usb

mount /dev/sdc1 /media/usb
\end{verbatim}

The prefix can explain the device types
\begin{itemize}
  \item /dev/fb* : frame buffer
  \item /dev/fd* : floppy disk
  \item /dev/hd* : hard-disk (or external hard-drive via USB)
  \item /dev/sd* : mass-storage driver (USB)
  \item /dev/tty* : terminal
\end{itemize}

If you use a SATA harddrive (SCSI) for backup, then you can use the script given
here:
\url{http://www.makestuff.eu/wordpress/sata-hotplug-in-ubuntu/}

References:
\begin{itemize}
  \item
  \url{http://askubuntu.com/questions/192461/how-do-i-get-ubuntu-server-to-recognize-and-automount-a-hotplugged-sata-drive}
\end{itemize}

\subsection{fdisk: partition manager}

\begin{verbatim}
fdisk /dev/sda
\end{verbatim}
Commands:
\begin{enumerate}
  \item n : to create a new partition (choose e: extended, p: primary partition
  (which then you choose the number from 1 to 4))
  \item t : choose partition type (select the partition number first (1-4), and
  then choose (1) fd: mark it as being part of a RAID array; (2) ext3)
  \item w : write the changes to disks
\end{enumerate}

To copy the setting from one harddrive to another we do
\begin{verbatim}
cfdisk -d /dev/sda | cfdisk /dev/sdb
\end{verbatim}

\subsection{hdparm: Harddrive check}
\label{sec:hdparm}


\begin{verbatim}
sudo hdparm -I /dev/sda
\end{verbatim}

Here are the important information for SAS disk
{\tiny
\begin{verbatim}
Model Number:       ST500DM002-1BD142                       
        Serial Number:      Z3T0E92Z            
        Firmware Revision:  HP73    
        Transport:          Serial, SATA Rev 3.0

Configuration:
        device size with M = 1024*1024:      476940 MBytes
        device size with M = 1000*1000:      500107 MBytes (500 GB)
        cache/buffer size  = 16384 KBytes
        Nominal Media Rotation Rate: 7200

\end{verbatim}
}

Here are the example for SATA disk
{\tiny
\begin{verbatim}
ATA device, with non-removable media
        Model Number:       WDC WD15EARX-00ZUDB0                    
        Serial Number:      WD-WMC1H0136695
        Firmware Revision:  80.00A80
        Transport:          Serial, SATA 1.0a, SATA II Extensions, SATA Rev 2.5, SATA Rev 2.6, SATA Rev 3.0

Configuration:
        device size with M = 1024*1024:     1430799 MBytes
        device size with M = 1000*1000:     1500301 MBytes (1500 GB)
        cache/buffer size  = unknown
        Nominal Media Rotation Rate: 5400
\end{verbatim}
}

Another tool is lshw
\begin{verbatim}
lshw -class disk -class storage
 
lshw -short -C disk

smartctl -d ata -a -i /dev/sda
\end{verbatim}

\subsection{RAID array}


Sect.\ref{sec:RAID}

\begin{verbatim}
smartctl
\end{verbatim}
\url{http://www.cyberciti.biz/faq/find-hard-disk-hardware-specs-on-linux/}

\subsubsection{SCSI}

For SCSI devices, we can use
\begin{verbatim}
scsi_id
sdparm
\end{verbatim}

\section{Auto answer 'yes'}

For a remote settup, you may want to avoid typing 'y' or 'yes' to your
installation, one way to do this in some tools is adding '-y' to the command
line. Another option is passing 'yes' to your command line via pipe
\begin{verbatim}
yes | sudo mdadm ...options and arguments...
\end{verbatim}

\section{Repeat a command (given interval)}

Use \verb!watch -n <time> <command>!. The drawback: 
\begin{enumerate}
  \item  it cannot use aliased command
  \item scrolling may not work properly (if the output is quite long)
  \item it may not work with very long interval more than 4200 seconds
  (about an hour). The reason is that
  the time being measured is stored in microseconds, so 4200 is $2^{32}$
  microseconds. (should use \verb!crontab! or \verb!at! or use the loop below)
  \item clearing the screen 
\end{enumerate}

The time is in seconds
\begin{verbatim}
watch -n 5 ls -l
\end{verbatim}
To stop, press Ctrl-C. Other options:
{\tiny
\begin{verbatim}
-t 	(--no-title) turn off showing time interval, command, ...
--cumulative     presenting a running display of positions that change
--difference     visual confirmation of changes
\end{verbatim}
}

IMPORTANT: Use escape special characters that are part of the command
\begin{verbatim}
watch mysql dbname -e \"show processlist\;\"
\end{verbatim}

Or Use the loop
\begin{verbatim}
// command is harder to kill
while true; do <your_command>; sleep <interval_in_seconds-or-hours>; done

// use this is preferred
while sleep SECONDS; do command; done
\end{verbatim}
We can use \verb!sleep 2h! or \verb!sleep 2! (in seconds by default).


Or (csh shell) Use \verb!repeat! if we know the fixed number of times
\begin{verbatim}
repeat 300 do <my first command here> && sleep 1.5
\end{verbatim}
This works for \verb!csh! (but not bash or sh).

Or create a script in \verb!~/.bash_aliases! (the first argument is the
number of times)
\begin{verbatim}
repeat() {
n=$1
shift
while [ $(( n -= 1 )) -ge 0 ]
do
    "$@"
done
}
\end{verbatim}
then run
\begin{verbatim}
repeat 5 echo Hello World !!!
repeat 15 ./myscript
\end{verbatim}


If you are monitoring the file system, then use \verb!inotifywait!. It runs and
report any change to the file system
\begin{verbatim}
// terminal 1
inotifywait .

// terminal 2 (some other users)
touch newfile

// immediately, on terminal 1, inotifywait report
./ CREATE newfile2
\end{verbatim}

\section{Remote shell login (resuming ssh connection)}
\label{sec:remote_shell-login}

A GUI option is using vnc server - Sect.\ref{sec:VNC-server}.
Here, we discuss the terminal option
\begin{enumerate}
  \item ssh - Sect.\ref{sec:ssh}
  
  \item mosh - Sect.\ref{sec:MOSH}
\end{enumerate}

\subsection{ssh}

You can customize SSH (Sect.\ref{sec:SSH}) to keep your session alive if the
connection get lost for minutes. On the server side, change
\begin{verbatim}
ClientAliveInterval = 60
TCPKeepAlive = yes
\end{verbatim}


\subsection{mosh (mobile shell)}
\label{sec:MOSH}

mosh (mobile shell) is a replacement for ssh, though MOSH still need ssh
(Sect.\ref{sec:SSH}). 

We need to insall a MOSH server on the server side, and MOSH client on the
client side. 

On server-side or client-side (Linux):
\begin{verbatim}
sudo apt-get install python-software-properties
sudo add-apt-repository ppa:keithw/mosh
sudo apt-get update
sudo apt-get install mosh
\end{verbatim}
It will install \verb!mosh-client! and \verb!mosh-server!; and a third
component, \verb!mosh! wrapper script. 

The client will call \verb!mosh! script which has the responsibility of using
ssh to connect to a remote machine, start mosh-server on the remote machine,
gather the necessary connection information (encryption key and remote IP), and
then finally starting mosh-client.
\url{http://mosh.mit.edu/#getting}

If the client-side is Windows, then you should use MobaXterm;
or Cygwin.

% In Ubuntu, start a mosh server
% \begin{verbatim}
% mosh-server new
% 
% ps aux | grep 'mosh'
% \end{verbatim}
% 
There's nothing to run in the server, mosh (the client)
will ssh, run "mosh-server" at the server-side and somehow exit and connect to
that new mosh-server via UDP. It keeps the session alive after days of
disconnection, changing IP, high latency, etc. MOSH use UDP, rather than TCP for
connection. As UDP is stateless, so there is no connection drop issue like with
SSH which uses TCP.

DEBUG: \url{https://github.com/mobile-shell/mosh/wiki/Debugging-and-Theory}

\subsection{freeNX and nomachine}
\label{sec:freeNX}
\label{sec:nomachine}

NoMachine provides remote connectivity and network computing software, and they
developed 'NX' technology - a proprietary compression protocol for remote
computer access. A free implementation of NX for server-side is freeNX.
 
FreeNX is not maintained, and you should try X2GO 
FreeNX is not part of Ubuntu
\begin{verbatim}
//Ubuntu 14.04
sudo apt-add-repository ppa:freenx-team/trusty
sudo apt-get update
sudo apt-get install freenx-server
\end{verbatim} 
The default port used by FreeNX is port 22 for SSH. If port 22 is not allowed,
modify and replace 22 with a different value, e.g. 8888
\begin{verbatim}
sudo nano /etc/ssh/sshd_config

sudo /etc/init.d/ssh restart
\end{verbatim}

NoMachine:
\url{https://www.nomachine.com/download/download}

\url{https://groups.google.com/forum/#!topic/alt.os.linux/GThI7FRk3BQ}



\subsection{X2GO}
\label{sec:X2GO}

X2Go is still incompatible with desktops that rely on 3D acceleration.
 
\begin{verbatim}
sudo apt-get install python-software-properties
sudo add-apt-repository ppa:x2go/stable
sudo apt-get update
sudo apt-get install x2goserver
\end{verbatim}
\url{https://askubuntu.com/questions/466309/unable-to-install-freenx-server-on-ubuntu-trusty-alternatives}

A client on Windows also available
\url{https://wiki.x2go.org/doku.php}

\url{http://googlefodder.blogspot.com/2012/09/nomachine-nx-freenx-neatx-which-should.html}


\section{VNC server + viewer}
\label{sec:VNC-server}

A deamon called a VNC server running on a remote machine can provides GUI access
to this machine. A VNC server establishes a type of virtual frame buffer (i.e.
RAM framebuffer, rather than graphics hardware-based framebuffer).

The RAM-based framebuffer manages a screen image (a two-dimensional array of
pixels), but does not display it directly on a screen, instead, it sends the
buffer to the local machine running a VNC viewer (Sect.\ref{sec:VNC-client}).

There are different options for VNC servers
\begin{enumerate}
  \item vino - Sect.\ref{sec:vino} - used GNOME to provides GUI

  \item tigerVNC - Sect.\ref{sec:tigervnc}
  
  \item tightVNC - Sect.\ref{sec:tightvnc}
  
  \item vnc4server - Sect.\ref{sec:vnc4server}
  
  \item x11VNC - Sect.\ref{sec:x11vnc}
  
  \item Gemsvnc - Sect.\ref{sec:gemsvnc}
  
  \item VNC-compatible server such as Xvnc or WinVNC
  
   vncserver is wrapper script that calls Xvnc. 
\end{enumerate}
whose choice depends on the server-side use GNOME 1/2 or GNOME 3
(Sect.\ref{sec:vncserver-GNOME-3})

Remember that you can uses tmux to provide command-line-based persistent session
(Sect.\ref{sec:tmux}). Similarly, VNC provides a GUI-based persistent desktop, I
do not need to log in or log out, but instead I just resume my session from
where I left off last time by running a VNC viewer application.

\begin{enumerate}
  \item RHEL 4-6: \url{https://www.redhat.com/magazine/006apr05/features/vnc/}

  \item RHEL 7: \url{https://access.redhat.com/articles/754933}

  \item Ubuntu:
\url{http://shebang.brandonmintern.com/tips-for-remote-unix-work-ssh-screen-and-vnc/}
\end{enumerate}

\subsection{TurboVNC (VirtualGL)}
\label{sec:TurboVNC}

To escape from fullscreen mode: Ctrl-Alt-Shift-F

To configure:
\begin{verbatim}
Connection tab
        Scaling factor:   Fixed Aspect Ratio
        Remote desktop size: choose proper resolution of the desktop on remote
                        machine
        Full-screen mode: checked
        Span mode:           automatic
\end{verbatim}

\url{https://virtualgl.org/}

VirtualGL is an open source toolkit that gives any Unix or Linux remote display
software the ability to run OpenGL applications with full 3D hardware
acceleration.
With VirtualGL, the OpenGL commands and 3D data are instead redirected to a 3D
graphics accelerator (AKA "graphics processing unit" or "GPU") in the
application server, and only the rendered 3D images are sent to the client
machine.

\url{https://sourceforge.net/projects/turbovnc/files/2.1.2/}

\subsection{vnc4server}
\label{sec:vnc4server}

\subsection{vino}
\label{sec:vino}

Vino is the default VNC server in Ubuntu to share your existing desktop with
other users. It uses GNOME to provide GUI.

The server can be a headless server, i.e. no graphical desktop installed. For
this, a graphics driver like \verb!xf86-video-dummy! must be installed.
\url{https://wiki.archlinux.org/index.php/Vino}


The problem with vino on Ubuntu 14.04 is the default authentication. This is a
current bug in 14.04 as of this post writing. There is an increased in 
security on server-side, and the vnc client has not been updated to work with
high security. The work-around solution is to disable security on server-side.

\url{http://www.libregeek.org/2014/04/29/getting-remote-access-work-ubuntu-14-04-vino/}

Besure to configure firewall on the server to enable the port
(Sect.\ref{sec:firewall})

\subsection{no matching security types}

The supported security type which is given for Ubuntu 14.04 is "type 18, TLS".
For Ubuntu 13.10 we see two types, "type 18, TLS" and "type 2, VNC".
However, The "type 2, VNC" is used by your VNC viewer. This can be analyzed
using wireshack (Sect.\ref{sec:wireshack}).

\begin{verbatim}
Jun 29 00:03:36 localhost /etc/gdm/Xsession[974]: 29/06/2013 00:03:36 [IPv4] Got connection from client nanawel
Jun 29 00:03:36 localhost /etc/gdm/Xsession[974]: 29/06/2013 00:03:36   other clients:
Jun 29 00:03:36 localhost /etc/gdm/Xsession[974]: 29/06/2013 00:03:36      stilgar-laptop
Jun 29 00:03:36 localhost /etc/gdm/Xsession[974]: 29/06/2013 00:03:36 Client Protocol Version 3.7
Jun 29 00:03:36 localhost /etc/gdm/Xsession[974]: 29/06/2013 00:03:36 Advertising security type 18
Jun 29 00:03:36 localhost /etc/gdm/Xsession[974]: 29/06/2013 00:03:36 Client nanawel gone
Jun 29 00:03:36 localhost /etc/gdm/Xsession[974]: 29/06/2013 00:03:36 Statistics:
Jun 29 00:03:36 localhost /etc/gdm/Xsession[974]: 29/06/2013 00:03:36   framebuffer updates 0, rectangles 0, bytes 0
\end{verbatim}

As most vnc viewers (except some commercial versions such as realvnc) do not
accept TLS security, we can disable this on server-side. To set vino to request
access each time, tick Allow other users to view your desktop in the Remote
Desktop configuration window.
\begin{verbatim}
$ pkill vino
$ export DISPLAY=:0.0
$ /usr/lib/vino/vino-server &
$ gsettings set org.gnome.Vino require-encryption false

gsettings set org.gnome.Vino require-encryption false
\end{verbatim}
Note that this schema (org.gnome.Vino) is not accessible with dconf-editor.

{\bf IMPORTANT}: A recent patch can help RealVNC 4.1.1 client to accept TLS
security on Unix, Windows, and Java. 
\url{https://www.auto.tuwien.ac.at/~mkoegler/index.php/tlsvnc}

{\bf IMPORTANT}: we can try ssvnc (Sect.\ref{sec:ssvnc})

\subsection{-- ssvnc}
\label{sec:ssvnc}

{\bf IMPORTANT}: ssvnc - a GUI wrapper for SSL and SSH VNC connections.
\url{http://www.karlrunge.com/x11vnc/ssvnc.html}

 ssvnc  is  a tcl/tk gui wrapper that runs on Unix, MacOSX, and Windows.
       It sets up an SSL or SSH tunnel to  the  remote  VNC  Server  and  then
       launches  the  VNC  viewer (either the one provided or another one that
       you have specified) to use that encrypted tunnel to connect to the  VNC
       Server.   
       
 \url{http://manpages.ubuntu.com/manpages/zesty/man1/ssvnc.1.html}

\url{https://bugs.launchpad.net/ubuntu/+source/vino/+bug/1281250}





\subsection{-- gemsvnc}
\label{sec:gemsvnc}


\subsection{--- tightvnc (Xvnc4) with SSH tunneling}
\label{sec:Xvnc}
\label{sec:tightvnc}
\label{sec:vncserver}
\label{sec:vncserver-Ubuntu}
\label{sec:vncserver-GNOME-1}
\label{sec:vnc-loopback-SSH-tunneling}

tightvnc was developped by Constantin Kaplinsky, by using and extending RFB
protocol of VNC. It effectively uses 'tight-encoding' techniques - a combination
of the JPEG and zlib compression mechanisms, to transmit screen-shots.
However, tight encoding is not supported by most other implementations, so it is
necessary to use TightVNC at both ends (both server and client) to gain the full
advantage of its enhancements.
In 2.0 beta, TightVNC supports auto scaling, which resizes the viewer window to
the remote users desktop size, regardless of the resolution of the host
computer.

Its forks
\begin{enumerate}
  \item TigerVNC - Sect.\ref{sec:tigervnc}
\end{enumerate}


Xvnc is an implementation of VNC protocol, and acts like an X11 server with
virtual display, i.e. this display is a buffer on RAM and be mapped to any
display (locally or remotely).
Thus, the display can be captured by a VNC viewer client program.
The wrapper is \verb!vncserver! script.
\url{http://tigervnc.org/doc/Xvnc.html}


\textcolor{red}{Step 1}: tightvnc only works with GNOME 1 or GNOME 2 only
(Sect.\ref{sec:vncserver-GNOME-1}), i.e. nautilus, and metacity. To work with
GNOME 3, read Sect.\ref{sec:vncserver-GNOME-3}

On server machine, create \verb!~/.vnc/xstartup! file
(Sect.\ref{sec:xstartup-vnc}).

Step 2: on server create \verb!/etc/init.d/vncserver! file
(simplest form)
\begin{verbatim}
#!/bin/bash
PATH="$PATH:/usr/bin/"
export USER="exampleuser"
DISPLAY="1"
DEPTH="16"
GEOMETRY="1024x768"
OPTIONS="-depth ${DEPTH} -geometry ${GEOMETRY} :${DISPLAY} -localhost"
. /lib/lsb/init-functions

case "$1" in
start)
log_action_begin_msg "Starting vncserver for user '${USER}' on localhost:${DISPLAY}"
su ${USER} -c "/usr/bin/vncserver ${OPTIONS}"
;;

stop)
log_action_begin_msg "Stopping vncserver for user '${USER}' on localhost:${DISPLAY}"
su ${USER} -c "/usr/bin/vncserver -kill :${DISPLAY}"
;;

restart)
$0 stop
$0 start
;;
esac
exit 0
\end{verbatim}

IMPORTANT: SSH tunneling: \verb!-localhost! feature is what limits VNC Server
connections to ONLY possible once connected via an SSH Tunnel.
\begin{verbatim}
sudo chmod +x /etc/init.d/vncserver

sudo service vncserver start
sudo service vncserver restart
sudo service vncserver stop
\end{verbatim}


Step 3: on server, check the listening port
\begin{verbatim}
netstat -plant | grep Xvnc


tmhoangt:/home/tmhoangt/nts>netstat -plant | grep Xvnc
(Not all processes could be identified, non-owned process info
 will not be shown, you would have to be root to see it all.)
tcp        0      0 127.0.0.1:5902          0.0.0.0:*               LISTEN      19062/Xvnc4     
tcp        0      0 127.0.0.1:5903          0.0.0.0:*               LISTEN      19338/Xvnc4     
tcp        0      0 0.0.0.0:6002            0.0.0.0:*               LISTEN      19062/Xvnc4     
tcp        0      0 0.0.0.0:6003            0.0.0.0:*               LISTEN      19338/Xvnc4 
\end{verbatim}

Step 3: from client, 
\begin{verbatim}
ssh -L 5901:127.0.0.1:5901 -N -f -l exampleuser ip_address
\end{verbatim}

Step 3: from client, if using Putty
\begin{verbatim}
Under Connection -> SSH -> Tunnels add:

Source port: 5901
Destination: localhost:5901
\end{verbatim}

Step 4: from client, once ssh is established, we just use any VNCclient to
connect to localhost with the designated port above
\begin{verbatim}
localhost:5901 
\end{verbatim}
via a VNC viewer such as TightVNC.

Step 5: fast for X11 tunneling
\url{http://xmodulo.com/how-to-speed-up-x11-forwarding-in-ssh.html}

% To use SSH tunneling, we modify /etc/vncserver/vncserver.conf file
% \begin{verbatim}
% VNCSERVERS="1:tmhoangt 2:tmhoangt 3:tmhoangt"                                                                                                                                                           
% VNCSERVERARGS[1]="-geometry 1920x1024 -nolisten tcp -SecurityTypes=VeNCrypt,TLSVnc"
% VNCSERVERARGS[2]="-geometry 1024x768  -nolisten tcp -SecurityTypes=VeNCrypt,TLSVnc"
% VNCSERVERARGS[3]="-geometry 1800x1024 -nolisten tcp -SecurityTypes=VeNCrypt,TLSVnc"
% \end{verbatim}
% NOTE: Valid values of \verb!-SecurityTypes! are a comma separated list of None,
% VncAuth, Plain, TLSNone, TLSVnc, TLSPlain, X509None, X509Vnc and X509Plain. Default is VncAuth,TLSVnc.

% It has built-in SSL/TLS encryption and 2048 bit RSA authentication, including
% VeNCrypt support; 
\url{https://www.liquidweb.com/kb/how-to-configure-a-vnc-server-to-use-an-ssh-tunnel-on-ubuntu-14-04-lts/}

\subsection{--- tigervnc}
\label{sec:tigervnc}

TigerVNC is a fork of tightVnC (Sect.\ref{sec:tightvnc}).
TigerVNC adds encryption for all supported operating systems (not just Linux),
but it removes scaling the remote display into the client window, file transfer,
and changing options while connected.


\subsection{-----------install tigervnc}

We can install tigervnc from pre-built package.

Full build instruction:
\url{https://gist.github.com/lucabelmondo/8b48506a686b830871bd}

However, to take advantage of GNOME-3, we need to install the latest tigervnc
(e.g. 1.8+):
\begin{itemize}
  
  \item have these packages ready
  
  fltk 1.3+ - Sect.\ref{sec:FLTK}; libjpeg-turbo 9+ - Sect.\ref{sec:libjpeg}
  
  \item modify \verb!CMakeList.txt! 
  
  
\begin{verbatim}
 //  
set(CMAKE_PREFIX_PATH /packages/libjpeg-turbo/9/;/packages/fltk/1.3/);

include_directories("/packages/fltk/1.3/include")
set (CMAKE_SHARED_LINKER_FLAGS "-ldl -Wl,--as-needed")
SET (CMAKE_CXX_FLAGS "${CMAKE_CXX_FLAGS} -ldl -lpthread ")
SET (CMAKE_CXX_LINK_EXECUTABLE "${CMAKE_CXX_LINK_EXECUTABLE}
      -L/packages/libjpeg-turbo/9/lib -ljpeg -L/packages/libpng/1.5/lib -ldl -lX11 -lXext
      -I/packages/fltk/1.3/include -L/packages/fltk/1.3/lib -l:libfltk.a
      -l:libfltk_forms.a -l:libfltk_gl.a -l:libfltk_images.a -l:libpng.a ")

include_directories("/packages/fltk/1.3/include")
set (CMAKE_SHARED_LINKER_FLAGS "-ldl -Wl,--as-needed")
SET (CMAKE_CXX_FLAGS "${CMAKE_CXX_FLAGS} -ldl -lpthread ")
set (CMAKE_CXX_LINK_EXECUTABLE "${CMAKE_CXX_LINK_EXECUTABLE} -L/packages/libpng/1.5/lib -ldl -lX11 -lXext -I/packages/fltk/1.3/include -L/packages/fltk/1.3         /lib -l:libfltk.a -l:libfltk_forms.a -l:libfltk_gl.a -l:libfltk_images.a -l:libpng.a")

set(FLTK_DIR /packages/fltk/1.3/)
find_package(FLTK NO_MODULE)
#find_package(FLTK NO_MODULE
  #HINTS "/packages/fltk/1.3/lib")
SET(FLTK_INCLUDE_SEARCH_PATH
      ${FLTK_DIR}/include
      /packages/fltk/1.3/include
   )

\end{verbatim}

  \item build

\begin{verbatim}
mkdir build
cd build
rm * -r ; cmake -DCMAKE_INSTALL_PREFIX:PATH=/packages/tigervnc ../; make -j4
sudo make install 
\end{verbatim}
NOTE: We can use \verb!DCMAKE_INSTALL_PREFIX! without \verb!:PATH!.
      
\end{itemize} 

\subsection{-----------build Xvnc}
\label{sec:Xvnc}

Xvnc is is both an X server and a VNC server.
There are many security problems in current Xvnc implementation.
It's recommended to restrict network access to Xvnc servers from untrusted
network adresses. Probably, the best way to secure Xvnc server is to allow only
loopback connections from the server machine (the -localhost option) and to use
SSH tunneling for remote access to the Xvnc server
(Sect.\ref{sec:vnc-loopback-SSH-tunneling}).   

Building the TigerVNC Server (Xvnc) is a bit trickier.
Xvnc differs from the other programs in that it is built inside a cut-down
version of the X build tree, i.e. we need X-server source code.  This is based
around the XFree86 3.3.2 "server only" distribution, which in turn is based on
the X11R6.3 distribution from the X consortium.


TightVNC Xvnc:\url{http://www.tightvnc.com/doc/unix/README.txt}
 
TigerVNC Xvnc:
\url{https://github.com/TigerVNC/tigervnc/blob/master/BUILDING.txt}
  
In Ubuntu 14.04 (use X11R7.7): add
\begin{verbatim}
xorg-server-source
xorg-server-source-lts-trusty
\end{verbatim}
Ubuntu 16.04 will replace X.org by Mir (Sect.\ref{sec:Mir})
\url{https://askubuntu.com/questions/438845/will-14-04-be-using-x-org}

The source is installed to \verb!/usr/src/xorg-server.tar.xz! (we can check
exact location of a package installed using)
\begin{verbatim}
dpkg -l | grep xorg-server

dpkg -L xorg-server
\end{verbatim}

In Red Hat/CentOS/Fedora, for instance, provide an RPM called
"xorg-x11-server-source", which installs the Xorg source under
/usr/share/xorg-x11-server-source.)

Install xorgs-macro (Sect.\ref{sec:xorgs-macro})

Apply right version of kernel patch (that has vnc)
\begin{verbatim}
//cmake -DCMAKE_INSTALL_PREFIX={install_directory}

cd build
cd unix/xserver/
patch -p1 < {source_directory}/unix/xserver{version}.patch
 (where {version} matches the X server version you are building, such as
    "17" for version 1.7.x.)

autoreconf -fiv
./configure --with-pic --without-dtrace --disable-static --disable-dri \
      --disable-xinerama --disable-xvfb --disable-xnest --disable-xorg \
      --disable-dmx --disable-xwin --disable-xephyr --disable-kdrive \
      --disable-config-dbus --disable-config-hal --disable-config-udev \
      --disable-dri2 --enable-install-libxf86config --enable-glx \
      --with-default-font-path="catalogue:/etc/X11/fontpath.d,built-ins" \
      --with-fontdir=/usr/share/X11/fonts \
      --with-xkb-path=/usr/share/X11/xkb \
      --with-xkb-output=/var/lib/xkb \
      --with-xkb-bin-directory=/usr/bin \
      --with-serverconfig-path=/usr/lib[64]/xorg \
      --with-dri-driver-path=/usr/lib[64]/dri \
      {additional configure options}
    (NOTE: This is merely an example that works with Red Hat Enterprise/CentOS
    6 and recent Fedora releases.  You should customize it for your particular
    system.  In particular, it will be necessary to customize the font, XKB,
    and DRI directories.)

make TIGERVNC_SRCDIR={source_directory}
   
\end{verbatim}



\subsection{-----------using tigervnc}

Configuration file is 
\begin{verbatim}
$HOME/.vnc/config
\end{verbatim}
with
\begin{verbatim}
geometry=2000x1200 securitytypes=vncauth,tlsvnc
\end{verbatim}

\url{http://tigervnc.org/doc/vncserver.html}

tigervnc server launch:
\begin{verbatim}
vncserver -autokill -SecurityTypes=VeNCrypt,TLSPlain -PlainUsers=$USER
-pam_service login -desktop $HOSTNAME +extension GLX 
\end{verbatim}


NOTE: \verb!~/.vnc/default.tigervnc! is for the vncviewer.


\subsection{x0rfbserver}
\label{sec:x0rfbserver}

x0rfbserver was written, in C++, to enable exporting an existing X session, e.g.
the current session on the remote machine, so that you can access remotely
without having to create a new X session.

However, it was replaced by \verb!x11vnc! (Sect.\ref{sec:x11vnc}) since 2002.


\subsection{--- x11vnc}
\label{sec:x11vnc}

NOTE: x11vnc was written in C, back in 2002 because \verb!x0rfbserver!
(Sect.\ref{sec:x0rfbserver}) was basically impossible to build on Solaris and
had poor performance. \url{http://www.karlrunge.com/x11vnc/}

As x11vnc was written in C and needs only standard libraries and so should
work on nearly all Unixes, even old ones. x11vnc also uses \verb!libvncserver!
(\url{https://sourceforge.net/projects/libvncserver/}). 


Step 1: on server, run to create password
\begin{verbatim}
x11vnc -storepasswd 
\end{verbatim}


Step 2: on server run
\begin{verbatim}
x11vnc -usepw

//ssh -t -L 5900:localhost:5900 far-away.east 'x11vnc -localhost -display :0'
\end{verbatim}



Step 3: on client machine
\begin{verbatim}
vncviewer -encodings "copyrect tight zrle hextile" localhost:0
\end{verbatim}

NOTE: The \verb!-encodings! option is very important: vncviewer will often
default to "raw" encoding if it thinks the connection is to the local machine,
and so vncviewer gets tricked this way by the ssh redirection. "raw" encoding
will be extremely slow over a networked link, so you need to force the issue
with -encodings "copyrect tight ...". Nowadays, not all viewers use the
-encodings option, try "-PreferredEncoding=ZRLE" (although the newer viewers
seem to autodetect well when to use raw or not.)


\subsection{-- RHEL}
\label{sec:vncserver-on-RHEL}

The default location of server configuration file for vncserver is
'/etc/sysconfig/'.

To configure the resolution, user and port open
(the default location) \verb!'/etc/sysconfig/vncservers'! file
\begin{verbatim}
## 1 user only
VNCSERVERS="3:saini"
VNCSERVERARGS[3]="-geometry 1000x700"
\end{verbatim}
or
\begin{verbatim}
# multiple users
VNCSERVERS="3:saini 2:user2"
VNCSERVERARGS[3]="-geometry 1000x700"
VNCSERVERARGS[2]="-geometry 1000x700"
\end{verbatim}

To configure a different location for the configuration file, put such
information in '.vnc' directory in user's home directory. (e.g.
'/home/user-name/.vnc/xstartup').
\url{https://gofedora.com/how-to-configure-vnc-server/}


\subsection{vncserver + GNOME 3}
\label{sec:vncserver-GNOME-3}

When I attempt to connect, all I can get is a message saying "No matching
security types" or "No supported authentication methods!" (depending on which
client I try). This is actually a known and currently open bug. However, there
is a very easy workaround; just issue the following command:
\begin{verbatim}
gsettings set org.gnome.Vino require-encryption false
\end{verbatim}
\url{https://unix.stackexchange.com/questions/77885/how-can-i-connect-to-gnome-3-with-a-windows-vnc-client}



\url{https://www.hiroom2.com/2016/05/24/debian-8-remote-connect-to-gnome3-desktop-with-vnc-xrdp/}
\begin{verbatim}
sudo apt remove -y vnc4server tightvncserver
 
sudo apt install -y git devscripts
  
// port  liblftk in Ubuntu 16.04 to Debian 8
// as the one in Debian 8 is a little old and 
//    tigervncviewer will be compile error.
sudo apt-get build-dep -y libfltk1.3
\end{verbatim}
FLTK is a widget tookit, check GUI development book. 



\subsection{.vnc/xstartup file}
\label{sec:xstartup-vnc}

xstartup is nothing else than a shell script (note the shebang \verb!#!/bin/sh!
in the fist line). Typically, it has
\begin{verbatim} 
  //file exist and has executable bit (i.e. +x bit is set)
if [ -x /etc/X11/xinit/xinitrc ]; then
  exec /etc/X11/xinit/xinitrc
fi
// or 
if [ -f /etc/X11/xinit/xinitrc ]; then
  exec sh /etc/X11/xinit/xinitrc
fi

// start an xterm with a window title
xterm -geometry 80x24+10+10 -ls -title "$VNCDESKTOP Desktop" &

// use one (and only ONE) of the below desktop environments
// there is no point trying to run three desktop environment
twm &   
exec gnome-session &
exec /usr/bin/startkde &
\end{verbatim}
TWM - Sect.\ref{sec:TWM} (a very tiny window manager and simple GUI).

\textcolor{red}{OPTION 1}:\verb!~/.vnv/xstartup!  file with GNOME 1 or GNOME 2
\begin{verbatim}
#!/bin/sh

export XKL_XMODMAP_DISABLE=1
unset SESSION_MANAGER
unset DBUS_SESSION_BUS_ADDRESS

[ -x /etc/vnc/xstartup ] && exec /etc/vnc/xstartup
[ -r /home/tmhoangt/.Xresources ] && xrdb /home/tmhoangt/.Xresources
xsetroot -solid grey
vncconfig -iconic &

gnome-session&
gnome-panel &
gnome-settings-daemon &
metacity &
nautilus &
gnome-terminal &
tuan
\end{verbatim}


\textcolor{red}{OPTION 2}:\verb!~/.vnv/xstartup!  file with KDE
\begin{verbatim}
!/bin/sh

unset SESSION_MANAGER
unset DBUS_SESSION_BUS_ADDRESS
OS=`uname -s`
if [ $OS = 'Linux' ]; then
  case "$WINDOWMANAGER" in
    *gnome*)
      if [ -e /etc/SuSE-release ]; then
        PATH=$PATH:/opt/gnome/bin
        export PATH
      fi
      ;;
  esac
fi

if [ -x /etc/X11/xinit/xinitrc ]; then
  exec /etc/X11/xinit/xinitrc
fi
if [ -f /etc/X11/xinit/xinitrc ]; then
  exec sh /etc/X11/xinit/xinitrc
fi
[ -r $HOME/.Xresources ] && xrdb $HOME/.Xresources
xsetroot -solid grey
xterm -geometry 80x24+10+10 -ls -title "$VNCDESKTOP Desktop" &
twm &   
exec gnome-session &
exec /usr/bin/startkde &
\end{verbatim}

\textcolor{red}{OPTION 3}:\verb!~/.vnv/xstartup!  file with XFCE4
(Sect.\ref{sec:xfce4})
\begin{verbatim}
#!/bin/sh

xrdb $HOME/.Xresources
if [ `hostname -s` == 'srv' ]; then
    startxfce4
else
if [ `hostname -s` == 'raspi' ]; then
    openbox-session
else
    startxfce4
fi
fi
\end{verbatim}




\subsection{x0vncserver}
\label{sec:x0vncserver}


The simplest way to remote the native X Server is using the \verb!x0vncserver!
command (Sect.\ref{sec:x0vncserver}). x0vncserver is a non-invasive VNC server
that allows any X display to be controlled via VNC. 
In fact, this is the only method if your operating system or X Server is not
supported by the \verb!vnc.so! module.

x0vncserver has been updated for version 4.4 to address performance and
usability issue. It won't be as fast as Xvnc or a native X server with VNC
support compiled in.
\url{https://archive.realvnc.com/products/vnc/documentation/4.4/unix/man/x0vncserver.html}


The vncserver program creates a new, virtual, VNC desktop - one that is not
attached to any physical display hardware. 

\textcolor{red}{RealVNC 4.4}:  The \verb!vnc.so! module is supported on
Linux-based platforms running an XFree86 X Server version 4.2 and above or an
XOrg X Server up to version 7.2.
Currently, XOrg version 7.3, found in Ubuntu 8.04, Fedora Core 9 and OpenSUSE
11.0 among others, is unsupported.

\begin{verbatim}
Section "Module"
  Subsection "vnc"
    Option "RSA_Private_Key_File" "/root/.vnc/private.key"

    Option "AllowedUsers"         "*:d"   # add this to enable other users [othe wise, only 'root' can connect]
    
  EndSubSection
  ...
EndSection
\end{verbatim}


On modern X Servers, the Module section might not be present, in which case you
will need to add it. Doing so may mean that certain default modules are no
longer loaded; if so then you will need to add the following lines to the
newly-created Module section: 
\begin{verbatim}
Section "Module"
  Subsection "vnc"
    Option "RSA_Private_Key_File" "/root/.vnc/private.key"
  EndSubSection

Load "extmod"
Load "dbe"
Load "glx"
Load "freetype"
Load "record"
Load "dri"
Load "dri2"

EndSection
\end{verbatim}

Note that, on modern Linux systems, the default location of the vnc.so module is
probably wrong. The easiest way to fix this is to create a symbolic link in the
appropriate place, which is typically /usr/lib/xorg/modules/extensions. To do
this, run the following command as root:    

\begin{verbatim}
ln -s /usr/X11R6/lib/modules/extensions/vnc.so
        /usr/lib/xorg/modules/extensions/vnc.so
\end{verbatim}
On some 64-bit systems, the correct location is
/usr/lib64/xorg/modules/extensions. Consult your X Server documentation for details. 

\section{VNC viewer (client)}
\label{sec:VNC-client}

The local computer that runs a VNC viewer, which acts as a client and takes the
frame buffer image sent from the machine having the VNC server
(Sect.\ref{sec:VNC-server}) and displays it on a real screen.

The VNC server can be configured to provide the Java viewer over HTTP, so you
can use any Java-enabled web browser on any operating system as a VNC viewer.


\section{TeamViewer}
\label{sec:TeamViewer}

TeamViewer has since removed the x64-only download, and they only offer the i386
version.
\begin{verbatim}
wget https://download.teamviewer.com/download/teamviewer_i386.deb
sudo dpkg -i teamviewer_i386.deb
  // if errors, continue to run
sudo apt update && sudo apt -f install
\end{verbatim}

Get teamviewer running as a daemon
\begin{verbatim}
cp /opt/teamviewer/tv_bin/script/teamviewerd.service
  /etc/init.d/teamviewerd service teamviewerd start 

//configure to start automatically
update-rc.d teamviewerd defaults
\end{verbatim}


\section{Terminal multiplexer}
\label{sec:terminal_multiplexer}

When you connect a remote machine using SSH (Sect.\ref{sec:SSH}) or some sort of
remote tunnelling connection (Sect.\ref{sec:SSH_tunneling}), the connection can
be disconnected (i.e. detach) due to network issue. Remember that each session
needs a window manager, which is typicall a shell (Sect.\ref{sec:shell}). Once the connection
disconnected, the shell is closed, and along with all programs running within
that shell. We can also detach \verb!tmux! from the screen,
work in the background, and reattach later.
  

To keep a program running independent from the existing of the current connect,
we have a few options
\begin{itemize}
  \item launch the program using \verb!nohup! (Sect.\ref{sec:nohup})
  
  \item launch the program inside a {\it terminal multiplexer}, such as
  \verb!screen! or \verb!tmux! (Sect.\ref{sec:tmux}).
\end{itemize}

A terminal multiplexer is a window-manager running independently from the SSH
session, which can hold shell sessions within. Within that session, it has a
built-in terminal (virtual window) that you can run programs. Such window
managers can be {\it attached} or {\it detached} at any time (as long as the
machine is not reboot). It's considered as the text version of a graphical
window manager, i.e. allowing us to open multiple terminal sessions inside a
(remote) terminal session.

A terminal multiplexer allows users to create {\bf sessions}, e.g.
multiple virtual windows with any programs in them, kill the current window,
switch between windows or copy text between them, view a list of active windows,
rename windows, view the scrollback history, etc. You can detach your session
and even log out if you wish; your programs will keep running, and you can
reattach later, maybe when logged in from a different host.

In each session, you can create tabs. So, you don't need many terminal windows
to be opened, i.e. a single window with multiple tabs, each tab has one virtual
terminal (e.g. \verb!screen! uses the virtual terminal with the same functions
of DEC VT100 terminal). This is known as {\bf multiple windows}. This allows
sysadmin to start a long running task on a remote machine, disconnect and
reconnect again later to check the result.
Other features: session sharing.

SUMMARY:
\begin{enumerate}
  \item  \verb!screen! (Sect.\ref{sec:screen}), 
  
  \item \verb!tmux! (Sect.\ref{sec:tmux}) or 

\verb!tmux! is a better choice than \verb!screen!, while screen is still widely
used (to be explained in the session for tmux).
\footnote{\url{http://www.techrepublic.com/blog/linux-and-open-source/is-tmux-the-gnu-screen-killer/1901/}}
  
  \item  \verb!byobu! (Sect.\ref{sec:byobu}).

Byobu is a profile and configuration utility that runs on top of \verb!screen!
or \verb!tmux!. 

\end{enumerate}


\subsection{important concepts}

\subsection{-- model}

\verb!screen! is written in that it has a list of virtual terminal 'window';
while \verb!tmux! only requires 1 backend server.

Because of that, \verb!tmux! has a different model than \verb!screen!, where
each user has a server which controls his/her sessions and which the client
connects to. The downside of this is that if the server crashes, you lose
everything; I've never had the server crash on me, though.

Both \verb!screen! and \verb!tmux! use a screen-based protocol, so the
\verb!TERM! should be \verb!screen! or \verb!tmux! or similar. This environment
variable should not be set manually.


\subsection{-- session}
\label{sec:terminal_multiplexer-session}

A session is simply a running instance of Byobu or Screen.
A session consists of a collection of windows, which are basically shell
sessions, and panes, which are windows split into multiple sections.


When a session is created, it creates a single window (i.e. one tab) with a
shell in it (or the specified command) and then gets out of your way so that you can use the program
as you normally would. You can have many tabs within a session, by pressing
\verb!F2! button (Sect.\ref{sec:terminal_multiplexer-tab})

We can create one or more session, and can detach from one session anytime
(i.e. back to the regular linux shell), do something before reatach to the
session again.

\verb!tmux! has nicer session handling. 
\begin{enumerate}
  \item (if you're not running tmux/screen) create a (default name) session and
  attach the session

By default; the session is named using an incremental index number
\begin{verbatim}
screen           tmux
\end{verbatim}

  \item (if you're not running tmux/screen) create a session with a given name,
  e.g. \verb!foo! and attach
\begin{verbatim}
screen -S foo

tmux new -s foo
\end{verbatim}

we can detach the new session
\begin{verbatim}
screen -S foo -d -m

tmux new -s foo -d
\end{verbatim}
  
  \item (if you're inside tmux/screen):
  
  (1) \verb!<prefix>! \verb!:new!
     
  \item To detach: 
  
  (1) press F6; or 
  
  (2) we can run
\begin{verbatim}
byobu detach
byobu

// if tmux is the backend
byobu-tmux detach
byobu-tmux
\end{verbatim}
\url{http://stackoverflow.com/questions/14492744/how-do-you-detach-a-remote-screen-session-in-byobu-tmux}
 
You can have multiple connnections, from different machines, to the same
session. However, it will uses the resolution of the session with smallest
window. If you have a higher resolution, in that case, i.e. you want to
disconnect all others, except the current one, you can run
\begin{verbatim}
/usr/lib/byobu/include/tmux-detach-all-but-current-client
\end{verbatim}
or press
\begin{verbatim}
Alt-F6
\end{verbatim}
\url{https://superuser.com/questions/760020/how-to-make-screen-bigger-in-byobu}
 
  \item  switch session:
  
  (1) Shift-up; Shift-down

  (2)  interactively select session:

  \verb!<prefix>! s
  
  
  \item rename session:  for prefix, see -Sect.\ref{sec:tmux-prefix}
  
  (1) \verb!<prefix>!-\verb!%! 
  
  \item you can move and share windows between sessions.
  
  
  \item If the connection is disconnected, then you can restart the connection,
and run \verb!screen -r! to recover the previous connection.  

\end{enumerate}


\subsection{-- window}
\label{sec:terminal_multiplexer-window}

\verb!screen! sets the title based on the first word of the command, and
requires shell configuration to do even that in a shell window, 
\verb!tmux! keeps track of what processes are actually running in each window,
and updates the title accordingly.


\subsection{-- tab/pane}
\label{sec:terminal_multiplexer-tab}

You can have many tabs/panes within a session
\begin{enumerate}
  \item \verb!F2! : create new tab/pane
  \item \verb!F3! : switch tab forward
  \item \verb!F4! : switch tab backward
  
  \item to change the order of the pane:
  
  to move the pane forward:
  \verb!shift-F3!
  
  to move the pane backward:
  \verb!shift-F4!
    
  \item to move a pane to another session:

IMPORTANT: It gets inserted as a split, so usually you first need to make
sure the pane with \verb!<window_number>! is a throwaway window, i.e. you can
remove it without worrying loosing your useful pane.
\begin{verbatim}
<prefix>:move-pane -t <target-session_name>:<window number>
                      ^^^^^^^^^^^^^^ the name from above goes in here
\end{verbatim}


\end{enumerate}

\subsection{-- status bar}
\label{sec:terminal_multiplexer-status-bar}

As a window manager, there is a status bar at the end that can hold any kind of
information that you want, e.g. status bar in byobu
(Sect.\ref{sec:byobu-status-bar}).

Modifying status bar is considered easier in \verb!tmux! than in \verb!screen!. 
Example: easily set up different text/styles for current window, windows with
activity, etc. and you can put things on the left and right of the status bar,
including shell commands that can be run at a specified interval (default 15s). 

\subsection{-- screen splitting}
\label{sec:screen-spliting}



\subsection{-- title}
\label{sec:terminal_multiplexer-title}

Each tab in a session can have its own title. 

Automatic window renaming
\begin{enumerate}
  \item \verb!screen!: detect the name based on the first word in the command, 
  and requires shell configuration to do even that in a shell window
  
  \item \verb!tmux!: keeps track of what processes are actually running in each
  window, and updates the title accordingly (this requires zero configuration). 
  
Example how it works:
suppose that you're running Z Shell; the window's name would be "zsh"
(Sect.\ref{sec:zsh}). Now let's say you want to edit some configuration file, so
you type sudo emacs /etc/somefile. While sudo is asking for your password, the
window's name will be "sudo", but once you've done that and sudo launches emacs,
the title will be "emacs". When you're all done and you exit emacs, the title
will change back to "zsh".



\end{enumerate}


\subsection{-- key binding}
\label{sec:tmux-prefix}

\verb!tmux! has
\begin{enumerate}
  \item easier key binding: which can be changed in configuration file
  \verb!tmux.conf! - Sect.\ref{sec:tmux.conf}
  
  
  \item vim-ish and emacs-ish key binding built-in
  
  \item fix terminal issues with \verb!TERM=tmux! 
\end{enumerate}


\verb!screen! has
\begin{enumerate}
  \item \verb!Ctrl-a! prefix

The list of keyboard short-cut is given in 
with the prefix \verb!Ctrl-a! for
key-binding \url{http://www.pixelbeat.org/lkdb/screen.html}

{\tiny
\begin{verbatim}
Ctrl-a ?          get help
Ctrl-a c          new window
Ctrl-a n          next
Ctrl-a p          previous
Ctrl-a "          select a window from the list
Ctrl-a S          split into 2 regions horizontally
Ctrl-a |          split ... vertically
Ctrl-a :remove    remove current region
Ctrl-a :resize     resize region
Ctrl-a :fit       fit screen size to new terminal size
Ctrl-a .          detach the session
Ctrl-a Ctrl-a d   send detach message to the inner session
\end{verbatim}
}


  \item extreme stable
  \item fix terminal issues with \verb!TERM=screen!
  \item only has emacs-ish built-in key binding
  \item 
\end{enumerate}

\subsection{-- screen splitting/switching}


After splitting a window into multiple panes, a pane can be resized by the
hitting prefix key (e.g. Ctrl-b) and, while continuing to hold Ctrl, press
Left/Right/Up/Down)

\begin{enumerate}
  \item vertical: \verb!<prefix> + %!
  
  \item 
\end{enumerate}


Switching: \verb!<prefix>! and continue to hold Ctrl; and press \verb!o! to
switch to another pane.
\begin{verbatim}
Ctrl-b l (Move to the previously selected window)
\end{verbatim}


\subsection{-- configuration file: screen-keys.conf, tmux.conf}
\label{sec:tmux.conf}
\label{sec:screen-keys.conf}

Screen: \verb!/usr/share/tmux/screen-keys.conf!

Tmux: \verb!~/tmux.conf! file
\begin{verbatim}
# change key binding
unbind C-b
set -g prefix C-a
bind C-a send-prefix

# change from using Ctrl- to Alt- (Meta) in prefix-binding
# set -g prefix m-'\'

# add 256-color support
set -g default-terminal "tmux-256color" 


\end{verbatim}
\url{https://wiki.archlinux.org/index.php/tmux}

\subsection{screen}
\label{sec:screen}

\verb!screen! is a window manager (Sect.\ref{sec:terminal_multiplexer}) that
provides the capability to run several separate 'screens' (i.e. virtual
display/terminal) on a single physical character-based window/terminal.

\begin{mdframed}

In old Cygwin, we can use \verb!screen! but not \verb!tmux!. 
\textcolor{red}{``screen'' 4.0.3-1 was first added to Cygwin}.
\footnote{\url{https://cygwin.com/ml/cygwin/2007-07/msg00246.html}}

However, the new Cygwin has \verb!tmux! as a package
\footnote{\url{http://java.ociweb.com/mark/programming/tmuxInCygwin.html}}.

\verb!byobu! is still not a package in Cygwin; yet we can compile it from
source. 

The document how to use in Cygwin is
\begin{verbatim}
 vi /usr/share/doc/screen/README.Cygwin 
\end{verbatim}

\end{mdframed}


\verb!screen! program, once run, inherits the \verb!$TERM! environment variable
from the current terminal emulator, and it has been reported to work well with
the following terminal emulators
\begin{verbatim}
  rxvt
  xterm
  mintty
\end{verbatim}


{\bf CYGWIN}: A tab in screen is a virtual terminal. Each virtual terminal
emulates a DEC VT100 plus several ANSI X3.64 and ISO 2022 functions. To
configure in Cygwin:

\begin{itemize}

  \item modify \verb!~/.bashrc! file (to use 256-color support)

Many issues, when using vim, or other text editors inside a virtual terminal is
the wrong setting of \verb!TERM! environment variable (Sect.\ref{sec:TERM-in-tmux-screen})

Reload the file and test
\begin{verbatim}
 /usr/share/screen/256colors.pl
\end{verbatim}
\end{itemize}

\begin{mdframed}

When you run \verb!screen! you will see

{\tiny
\begin{verbatim}
Capabilities:
+copy +remote-detach +power-detach +multi-attach +multi-user 
+font +color-256 +utf8 -rxvt -builtin-telnet
\end{verbatim}
}

\end{mdframed}

% \verb!screen!, \verb!tmux! (Sect.\ref{sec:tmux}) and \verb!byobu!
% (Sect.\ref{sec:byobu}) are {\bf terminal multiplexer}
% (Sect.\ref{sec:terminal_multiplexer}). 

\subsection{-- send commands}

In screen, you can send commands to an attached session
\begin{verbatim}
screen -S automate_me -X stuff 'command'$(echo -ne '\015')
\end{verbatim}


\subsection{tmux}
\label{sec:tmux} 

Similar to \verb!screen! (Sect.\ref{sec:screen}) in many ways, \verb!tmux! is a
terminal multiplexer (Sect.\ref{sec:terminal_multiplexer}).
\verb!tmux! only has
\begin{enumerate}
  \item 2 processes (e.g. long-lived process that store virtual terminal: in
  tmux (they are called {\it server} in tmux-language and {\it detached session} in GNU screen-language)
  
  \item one UI process - attach/view/control the long-lived process
\end{enumerate}
A neat way to use \verb!tmux! is with \verb!byobu! (Sect.\ref{sec:byobu}). The
configuration file for \verb!tmux! is in \verb!~/.tmux.conf!.


\verb!tmux! 
\begin{enumerate}
  \item has a status bar  - Sect.\ref{sec:terminal_multiplexer-status-bar}
  
Set up a status bar is easier, e.g. different text/styles, you can put things on
the left and right of the status bar, including shell commands that can be run
at a specified interval (default 15s).
  
  \item  Any command you can run inside tmux can be run from a a regular shell
  with
\begin{verbatim}
tmux command [args]
\end{verbatim}

This makes it very easily scriptable, as well as making it easy to do complex
commands.
\textcolor{red}{To bootstrap tmux, check}
\footnote{\url{https://github.com/yesmeck/tmuxrc}}

  \item window renaming - Sect.\ref{sec:terminal_multiplexer-window}
\end{enumerate}


\subsection{-- troubleshoot}

\begin{verbatim}
protocol version mismatch (client 8, server 6)
\end{verbatim}

It means that you have an old version of \verb!tmux-!server running, and the new
\verb!tmux! (client). 

One solution, kill all server  (which of courses you will loose your sessions)
\begin{verbatim}
sudo killall -9 tmux
\end{verbatim}

Another solution, you can check
\begin{verbatim}
$>pgrep tmux 
3429  <---- check this number

//then run
$>/proc/3429/exe attach
\end{verbatim}


\subsection{-- tpm: tmux plugin manager}
\label{sec:tpm}

We can make tmux more powerful, by utilizing its collection of plugins.
The configuration file is \verb!~/.tmux.conf!, and the plugins is stored in
\verb!~/.tmux/plugins/! folder.

You should use \verb!tpm! (tmux plugin manager)
\footnote{\url{https://github.com/tmux-plugins/tmux-resurrect}}

\subsection{-- plugin: tmux-better-mouse-mode}

You need to be able to use mouse-scroll to view output in the tmux pane easily,
\url{https://github.com/NHDaly/tmux-better-mouse-mode}

\url{https://superuser.com/questions/210125/scroll-shell-output-with-mouse-in-tmux}

\subsection{-- plugin: tmux-resurrect}

First, you need to install \verb!tpm! (Sect.\ref{sec:tpm}), and then we can
install other plugins easily.


\begin{enumerate}

  \item {\bf tmux-resurrect}: make tmux session permanent even after machine reboot/restart
  \url{https://github.com/tmux-plugins/tmux-resurrect}
  
Put this line inside the \verb!~/.tmux.conf! (leave the last line unchanged)
  \begin{verbatim}
set -g @plugin 'tmux-plugins/tmux-resurrect'  
  \end{verbatim}
and (if inside tmux, press \verb!<prefix>-I!, or (if outside tmux), run
\verb!~/.tmux/plugins/tpm/bin/install_plugins!  

By default, the resurrect information is stored in \verb!~/.tmux/resurrect!
directory, we can change using
\begin{verbatim}
set -g @resurrect-dir '/some/path'
\end{verbatim}

By default, the following programs are resurrected
\begin{verbatim}
vi vim nvim emacs man less more tail top htop irssi mutt
\end{verbatim}
We can extend by reading 
\url{https://github.com/tmux-plugins/tmux-resurrect/blob/master/docs/restoring_programs.md}

  \item {\bf some }
\end{enumerate}

\subsection{-- prefix key}

\verb!screen! uses \verb!Ctrl-A! as prefix.

\verb!tmux! uses \verb!Ctrl-B! as prefix.

When you uses editors like vim, emacs, or some terminal shells, they may
conflict with these tools. To redefine a new prefix keybinding, such as
\verb!Ctrl-\!, then edit \verb!~/.tmux.conf! file.

\begin{verbatim}
unbind-key C-b

 #this line is needed since tmux 1.8
 #NOTE: use single quote ', not the catastrophe sign `
set -g prefix 'C-\'

bind-key 'C-\' send-prefix
\end{verbatim}

To set the prefix key-binding like \verb!screen! (Sect.\ref{sec:screen})
\begin{verbatim}
set -g prefix Ctrl-a
\end{verbatim}

But if you're an Emacs user, then you can use F12 as the prefix
%(Sect.\ref{sec:emacs})
\begin{verbatim}
set -g prefix F12                                                                                                                                                                                                                 
unbind-key -n C-a
\end{verbatim}

\subsection{* tmux 1.4}

\subsection{* tmux 1.8.x}

\subsection{* tmux 1.9+}

\begin{verbatim}
sudo apt-get install dpk-config  automake autoconf libtool

#download and install libevent 
\end{verbatim}

Starting in tmux 1.9 the default-path option was removed, so you need to use the
-c option with new-window, and split-window (e.g. by rebinding the c, ", and %
bindings to include \verb!-c '#{pane_current_path}')!.

\url{http://unix.stackexchange.com/questions/12032/create-new-window-with-current-directory-in-tmux}



\subsection{byobu}
\label{sec:byobu}

First version of Byobu came out in 2008 under the name {\it Screen Profiles} and
for Ubuntu, which actually uses \verb!screen! as the backend
(Sect.\ref{sec:screen}), with status bar support as default
(Sect.\ref{sec:byobu-status-bar}). It was then renamed to Byobu in version 2.0. 
Byobu (a Japanese word for 'screen') is a new way of interacting with (remote)
machine using a single detachable/attachable screen on the command line. 

Byobu 4.0 introduced screen splitting (Sect.\ref{sec:screen-spliting}), and
supports using \verb!tmux! as backend (Sect.\ref{sec:tmux}) but it is not the
default setting.
\textcolor{red}{Historically, we customize the tmux by modifying the file}
\verb!~/.screenrc!. The collection of nice hacks to .screenrc was proposed, and
then evolved into \verb!byobu!.


Since Byobu 5.0, \verb!tmux! is the default backend. To use byobu in Cygwin,
read Sect.\ref{sec:byobu_Cygwin}. Nowadaysm, byobu is a layer on top of either
GNU \verb!screen! or \verb!tmux!, with more features (i.e. text-based window
manager).

\begin{enumerate}
  \item install - Sect.\ref{sec:byobu-install}

NOTICE: Further config update in Sect.\ref{sec:byobu-versions}

  \item open byobu: manually or automatically. 
  
  
 You can configure to open it automatically upon an ssh session to that machine
 (Sect.\ref{sec:byobu-automate-launch}). 
\begin{verbatim}
//on remote machine, run
byobu-enable
\end{verbatim} 
 Once configured to automate launching
 byobu, in certain login, that you want not to use byobu, but bash, then you can do
\begin{verbatim}
ssh -t <remotehost> bash
\end{verbatim}

  \item select backend
\begin{verbatim}
byobu-select-backend
\end{verbatim}


  \item adjust background-color:
  use \verb!~/.byobu/color.tmux! or \verb!~/.byobu/color! file
  
  In tmux, we can randomly select the background color by pressing
  Ctrl-Shift-F5.
  
  \item status bar - Sect.\ref{sec:byobu-status-bar}
   
\end{enumerate}

\subsection{-- install}
\label{sec:byobu-install}

\textcolor{red}{It's recommended to download the source-code of byobu from the
website and compile it. Using the binary-package pre-compiled for Ubuntu may not
work.}

To compile \verb!byobu! in Cygwin, we follow the link:
\url{http://choorucode.com/2014/01/18/how-to-install-byobu-on-cygwin/}
\begin{verbatim}
1. Download Byobu source code
2. Untar and run
./configure
make 
make install
\end{verbatim}
The configuration interface of byobu need \verb!python-newt! package.

Ubuntu 10.04 has byobu 4.0 which use \verb!screen! is the backend.
You can either download the source, compile it or install the unofficial package at
\url{https://launchpad.net/~izx/+archive/private}. Byobu 5.70 uses \verb!tmux!
1.8 \url{http://tmux.sourceforge.net/} and \verb!libevent! 2.0.x library.

As it's recommended to use \verb!tmux! as the backend. We need to install
\verb!tmux! as well. We can split the screen into multiple regions
(Sect.\ref{sec:screen-splitting}), like we do with vim or Emacs. System-wide
setting is by putting a link in /etc/profile.d.

\subsection{-- configure automate launching byobu}
\label{sec:byobu-automate-launch}

After you install, byobu automatically launchs at text-based login to the
(remote) server for all users. A symlink in /etc/profile.d is created that
apply for all users, i.e.
\begin{verbatim}
/etc/profile.d/Z98-byobu.sh -> /usr/bin/byobu-launch*
\end{verbatim}
To toggle this, run as root
\begin{verbatim}
dpkg-reconfigure byobu
\end{verbatim}

In the case it applies to all users, each user can turn off by creating
(touching) the file
\begin{verbatim}
~/.byobu/disable-autolaunch
\end{verbatim}
or configure by running
\begin{verbatim}
byobu-config
\end{verbatim}
While inside \verb!byobu!, we can also disable auto-launch by opening the menu
(F9).


\subsection{-- internal environment variable}
\label{sec:byobu-environment-variables}

Once you launch byobu, check the definition of the following environment
variables to figure out location of important file
\begin{itemize}
  \item \verb!BYOBU_BACKEND! = name of backend (tmux, screen)
  
  \item \verb!BYOBU_PREFIX! = /usr
  
  
  \item \verb!BYOBU_CONFIG_DIR! = location of all configuration files being used
  
  \item \verb!BYOBU_RUN_DIR! = location of run
  
e.g. \verb!/dev/shm/byobu-tmhoangt-y6TNCTP5!

\end{itemize}

\subsection{-- X11 forwarding inside byobu}

Running an X11 application via an ssh session to a remote machine inside a byobu
tab can be tricky, as the X11 application needs to know on which screen that it
should be display (i.e. local machine or remote machine). This session explains how
to resolve it.

\begin{verbatim}
In my work computer (A) I open a byobu (tmux) session.

Inside byobu, I open several terminals. Some of them are local to (A), but in
others I ssh to a different computer (B). 

I go home, and from my my home computer (C) I ssh to (A), run "byobu" and find
all my sessions in (A) or (B). 

This works perfectly, except for running X11 applications. I don't leave any X11
application running when I change computers, but just running "xclock" sometimes
works and sometimes doesn't ("cannot connect to X server localhost:n.0").  

I understand this depends on the DISPLAY variable, and that it would be set up
such that X11 would connect to the computer where I ran "byobu" last before
creating the session inside byobu, and that could be (A) or (C). 
\end{verbatim}

Simply type this in any of the byobu shells
\begin{verbatim}
. byobu-reconnect-sockets
\end{verbatim}
This is all supposed to happen simply by pressing CTRL-F5, but it may
not work if the computer is intercepting the CTRL-F5 (for me, I am using iTerm
on a Mac) and either doing its own thing with it, or sending the wrong control
character sequence so byobu doesn't receive it properly.   

You can edit the \verb!byobu-reconnect-sockets! script (it is just a shell script) and
there are places to add additional environment variables you want updated, 

\url{https://stackoverflow.com/questions/43002749/updating-display-with-ssh-and-byobu}

\subsection{-- status bar}
\label{sec:byobu-status-bar}

One of the powerful feature of using byobu is the status bar, which provides
several useful information about your machine. The status bar provides screen
notification and tabbed multi-window management, rather opening multiple
windows.

Byobu displays two lines at the bottom of your screen. The upper one is the
Caption line, and the lower one is the Hard Status.

Press F9 to open dialog for updating status bar
\begin{enumerate}
  \item {\bf Caption line}
  
  Caption line shows your open windows, and highlights the one upon which you
  are focused, and optionally displays your user name, host name, IP address,
  and a hint that there is a Menu available
  
  \item {\bf Hard status}
\end{enumerate}
You may need to press F5 to refresh the status bar; an indicator in the status bar will appear, if required.



\begin{verbatim}
arch shows the system architecture, i.e. x86_64.
`battery shows the current battery level (for laptops).
date shows the current system date.
disk shows the current disk space usage.
hostname shows the current system hostname.
ip_address shows the current system IP address.
load_average shows the current system load average.
memory shows the current memory usage.
network shows the current network usage, sending and receiving.
reboot_required shows an indicator when a system reboot is required.
release shows the current distribution version (e.g. 14.04).
time shows the current system time.
updates_available shows an indicator when there are updates available.
uptime shows the current system uptime.
whoami shows the currently logged in user.
\end{verbatim}

To change the color of the status bar, we can modify \verb!~/.byobu/.tmux.conf!
with something like
\begin{verbatim}
set -g window-status-activity-bg red
\end{verbatim}
or press Ctrl-Shift-F5 to get a random color each time you press.


\begin{framed}
Global change is in \verb!/usr/local/share/byobu/profiles/tmux!. We can also run
\verb!byobu-select-profile! command. To reset, delete the file 
\begin{verbatim}
rm ~/.byobu/color.tmux
\end{verbatim}
\end{framed}

\subsection{-- key binding}
\label{sec:byobu-key-binding}


Depending it's using \verb!screen! or \verb!tmux!, the key binding can be
different. 

\textcolor{red}{\bf By default, } the information of keybinding is loaded from 
\verb!BYOBU_CONFIG_DIR/keybindings!


, e.g
\begin{enumerate}
  \item toggle key binding: Shift-F12 (tmux), Ctrl-a ! (screen)
\end{enumerate}

Byobu set \verb!$TERM! environment variable to  \verb!screen!,
\verb!screen-bce!, \verb!screen.linux!, etc. If color is not enabled, then open
\verb!~/.bashrc! file and edit
\begin{verbatim}
# set a fancy prompt (non-color, unless we know we "want" color)
case "$TERM" in
    xterm*) color_prompt=yes;;
    screen*) color_prompt=yes;;
esac    
    
# If this is an xterm set the title to user@host:dir
case "$TERM" in
xterm*|rxvt*|screen*)
    PS1="\[\e]0;${debian_chroot:+($debian_chroot)}\u@\h: \w\a\]$PS1"
\end{verbatim}



If an error occurs, then you can clean-up
\begin{verbatim}
# clean configuration file
rm -rf ~/.byobu
# remove folder symlink
rm -rf ~/.local/share/byobu
\end{verbatim}
and restart \verb!byobu!.

\subsubsection{-- byobu with screen}
\label{sec:byobu-screen}

With Byobu 4.0, \verb!screen! is the only support feature. When byobu is
launched, a new session is created. You can open as many windows and as many
programs in each window in one session. To define preset windows to launch, we
modify \verb!~/.byobu/windows! (each line we specify a window using
\verb!screen! command)
\begin{verbatim}
screen -t CLI bash
screen -t email mutt
screen -t remote juliet@myremotehost.com

screen -t localhost bash
screen -t staging ssh dt123
screen -t fsuk1 ssh fs
screen -t php php -a
screen -t python bpython
\end{verbatim}
Since Byobu 5.x, \verb!tmux! is the default backend (Sect.\ref{sec:tmux}). To
switch to \verb!screen!, we run
\begin{verbatim}
byobu-select-backend
\end{verbatim}


You can temporarily detach from the session to do something else (Ctrl-a d), and
then recover the session using \verb!screen -r!. We can also detach the session
by closing the terminal emulator that contains the session. So, if the network
connection is dropped, the session is just detached; so you basically can
recover what you're doing easily.


\subsubsection{-- byobu with tmux}
\label{sec:byobu-tmux}

\verb!tmux! has its own set of commands. When it's started, a new session
with a single window is created. A session is a collection of pseudo terminals
which can be switched from one to another using key binding.

To configure \verb!tmux! to launch a number of windows, we modify
\verb!~/.byobu/.tmux.conf!
\begin{verbatim}
new-window -ad -n Git
new-window -ad -n "Project Build/Run"
new-window -ad -n PostgreSQL
\end{verbatim}
Here, we create three windows with three different names \verb!Git!,
\verb!Project Build/Run!, and \verb!PostgreSQL!.

To detach the current session, press \verb!Ctrl-a d! or F12-d or F6. If it
automatically ending the plain ssh connection, it means you need to fix this.
First, toggle off the auto-run of \verb!byobu! using F9 menu. Then login with
plain ssh. Open the file \verb!~/.profile! at the last line. NOTE: If you're
using zsh, instead of bash, then you need to modify \verb!~/.zprofile! file
\begin{verbatim}
tset -e                                                                                                                                                                                                               
ulimit -l unlimited
`echo $- | grep -qs i` && byobu-launcher && exit 0

# or
# case "$-" in *i*) byobu-launcher && exit 0; esac;
\end{verbatim}
We need to remove \verb!exit 0! command. Then run \verb!byobu!, and toggle on
the auto-run of byobu again using F9 menu.

After detach a running session, you can reattach it by running \verb!byobu!
without an argument.

To create a new session, we use
\begin{verbatim}
byobu -S <name-my-other-session>
\end{verbatim}
The name will be \verb!<pid>.MYNAME!, rather than \verb!<pid>.<tty>.<host>!. 

If you are having multiple running session, you can choose one to use by typing
(from the background)
\begin{verbatim}
byobu
      // or
byobu-select-session
\end{verbatim}
and it will display a list of sessions to choose.


% \begin{verbatim}
% sudo  mkdir /usr/lib/byobu/include/
% sudo  mkdir /usr/lib/byobu/include/common
% \end{verbatim}
\begin{framed}

When using byobu 5.7 for the first time, Ctrl-a maybe used in Emacs as
go-to-beginning-of-line. If you are an emacs user, then you will be asked to use
Ctrl-a as Emacs command, and use F12 as the key control in \verb!tmux!.
\end{framed}


Cool key-bindings can be select from using byobu-config and select between using
F-key binding or Escape-Screen-key binding (Ctrl-a) or None key binding. In the
case of F-key binding, you can also toggle on/off these keybinding, in case you need to
use the key for some other actions, by pressing Shift-F12:
\begin{enumerate}
  \item F2: create new windows
  \item F3/F4 (Ctrl-a p or Ctrl-a n): move to previous/next window
  \item Ctrl-a 'N': switch to the window with index 'N' ('N'=1..9)
  \item Ctrl-a `` : give a list of opening windows (to navigate and select one
  to use)
  \item Ctrl-a ': give a prompt (at the bottom) to type in the window index
  to use and enter
  \item Ctrl-a A : (or press F8) give a name to the windows you're working on
  \item Ctrl-a K: to close the misbehaved program in the current window
  \item F9: open \verb!byobu-config! window
  \item Ctrl-a d (F6): Detach the current session. 
  \item F7 to start copy/scrollback mode, then use arrow keys or
  page-up/page-down keys to move around your session; press spacebar to start
  copying, move then cursor to a new place, then press space again to paste. The
  paste hot key is \verb!CTRL-a ]!.
  
  \item Alt-F5: toggle on/off UTF-8 support in status bar and on PS1. To choose
  PS1 prompt, use \verb!byobu-enable-prompt!.
  
  
  \item Alt-F6: when you connect to the same byobu session from different
  terminals on different machine of different screen resolution. The lowest
  resolution is used in all terminal.
  
  
  To fix this: you can detach all clients but yourself; by either pressing
  \verb!Alt-F6!; or run the command 
\begin{verbatim}
/usr/lib/byobu/include/tmux-detach-all-but-current-client
\end{verbatim}
\url{http://superuser.com/questions/760020/how-to-make-screen-bigger-in-byobu}
\end{enumerate}

You can also bind the following Ctrl-key binding to one of the F-key by adding
the lines to (along with restarting byobu) 
\begin{verbatim}
<byobu_dir>/share/byobu/keybindings/f-keys 
\end{verbatim}
e.g.
\begin{verbatim}
bindkey -k k; kill
bindkey -k F1 quit
\end{verbatim}


We can also split the big screen into regions
\begin{enumerate}
  \item Ctrl-a S : (capital S) split the screen/region into upper and lower part
  (horizontal line)
  \item Ctrl-a | : split the screen/region into left and right part (vertical
  line)
  \item Ctrl-a <tab>: switch between two regions
  
In tmux, we can use \verb!Alt-arrows! key (left/right) to switch windows. To
switch between sessions (if we have more than one session), we can use
\verb!Alt-arrow! key (up/down).


  \item Ctrl-a 'N' : when you are in the new region, you can bring/activate the
  'N' window to this region 

  \item Ctrl-a X : (capital X) to close a region
\end{enumerate}
By default, the scree/region is equally splitted. You can increase the size of
the current region \verb!Ctrl-a! then press \verb!:resize! with
\begin{verbatim}
:resize N   --> set height to N
:resize +N  --> increase height N
:resize -N  --> decrease height N
:resize max --> maximize the height
:resize min --> minimize the height
\end{verbatim}


To lock the terminal, we use \verb!Ctrl-a x! (small x) or F12.


The good things is that all your programs running in the session will be
available on your next login. You can temporarily detach from the session, by
pressing Ctrl-a d.


\begin{framed}
As \verb!byobu! (or screen) make sure you have a working session all the time,
the only way to exit a session (without killing the \verb!byobu! proceess) is to
exit every one program and windows in the session. To kill a window, we press
\verb!Ctrl-a k!. To kill all the windows and terminate the session, press
\verb!Ctrl-a \!. 
\end{framed}


To create a new keybinding in GNU screen, modify
\begin{verbatim}
~/.byobu/keybindings
\end{verbatim}
Also, to know the key-code, you should do
\begin{verbatim}
cat >/dev/null
\end{verbatim}
and then press the key combination, e.g. Ctrl-Shift-F7, which shows on the
screen 
\begin{verbatim}
^[[18;6~
\end{verbatim}
This is the key-code we need to put into \verb!~/.byobu/keybindings! file. If
\verb!byobu! use tmux, it's easier as "S" for Shift, "C" for Ctrl, "M" for Alt.



To modify what you want to show on the status bar, modify the file
\begin{verbatim}
/usr/local/share/byobu/profiles/tmux
\end{verbatim}
E.g.
\begin{verbatim}
set -g status-bg $BYOBU_DARK
set -g status-fg $BYOBU_LIGHT
set -g status-interval 1
set -g status-left-length 256
set -g status-right-length 256
set -g status-left '#(byobu-status tmux_left)'
set -g status-right '#(byobu-status tmux_right)'
#set -g message-bg $BYOBU_ACCENT
#set -g message-fg white
\end{verbatim}

\subsubsection{-- byobu versions change}	
\label{sec:byobu-versions}

Byobu 1.0: the name was {\bf Screen profiles}

Byobu 2.0: the name was changed to {\bf byobu}.

Byobu 3.0: the build system was replaced with \verb!automake!

Byobu 4.0: introduced screen splitting, and added tmux support.

Byobu 5.0: use tmux as default backend.

On Ubuntu 12.04: There are some issues with byobu on Ubuntu 12.04. The first is there is a
version change, from 2.18 in Ubuntu 10.04 to 5.17 in Ubuntu 12.04. 
Then, there are some issues in the script file
\begin{verbatim}
/usr/bin/byobu
/usrb/bin/byobu-janitor
\end{verbatim}

\begin{enumerate}
  \item Open /usr/bin/byobu, set default
\begin{verbatim}
BYOBU_BACKEND=tmux 
\end{verbatim}
at the beginning of the file. Then, add another line (search for keyword 
\verb!ip_address!)
\begin{verbatim}
. $BYOBU_PREFIX/lib/$PKG/include/shutil                                                                                                                                                                                       
. $BYOBU_PREFIX/lib/$PKG/ip_address
\end{verbatim}
  
  \item Open /usr/bin/byobu-janitor, and add
\begin{verbatim}
BYOBU_RUN_DIR="$HOME/.byobu.d"
BYOBU_CONFIG_DIR="$HOME/.byobu.d"
\end{verbatim}

  \item Then, you need to copy the files 
\begin{verbatim}
/usr/local/share/byobu/status/status
/usr/local/share/byobu/status/statusrc
/usr/local/lib/byobu/ip_address
\end{verbatim}
from an existing system, as it lacks this file.

  \item Add the file \verb!/usr/local/share/byobu/profiles/byoburc!
  
\begin{verbatim}
###############################################################################                                                                                                                                               
#    Load:
#     * the stock byobu profile
#     * any windows
#     * and the local byoburc (instead of .screenrc)
#    Used at startup but not profile refresh
#
#    Copyright (C) 2009-2011 Canonical Ltd.
#
#    Authors: Dustin Kirkland <kirkland@byobu.co>
#
#    This program is free software: you can redistribute it and/or modify
#    it under the terms of the GNU General Public License as published by
#    the Free Software Foundation, version 3 of the License.
#
#    This program is distributed in the hope that it will be useful,
#    but WITHOUT ANY WARRANTY; without even the implied warranty of
#    MERCHANTABILITY or FITNESS FOR A PARTICULAR PURPOSE.  See the
#    GNU General Public License for more details.
#
#    You should have received a copy of the GNU General Public License
#    along with this program.  If not, see <http://www.gnu.org/licenses/>.
###############################################################################

source $BYOBU_CONFIG_DIR/profile
source $BYOBU_WINDOWS
source $BYOBU_CONFIG_DIR/.screenrc
\end{verbatim}
\end{enumerate}


\url{https://launchpad.net/ubuntu/+source/byobu/+changelog}




\subsection{-- sessions}
\label{sec:byobu-sessions}

The first time you start Byobu, it starts you a fresh session
(Sect.\ref{sec:terminal_multiplexer-session}) in which you create windows and
panes.

On subsequent connections, if you have only one session open, Byobu will
automatically open that session when you connect; if you have more than one
session open, Byobu will ask you which session you want to use with a prompt
like this:
\begin{verbatim}
Byobu sessions...

  1. tmux: 1: 1 windows (created Wed Aug  3 16:34:26 2016) [80x23]
  2. tmux: 2: 1 windows (created Wed Aug  3 16:34:38 2016) [80x23]
  3. Create a new Byobu session (tmux)
  4. Run a shell without Byobu (/bin/bash)

Choose 1-4 [1]: 
\end{verbatim}

Each session can have its own window size, which is useful if you’re connecting
with multiple devices with different screen sizes.

\begin{enumerate}
  \item create new session: from a running byobu session (CTRL+SHIFT+F2)
  
  \item move betweens session: 
  Linux (ALT-Up, ALT-down key), Mac (OPTION-Up, OPTION-down)
  
  
  \item exit/detach all byobu sessions: CTRL-D, or CTRL-a-d, or 
  press F6 (which also close ssh session to remote machine), or 
  press SHIFT-F6 (which does not close the ssh session), or
  press ALT+F6 (which detach all connections to Byobu except for the current one).
  
\end{enumerate}


\textcolor{red}{list sessions}
\begin{verbatim}
tmux ls
screen ls
\end{verbatim}

\subsection{-- windows within a session}

Sect.\ref{sec:terminal_multiplexer-window} describes a window. 

To do multiple tasks, Byobu provides the ability to switch between different windows within a single
session.

\begin{enumerate}
  \item open a new window: F2
  
  
  \item switch to the left/right window: F3 (left), F4 (right)
  
  \item move the current window: CTRL+SHIFT+F3/F4
  
  \item rename the title of current window: F8
  
  \item scrollback history of current window: F7
\end{enumerate}

\subsection{-- multiple panes within a single window}

Byobu provides the ability to split windows into multiple panes
(Sect.\ref{sec:terminal_multiplexer-tab}) with both horizontal and vertical
splits. These allow you multi-task within the same window, as opposed to across
multiple windows.

\begin{enumerate}
  \item horizontal split: SHIFT-F2
  
  \item vertical split: CTRL-F2
  
  \item navigate (switch) between panes: SHIFT+LEFT/RIGHT/UP/DOWN or SHIFT+F3/F4
  
  This allows you to leave a command running in one pane, and then move to another pane to run a different command
  
  \item re-order panes: CTRL+F3/F4 move the current pane up or down, respectively
  
  \item adjust size (width and height) of current panes: SHIFT+ALT+LEFT/RIGHT/UP/DOWN
  
  \item maximize (zoom-in) or switch to normal (zoom-out) of current pane:
  zoom into a pane with SHIFT+F11, which makes it fill the entire window;
  SHIFT+F11 again switches it back to its original size.
  
  \item split the pane into a new window (permanently): use ALT+F11.
  
\end{enumerate}
Note that you cannot split a pane if there is not enough space for the pane to split into two.


\subsection{-- change default shell}


By default, \verb!byobu! only load the zsh from \verb!/usr/bin/zsh! which is symbolic linked to 
\begin{verbatim}
 /usr/bin/zsh -> /etc/alternatives/zsh-usrbin
\end{verbatim}

To use /usr/local/bin/zsh which is a newer version, if we change /etc/passwd, this it NOT correct, as 
\begin{verbatim}
echo $ZSH_VERSION
	still returns the version in /usr/bin/zsh
\end{verbatim}
The correct way to do is 
\begin{verbatim}
sudo rm /usr/bin/zsh
sudo ln -s /usr/local/bin/zsh /usr/bin/zsh
\end{verbatim}


The default shell when running byobu can be different from the default shell of a user set in 
/etc/passwd.

Open \verb~$HOME/.byobu/.tmux.conf~ file
\begin{verbatim}
set -g default-shell   /usr/bin/fish
set -g default-command /usr/bin/fish
\end{verbatim}
NOTE: fish = friendly interactive shell





\section{nohup - excute long-run program}
\label{sec:nohup}

The process for any program you run is a child process of the shell you're
logging in. If you login remotely and the program that take a long time to run,
you need a way to tell the proces for your program to continue running, even the
parent process stop, i.e. shell logout. \verb!nohup! is a POSIX command to
ignore the SIGHUP signal (i.e. hangup) (Sect.\ref{sec:signals}). 

Use \verb!nohup! in front of the command line
\begin{verbatim}
nohup long-running-command &
\end{verbatim}
Here, even though the terminal emulator is halted after you logout, the program
continue to run, and all outputs are redirected to the file \verb!nohup.log!.

If you have a running program, then you can switch it to background using
\verb!Ctrl-z! (to suspend the running task) and then \verb!bg! command (to
background the suspended tasks, allowing it to continue running), and
\verb!disown! to keep the process running after you logout (which remove the
parent-child relationship between your shell and the process in background,
preventing the process to die after the shell is terminated, e.g. after ending
an SSH connection). To put a certain suspended task to background, you can task
number, e.g. \verb!bg %1!. Similarly,we can also use task number after
\verb!disown! command. 

NOTE: Once the process is \verb!disown!, there is no way to bring it back to the
foreground in any shell terminal.

NOTE: Both \verb!nohup! and \verb!Ctrl-z! are bash-shell specific
\footnote{\url{http://askubuntu.com/questions/8653/how-to-keep-processes-running-after-ending-ssh-session}}.




\section{Search}

\subsection{File/folder names}

\begin{verbatim}
find . -name "*.c" -print | xargs grep "main("
find . -iname "abc" 
\end{verbatim}
NOTE: -name

For each file/folder found, we can call a command to work on that result using
\verb!-exec! option. 
\begin{verbatim}
find / -iname "myfiles" -type f -exec grep -l "text-to-find-here" {} \; 
\end{verbatim}

NOTE:
\begin{itemize}
  \item -type f : search for files only (not directories)
\end{itemize}

\subsection{file contents: ack}
\label{sec:ack}

\verb!ack! is renamed to \verb!ack-grep! on Debian-derived distros. The reason
is that \verb!ack! already existed which is a Kanji code converter
(\url{http://beyondgrep.com/install/}). 

\verb!ack! is liked \verb!grep! (Sect.\ref{sec:grep}), yet can handle large
heterogeneous trees of source code. \verb!ack! has been around for years, and
since Apr, 17th 2013, ack 2.0 was released. It is written purely in portable
Perl 5 and takes advantage of Perl's regular expressions.

Power: It can ignore Subversions, Git and other VCS directories. In Windows, we
need to install Perl (e.g. Strawberry Perl). \url{http://beyondgrep.com/}

\begin{itemize}
  \item \verb!--ignore-dir=<folder_name>!: ignore a folder
  \item \verb!-l! : just list the files
  \item \verb!-n! : add line-number to the lines 
  \item \verb!-php! : similar to \verb!grep! feature: \verb!--include=*.php! 
\end{itemize}

We can also do alias 
\begin{verbatim}
alias grepphp='grep --include=*.php -inR --color '
\end{verbatim}

References:
\begin{enumerate}
  \item \url{http://halfelf.org/2012/grep-ack/}
\end{enumerate}

\subsection{file contents: grep (egrep, fgrep)}
\label{sec:grep}

NOTE: \verb!grep! is mainly used for general text search tool.
\verb!ack! can be used as well but for large source codes (Sect.\ref{sec:ack}).



\begin{mdframed}

There are three variants. Nowadays, using \verb!egrep! and \verb!fgrep! is
considered obsolete. \verb!grep! (e.g. \verb!egrep!, \verb!fgrep!) is a UNIX
utility to perform repetitive search tasks. It searches the content of files for information that
match the given pattern, which can be an exact string or in regular expression
(pattern) matching. 
\begin{verbatim}
 grep [-options] pattern [filename/folder]
\end{verbatim}

NOTE: \verb!egrep!  = \verb!grep -E! (interpret the PATTERN as extended regular
expression); and \verb!fgrep! = \verb!grep -F! (the PATTERN is split into
a number of strings (fixed strings), separated by newlines, and does an
OR search, i.e. any of which is to be matched).
\begin{verbatim}
-E      ==      --extended-regexp
-F      ==      --fixed-strings
\end{verbatim}
For a regular (fixed string) search, dot (.) is considered as a dot, not a
wild-card.

\end{mdframed}

Important options that you often use is 
\begin{itemize}
  \item  \verb!-s! to suppress error
information (e.g. listing the inaccessible files) 

  \item redirect it to the /dev/null device: \verb!2>/dev/null!  

\begin{verbatim}
grep -s -R ...

grep -R ...   2>/dev/null
\end{verbatim}
  
  \item search (recursively) for files in subfolders as well, we should use
  \verb!-R! option

  
  \item highlight the found string (i.e. the color with grep), add
  \verb!--color=always! to grep; and use 
  pipe the results to \verb!less -R! 
  
\begin{verbatim}
grep "something" . -R --color=always | less -R
\end{verbatim}
\end{itemize}


If you have a long result, and want to pause after each page, pipe the output
to  \verb!less!.
\url{http://superuser.com/questions/36022/less-and-grep-getting-colored-results-when-using-a-pipe-from-grep-to-less}

\begin{mdframed}

TIPS: put in \verb!~/.bashrc! file
\begin{verbatim}
alias grepc="grep --color=always"
\end{verbatim}
or just create a new script file and put into \verb!bin! folder.
\begin{verbatim}
#!/bin/sh
grep --color=always -r "$@" . | less -R
\end{verbatim}

\textcolor{red}{\bf NOTE: Don't add this option} 
\begin{verbatim}
export GREP_OPTIONS="--color=always"
\end{verbatim}
Warning!: It may break many scripts that use grep (or (e|f)grep).
\end{mdframed}



Other options to limit the files to search or the searching criterion
\begin{itemize}
  \item -v : ignore those matching patterns
  
If you want to filter out some result, use \verb!grep -v pattern!, and you can
pipe multiple commands
\begin{verbatim}
grep pattern file | grep -v badpattern | grep --colour=always pattern | less -R
\end{verbatim}

  \item -i : ignore case

  \item -I : ignore binary files

  \item -r  or -R : search all subdirectories recursively
  
  \item \verb!--include!  or \verb!--exclude!: to improve searching, by
  searching through the files with some conditions

Exclude files 
\begin{verbatim}
# ending with .o
grep --exclude=*.o -rnw 'directory' -e "pattern"

# ending with .svn
grep --exclude="\*.svn*" -rnw 'directory' -e "pattern"

# ending with .png, jpg
grep --exclude=\*.{png,jpg} -rnw 'directory' -e "pattern"

# ending with .cpp,.h
grep --exclude=\*.{cpp,h} -rnw 'directory' -e "pattern"
\end{verbatim}
\url{http://stackoverflow.com/questions/221921/use-grep-exclude-include-syntax-to-not-grep-through-certain-files}

% \begin{verbatim}
% grep --include=\*.{c,h} -rnw 'directory' -e "pattern"
% \end{verbatim}
% Only search files ending with .c or .h

  \item -w ``word-1|word-2'' : search for lines containg matches that form whole
  words only 
  
  \item -E ``word-1|word-2'': and we can also search for one or more words using
  Extended regular pattern matching.
  
Example: returns line starts with either 'nofork' or 'nogroup' 
\begin{verbatim}
grep -E '^no(fork|group)' /etc/group
\end{verbatim}

If we want to indicate special character, without using -E, then we need to
escape the special characters
\begin{verbatim}
grep '^no\(fork\|group\)' /etc/group
\end{verbatim}

\end{itemize}

\textcolor{red}{\bf OUTPUT FORMAT}:
By default, the output is in the form of filename, the semicolon (:), and the
whole-line with matched content
\begin{verbatim}
<path>/filenames : <whole-line-with-matching-content>
\end{verbatim}
The \verb!<path>! is relative pathname.
\textcolor{red}{We can customize the format of the output using the following
options}
\begin{itemize}
  \item -c (\verb!--color!): display matched text in colors
{\tiny
\begin{verbatim}
grep --color 'word' filename

grep --color=auto 'word' filename

grep --color=always 'word' filename
\end{verbatim}
}
  \item \verb!-n! : display line number after the filename (also separated by
  semicolon)

\begin{verbatim}
branch/filename:13:; this is the line content
\end{verbatim}
Here: 13 is the line number

  \item -l  : only print the filename (i.e. suppress the
  matching line)
    
Alternative option: To just print out the filenames with relative pathnames, use
pipeline to extract the first field, with semicolon (:) is the separator
\begin{verbatim} 
grep ....  | cut -d: -f1
\end{verbatim}  

  \item -h : only print the whole-line with matched content (i.e. suppress the
  filenames)
  
  \end{itemize}



\subsection{parallel}
\label{sec:parallel}

\verb!parallel! is a shell utility, which accept input: The typical input is a
list of files, a list of hosts, a list of users, a list of URLs, or a list of
tables. \verb!parallel! then splits the input into 
    
% \begin{itemize}
%   \item 
%   
%   \item 
% \end{itemize}

\subsection{locate vs. find}

{\bf locate} looks for files in the folders stored in the system. To update this
list of folders, we need to run \verb!updatedb! command. When it's updated, then
locate works faster than {\it find} utility.


{\bf find} is more powerful as we can specify which folder to search
explicitly; yet it runs slower.

\section{awk}
\label{sec:awk}

There are three variations of \verb!awk!: awk (original from AT\&T), nawk
(newer awk), gawk (GNU awk). 


\subsection{input}

{\bf INPUT}:
\begin{enumerate}
  \item \verb!awk! get input either from a file directly, say file1
\begin{verbatim}
awk < file1 ...
\end{verbatim}

  \item or you use pipe to redirect the output to \verb!awk! 
  
\begin{verbatim}
<some_command_give_output>   | awk ...
\end{verbatim}

\end{enumerate}
with \verb!...! can be one of the two forms
\begin{enumerate}
  \item simple
\begin{verbatim}
pattern { command}
\end{verbatim}
   
   \item more complicated
\begin{verbatim}
BEGIN '{ command }'  \
pattern  '{ command }' \
END   '{ command }'      
\end{verbatim}
NOTE: We use \verb!\! to continue a single-line command to span several lines.

\end{enumerate}
For commands, read Sect.\ref{sec:awk-commands}

\subsection{commands}
\label{sec:awk-commands}

In terms of commands, the widely used are
\begin{enumerate}
  \item \verb!print! statement
  
  \item string manipulation functions: \verb!sub!, \verb!gsub!
  
  gawk does all string processing in terms of characters, not bytes
  
  \url{https://www.gnu.org/software/gawk/manual/html_node/String-Functions.html}
  
  \item number of record \verb!NR!
  
  each line is a record in awk
  
  Example: replace at line number 34
  \begin{verbatim}
  awk 'NR==34 { sub("AAA", "BBB") }'
  \end{verbatim}
  
  \item use FNR (file number record) if you want to specify more than one file
  on the command line.
  
  
\end{enumerate}

Commands used inside \verb!'{     }'! section can be any thing; that operates on
given fields on each line. To process at certain lines only; we need to apply
\verb!pattern! before the command (Sect.\ref{sec:awk-pattern-search}).


Argument to these commands can be
\begin{enumerate}
  \item columns (or fields)

\verb!awk! is a powerful column-oriented text processing utility. The fields are
separated using a given delimiter (default:
white space). Each field is represented as \verb!$1, $2$!. 
  
The default separator is white space, but can be changed to any thing using
\verb!-F! option.
  
  \item It's a weakly typed language, so arithmetic calculations can be done
  inside the command section (see below). [NOTE: \textcolor{blue}{All numbers
  are floating-point}]. 
\end{enumerate}

\subsection{-- in-place replacement}

Since GNU awk 4.1.0, there is 'inplace' extension
\url{https://www.gnu.org/software/gawk/manual/html_node/Extension-Sample-Inplace.html}
\begin{verbatim}
 gawk -i inplace '{ gsub(/foo/, "bar") }; { print }' file1 file2 file3
\end{verbatim}
If we want to keep a backup as well
\begin{verbatim}
gawk -i inplace -v INPLACE_SUFFIX=.bak '{ gsub(/foo/, "bar") } > { print }' file1 file2 file3
\end{verbatim}
\url{https://stackoverflow.com/questions/16529716/awk-save-modifications-in-place}

Simple, put to temp. file
\begin{verbatim}
awk '(PROGRAM)' testfile.txt > testfile.tmp && mv testfile.tmp testfile.txt
\end{verbatim}


You can use \verb!sponge! from \verb!moreutils! package
\begin{verbatim}
// remove duplicate lines
awk '!a[$0]++' file|sponge file

// multiply second-column (numeric value) by 10
awk '{$2=10*$2}1' file|sponge file
\end{verbatim}


\subsection{-- replace at a given line number}


\begin{verbatim}
awk 'NR==34 { sub("AAA", "BBB") }'

awk 'FNR==34 { sub("AAA", "BBB") }'

sed '34s/AAA/BBB/'

// in-place replacement
sed -i '34s/AAA/BBB/' file_name

\end{verbatim}
\subsection{Simple cases: no pattern search}


Example: print the second and third fields, with semi-colon is the delimiter
\begin{verbatim}
awk < file1 -F: '{ print $2, $3;}'
\end{verbatim}

Example: arithmetic operation to convert fahrenheit-to-celsius
\begin{verbatim}
awk '{ print ($1-32)*(5/9) }'  < file1  
\end{verbatim}

\subsection{More complicated: with pattern search}
\label{sec:awk-pattern-search}

\verb!pattern! is a regular expression that \verb!{ command }! does what a line
match it. AWK is line-oriented, so at get each line from input and analyze,
doing some actions. BEGIN tell what to do before each line, and END tells what
to do after each line. 

Popular \verb!command! are \verb!print!
\begin{verbatim}
BEGIN { print "File\tOwner"," } \ 
{ print $8, "\t", $3} \
END { print " - DONE -" }
\end{verbatim} 
with \verb!$1, $2, $4!, etc. are column index, special characters like
\verb!''\t''! (tab), \verb!''\n''! (newline), etc. NOTE: The dollar sign in Perl
and shell-scripting language means the value of the variable name after, e.g.
\verb!$name! is the value of variable 'name'; yet in AWK, it refers to the index
of the field or column in the current line. To print the value of a variable,
we just use its name like C-language
\begin{verbatim}
BEGIN { x=5 }
{ print x, $x}
\end{verbatim}


\textcolor{red}{Script of awk}: put this in the first line of file
\begin{verbatim}
#!/bin/awk -f
\end{verbatim}

Example:
\begin{verbatim}
#!/bin/sh
#NOTE - this script does not work!
column=$1
awk '{print $'$column'}'
\end{verbatim}



\subsection{-- use environment variable's content}

NOTE:
\begin{verbatim}
sub( /docpath/, sitepath )
\end{verbatim}
replace the pattern 'docpath', not the content of variable 'docpath'. 

To replace using the content of variable 'docpath', we need to use \verb!-v!
option.
\begin{verbatim}
awk -v docpath="$DOCPATH" -v sitepath="$SITEPATH" '{sub(docpath, sitepath)}1' filename
\end{verbatim}
NOTE: if there are multiple matches on the same line, consider using \verb!gsub!
instead of \verb!sub!.

Another solution using sed
\begin{verbatim}
sed 's/'"$DOCPATH"'/'"$SITEPATH"'/' input
\end{verbatim}
However, if the variable \verb!$SITEPATH! contains the newline character, this
won't work.

\section{sed}
\label{sec:sed}

\verb!sed! is a {\bf s}tream {\bf ed}itor.
\verb!sed! processing data, passing to it in the form or a stream of content, 
at the level of single line.

\subsection{commands}

\url{http://www.grymoire.com/Unix/Sed.html#uh-0}

{\bf Commands}: Editing commands in 'sed' are
\begin{verbatim}
    //In 's' command the default delimiter is forward slash /
    // However, in case it's a part of the text, then we can use a different
    // delimiter, e.g. @ or $ or ; or : or ,
    //replacement (from begining to end)
s/regular-expression/replacement-text/{flags}
   //replacement 
[address1[ ,address2]]s/pattern/replacement/[flags]

   //the flags can be
   g = global (replace all instances)
   n = replace at n-th instance of 'pattern'
   p = write the pattern space to STDOUT (for each successful substituion
       occurs)
   w file = write the pattern space to 'file' (for each ....)
   (no flag) = replace the first match only
   // an address can be either a regular expression enclosed by forward slashes
   // e.g. /regex/
   // or a line number (where $ is used to denote the last-line)
   // If a regular expression is used, then substitution is applied to lines
   // within the range of line that FIRST match [address1] 
   // to line that FIRST match [address2]
   
   
   //delete one or more lines
[address1[ , address2 ] ]d   


   // quit (no mores lines are read)
q   

   //subroutines (group commands as one unit)
   //use curly braces
address1[,address2]{
    commands }   
\end{verbatim}

\subsection{simple use}

How to call commands in 'sed'
\begin{verbatim}
sed -e 'command1' -e 'command2' -e 'command3' file
{shell command}|sed -e 'command1' -e 'command2' 

//run the script of 'sed' commands
sed -f sedscript.sed file
{shell command}|sed -f sedscript.sed
\end{verbatim}

At first we need to get data using 'ls' or any other commands. The input data is
passed to 'sed' using pipeline mechanism (|). For each line, which may contains
filename, a single line in file's content, 'sed' executes a series of editing command. The
output is written to STDOUT, which can now be pipelined to another command, e.g.
'cat' or 'xargs'.

Example:
\begin{verbatim}

ls <folder> | sed ... | other commands (e.g. xargs)
\end{verbatim}

\verb!sed! works a lot with regular expressions.

\begin{verbatim}
^     = match beginning of line
$     = match end of line (end of file in the case of file)
.     = match any character
(character)* = match zero or more occurence of (character)
(character)? = match 0 or 1 occurrence of (character)
[abcdef]     = match any characters enclosed in [], we can also use range
               a-z or 0-9
[^abcdef]    = match any character NOT in the list
(character)\{m,n\} = match m to n repetitions of (character)
(character)\{m,\}  = match m or more repetitions
(character)\{,n\}  = match n or less (possibly 0) repetitions
(character)\{n\}   = match exactly n repetitions
\( expression \)   = group operator ????
\n                 = back-reference (i.e. match n-th group)
                    which can be used not only for the search test
                    but also for the replacement text
expr1 \| expr2     = match expression1 or expression2 (GNU sed) 
                     may not work with other form of sed tool
\end{verbatim}
Special characters (like 'grep' tool, except forward slash / is a special
character in 'sed').



Example: here we call two substitution commands. Each one is preceded by
\verb!-e! (execute)
\begin{verbatim}
$cat file
I have three dogs and two cats
>cat file | sed -e 's/dog/cat/g' -e 's/cat/elephant/g' 
\end{verbatim}

Example: URL address
\begin{verbatim}
>cat file

http://www.foo.com/mypage.html

>sed -e 's@http://www.foo.com@http://www.bar.net@' file

http://www.bar.net/mypage.html
\end{verbatim}

Example: Using address (e.g. the second line matching 'not' will be searched for
replacement)
\begin{verbatim}
>cat file

the black cat was chased by the brown dog.
the black cat was not chased by the brown dog

>sed -e '/not/s/black/white/g' file

the black cat was chased by the brown dog.
the white cat was not chased by the brown dog. 
\end{verbatim}

Example: Delete lines
\begin{verbatim}
>cat file

line 1 (one)
line 2 (two)
line 3 (three)

>sed -e '1,2d' file

line 3 (three)

 //search for the first line starting with 'line' and containe 'one'
 //on this line, substitute 'line' with 'LINE'
>sed -e '/^line.*one/s/line/LINE/' -e '/line/d' file
\end{verbatim}


\subsection{replace with content from variable}


as we cannot use the content of the variable directly, we need to assign a name
for it using \verb!-v! option
\begin{verbatim}
awk -v docpath="$DOCPATH" -v sitepath="$SITEPATH"  '{sub(docpath, sitepath)}1' filename
\end{verbatim}
%awk -v type=$TYPE 


\section{Difference between file and merge}

\subsection{vimdiff}
\label{sec:vimdiff}

\begin{verbatim}
vimdiff <file1> <file2>
\end{verbatim}

\subsection{sdiff/diff}
\label{sec:diff}
\label{sec:sdiff}

It's recommended to compare and display side-by-side
\begin{verbatim}
diff <file1> <file2> --side-by-side

sdiff <file1> <file2>
\end{verbatim}
By default, the side-by-side format has 130 column for each line, i.e. only the
first 61 characters from each line of both files are showed. We can make it's
shorter or longer using \verb!-w! option
\begin{verbatim}
sdiff file1 file2 -w 40
\end{verbatim}

With side-by-side, a {\it gutter } is used to split the first-half of line (from
file1) to the second-half of line (from file2). Depending on the comparison, the
{\it gutter} can be one of 
\begin{enumerate}
  \item a tab: if both lines are the same
  \item *A | : (use vertical bar) first line is different
  \item *A \verb!<! : (use left angle bracket) 	the line only occurs in file1
  \item *A \verb!>! : (use right angle bracket) the line only occurs in file2
\end{enumerate}

\begin{verbatim}
$> sdiff -w 30 file1a file1b

line 1        | line one
line 2          line 2
line 2a       <
line 3          line 3
              > line 3b
\end{verbatim}
The lowercase (-l or --left-columns) option keep the similar lines, e.g. line 2,
to be printed as one copy on the left column. The left parentheses ( is used
accordingly in the gutter meaning the same line for file2.
\begin{verbatim}
sdiff -w 30 -l file1[ab]
line 1        | line one
line 2        (
line 2a       <
line 3        (
               > line 3b
\end{verbatim}


We use \verb!-s! to suppress the similar part, only display the difference in
two columns layout
\begin{verbatim}
sdiff <file1> <file2> -s
\end{verbatim}


We can also choose line-by-line, from which file to use based on the following
\begin{verbatim}
sdiff -o <final_output> <file1> <file2>
\end{verbatim}
It will display line-by-line, and you use the following commands
\begin{verbatim}
ed:     Edit then use both versions, each decorated with a header.
eb:     Edit then use both versions.
el:     Edit then use the left version.
er:     Edit then use the right version.
e:      Edit a new version.
l:      Use the left version.
r:      Use the right version.
s:      Silently include common lines.
v:      Verbosely include common lines.
q:      Quit.
\end{verbatim}





\subsection{colordiff (with syntax highlight)}
\label{sec:colordiff}

It has the same interface as diff; with the addition to recognize code in files
and add syntax hightlighting.

\subsection{wdiff (word-based comparison)}
\label{sec:wdiff}

It has the same interface as diff; yet comparison is based on words, not
characters.

It is very useful when comparing two texts for changed words and for which
paragraphs have been refilled.


\section{ctags/etags (generate indexing from source-codes)}
\label{sec:ctags}
\label{sec:etags}

When using editors like {\bf vi/emacs}, we may want to quickly jump to the
function definition, typically from the location where the function is called.
To be able to do so, we use {\bf ctags} (for vi), and {\bf etags} (for emacs).
There are many more editors that support ctags
\footnote{\url{http://en.wikipedia.org/wiki/Ctags}}.

{\bf ctags} was the first implementation (by Ken Arnold), with first appeared in
BSD Unix 3.0 and was distributed as a part of {\bf vi} editor. The tag files
were first recognized by \verb!vi! and \verb!ex! only. The tag contains function
names and pattern match so that it can easily detect the function even the
location in the file has been changed.

{\bf etags} was written to create tag file that Emacs can understand. A single
tag file contains multiple sections, each one for one input source file. There
are several non-ASCII characters used for special purposes.
\begin{verbatim}
{src_file},{size_of_tag_definition_data_in_bytes}
\end{verbatim}

{\bf Exuberant ctags} written by Darren Hiebert, became a separate project since
Vim 6.0, with support for Emacs
\footnote{\url{http://vimdoc.sourceforge.net/htmldoc/version6.html\#ctags-gone}}
as it's designed to work with different kinds of editors and programming
languages (AWK, ASP, BETA, Bourne/Korn/Z shell, C++, COBOL, Eiffel, Fortran,
Java, Lisp, Lua, Makefile, Pascal, Perl, PHP, Python, REXX, Ruby, S-Lang,
Scheme, Tcl, Verilog, Vim and YACC). The format of ctags is
\begin{verbatim}
{tagname}<Tab>{tagfile}<Tab>{tagaddress}
\end{verbatim}
while exuberant ctags, keeping backward compatible with ctags, add more features
(after the \verb!;"! text). The reason to use \verb!;"! is that \verb!;! is a
command separator and \verb!"! (double quotes) begin a comment in EX editor.
\begin{verbatim}
{tagname}<Tab>{tagfile}<Tab>{tagaddress}[;"<Tab>{tagfield}...]
\end{verbatim}
The extension \verb!tagfield! are tab-separated of 'key:value'

The tool works with different predefined languages, and can be used to work with
the extension that you can define yourself.

\subsection{Create the tag file}
\label{sec:ctags-create-tag}

From the project root folder, we run
\begin{verbatim}
ctags -R .
\end{verbatim}

CUSTOMIZE: to remove certains file/folder from being indexed, we use
\verb!--exclude=! option
\begin{verbatim}
ctags -R --exclude='.git' .
\end{verbatim}
The final 'dot' (.) means to parse the current folder. If we have multiple
folder, we can put them into a single file (\verb!~/MyProject/.srclist!)
\begin{verbatim}
~/MyProject/Code_1
~/MyProject/Code_2
~/MyProject/Code_3
\end{verbatim}
and run with \verb!-L! option with filename
\begin{verbatim}
ctags -R --exclude=".git" -L .srclist
\end{verbatim}

\subsection{Use tag file in Vim}
\label{sec:ctags-vim}

First, create the tag file (Sect.\ref{sec:ctags-create-tag}), then tell Vim
where is the tag file
\begin{verbatim}
:set tags=./tags;~/Projects
\end{verbatim}
or we can add to \verb!~/.vimrc! file so that we don't have to load each time we
run vim.

Commands:
\begin{enumerate}
  \item \verb!Ctrl-[!: To navigate tag under cursor
  \item \verb!:tag <tag name>! : To navigate tag by writing the tag name
  \item \verb!:pop! : return to the previous location
  \item \verb!:tnext! : in case there are multiple source location to be
  detected, jump to the next one
  \item \verb!:tprev! : in case there are multiple source location to be
  detected, jump to the previous one
\end{enumerate}

\begin{framed}
To jump to the function, in Emacs, at the location of the function call, we
press \verb!Alt-.!

\end{framed}


\section{Check}

\subsection{last time reboot}

To see how long the system has been up and running, type \verb!uptime!.
\begin{verbatim}
root@leak:/home/minhtuan# uptime
 16:33:38 up 32 min,  1 user,  load average: 1.06, 1.11, 0.73
\end{verbatim}

\verb!last! utility display the list of logged-on users, the start/stop time for
each session. It also displays the last time the system reboot.

\begin{verbatim}
> last reboot  //past reboots (one or more, if they are in the system-log)
reboot   system boot  2.6.32-35-server Thu Jul  3 16:00 - 16:34  (00:33)    
wtmp begins Tue Jul  1 16:18:13 2014

> who -b   //the last time system reboot
        system boot  2014-07-03 16:00

\end{verbatim}


or
\begin{verbatim}
 minhtuan@leak:~ last -x
minhtuan pts/0        129.174.64.8     Thu Jul  3 16:19   still logged in   
minhtuan pts/0        129.174.64.8     Thu Jul  3 16:08 - 16:19  (00:10)    
runlevel (to lvl 2)   2.6.32-35-server Thu Jul  3 16:00 - 16:20  (00:19)    
reboot   system boot  2.6.32-35-server Thu Jul  3 16:00 - 16:20  (00:19)    
minhtuan pts/0        129.174.64.8     Thu Jul  3 14:32 - crash  (01:28)    
minhtuan pts/0        10.196.76.27     Wed Jul  2 14:47 - 17:58  (03:11)    
minhtuan pts/0        10.196.76.27     Wed Jul  2 10:57 - 12:37  (01:39)    
minhtuan pts/0        10.196.76.27     Tue Jul  1 16:18 - 17:08  (00:50)    

\end{verbatim}

Only root previleged programs can gracefully shutdown a system. Thus, it can
either be done by a user with 'root' privileges	or an ACPI script.
An ACPI shutdown can be caused by either power-button press, overheating or low
battery (laptop).

\subsection{memory usage}

\begin{verbatim}
ps aux --sort -rss | less
\end{verbatim}
NOTE: 
\begin{enumerate}
  \item \verb!-rss! sort resident by non-swapped physical memory usage. However,
  RSS can be misleading as it shows a higher value than the real one if the
  pages are shared (e.g. several threads or dynamically linked libraries).
  
  \item \verb!-vsz! (virtual size) the amount of memory reserved for the program
  (which include RSS value).
\end{enumerate} 

Then we can kill the program using its PID
\begin{verbatim}
kill -9 <PID>
\end{verbatim}


\subsection{running processes}

We can use \verb!top! or \verb!htop! (can scroll the list of processes
vertically or hotizontally).




\section{Alien (software)}
\label{sec:alien}

\verb!alien! is a Perl script written by Joey Hess
\footnote{\url{https://joeyh.name/code/alien/}} to convert between package
formats (e.g. *.rpm, *.deb, *.slp, *.pkg).
\begin{verbatim}
  apt-get install alien alien-extra
  
  alien --veryverbose -c -d -g -k pbeagent-9.0.0-613.i386.rpm
\end{verbatim}

Example: convert from .deb to .rpm
\begin{verbatim}
# alien --to-rpm --scripts ./mypkg.deb
\end{verbatim}

Example: convert from .rpm to .deb (the system needs to have dpkg, dpkg-dev,
debhelper, gcc, make packages), and then install the .deb file
\begin{verbatim}
alien -k pbeagent-9.0.0-613.i386.rpm

dpkg -i <name.deb> 
\end{verbatim}

We can also use \verb!alien!

\subsection{Coding Alien}

\verb!alien! needs Perl 5.0004 or greater. 

\section{Add GUI to Ubuntu server}


The Ubuntu server is just the Ubuntu distro without the default Desktop
environment. To install the default Desktop environment, we do
\begin{verbatim}
sudo apt-get install ubuntu-desktop
\end{verbatim}
However, GUI will leads to utilize high resource and still you want GUI , you
can choose lightweight packages. Install ubuntu-desktop with
--no-install-recommends.
\begin{verbatim}
 // the below command adds about 800MB to Ubuntu Server 14.04
sudo apt-get install --no-install-recommends ubuntu-desktop

// Old Gnome only
sudo apt-get install gnome-core
\end{verbatim}
This will exclude major softwares and tools like libreoffice, firefox ,
thunderbird etc.

Other options:
\begin{enumerate}
  \item Gnome 3: 
  \begin{verbatim}
  sudo apt-get install gnome-shell
  \end{verbatim}
  
  \item KDE (Kubuntu) kubuntu-desktop
  
  \item XFCE: \verb!xubuntu-desktop!, or less resource \verb!xfce4!
  
  LXDE is lighter and thus faster than XFCE. Xfce will be ported to Gtk3, which
  is not exactly lightweight, so the difference will be much larger between
  Lubuntu and Xubuntu. 
  
  \item LXDE: \verb!lxde!, or less resource \verb!lxde-core!
  
  LXDE using GTK-3, and a newer replacement is LXQt (LXDE using Qt).
   
  
  \item LXQt: formed by (LXDE and RazorQt teams).
   Openbox is the default window manager for LXQt, but can also be replaced by a
   different one.
   
  \begin{verbatim}
sudo apt-get install   
sudo add-apt-repository ppa:lubuntu-dev/lubuntu-daily 
sudo add-apt-repository ppa:gilir/q-project
sudo apt-get update
sudo apt-get install lxqt-metapackage lxqt-panel openbox
  \end{verbatim}
  This takes about 800MB. 
  \url{http://askubuntu.com/questions/464741/how-to-install-lxqt-de-in-ubuntu-14-04-lts}
  
  MoonLightDE is a fork of LXQt.
  
  \item OpenBox: openbox
  
  \item Gnome classic (a Gnome-3 but looks like Gnome-2)
  
\begin{verbatim}
sudo apt-get install gnome-session-fallback
\end{verbatim}
\end{enumerate}
\url{http://askubuntu.com/questions/53822/how-do-you-run-ubuntu-server-with-a-gui}

\section{Remote control your server via Web}

We can use web-based solution
\begin{enumerate}
  \item Webmin
  \item Zentyal: 
\end{enumerate}
\url{http://askubuntu.com/questions/53822/how-do-you-run-ubuntu-server-with-a-gui}


\section{Web Browsers}

\subsection{Qupzilla}


Qupzilla use WebKit
\begin{verbatim}
sudo add-apt-repository ppa:nowrep/qupzilla

sudo apt-get update

sudo apt-get install qupzilla
\end{verbatim}


\section{Undelete file (recover file)}
\label{sec:debugfs_utility}

Deleted files can be recovered on ext3 file systems using the debugfs program.
\url{http://www.cyberciti.biz/tips/linux-ext3-ext4-deleted-files-recovery-howto.html}




You can use {\bf PhotoRec} which is a companion program to TestDisk, an app for
recovering lost partitions on a wide variety of file systems and making
non-bootable disks bootable again.

\url{http://www.cs.montana.edu/courses/309/topics/4-disks/debugfs_example.html}

\section{Bring an orphan process (background process) back to life}
\label{sec:reptyr}

Once a terminal from that a process was launched, if the terminal die, the only
way to bring the process back to life is to attach it to a new terminal

\verb!reptyr! is a utility for taking an existing running program and attaching
it to a new terminal, i.e. making it eligble for running \verb!fg! command on
that process. reptyr depends on the ptrace(2) system call to attach to the
remote program. \url{https://wiki.ubuntu.com/Security/Features}

NOTE: reptyr only attaches a process to another terminal. This does not mean the
process becomes a child of your current shell.
\url{https://github.com/nelhage/reptyr}

\begin{verbatim}
sudo apt-get install reptyr

reptyr PID
\end{verbatim}
with PID is the process ID of the process you want to attach. Then, just run
\verb!fg! to reopen that process.
\url{http://serverfault.com/questions/55880/moving-an-already-running-process-to-screen}

\textcolor{red}{IMPORTANT}: To know the location of the folder from which the process runs
\begin{verbatim}
readlink /proc/<PID>/cwd

pwdx <PID>

lsof -p PID | grep cwd
\end{verbatim}
this can help to tell which is the right one you want to bring back. 

\textcolor{red}{On UBUNTU}: A troubling weakness of the Linux process interfaces
is that a single user is able to examine the memory and running state of any of
their processes.
Example: if one application was compromised, it would be possible for an
attacker to attach to other running processes using \verb!ptrace! to extract
additional credentials and continue to immediately expand the scope of their
attack without resorting to user-assisted phishing or trojans.

On Ubuntu Maverick (11.10) and higher, this ability is disabled by default for
security reasons, i.e. users cannot ptrace processes that are not a descendant
of the debugger. You can enable it temporarily by doing
\begin{verbatim}
sudo echo 0 > /proc/sys/kernel/yama/ptrace_scope
\end{verbatim}
or permanently by editing the file
\begin{verbatim}
sudo vi  /etc/sysctl.d/10-ptrace.conf
\end{verbatim}
, which also contains more information about this setting.
\url{https://superuser.com/questions/1231023/how-to-foreground-a-process-started-by-upstart}

\subsection{ptrace}
\label{sec:ptrace}

ptrace() system call provides a means by which one process (the "tracer") may
observe and control the execution of another process (the "tracee"), and examine
and change the tracee's memory and registers.
It can single-step through the target's code, can observe and intercept system
calls and their results, and can manipulate the target's signal handlers and
both receive and send signals on its behalf.

It is primarily used to implement breakpoint debugging and system call tracing.

ptrace is used 
\begin{enumerate}
  \item by debuggers (such as gdb and dbx), 
  
  \item by tracing tools like strace and ltrace, and 
  
  \item by code coverage tools.
  
  \item by specialized programs to patch running programs, to avoid unfixed bugs or to overcome security features
  
  \item as a sandbox[1][2] and as a run-time environment simulator (like emulating root access for non-root software[2][3]).
  
\end{enumerate}


For security reason, the O/S typically is configured so that ptrace, run by a
specific user, can attach only to processes that the owner can send signals to
(typically only their own processes).

the superuser account can ptrace almost any process (except init on kernels
before 2.6.26).

In Linux, the ability to ptrace is further limited by the \verb!CAP_SYS_PTRACE! capability,
or by the YAMA Linux Security Module (LSM - Sect.\ref{sec:LSM}).

In FreeBSD, it's limited by FreeBSD jails and Mandatory Access Control policies.

\subsection{Linux Security Module (LSM)}
\label{sec:LSM}

Linux Security Modules (LSM) is a framework that allows the Linux kernel to
support a variety of computer security models while avoiding favoritism toward
any single security implementation.

LSM is a standard part of the Linux kernel since Linux 2.6. 

LSM is an interface for the Linux kernel that provides sufficient "hooks"
(upcalls) from within the Linux kernel to a loadable module


LSM framework supports the following loadable modules of security
implementation.
IMPORTANT: One can have SELinux, Smack, and TOMOYO enabled in a kernel, but only
one—chosen at boot time—can be active. The reason is that LSM does not support
'stack' or 'chaining' of multiple security modules.
There have been discussions and proposals for LSM stacking (or chaining) along
the way, but nothing has ever been merged \url{https://lwn.net/Articles/114588/}


\begin{enumerate}
  \item  AppArmor  (version 2.6.36) 
  
  \item SELinux, 
  
  \item  Smack (added into Linux kernel version 2.6.25),
  
  Simplified Mandatory Access Control Kernel) is a Linux kernel security module
  that protects data and process interaction from malicious manipulation using a
  set of custom mandatory access control (MAC) rules, with simplicity as its
  main design goal.
  
  It is the main access control mechanism used by  MeeGo mobile Operating
  System, by Intel's Ostro OS for IoT devices; as  sandbox HTML5 web
  applications in the Tizen architecture.
  \url{https://en.wikipedia.org/wiki/Smack_(software)}
  
  \item TOMOYO Linux (first added into Linux kernel version 2.6.30, June 2009):
  
  Tomoyo Linux allows each process to declare behaviours and resources needed to
  achieve their purpose.
  When protection is enabled, Tomoyo Linux restricts each process to the
  behaviors and resources allowed by the administrator.
  \url{https://en.wikipedia.org/wiki/Tomoyo_Linux}
  
  NOTE: 2.5.0 included in Linux kernel 3.2
  
  
  \item Yama:
  
  Yama provides security by restricting the ptrace() system call.
  It it creates a sysctl called 
  \verb!ptrace_scope! that defaults to zero, which
  chooses the existing behavior. If it is set to one, though, it only allows
  ptrace() to be called on descendants of the tracing process.
  
  
  \url{https://lwn.net/Articles/398607/}
  
\end{enumerate}
 

\subsection{process tree}
\label{sec:process-tree}

When you launch a process, it belongs to a particular shell terminal process.

A process tree shows the tree of hierarchical structure of all process, with
root is \verb!init! process (Sect.\ref{sec:init}). The \verb!pstree! command
shows that process tree of running processes, better than \verb!ps! command
\begin{verbatim}
pstree
\end{verbatim}

fg and bg only work with processes on the same process tree. 


When SIGSTOP or SIGTSTP is sent to a process, the usual behaviour is to pause
that process in its current state. The process will only resume execution if it
is sent the SIGCONT signal.
\begin{verbatim}
   //When SIGSTOP is sent to a process, the usual behaviour is to pause that process in its current state. 
   //The process will only resume execution if it is sent the SIGCONT signal.
kill -SIGSTOP [pid]

kill -SIGCONT [pid]
\end{verbatim}
\url{https://unix.stackexchange.com/questions/317492/list-of-kill-signals}

\url{https://www.techonthenet.com/linux/commands/kill.php}

The following command show you a complete table with a brief summary of the
meaning of each signal

\begin{verbatim}
man 7 signal
\end{verbatim}


\section{Troubleshoot}

\begin{verbatim}
Some packages could not be installed. This may mean that you have
requested an impossible situation or if you are using the unstable
distribution that some required packages have not yet been created
or been moved out of Incoming.
The following information may help to resolve the situation:

The following packages have unmet dependencies:
 python-dev : Depends: libpython-dev (= 2.7.5-5ubuntu3) but it is not going to be installed
              Depends: python2.7-dev (>= 2.7.5-1~) but it is not going to be installed
E: Unable to correct problems, you have held broken packages.
\end{verbatim}

The problem is that python-dev depends on 2.7.5, but your python is 2.7.6
\begin{verbatim}
apt-cache policy python-dev
apt-cache policy python2.7-dev
\end{verbatim}
You can see that they are not the same
\begin{verbatim}
thoangtr@thoangtr:~$ apt-cache policy python2.7-dev
python2.7-dev:
  Installed: (none)
  Candidate: 2.7.6-8
  Version table:
     2.7.6-8 0
        500 http://us.archive.ubuntu.com/ubuntu/ trusty/main amd64 Packages
thoangtr@thoangtr:~$ apt-cache policy python-dev
python-dev:
  Installed: (none)
  Candidate: 2.7.5-5ubuntu3
  Version table:
     2.7.5-5ubuntu3 0
        500 http://us.archive.ubuntu.com/ubuntu/ trusty/main amd64 Packages
\end{verbatim}
It suggests that you are using some mismatched repositories, or have some
apt-pins in place keeping the version dependencies from lining up.

To resolve the problem, add
\begin{itemize}
  \item Ubuntu 12.04 (precise)
\begin{verbatim}
echo "deb http://us.archive.ubuntu.com/ubuntu/ precise-updates main restricted"
| sudo tee -a /etc/apt/sources.list.d/precise-updates.list
\end{verbatim}

  \item Ubuntu 14.04 (trusty)
\begin{verbatim}
echo "deb http://us.archive.ubuntu.com/ubuntu/ trusty-updates main restricted"
| sudo tee -a /etc/apt/sources.list.d/trusty-updates.list
\end{verbatim}
\end{itemize}
and then run
\begin{verbatim}
sudo apt-get update
sudo apt-get install python2.7-dev
\end{verbatim}
\url{http://askubuntu.com/questions/275861/problem-installing-python-dev}

\url{http://manpages.ubuntu.com/manpages/utopic/man5/sources.list.5.html}

\section{sendmail vs. mail vs. qmail vs. mailx vs. mutt}
\label{sec:sendmail}
\label{sec:mail}
\label{sec:mailx}

\verb!sendmail! is a professional mail server service allowing your system to
handle incoming and outgoing mail requests. However, Sendmail is complicated to
configure and overused for most users needs. If you are not looking to receive
mail and are instead looking to simply send mail (i.e. from a script or
whatever), you don't need sendmail.

Using \verb!mailx! (nail) is much simpler by not setting up a mail server at
all.
\begin{itemize}
  \item On openSUSE, /bin/mail is a symlink that points to /usr/bin/mailx
  
  \item Debian systems do things the other way around: /usr/bin/mailx points to
  /usr/bin/mail.
\end{itemize}
For attachment, we use \verb!-A! option.
\begin{verbatim}
//multiple files
$ uuencode hej.jpg hej.jpg > attach_image.txt
$ mailx -s 'photo' petronella@huhnix.org < attach_image.txt

  // or
$ mailx -s 'photo' petronella@huhnix.org -A "file1" -A "file2"
\end{verbatim}

The \verb!mutt! mail client not only offers -s for subject, -c for copy, and -b
for blind copy options referred to in this article, but also the -a parameter
for simple attachment sending. 
\begin{verbatim}
mutt -s 'Photo' -a hej.jpg petronella@huhnix.org < mail.txt
\end{verbatim}

\subsection{sendmail} 

\begin{verbatim}
 // Ubuntu
sudo apt-get install sendmail
\end{verbatim}

In short, 'mail' is a client program wheres 'sendmail' is a server program. They
are also called MUA and MTA (Mail User Agent vs. Mail Transfer Agent). 
\verb!sendmail! implements the SMTP listening on port 25, and it is pretty much
the same program in all Unixes. \verb!mail! may be rewritten by the vendor
(Solaris, AIX, etc.).

NOTE: The difference
\begin{verbatim}
mail [emailid] <filename1 

sendmail -t -i <filename2 
\end{verbatim}
\begin{enumerate}
  \item \verb!mail! accepts a very basic input, i.e. filename1 whose content is
  treated as the body of the email
  
  Other part of the email can be passed using options, e.g. \verb!-s! (subject), 
  \verb!-f! (email-from)
  
  \item \verb!sendmail! accepts filename2 whose content all the headers
  including to,from,subject,contenttype etc.
\end{enumerate}

Example:  
\begin{verbatim}
echo "testing" | /usr/sbin/sendmail -f fromuser@coldmail.com touser@somemail.com

echo "testing" | /bin/mail -s "haha" touser@somemail.com -- -f fromuser@coldmail.com
\end{verbatim}

However, sendmail is not without competitors, and perhaps the most prominent is
\verb!qmail!, developed by Dan Bernstein. 
\url{http://www.linuxplanet.com/linuxplanet/reviews/1640/1}

In general, sendmail has some significant advantages. It is very mature, being
one the of oldest MTAs around. Its wide use and availability has given rise to a
vast pool of experienced sendmail administrators and consultants. sendmail is
well documented in O'Reilly's sendmail book written by Bryan Costales
with Eric Allman.

\section{pkg-config, PKG\_CONFIG\_PATH}
\label{sec:pkg-config}
\label{sec:PKG_CONFIG_PATH-env}


A script that helps to install a software typically requires checking the
presence of or installed packages. It oftens use \verb!pkg-config!.

pkg-config is a computer program that provides a unified interface for querying
installed libraries. pkg-config was originally designed for Linux, but it is now
also available for the various BSDs, Microsoft Windows, Mac OS X, and Solaris.


IDEA: When a package (e.g. library) is installed, via a package manager (e.g.
RPM, .deb - Sect.\ref{sec:package-manager}), a  .pc file should be included and
placed into a directory with other .pc files. 
These files are named after the package, with the extension .pc. To locate these
files, pkg-config will look for certain folders (as being described later in
this section).

\begin{verbatim}
/packages/pcre2/lib/pkgconfig/libpcre2-posix.pc
/packages/pcre2/lib/pkgconfig/libpcre2-8.pc
\end{verbatim}
This .pc file has several entries

Example of the content of a .pc file
\begin{verbatim}
# Package Information for pkg-config

prefix=/packages/pcre2
exec_prefix=${prefix}
libdir=${exec_prefix}/lib
includedir=${prefix}/include

Name: libpcre2-8
Description: PCRE2 - Perl compatible regular expressions C library (2nd API) with 8 bit character support
Version: 10.21
Libs: -L${libdir} -lpcre2-8
Libs.private:  
Cflags: -I${includedir} 
\end{verbatim}

\textcolor{red}{List all installed libraries}
\begin{verbatim}
pkg-config --list-all
\end{verbatim}

NOTE: The location of pkg-config can be specified using \verb!PKG_CONFIG!
environment variable.

NOTE: On a typical Unix system, it will search in the directories
/usr/lib/pkgconfig and /usr/share/pkgconfig. This built-in search pathes can be
overriden by modifying \verb!PKG_CONFIG_LIBDIR! environment variable.

NOTE: \verb!PKG_CONFIG_PATH! is a environment variable that specifies additional
paths in which pkg-config will search for its .pc files. So, when using
\verb!module! for package versin management (Sect.\ref{sec:modules}), remember
to update this environment variable as well, e.g.
\begin{verbatim}
prepend-path PKG_CONFIG_PATH /packages/pcre2/lib/pkgconfig
\end{verbatim}

Example: to get compilation flags to libpng
\begin{verbatim}
pkg-config --libs --cflags libpng

// so we can automate
gcc -o test test.c $(pkg-config --libs --cflags libpng)
\end{verbatim}


\section{ldconfig, LD\_LIBRARY\_PATH}
\label{sec:ldconfig}
\label{sec:LD_LIBRARY_PATH-env}

When you run a program, or you want to install a package, the program or the
build script needs to find locations of shared libraries that need to be
dynamically loaded and linked to the program. By default, the shared library
files in /lib, /usr/lib and a few other standard directories are maintained in
the cache by ldconfig.

ldconfig is a program that is used to maintain the shared library cache. This
cache is typically stored in the file /etc/ld.so.cache and is used by the system
to map a shared library name to the location of the corresponding shared library
file.  

If a new program or library might be installed in a non-standard
directory, you need to tell the system where to find it. Choose one of the two
options
\begin{enumerate}
  \item (without root privilege): update \verb!LD_LIBRARY_PATH! environment
  variable
  
  \item (with root privilege): rebuild the cache
  
\begin{verbatim}
sudo /etc/ld.so.conf
\end{verbatim}
add a new line with the new library directory.

Finally, run
\begin{verbatim}
sudo ldconfig
\end{verbatim} 

\end{enumerate}
Check if the shared library cache now includes the new library location
\begin{verbatim}
ldconfig -p | grep <package-name>
\end{verbatim}



\section{m4 macros}
\label{sec:m4}

Macros are either builtin or user-defined, and can take any number of arguments.
GNU m4 is a general purpose macro processor. Besides just doing macro expansion,
m4 has builtin functions for including named files, running shell commands,
doing integer arithmetic, manipulating text in various ways, performing
recursion, etc.... m4 can be used either as a front-end to a compiler, or as a
macro processor in its own right.

{\it The m4 macro processor is widely available on all UNIXes, and has been
standardized by POSIX.}
\begin{itemize}
  \item  Autoconf (Sect.\ref{sec:autoconf}), which requires GNU m4 for
  generating configure scripts (Sect.\ref{sec:configure})
  
  \item 
\end{itemize}

\url{https://www.gnu.org/software/m4/manual/m4.html#Preliminaries}



\url{https://geeksww.com/tutorials/libraries/m4/installation/installing_m4_macro_processor_ubuntu_linux.php}


\section{gperf utility}
\label{sec:gperf}

gperf helps to parse command-line input easily.  The input has to be passed in
via a file which contains a set of keywords; and then
it generates C/C++ sources for the hash table, hashing, and lookup methods.
\begin{verbatim}
gperf  -L C++ command_line_options.gperf > perfecthash.hpp
\end{verbatim}

The \verb!*.gperf! file has the format
\begin{verbatim}
%{
/* C code that goes verbatim in output */
%}
declarations
%%
keywords
%%
functions
\end{verbatim}
Similar to the syntax of the grammar file used by Flex and Bison.

\url{https://www.ibm.com/developerworks/library/l-gperf/index.html}


\section{wireshack}
\label{sec:wireshack}

\url{https://www.wireshark.org/}

{\bf Wireshack} is  widely-used network protocol analyzer, started in 1998. It
lets you see what's happening on your network at a microscopic level and is the
de facto (and often de jure) standard across many commercial and non-profit
enterprises, government agencies, and educational institution.


\section{libjpeg-turbo}
\label{sec:libjpeg-turbo}

We may need to install nasm (Sect.\ref{sec:NASM}).

\begin{verbatim}
cd libjpeg-turbo
autoreconf -fiv

mkdir build
cd build
sh ../configure --prefix=/packages/libjpeg-turbo/9
\end{verbatim}

\section{gnulib}
\label{sec:gnulib}

\verb!pty.h! can be missing in some platforms. To fix, install \verb!gnulib! package.

\url{https://www.gnu.org/software/gnulib/manual/html_node/pty_002eh.html}

\url{http://manpages.ubuntu.com/manpages/bionic/man1/gnulib-tool.1.html}

\section{NASM, YASM}


\subsection{NASM}
\label{sec:NASM}

The Netwide Assembler, NASM, is an 80x86 and x86-64 assembler designed for
portability and modularity. It supports a range of object file formats,
including Linux and *BSD a.out, ELF, COFF, Mach-O, 16-bit and 32-bit OBJ (OMF)
format, Win32 and Win64.  

\url{http://www.nasm.us/xdoc/2.13.01/html/nasmdoc1.html#section-1.1}

\url{http://www.nasm.us/pub/nasm/releasebuilds/2.13.01/linux/}

\subsection{YASM}
\label{sec:YASM}

Yasm is a complete rewrite of the NASM assembler under the "new" BSD License.

Yasm currently supports the x86 and AMD64 instruction sets, accepts NASM and GAS
assembler syntaxes, outputs binary, ELF32, ELF64, 32 and 64-bit Mach-O, RDOFF2,
COFF, Win32, and Win64 object formats, and generates source debugging
information in STABS, DWARF 2, and CodeView 8 formats.

\section{gnulib}
\label{sec:gnulib}

Gnulib is a central location for common GNU code, intended to be shared among
GNU packages. It is a {\bf source code library} that provides basic
functionality to programs and libraries. As a source code library, its
components are intended to be shared at the source level, rather than being a
library that gets built, installed, and linked against. Thus, there is no
distribution tarball; the idea is to copy files from Gnulib into your own source
tree.

To use GNUlib, just wrap the code
\begin{verbatim}
git clone git://git.savannah.gnu.org/gnulib.git

cd gnulib
 ./gnulib-tool --help
\end{verbatim}

The easiest, and recommended, way to do this is to use the gnulib-tool script
\begin{verbatim}
ln -s $HOME/gnu/src/gnulib.git/gnulib-tool $HOME/bin/gnulib-tool
\end{verbatim}

\section{libiberty}
\label{sec:libiberty}


The libiberty library is a collection of subroutines used by various GNU
programs.

IMPORTANT: To date, libiberty is generally not installed on its own. It has
evolved over years but does not have its own version number nor release schedule.

The easiest way to use libiberty in your projects is to drop the libiberty code
into your project's sources, and to build the library along with your own
sources; the library would then be linked in at the end. This prevents any
possible version mismatches with other copies of libiberty elsewhere on the
system.

You can also choose libiberty that installed as part of GCC, and use that for
all software built with that compiler. To link, use \verb!-liberty! option.

\subsection{install}

Pass \verb!--enable-install-libiberty! to the configure script when building
libiberty causes the header files and archive library to be installed when make install is run.
Also, consider using \verb!--prefix! option.



\section{Apport (crash report utility)}
\label{sec:apport}

Ubuntu has a program called Apport that is responsible for detecting such
crashes and upon user consent, report these crashes to developers. This process
intends to get the problem fixed by the developers.

However it can be very annoying to common users, and there is no point in
showing errors to users when they cannot do anything about it themselves. So you
might want to disable them.


\begin{verbatim}
gksudo gedit /etc/default/apport

# set 
enabled=0

# then (Ubuntu 14.04)
sudo service apport restart


# or (Ubuntu 16.04+) which use systemd to manage services
sudo systemctl restart apport
\end{verbatim}
\url{https://askubuntu.com/questions/615478/constant-startup-error-issue-what-is-com-ubuntu-apport-support-gtk-root}