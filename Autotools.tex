\chapter{Autotools}

\url{https://www.lrde.epita.fr/~adl/dl/autotools.pdf}


Autotools comprises:
\begin{enumerate}
  \item GNU Autoconf
  
  \item GNU Automake

automatic build both shared and static libraries. 
  
  \item GNU Libtool

uninstall targets, e.g. uninstall the program you install
  
  \item GNU Gettext: ideal to work with tasks that require lots of translation
  and translated source code.  
\end{enumerate}
The GNU Autotools framework is comprised of three main packages, each of which
provides and relies on several smaller components (Autoconf, Automake and
Libtool). Additionally, the tools in the Autotools packages can depend on or use
utilities and functionality from the gettext, m4, sed, make and perl packages,
as well as others. 

Nowadays, we have other options, e.g. CMake, SCons, bjam (Boost build, C++)


The Autotools are NOT supposed to make projects simpler for the maintainer.
Instead, it aims to
\begin{enumerate}
  \item  make life easer for your users, and second, 
  
  \item to make your project more  portable
\end{enumerate}

Autotools were designed by GNU people to manage GNU projects.
So, it is ideal to be used in projects written in C, C++, Objective C, Fortran,
Fortran 77 and Erlang.
By "native support", I mean that Autoconf will compile, link and run
source-level feature checks in these languages.
Efforts to support Java using  both GNU (gcj) and non-GNU
Java compilers and VM's.


\section{History}

 The original author was David MacKenzie, who started the Autoconf project in
1991. While configuration scripts were becoming longer and more complex, there
were really only a few variables that needed to be specified by the user.  

Most of these were simply choices to be made regarding components, features and
options: Where do I find libraries and header files? Where do I want to install
my finished product? Which optional components do I want to build into my products? 

With Autoconf, instead of modifying, debugging and losing sleep over literally
thousands of lines of supposedly portable shell script, developers can write a
short meta-script file, using a concise macro-based language, and let Autoconf
generate a perfect configuration script.  



\section{configure.ac file (./configure): autoreconfig}
\label{sec:configure.ac}
\label{sec:autoreconf-tool}

If you have issue, call (to create m4/pkg.m4 from /usr/share/local/pkg.m4 file)
\begin{verbatim}
aclocal --install
\end{verbatim}
before calling \verb!autoreconf!.

To generate configure script:
\begin{verbatim}
autoreconfig -i
\end{verbatim}

\verb!autoreconf -ivf! calls \verb!aclocal --force!, which checks the
\verb!AC_CONFIG_MACRO_DIR! (which is obsoleted -
Sect.\ref{sec:AC_CONFIG_MACRO_DIR}).
\begin{verbatim}
// the -v option shows the calling sequence/order

autoreconf-2.69  : 
    enter directory '.'

aclocal-1.14
    run:   aclocal --force '-I m4'

autoreconf-2.69: 
   configure.ac 
       
libtoolize:
    run: libtoolize --copy --force
    (to put *.m4 files into ./m4/ folder)
    
    run: /usr/bin/autoconf-2.69 --force
    run: /usr/bin/autoheader-2.69 --force
    run: automake --add-missing --copy --force-missing

add    
    ./compile
    ./missing
    ./decomp
\end{verbatim}



\section{aclocal.m4 file (aclocal utility)}
\label{sec:aclocal.m4}

Autoconf was designed and written first, and then a few years later, the idea
for Automake was conceived as an add-on for Autoconf. But Autoconf was really
not designed to be extensible on the scale required by Automake.

Automake adds an extensive set of macros to those provided by Autoconf. The
originally documented method for adding user-defined macros to an Autoconf
project was to create a file called \verb!aclocal.m4! in the same directory as
\verb!configure.ac! (Sect.\ref{sec:configure.ac}).

The aclocal utility is actually documented by the GNU manuals as a temporary
work-around for a certain lack of flexibility in Autoconf. 
\verb!aclocal! utility was designed to create a project's aclocal.m4 file,
containing all the required Automake macros. 

Any user-provided extension macros were to be placed in \verb!aclocal.m4! file,
and Autoconf would automatically read it while processing configure.ac

\textcolor{red}{OLD}: Automake was also designed to read a new user-provided
macro file called acinclude.m4. Automake's manual originally suggested that you should rename
aclocal.m4 to acinclude.m4 when adding Automake to an existing Autoconf project.
This method is still followed rigorously in new projects. 

\textcolor{red}{CURRENT}: However, the latest documentation from both sets of
tools suggests that the entire aclocal/acinclude paradigm is now obsolete, in
favor of a newer method of specifying a directory containing m4 macro files.
The current recommendation is that you create a directory in your project
directory called simply  \verb!m4! (though \verb!acinlucde! has a better name
to reflect the purpose being used by aclocal utility). 
All files in this directory will be gathered into aclocal.m4 before Autoconf
processes your configure.ac file.


\begin{verbatim}
4 inputs (files/folders):
     configure.ac   (ac-m4/shell)
     acinclude.m4   (ac-m4)
     m4/*.m4        (ac-m4)
     acsite.m4      (ac-m4)
\end{verbatim}
will be read-in by \verb!aclocal! utility (perl script), which then use those
information to generate \verb!aclocal.m4! file.
All files in \verb!./m4/! directory will be gathered into aclocal.m4 before
Autoconf processes your configure.ac file




\section{m4 folder (*.m4 file): libtoolize}
\label{sec:libtoolize}
\label{sec:AC_CONFIG_MACRO_DIR}

The libtoolize program provides a standard way to add libtool support to your
package.
\url{https://www.gnu.org/software/libtool/manual/html_node/Invoking-libtoolize.html}

In \verb!configure.ac! file, we can specify where the \verb!*m4! files should
reside using \verb!AC_CONFIG_MACRO_DIRS!
\begin{verbatim}
  # it's a list, so we can specify more than one location
AC_CONFIG_MACRO_DIRS([m4])

AC_CONFIG_HEADERS([src/include/config.h])
\end{verbatim}
The purpose of \verb!AC_CONFIG_MACRO_DIR! (which is obsoleted by
\verb!AC_CONFIG_MACRO_DIRS! in new autotools) has purpose of telling to
'aclocal' where to look for user defined  *.m4 files.
\verb!AC_CONFIG_MACRO_DIRS! won't work until autoconf 2.70 is released upstream
(I'm still working on that part).  But \verb!AC_CONFIG_MACRO_DIR! will continue
to work, and is NOT obsolete.
\url{https://bugzilla.redhat.com/show_bug.cgi?id=901333}

So that when you run
\begin{verbatim}
libtoolize
\end{verbatim}
it will put macro files
\begin{verbatim}
m4/libtool.m4
m4/ltoptions.m4
m4/ltsugar.m4
m4/ltversion.m4
m4/lt~obsolete.m4
\end{verbatim}

If libtoolize doesn't see \verb!AC_CONFIG_MACRO_DIRS!, it too will honour the
first '-I' argument in \verb!ACLOCAL_AMFLAGS! (defined in
Sect.\ref{sec:Makefile.am}) when choosing a directory to store libtool
configuration macros in. It is perfectly sensible to use both
\verb!AC_CONFIG_MACRO_DIRS! and \verb!ACLOCAL_AMFLAGS!, as long as they are kept
in synchronisation. These libtool's suggestions are bad as both macros should be
legacy now.  For now, I'd prefer to ignore this.
Reading \url{http://www.flameeyes.eu/autotools-mythbuster/autoconf/macros.html}
it suggests that \verb!ACLOCAL_AMFLAGS! will be ignored in the future too, so
libtool's suggestion is even more confusing.


\section{Makefile.am}
\label{sec:Makefile.am}

\begin{verbatim}
Consider adding '-I m4' to ACLOCAL_AMFLAGS in Makefile.am
\end{verbatim}

In the future other Autotools will automatically check the contents of
\verb!AC_CONFIG_MACRO_DIRS! (Sect.\ref{sec:libtoolize}), but at the moment it is
more portable to add the macro directory to \verb!ACLOCAL_AMFLAGS! in
Makefile.am, which is where the tools currently look. 


