\chapter{Troubleshoot: Network}
\label{chap:troubleshoot_network}

\section{Switch vs. Router vs. Hub}

A switch connects computers, printers, and servers within a building. A switch
can be unmanaged or managed. 
\begin{enumerate}
  \item An unmanaged switch works right out-of-the-box, i.e. without
  configuration. Thus, it has less network capability.
  \item A managed switch is configurable. It can be controlled locally or
  remotely. 
\end{enumerate}
A switch is designed to form a network by connecting different machines. A
switch can send and receive at the same time. A swtich also analyze the packet
and decide which machine it should send to.

A hub is simpler than a switch. It can NOT identify the source or intended
destination of the packets, so it sends to all the computers connecting to it,
including the one that send it. A hub can send/receive, but not both at the same
time. So a hub is slower than a switch. 

A router is similar to a switch, yet it's more complicated. It's designed to
link different network. A router can be wired or wireless. A router will allows
your machines to connect to the Internet. A router also have a built-in firewall
for
security\footnote{\url{http://helpdeskgeek.com/networking/router-vs-switch-vs-hub/}}.
As a result, each router is assigned an IP address. However, they all have the
same IP address: 192.168.0.1 This IP address is a non-routable IP address, i.e.
a private IP address and is not assigned to any particular organization. 

\section{Private vs. Public IP address}

Typically, the number of IP address are limited. So, each organization is
assigned a certain number of public IP. These are IP addresses that can be seen
from outside, i.e. the Internet. However, that organization may have many PC. A
solution is to assigned each machine a private IP address. These address cannot
be seen from outside. In order for these machine to connect to the Internet,
they are routed to NAT gateway (network address translation) or proxy server. A
router can also be a NAT device. 

There are three ranges that can be used to assign private IP addresses, you can
choose one class depending on the size of your LAN, Table.\ref{tab:private_IP}.
Most routers come with a setup of class C network, and so the first usuable IP
address is 192.168.0.1. NOTE: If you want, you can still change your router's IP
address to class B or class A and it still work
fine\footnote{\url{http://helpdeskgeek.com/networking/what-is-192-168-0-1/}}.

\begin{table}[hbt]
\begin{center}
\caption{Private IP address range}
\begin{tabular}{ccc} 
  \hline
IP address range & Number of addresses & Class \\ 
  \hline\hline
10.0.0.0 - 10.255.255.255	& 16,777,216 & Class A \\
172.16.0.0 - 172.31.255.25 &  1,048,576 & Class B \\
192.168.0.0 - 192.168.255.255 & 65,536 & Class C \\
\end{tabular}
\end{center}
\label{tab:private_IP}
\end{table}

\begin{framed}
There are other special private IP address ranges, e.g. 
1.0.0.0/8 and 2.0.0.0/8, but they are not being used. The other private IP
address 169.254/16 are called {\bf link-local addresses}
and are only used when there is no DHCP server to assign IP addresses. In such
case, the devices will automatically assign an IP address to themselves in the
range 169.254.0.0 to 169.254.255.255.
\end{framed}



\section{Static IP and dynamic IP (DHCP, BOOTP)}
\label{sec:static-IP}

In Linux, all network configuration is given in \verb!/etc/network/interface!
file (Sect.\ref{sec:network-interface}).

A dynamic IP is requested by \verb!dhclient! which is a Internet Systems
Consortium DHCP Client. \verb!dhclient! provides a means for configuring one or
more network interfaces using the Dynamic Host Configuration Protocol, BOOTP
protocol, or if these protocols fail, by statically assigning an address. 

\textcolor{red}{To request a new DHCP IP} in Linux; there is a number of ways
\begin{enumerate}
  \item restart network service (works on any Linux distro - RHEL, Fedora,
  CentOS, Ubuntu)

\begin{verbatim}
dhclient -r && dhclient
\end{verbatim}
NOTE: \verb!-r! option explicitly releases the current lease, and thus the
dhclient process also exit. 

or pass multiple interface names to dhclient
\begin{verbatim}
dhclient -r eth0 eth1 && dhclient eth0 eth1
\end{verbatim}

  \item restart for the given interface
\begin{verbatim}
# ifdown eth0
# ifup eth0
### RHEL/CentOS/Fedora specific command ###
# /etc/init.d/network restart

### Debian / Ubuntu Linux specific command ###
# no longer works since Ubuntu 14.04 
# due to a change in 'networking' file
#  (need to use ifdown/ifup)
# /etc/init.d/networking restart
\end{verbatim}

  \item \verb!nmcli! = network manager daemon
  
The NetworkManager daemon attempts to make networking configuration and
operation as painless and automatic as possible by managing the primary network
connection and other network interfaces, like Ethernet, WiFi, and Mobile
Broadband devices command-line tool for controlling NetworkManager.

For network connection named \verb!nixcraft_5G!
\begin{verbatim}
nmcli con
nmcli con down id 'nixcraft_5G'
nmcli con up id 'nixcraft_5G'
\end{verbatim}

\end{enumerate}

NOTE:
{\tiny
\begin{verbatim}
dhclient -1 -v -pf /run/dhclient.eth0.pid -lf /var/lib/dhcp/dhclient.eth0.leases eth0
\end{verbatim}
}
EXPLAIN:
\begin{itemize}
  \item -1 = means try once to get a lease; if fails, then exit with code 2
  
  \item -v = verbose message
  
  \item -pf = indicate path to process ID file.
  
NOTE: If unspecified, the default /var/run/dhclient.pid is used.

  \item -lf = pah to the lease database file. 
  
NOTE: If unspecified, the default /var/lib/dhclient/dhclient.leases is used.

\end{itemize}
\url{https://linux.die.net/man/8/dhclient}

\textcolor{red}{To request a new DHCP IP } in Windows
\begin{verbatim}
ipconfig /renew
\end{verbatim}

\textcolor{red}{If you experience static IP switch to dynamic IP}, maybe a
DHClient is automatically requesting IP unexpectedly
\begin{itemize}
  \item check for dhclient 
  
\begin{verbatim}
ps -aux | grep dhclient
\end{verbatim}

  \item then kill the process that keep requesting dynamic IP for the interface
  that you want static IP.

Example: the interface is eth1
\begin{verbatim}
sudo kill <pid> -9
sudo ifdown eth1 ; sudo ifup eth1
\end{verbatim}
\end{itemize}

\subsection{/etc/network/interfaces file}
\label{sec:network-interface}

\begin{verbatim}
auto lo
iface lo inet loopback

# The primary network interface
auto eth0
iface eth0 inet static
netmask 255.255.0.0
address 10.10.130.128
gateway 10.10.1.1
dns-nameserves IP1 IP2
dns-domain domain1.com
dns-search domain1.com sub.domain1.com

auto bond0
iface bond0 inet manual
        down ip link set $IFACE down
        post-down rmmod bonding
        pre-up modprobe bonding mode=4 miimon=200
        up ip link set $IFACE up mtu 9000
        up udevadm trigger

allow-hotplug eth0
iface eth0 inet manual
        up ifenslave bond0 $IFACE
        down ifenslave -d bond0 $IFACE 2> /dev/null

allow-hotplug eth1
iface eth1 inet manual
        up ifenslave bond0 $IFACE
        down ifenslave -d bond0 $IFACE 2> /dev/null
\end{verbatim}

Link-layer commands (the first command to use in each block)
\begin{enumerate}
  \item \verb!auto! <interface-name> = means start the network of given
  interface at boot
  
  \item \verb!allow-auto! <interface-name> = the same as \verb!auto! 
  
  \item \verb!allow-hotplug! <interface-name> = enable the device of given
  interface to be detected not at the begining of the boot; but as 'hotplug'
 
\end{enumerate}  

Layer-3 (routing and addressing) commands
\begin{enumerate}
  \item \verb!iface! <interface-family> : 
  
  iface eth0: creates a stanza called eth0 on an Ethernet device
  
  iface ppp0: create a point-to-point interface
  
  On the same line of \verb!iface!; or after that.
  
  \item \verb!pre-up! : actions before the interface is up
  
  \item \verb!post-down! : actions right after the interface is up
  
  \item \verb!gateway! : default gateway of a server. Be careful to use only one
  of this guy.
  
  \item \verb!address! : put static IP after this
  
  \item \verb!netmask! : provide net mask
  
  \item \verb!inet! or   \verb!inet6! + options
  
  It defines the version of the IP protocol that will be used and the way this
  address will be configured (static, dhcp, scripts...)
\begin{verbatim}
inet static

inet manual      # do not define address on that interface

inet dhcp

inet6 static    # static IP using IPv6
\end{verbatim}  

  \item \verb!hwaddress ether 00:00:00:00:00:00! : Change the mac address of the
  interface instead of using the one that is hardcoded into rom, or generated by algorithms
  
  \item \verb!dns-nameservers! = IP addresses of nameservers.
  
  Requires the \verb!resolvconf! package. This utility 
  aggregates all the information in /etc/network/interfaces
  and auto-generate the content of /etc/resolv.conf for
  DNS-related configurations.
  
  \item \verb!dns-search! : 
  append the name given as domain to queries of host, creating the FQDN. 
  
  It becomes \verb!domain! option in \verb!/etc/resolv.conf!
  
  \item \verb!wpa-ssid! = Wireless: Set a wireless WPA SSID.
  
  \item \verb!mtu! MTU size
  
If the workload is only sending small messages, then the larger MTU size will
not help.

The default MTU size of 9180 is much more efficient than using a MTU size of
1500 bytes (normally used by LAN Emulation).

 With Gigabit and 10 Gigabit Ethernet, if all of the machines on the network
have Gigabit Ethernet adapters and no 10/100 adapters on the network, then it
would be best to use jumbo frame mode (i.e. greater than 1500).
\begin{verbatim}
mtu 9000 # Jumbo Frame.
\end{verbatim}

The use of large MTU sizes allows the operating system to send fewer packets of
a larger size to reach the same network throughput. The larger packets greatly
reduce the processing required in the operating system, assuming the workload
allows large messages to be sent. 



   \item \verb!wpa-psk! = Wireless: Set a hexadecimal encoded PSK for your SSID.
   
   
\end{enumerate}

{\bf Interface-names}
\begin{enumerate}
  \item \verb!lo! = an interface name representing a loopback device
  
  NOTE:  imagine it as a virtual network device that is on all systems, even if
  they aren't connected to any network. It has an IP address of 127.0.0.1 and
  can be used to access network services locally.  
  Example: if you run a webserver on your machine and browse to it with Firefox
  or Chromium on the same machine, then it will go via this network device.

  
  \item \verb!eth0, eth1! = regular NIC
  
  \item \verb!en0, eno1! = embedded NIC (onboard network interface card)
  
  \item 
\end{enumerate}

\section{hostname-2-IP}

The only way to uniquely identify a machine is via IP address (Internet
Protocol). DNS (Domain Name Server) serves as a directory service to look up the
IP using the hostname (user-friendly name). Similar services being used
for LAN include WINS (in Windows), NIS (in Unix). NOTE: NDS can be used both for
LAN and Internet. For small LAN with a few machines, we can also use hostfiles
(/etc/hosts), instead of setting up a server running NIS, DNS or WINS.

The three popular commands
\begin{verbatim}
ifdown eth0
ifup eth0
ifconfig eth0
\end{verbatim}
The MAC address is stored in the file
\begin{verbatim}
/etc/udev/rules.d/*-net.rules
\end{verbatim}
If you replace the mainboard, you may need to wipe-out the content of this file,
shutdown the machine, and then restart it. Otherwise, you may get the error
\begin{verbatim}
SCIOCSIFADDR: no such device
\end{verbatim}
\url{http://blog.kabisa.nl/2009/12/11/xen-how-to-fix-siocsifaddr-no-such-device/}

\section{Cables (cat-5, cat-6)}
\label{sec:cable}

If you want to connect two machines directly for data transfer, use Cat-5 {\bf
crossover cables}. Other than that, i.e. connect the machines to the
hub/switch/router, we use Cat-5 patch cables. If you want to make the cable your
self, read
\url{http://helpdeskgeek.com/networking/lan-cable-installation-tips/}. Cat-5
supports 10Base-T and 100Base-T network standards, running at 10Mbps and 100
Mbps, respectively. The latest verion of Cat-5 is Cat-5e which supports
1000Base-T network, i.e. 1000 Mbps. Cat-5e also reduce crosstalk, i.e. the
``bleeding'' of signal between one cable to another, which is believed to reduce
the signal quality and speed. 


Faster cables is Cat-6. Cat-5 use 100MHz, Cat-5e use 350MHz while Cat-6 use
200MHz. This provides a higher signal-to-noise ratio, providing higher
reliability network connection.


\section{Error: Unable to bind to localhost}


When you launch a webserver on the local machine, you may get the error
\begin{verbatim}
unable to bind to localhost
\end{verbatim}

Example: in Google App engine \verb!dev_appserver.py!
\begin{verbatim}
raise BindError('Unable to bind %s: %s' %self.bind_addr
\end{verbatim}


The problem is there is an existing socket connecting to 8080 port, first
identify the process binding to that port, and then kill it
\begin{verbatim}
$ netstat -ano | grep 8080
  TCP    127.0.0.1:8080         0.0.0.0:0              LISTENING       18708
  TCP    [::1]:8080             [::]:0                 LISTENING       18708
// TCP [the IP address]:[port number] .... #[target_PID]#

$ kill <process-ID>
\end{verbatim}
\url{http://stackoverflow.com/questions/8688949/how-to-close-tcp-and-udp-ports-via-windows-command-line}

In Windows, we can 
\begin{itemize}
  \item use Task Manager to kill the process (click on Process  tab,
enable PID by going to View/Select Columns/ check PID), and click 'End Process'
for the selected process.

  \item using a tool name \verb!TCPView! 
  \url{http://technet.microsoft.com/en-us/sysinternals/bb897437.aspx}
  
  \item using network monitoring program CurrPorts that 
  list of all currently opened TCP/IP and UDP ports on your local computer.
  \url{http://www.nirsoft.net/utils/cports.html}
  
  \item \verb!wkillcx! command to kill TCP connections from the command
  line (Windows XP/Vista/7 as well as Windows Server
  2003/2008.).\url{http://wkillcx.sourceforge.net/}
  
  \item sending a RST TCP packet  to the process asking it to close the  socket
  using  RST scanning by \verb!nmap! utility \url{http://nmap.org/}
  
  
\end{itemize}

In Mac OS, we can use
\begin{itemize}
  \item \verb!lsof! command
\begin{verbatim}
lsof -nl | egrep "TCP|UDP"
\end{verbatim}
\end{itemize}

\url{http://www.tech-recipes.com/rx/227/find-out-which-process-is-holding-which-socket-open/}

\section{nmap: }

\subsection{in windows}

Cygwin does not have package for nmap. However, you can install the binary
package and put the alias in \verb!~/.bash_profile! file
\begin{verbatim}
alias nmap="/cygdrive/c/Program\ Files\ \(x86\)/Nmap/nmap.exe"
\end{verbatim}

\url{https://www.scivision.co/nmap-with-cygwin/}

\section{RTNETLINK answers: File exists}

ERROR:
\begin{verbatim}
RTNETLINK answers: File exists
Failed to bring up eth0
\end{verbatim}
The basic problem is some previous configuration, automatic or manual (such as
running ifconfig from the cmd line) still lingers. The flush command fixes that
situation. 

Solution:
\begin{verbatim}
sudo ip addr flush dev eth0

sudo ifdown eth0
sudo ifup eth0
\end{verbatim}
\url{https://raspberrypi.stackexchange.com/questions/13895/solving-rtnetlink-answers-file-exists-when-running-ifup}