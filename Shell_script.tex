\chapter{Shell script}
\label{chap:shell_and_shell_scripting}

There are situations that you want to automate your tasks using
scripting language via shell programming. Shell programming is convenient to
automate many processes, but not aimed to solve heavy computational tasks.
There's a nice tool to convert script form one language to another
\url{http://sourceforge.net/projects/env2/}.

The beginning of a shell script is sha-bang (\verb.#!.) which is a two-byte
magic number (type ``man magic'' for further information). After that, on the
same line, is the path to the command interpreter. Some examples are:
\begin{verbatim}
#!/bin/sh
#!/bin/bash
#!/usr/bin/perl
#!/usr/bin/tcl
#!/bin/sed -f
#!/bin/awk -f 
\end{verbatim}

A common initial structure
\begin{verbatim}
E_WRONG_ARGS=85
script_parameters="-a -h -m -z"
#                  -a = all, -h = help, etc.

if [ $# -ne $Number_of_expected_args ]
then
  echo "Usage: `basename $0` $script_parameters"
  # `basename $0` is the script's filename.
  exit $E_WRONG_ARGS
fi
\end{verbatim}

\section{.bashrc}
\label{sec:bashrc}

An example of \verb!~/.bashrc!
{\small
\begin{verbatim}
# ~/.bashrc: executed by bash(1) for non-login shells.
# see /usr/share/doc/bash/examples/startup-files (in the package bash-doc)
# for examples

# If not running interactively, don't do anything
[ -z "$PS1" ] && return

#########################
###  SECTION 1 ##########
# don't put duplicate lines in the history. See bash(1) for more options
# don't overwrite GNU Midnight Commander's setting of `ignorespace'.
#export HISTCONTROL=$HISTCONTROL${HISTCONTROL+,}ignoredups
# ... or force ignoredups and ignorespace
#export HISTCONTROL=ignoreboth

# append to the history file, don't overwrite it
# This is good, if you have multiple-session, and you don't want only the last
# session commands are saved
shopt -s histappend

## HISTIGNORE:
#    & = special pattern (suppress duplicate entries)
#   '( *)'  = ignore command starts with space    
# for setting history length see HISTSIZE and HISTFILESIZE in bash(1)
export HISTSIZE=1000
export HISTFILESIZE=3000
export HISTCONTROL=ignorespace:ignoredups
export HISTIGNORE='ls:bg:fg:history:&'
export HISTTIMEFORMAT='%F %T '

  # to avoid crashed session, which you may loose the history 
  # NOTE: you don't need to worry if you're using byobu. 
export PROMPT_COMMAND='history -a'

  ### rebind terminal flow control to Ctrl-X
  ### so that we can use Ctrl-S for forward search
stty stop ^X
 
  ## make sure command that you entered on multiple lines are fit into a single
  # line. This allows you to parse the history file easier
shopt -s cmdhist



# check the window size after each command and, if necessary,
# update the values of LINES and COLUMNS.
shopt -s checkwinsize

# make less more friendly for non-text input files, see lesspipe(1)
[ -x /usr/bin/lesspipe ] && eval "$(SHELL=/bin/sh lesspipe)"

# set variable identifying the chroot you work in (used in the prompt below)
if [ -z "$debian_chroot" ] && [ -r /etc/debian_chroot ]; then
    debian_chroot=$(cat /etc/debian_chroot)
fi

# set a fancy prompt (non-color, unless we know we "want" color)
case "$TERM" in
    xterm*) color_prompt=yes;;
    screen*) color_prompt=yes;;
esac

# uncomment for a colored prompt, if the terminal has the capability; turned
# off by default to not distract the user: the focus in a terminal window
# should be on the output of commands, not on the prompt
#force_color_prompt=yes

if [ -n "$force_color_prompt" ]; then
    if [ -x /usr/bin/tput ] && tput setaf 1 >&/dev/null; then
 # We have color support; assume it's compliant with Ecma-48
 # (ISO/IEC-6429). (Lack of such support is extremely rare, and such
 # a case would tend to support setf rather than setaf.)
 color_prompt=yes
    else
 color_prompt=
    fi
fi
if [[ "$color_prompt" = yes && "$USER" = root ]]; then
    PS1='${debian_chroot:+($debian_chroot)}\[\033[01;32m\]\u@\h\[\033[00m\]: \[\033[01;34m\]\w\[\033[00m\] \$ '
elif [[ "$color_prompt" = yes ]]; then
    PS1='${debian_chroot:+($debian_chroot)}\[\033[00;33m\]\u@\h\[\033[00m\]: \[\033[01;34m\]\w\[\033[00m\] \$ '
else
    PS1='${debian_chroot:+($debian_chroot)}\u@\h:\w \$ '
fi
unset color_prompt force_color_prompt

# If this is an xterm set the title to user@host:dir
case "$TERM" in
xterm*|rxvt*|screen*)
    PS1="\[\e]0;${debian_chroot:+($debian_chroot)}\u@\h: \w\a\]$PS1"
    ;;
*)
    ;;
esac

# Alias definitions.
if [ -f ~/.bash_aliases ]; then
    . ~/.bash_aliases
fi

# enable programmable completion features (you don't need to enable
# this, if it's already enabled in /etc/bash.bashrc and /etc/profile
# sources /etc/bash.bashrc).
if [ -f /etc/bash_completion ]; then
    . /etc/bash_completion
fi

export EDITOR='vim'
export GREP_COLOR=';33'
#export PATH=$PATH:/usr/local/mysql/bin:/usr/local/mysql/sbin
\end{verbatim}

When you make some changes, a quickly refresh the session without logout is to
run
\begin{verbatim}
source ~/.bashrc

// or

. ~/.bashrc
\end{verbatim}
}

\textcolor{red}{Control history commands}: The list of typed commands is stored
in \verb!~/.bash_history! file. You can control what to put in here, how many
commands to store for each session, and for all sessions using the following
environment variables

\begin{itemize}
  \item HISTZIE = number of lines (commands) saved in the memory during the bash
  session
  \item HISTFILESIZE = number of lines (commands) saved in the file 
  \verb!~/.bash_history! (which is used to save all the commands during the
  session that can be reused next time)
\end{itemize}
To avoid storing duplicate commands we use HISTCONTROL (ignorespace,
ignoredups, ignoreboth)
\begin{verbatim}
export HISTCONTROL=ignorespace:ignoredups
\end{verbatim}
This is used as to run commands that you don't want to record in the history,
add one or more spaces in the front, and enable \verb!ignorespace!. Also, it's
useful to ignore duplicated commands.


You can add a number of commands that should not be added to the history file
\begin{verbatim}
export HISTIGNORE='ls:bg:fg:history'
\end{verbatim}

Put the data-time in format, that you can sort easily with \verb!cut! and
\verb!awk!
\begin{verbatim}
export HISTTIMEFORMAT='%F %T '
\end{verbatim}


\subsection{History of commands}

You don't have to retype the commands every time, but can reuse the one you type
in the past via \verb!history! command. Each command is saved in a buffer (a
array of string), with an integer to refer to the index of the command in the
buffer. To revoke the command at a given index \verb!<number>!, we use the
exclamation mark (!)
\begin{verbatim}
!<number>
\end{verbatim}

To search a command, use \verb!Ctrl-r! (reverse-search) and \verb!Ctrl-s!
(forward-search). However, the forward-search key-binding is shadowed by
terminal flow control key binding (XON/XOFF flow control, e.g. in Konsole),
which freeze or 'hang' the terminal session (and need to press \verb!Ctrl-c! to
unfreeze it). 

To be able to use \verb!Ctrl-s! for forward-search, we can 
\begin{itemize}
  \item Disable terminal flow control: \verb!stty -ixon!
  \item Keep terminal flow control, and re-bind the terminal flow control to
  another key, e.g. Ctrl-X
  \begin{verbatim}
  stty stop ^X 
  \end{verbatim}
  
  \item Keep terminal flow control, and re-bind the forward-search to another
  key, e.g. 
  
  \begin{enumerate}
    \item Alt-S
  \begin{verbatim}
  bind '"es":forward-search-history'
  \end{verbatim}
  or put in the file \verb!~/.inputrc!. This option, however, doesn't work
  nicely with
  Ctrl-R.\footnote{\url{http://ruslanspivak.com/2010/11/25/bash-history-incremental-search-forward/}}
  
     \item Use \verb!pgup! and \verb!pgdown! for reverse- and forward-search
  \begin{verbatim}
"\e[5~": history-search-forward
"\e[6~": history-search-backward
  \end{verbatim}
NOTE: Ctrl-R search all lines containing words, while history-search-forward
only searches lines beginning with words
    \end{enumerate}
\end{itemize}

\begin{mdframed}

To list the current key binding, we run \verb!stty -a! 

{\tiny
\begin{verbatim}
>>> stty -a
speed 38400 baud; rows 86; columns 319; line = 0;
intr = ^C; quit = ^\; erase = ^?; kill = ^U; eof = ^D; eol = M-^?; eol2 = M-^?; 
swtch = <undef>; start = ^Q; stop = ^S; susp = ^Z; rprnt = ^R; werase = ^W; 
lnext = ^V; flush = ^O; min = 1; time = 0;

-parenb -parodd cs8 -hupcl -cstopb cread -clocal -crtscts
-ignbrk -brkint -ignpar -parmrk -inpck -istrip -inlcr -igncr icrnl ixon -ixoff 
-iuclc -ixany -imaxbel -iutf8

opost -olcuc -ocrnl onlcr -onocr -onlret -ofill -ofdel nl0 cr0 tab0 bs0 vt0 ff0
isig icanon iexten echo echoe echok -echonl -noflsh -xcase -tostop -echoprt echoctl echoke
\end{verbatim}
}

\end{mdframed}

To increase the buffer size, we use HISTSIZE and HISTFILESIZE (default: 500) - Sect.\ref{sec:bashrc}.


\section{Shell}
\label{sec:shell}

A {\bf shell} is a window manager providing a single text-based terminal.
It takes your commands from the keyboard and gives them to the operating system
to perform. We have
\begin{enumerate}
  \item terminal shell (i.e. command-line interfaces): This is the only user
  interface available on a Unix computer. - Sect.\ref{sec:shells-CLI}.

NOTE: Even though the shell is a command line interface, you can still use the
mouse for several things: scroll backwards and forwards through the output of
the terminal window.
  
  \item GUI shell: Sect.\ref{sec:shell-GUI}
\end{enumerate}

Also, there are programs called "terminal emulators" -
Sect.\ref{chap:terminal-emulator}. They are programs that put a window up and
let you interact with the terminal shell.


\subsection{GUI shells}
\label{sec:shell-GUI}

%Read Chap.\ref{chap:GUI-shells}.
\begin{enumerate}
  \item GNOME shell (written in C, Javascript): %- Sect.\ref{sec:gnome-terminal} 
  
  It provides basic functions like launching applications, switching between
  windows and is also a widget engine.
  
  \item 
\end{enumerate}


\subsection{CLI shells (terminal shell): bash, zsh, fish, csh, tcsh, ksh}
\label{sec:shells-CLI}

A terminal shell (or command-line interface) is a shell program
(Sect.\ref{sec:shell}) that provides a simple UI environment, in that a user can
type in command, and the shell will launch the given command so that the user
can interact with the system via these commands.

There are different terminal shell programs in a Linux-based system:
BASH, CSH, KSH, TCSH.  Remeber that a process is created by launching a
particular terminal shell programm (Sect.\ref{sec:shell-process}).

As all terminal shell program provides a command-line environment,
to know which shell environment is currently being used, type (and hit ENTER)
\begin{verbatim}
echo $0
\end{verbatim}

An important aspect of using terminal shell is that each shell program comes
with a built-in support of scripting language so that the user can write a batch
file or script file which can be used as a regular executable file to automate
or perform certain task (as needed). Depending on the shell program being used,
you may have different level of complexity of the supported scripting language.

REMINDER: The wording like "shell" or "default shell", it should refer to the
system's default shell in that the system can be Ubuntu, RedHat Linux, MacOS,
etc.

\textcolor{red}{The first Unix shell is named Thompson shell - with the
executable file is {\it sh}} (introduced in 1971, first version of Unix; and
become obsolete since Unix version 6). Thompson shell was simple, and was not
designed for scripting. It supports feature like:
\begin{itemize}
  \item redirect output from one command (as input) to the next command
  
  Use \verb!<! for input; and \verb!>! for output. 
  
  \item pipe syntax is added to also support redirect output, as the syntax
  above is considered too ambiguous (as it is not clear if command2 refers to
  the text filename or binary program)
  
\begin{verbatim}
command1  | command2
command1  ^ command2

command1 > file1
\end{verbatim}

  \item no support for control statements like \verb!if! or \verb!goto!
\end{itemize}
Thompson shell is replaced by {\bf Bourne shell}, also name \verb!/bin/sh! as
excutable.

\textcolor{red}{Bourne shell is used in Unix version 7}; and \textcolor{red}{C
shell (csh) is used in 2BSD O/S} [both were released in 1979].

\begin{enumerate}
  \item BASH - Sect.\ref{sec:bash}

BASH is the most common shell; but many people like using C-style programming
language of CSH and TCSH.

  \item CSH, TCSH (Sect.\ref{sec:csh-shell}): it has many C-style functions to
  do shell-scripting.
  
Of course, we can use other shell terminal to do shell-scripting as well
(Sect.\ref{sec:shell_scripting}). TCSH is an enhanced version of CSH.

  \item ZSH - Sect.\ref{sec:zsh}
  
  \item FISH - Sect.\ref{sec:fish}
  
  \item DOS:

NOTE: MS-DOS is a very limited shell environment in Windows.

\end{enumerate}

Each shell also support different levels of search capability, e.g. search
previous commands
\begin{enumerate}
  \item BASH - Ctrl-R (backward search), TAB (for auto-completion), support
  syntax rule, e.g. using wildcard (asterisk), variables, or brackets
  
  
  \item ZSH: use oh-my-zsh (Sect.\ref{sec:oh-my-zsh})
  
 
  
  \item FISH: built-in 'search-as-you-type' (with automatic suggestions)
  
Fish has few syntactic rules, preferring features as commands rather than syntax. 

The only problem with Fish is that it's not entirely POSIX-compliant. So
sometimes my one-liner bash commands just brake

\end{enumerate}

To run more than one command on one line, we use semi-colon (;) to combine them.

To run a script in background, we typically add an ampersand (\&) to the end,
and the control is returned back to the shell program.
However, the backgrounded program will terminate when the terminal emulator
(Sect.\ref{sec:terminal-emulators}) is closed. This is because the operating
system sends SIGHUP (hang up) signal to background jobs when a terminal close;
and most programs terminate when receiving this signal. To prevent the signal
being sent, we add \verb!nohup! (Sect.\ref{sec:nohup}).

\begin{verbatim}
nohup <command  args > & 
\end{verbatim}

\subsection{-- FISH shell}
\label{sec:fish}

\url{https://en.wikipedia.org/wiki/Friendly_interactive_shell}

\subsection{-- Bash shell}
\label{sec:bash}

There are so many different files — .bashrc, .bash_profile, .bash_login,
.profile etc.
The issue is that Bash sources from a different file based on what kind of shell
it thinks it is in.
\begin{enumerate}
  \item interactive non-login shell:
  
  if \verb!--rcfile <file>! option is used: then only load that file
  
  otherwise, check if \verb!--norc! option is used: if yes, then loads nothing
  
  otherwise, load \verb!/etc/bash.bashrc! file, then load \verb!~/.bashrc! file.
  
  \item interactive login shell:
  
  if \verb!--noprofile! option is used: not loading any file
  
  otherwise, load \verb!/etc/profile! file
  
  \item non-interactive non-login shell
  
  only load \verb!$BASH_ENV! 
  
  \item non-interactive login shell, i.e. \verb!--login! option is used
  
  if \verb!--noprofile! is used, then the same order as in interactive login shell
  
\end{enumerate}

\label{sec:bash-loading-order-of-files}
\textcolor{red}{\bf When you open an interactive non-login shell}, the order of resouce files to be
read-in when booting a terminal BASH shell is
\begin{verbatim}
/etc/environment -> 
	[if using --rcfile <file> option]
		-> load that given file only
		
	[if not using --rcfile <file> option]
		[if --norc option is used]
			load nothing
		[else]
         -> /etc/bash.bashrc   ... (next line) 
         -> ~/.bashrc
\end{verbatim}

\textcolor{red}{For interactive login shell} (i.e. shell created by ssh login
session), it does not read \verb!.bashrc!, but read \verb!.bash_profile!.
\begin{verbatim}
/etc/environment -> 
	[if not using --noprofile option]
		 /etc/profile 
         -> ~/.bash_profile 
         -> ~/.bash_login 
         -> ~/.profile 
         [these files may actually load -> /home/<user>/.bashrc]

\end{verbatim}
However, the default content of \verb!.bash_profile! calls to load
\verb!.bashrc!. Example:
\begin{verbatim}
if [ -f ~/.bashrc ]; then
   source ~/.bashrc
fi
\end{verbatim}

The reason to have two files is that there are some information 
you only want to see it on login (e.g. load average, memory usage, current
users, etc.), so you only want to place this in your \verb!.bash_profile!.

\begin{mdframed}
EXCEPTION: Mac O/S X terminal window always call \verb!.bash_profile!
\end{mdframed}

\begin{verbatim}
/bin/bash
       The bash executable
/etc/profile
       The systemwide initialization file, executed for login shells
~/.bash_profile
       The personal initialization file, executed for login shells
~/.bashrc
       The individual per-interactive-shell startup file
~/.bash_logout
       The individual login shell cleanup file, executed when a login shell exits
~/.inputrc
       Individual readline initialization file
\end{verbatim}


Now, the location of these files are, by default, under the \verb!$HOME! folder. 
You can use the file in a different folder, and even in a different name.
\begin{enumerate}
  \item from \verb!.bash_profile! to load \verb!.bashrc!
  
\begin{verbatim}
BASHRC_CONFIG_DIR=~/config/bash

if [ -f "$BASHRC_CONFIG_DIR/.bashrc" ]; then
    . "$BASHRC_CONFIG_DIR/.bashrc"
fi
\end{verbatim}


  \item load \verb!.bashrc! 
  
  As \verb!bash! shell reads \verb!/etc/bash.bashrc! which is a system-wide
  bashrc, we can modify this file at the end

\begin{verbatim}
 if [ -s "${XDG_CONFIG_HOME:-$HOME/.config}/bash/bashrc" ]; then
    . "${XDG_CONFIG_HOME:-$HOME/.config}/bash/bashrc"
 fi
\end{verbatim}

\end{enumerate}

\subsection{--** bash 4.4 (2016)}
\label{sec:bash-4.4}

\url{https://lwn.net/Articles/700982/}

\url{http://sourcedigit.com/20892-bash-4-4-released-install-bash-shell-ubuntu/}

New 
\begin{enumerate}
  \item \verb.$!.  : represent the most recent process  PID
  
  
  \item \verb!BASH_SUBSHELL!
  
  
  \item 
\end{enumerate}

Remove feature to be the same as other shells
\begin{enumerate}
  \item \verb!set -i! 	: no longer valid
  
  
  \item 
\end{enumerate}


New behavior:
\begin{enumerate}
  \item bash checks \verb!$INSIDE_EMACS! as well as \verb!$EMACS!
  to know  whether or not bash is being run in a GNU Emacs shell window.
  
  
   
\end{enumerate}

\subsection{-- C shell: csh or tcsh}
\label{sec:csh-shell}
\label{sec:tcsh-shell}

C shell (csh or the improved version, tcsh) is a terminal program developped by
Bill Joy (a graduate student at UC Berkeley) in 1979, and was first used in 2BSD
O/S. 
\begin{enumerate}
  \item filename wildcard
  
  \item pipe
  
  \item here document 
  
  \item command substitution 
  
  \item variables
  
  \item control statements for condition-testing and iteration, using C-like
  syntax. 
  
  \item (since tcsh): add  filename and command completion and command line
  editing concepts
\end{enumerate}

NOTE: tcsh is now the main branch for ongoing development.

\subsection{Bourne shell: sh, zsh}
\label{sec:bourne-shell}
\label{sec:zsh-shell}

Bourne shell (also /bin/sh) is a replacement for the first Thompson shell
(/bin/sh); and was created by Stephen Bourne at AT\&T Bell Labs in 1979.
\begin{itemize}
  \item PWB shell (Mashey shell) is an extended version of Thompson shell; and
  was developped by  John Mashey.
  
  PWB shell is soon superceded by Bourne shell.
  \begin{itemize}
    \item \verb!if! and \verb!goto! command is part of the shell.
    
    \item \verb!if-then-else-endif! command
    
    \item \verb!switch! 
    
    \item \verb!while! statement
    
    \item \verb!onint! statement to ignore interrupts, or catch them to perform
    cleanup
    
    \item variable: is supported, but only one character is allowed; with some
    is reserved for special purpose
    
 \begin{verbatim}
 $s      the ancestor of $HOME
 $p      the ancestor of $PATH
 \end{verbatim}
 
    \item  if a non-binary file marked executable, it would fork another shell
    instance to read that file as a shell script.
 
\begin{verbatim}
//previously (a shell is forked explicitly)
sh excutable_script_file arguments

// now (the shell is implicitly forked)
// this is done by 'pexec' function of the shell
excutable_script_file arguments
\end{verbatim}
This makes use of the system call \verb!pexec()! which is replaced by
\verb!execvp()! function.
\url{http://www.csl.mtu.edu/cs4411.ck/www/NOTES/process/fork/exec.html}

   \item 
  \end{itemize}
  
    \item 
\end{itemize}

Most Unix-like systems continue to
have /bin/sh - which will be the Bourne shell.


\subsection{-- zsh shell script}
\label{sec:zsh}

As we cannot use \verb!source! command on zsh, it is hard to load initialization
file at log-on or ssh log-in shell. 

If we can't have a proper login session, Ubuntu will keeps logging you out of
the session. To get \verb!zsh! works in Ubuntu
\begin{verbatim}
apt-get install zsh
apt-get install git-core

wget https://github.com/robbyrussell/oh-my-zsh/raw/master/tools/install.sh -O - | zsh
//docker Debian based
bash -c "$(wget
https://raw.githubusercontent.com/robbyrussell/oh-my-zsh/master/tools/install.sh -O -)"


// chsh: PAM authentication failed
chsh $USER -s $(which zsh); 

sudo shutdown -r 0
\end{verbatim}

\url{https://github.com/robbyrussell/oh-my-zsh/issues/227#issuecomment-825773}
\url{https://gist.github.com/tsabat/1498393}
\url{https://github.com/robbyrussell/oh-my-zsh/issues/1224}

\url{https://github.com/robbyrussell/oh-my-zsh/issues/3061}

Use \verb!~/.zshrc! from oh-my-zsh 
\begin{verbatim}
bindkey "\033[1~" beginning-of-line
bindkey "\033[4~" end-of-line
bindkey '^[[3~' delete-char
\end{verbatim}

To backward delete, use Ctrl-w 

\subsection{--** zsh 5.5}

\url{http://zsh.sourceforge.net/releases.html}

\subsection{--** zsh 5.1.1 }

\subsection{--** oh-my-zsh}
\label{sec:oh-my-zsh}

  Oh-my-zsh is community-driven framework for managing zsh configuration. It
  includes 200+ optional plugins (rails, git, OSX, hub, capistrano, brew, ant,
  php, python, etc), over 140 themes to spice up your morning, and an
  auto-update tool so that makes it easy to keep up with the latest updates from
  the community. Oh-my-zsh framework will give you auto-completion, tab
  completion, several cool themes, programming language integration and like
  bash shell, ZSH is POSIX-compliant.

  

\section{Script overview}
\label{sec:shell_scripting}

Commands can be combined into a single plain-text file to run on shell, known as
{\it shell script}. It can automate tasks by avoiding retyping the same
commands, or using different parameters for the same commands.

\begin{enumerate}
\item Comment line: start with \#
 
\item Define a variable: \verb!varname=value! (no space on either
  side of the equal sign), value can be a number of a string,
  \verb!varname! is case sensitive.

  To access a variable, we add dollar sign (\$) in front. 

\item To perform arithmetic operations, we use \verb!expr! statement
\begin{verbatim}
expr op1 math-operator op2
\end{verbatim}
E.g.
\begin{verbatim}
expr 10 \* 3

echo `expr 10 + 3
\end{verbatim}
NOTE: When we pass a shell statement to echo, the shell statement need
to have a back quote (\verb!`! sign) in front and end with another
back quote. Otherwise, if we use single quotes or double quotes, it
will interpret as a regular string. 

\item Useful variables
\begin{verbatim}
$#	Number of command line arguments. Useful to test no. of 
        command line args in shell script.
$*	All arguments to shell
$@	Same as above
$-	Option supplied to shell
$$	PID of shell
$!	PID of last started background process (started with &)
\end{verbatim}
To access each argument, we use \verb!$0, $1, $2...!
\begin{verbatim}
echo "Total number of command line argument are $#"
echo "$0 is script name"
echo "$1 is first argument"
echo "$2 is second argument"
echo "All of them are :- $* or $@"
\end{verbatim}


\item Logical operators
\begin{verbatim}
>
<
!=
==
\end{verbatim}

\item Bitwise operator
\begin{verbatim}
! expression	Logical NOT
expression1  -a  expression2	Logical AND
expression1  -o  expression2	Logical OR
\end{verbatim}


\item \verb!test! command
\begin{verbatim}
if test $1 -gt 0
then
echo "$1 number is positive"
fi
\end{verbatim}

NOTE: \verb!test expr! is equivalent to \verb! [ expr ] ! (there is
space after [ and before ])


\item Testing:

In shell script, we often use the following mathematical operators
\begin{verbatim}
-eq	is equal to	5 == 6	if test 5 -eq 6	if [ 5 -eq 6 ]
-ne	is not equal to	5 != 6	if test 5 -ne 6	if [ 5 -ne 6 ]
-lt	is less than	5 < 6	if test 5 -lt 6	if [ 5 -lt 6 ]
-le	is less than or equal to	5 <= 6	if test 5 -le 6	if [ 5 -le 6 ]
-gt	is greater than	5 > 6	if test 5 -gt 6	if [ 5 -gt 6 ]
-ge	is greater than or equal to
\end{verbatim}

For string, we can use
\begin{verbatim}
string1 = string2	string1 is equal to string2
string1 != string2	string1 is NOT equal to string2
string1	string1 is NOT NULL or not defined 
-n string1	string1 is NOT NULL and does exist
-z string1	string1 is NULL and does exist
\end{verbatim}

For file, we can use
\begin{verbatim}
-s file   	Non empty file
-f file   	Is File exist or normal file and not a directory 
-d dir    	Is Directory exist and not a file
-w file  	Is writeable file
-r file   	Is read-only file
-x file   	Is file is executable
\end{verbatim}

\end{enumerate}

\section{Script examples}

A script is a plain text file. To make a script executable, we do
\begin{verbatim}
chmod +x <script_name>
\end{verbatim}
We can use any editor to edit the script file, e.g. VIM, Emacs, PICO, etc. The
end of the script, should be 
\begin{verbatim}
exit 0
\end{verbatim}
to indicate that the script run successfully.

A script can receive arguments through using 
\begin{verbatim}
$0 = script name itself
$1 = first argument
$2 = second argument
\end{verbatim}

A script or any shell command return two type of values: zero (success),
non-zero (failure). To check the error status of the last command, we do
\begin{verbatim}
echo $?
\end{verbatim}

\subsection{Automate answer}

If you need to type in the answer to the command line, you can automate the
process by using \verb!echo! command and pipeline.

Example:
\begin{verbatim}
(echo n; echo p; echo 1; echo 1; echo 200; echo w) | fdisk /dev/sdc
\end{verbatim}
to create a 200MB partition.



\subsection{System Variables}

We can test the values of the following system variables, if these settings
affect how the script behaves
\begin{verbatim}
BASH=/bin/bash	Our shell name
BASH_VERSION=1.14.7(1)	Our shell version name
COLUMNS=80	No. of columns for our screen
HOME=/home/vivek	Our home directory
LINES=25	No. of columns for our screen
LOGNAME=students	students Our logging name
OSTYPE=Linux	Our Os type
PATH=/usr/bin:/sbin:/bin:/usr/sbin	Our path settings
PS1=[\u@\h \W]\$	Our prompt settings
PWD=/home/students/Common	Our current working directory
SHELL=/bin/bash	Our shell name
USERNAME=vivek
\end{verbatim}

\subsection{Special variables}

\begin{verbatim}
$?  = the last error code
$#  = the number of argument passed to the script
$*  = all arguments passed to the script
$@  = $*
$0  = zero-th argument (the script name)
$1  = first argument 
\end{verbatim}

\begin{verbatim}
# Check for proper number of command line args.

EXPECTED_ARGS=1
E_BADARGS=65

if [ $# -ne $EXPECTED_ARGS ]
then
  echo "Usage: `basename $0` {arg}"
  exit $E_BADARGS
fi
\end{verbatim}


\subsection{User-defined Variables}

A script is like a small program, we can define variables which can receive
either numeric or character string
\begin{verbatim}
no=10

vech=Bus
\end{verbatim}
NOTE: There is NO space in front and after the equal sign (=).

A variable is case-sensitive. We can also define a NULL variable, which has no
value at the time of definition.
\begin{verbatim}
vech=

vech=""
\end{verbatim}

\subsection{-- Value of a variable}

To check the value of one variable (user-defined or system) in the script, we
need to add dollar sign (\$) in the front.
\begin{verbatim}
echo $BASH

yourname=Tuan
echo "Hello, $yourname"
\end{verbatim}

\subsection{Script input (command-line argument)}

For the shell script, to pass input  
\begin{enumerate}
  \item as command-line argument: then we can access using macros
  \verb!$1, $2!
  

\verb!$@! refer to all arguments.

To get the last command-line argument:
\begin{verbatim}
// bash-only
echo "${@: -1}"

echo ${@:${#@}} 
echo ${@:$#}

// all shelles
for last; do true; done
echo $last
\end{verbatim}

  \item From the content of another file: use \verb!<!
  \begin{verbatim}
  script < file
  \end{verbatim}
  
  \item We can read value from the terminal
\begin{verbatim}
read var1, var2, ..., varN
\end{verbatim}

Example:
\begin{verbatim}
echo "Your first name please:"
read fname
echo "Hello $fname, Lets be friend!"
\end{verbatim}
\end{enumerate}


%\subsection{-- Command-line argument}

\begin{verbatim}
$1 is the first argument.

$2 is the first argument.

..

$@ is all of them.
\end{verbatim}

Get the last argument:
\begin{enumerate}
  \item  This one is also pretty portable (again, should work with bash, ksh and
  sh) and it doesn't shift the arguments, which could be nice.
\begin{verbatim}
for last; do true; done
echo $last
\end{verbatim}

It uses the fact that for implicitly loops over the arguments if you don't tell
it what to loop over. 
Also,  for loop variables aren't scoped, i.e. they keep the last value they were
set to, which is \verb!$last! in this case.
\end{enumerate}

\subsection{-- short argument without a value}

\verb!getopts! is a bash built-in, but there is also an external \verb!getopt!
program that should work with any POSIX-compatible shell.
However, the external command getopt(1) is never safe to use, unless you know it
is GNU getopt, you call it in a GNU-specific way.
\url{http://wiki.bash-hackers.org/howto/getopts_tutorial}

IMPORTANT: It can only parse short option (e.g. -c); but not for long option
(e.g. --myoption).

You use \verb!getopts! but don't put the colon at the end of the character
repsenting the option.
\begin{verbatim}
optspect = ":abc"
while getopts "$optspec" opt; do
...
\end{verbatim}

\url{http://stackoverflow.com/questions/16483119/example-of-how-to-use-getopts-in-bash}

\subsection{-- short argument with value}

IMPORTANT: It can only parse short option (e.g. -c); but not for long option
(e.g. --myoption).

To indicate the an option as an associated value to be passed in with it, put a
colon after the character representing the option.

Example: 'a', 'b' have no values; 'c' and 'd' have the values
\begin{verbatim}
#!/bin/bash
optspec=":ab:c:d:"
while getopts ":a:" opt; do
  case $opt in
    a)
      echo "-a was triggered, Parameter: $OPTARG" >&2
      ;;
    \?)
      echo "Invalid option: -$OPTARG" >&2
      exit 1
      ;;
    :)
      echo "Option -$OPTARG requires an argument." >&2
      exit 1
      ;;
  esac
done
\end{verbatim}


\subsection{-- long argument with value}

There is a trick; first we use \verb!-! as the character option, and then
continue parsing it.

\begin{verbatim}
optspec=":hv-:"
while getopts "$optspec" optchar; do
    case "${optchar}" in
        -)
            case "${OPTARG}" in
                loglevel)
                    val="${!OPTIND}"; OPTIND=$(( $OPTIND + 1 ))
                    echo "Parsing option: '--${OPTARG}', value: '${val}'" >&2;
                    ;;
                loglevel=*)
                    val=${OPTARG#*=}
                    opt=${OPTARG%=$val}
                    echo "Parsing option: '--${opt}', value: '${val}'" >&2
                    ;;
                *)
                    if [ "$OPTERR" = 1 ] && [ "${optspec:0:1}" != ":" ]; then
                        echo "Unknown option --${OPTARG}" >&2
                    fi
                    ;;
            esac;;
       h)
            echo "usage: $0 [-v] [--loglevel[=]<value>]" >&2
            exit 2
            ;;
    esac
done
\end{verbatim}
\url{http://stackoverflow.com/questions/402377/using-getopts-in-bash-shell-script-to-get-long-and-short-command-line-options/7680682#7680682}
\subsection{-- number of arguments}

\verb!$#! returns the number of command-line input arguments passed to the
script.

\begin{verbatim}
if [ $# -eq 0 ]
  then
    echo "No arguments supplied"
fi

 // or equivalent: 
 // The -z switch will test if the expansion of "$1" is a null string or not. 
if [ -z "$1" ]
  then
    echo "No argument supplied"
fi
\end{verbatim}


\subsection{Integer arithmetic operations}

There are different ways of doing this:
\begin{enumerate}
  \item We can do arithmetic operation using \verb!expr! command.

\begin{verbatim}
expr op1 math-operator op2
\end{verbatim}
With math-operator can be \verb!\*! (multiply), /, +, -, \% (modulus: the
remainder).

Example:
\begin{verbatim}
sum = $(expr 6 + 3)
\end{verbatim}

Example: we use the back-quote (found under the tilde (\~) key) to execute a
command
\begin{verbatim}
x = `expr 6 + 3`
\end{verbatim}


   \item Put them in \verb!$(( ))!

\begin{verbatim}
x = $(( 4 + 3 ))

NPROCS=4
x = $(( $(NPROCS) + 3 ))
\end{verbatim}

Becareful, as \verb!((  ))! expects numbers, and it treats all string as 0, and
it can happen that NPROCS gets a string value.

   \item Use \verb!let! command
\begin{verbatim}
let x = 4 + 3
\end{verbatim}
\end{enumerate}


\subsection{Boolean operations}

To use in conditionals, put the expression inside \verb!(( ))!, which returns 
0 for true (non-zero) and 1 for false
\begin{verbatim}
(( 4 + 3))
\end{verbatim}


\subsection{Float arithmetic operations}

 If you need more maths power, use tools like bc/awk/dc

\begin{verbatim}
var=$(echo "scale=2;3.4+43.1" | bc)
var=$(awk 'BEGIN{print 3.4*43.1}')

awk '{printf "%.2f\n",$0/59.5}' ball_dropping_times >bull_velocities
\end{verbatim}

Example: we use pipe to direct the expression to \verb!bc! utility
\begin{verbatim}
echo 3 + 4 | bc
\end{verbatim}

Example:
\begin{verbatim}

#!/bin/bash

for i in `cat ball_dropping_times`
do
echo "scale=20; $i / 59.5" | bc 
done > ball_velocities
\end{verbatim}

\url{http://stackoverflow.com/questions/2552130/bash-evaluate-a-mathematical-term}
\subsection{Parameter substitution}

To substitute the value of a variable, and concatenate it into a given string,
we use \verb!$(var-name)!
\begin{verbatim}
x = "the string $($ echo 'expr 6 + 3')"
\end{verbatim}
to print out ``the string 9''.




\subsection{Loops}

In BASH, we can have {\it for loop} and {\it while loop}.

\begin{verbatim}
for { variable name } in { list }
do
         execute one for each item in the list until the list is
         not finished (And repeat all statement between do and done)
done

for (( expr1; expr2; expr3 ))
do
      .....
			   ...
      repeat all statements between do and 
      done until expr2 is TRUE
done
\end{verbatim}

E.g.:
\begin{verbatim}
for i in 1 2 3 4 5
do
echo "Welcome $i times"
done
\end{verbatim}

E.g.:
\begin{verbatim}
for ((  i = 0 ;  i <= 5;  i++  ))
do
  echo "Welcome $i times"
done
\end{verbatim}


Example:
\begin{verbatim}
for ((  i = 0 ;  i <= 5;  i++  ))
do
  echo "Welcome $i times"
done
\end{verbatim}

NOTE: It's important to have space before and after parantheses, operator;
semicolon. 

\begin{verbatim}
   while [ condition ]
   do
         command1
         command2
         command3
         ..
         ....
   done
\end{verbatim}

E.g.:
\begin{verbatim}
n=$1
i=1
while [ $i -le 10 ]
do
  echo "$n * $i = `expr $i \* $n`"
  i=`expr $i + 1`
done
\end{verbatim}

\begin{verbatim}
fun ()
{ # A somewhat more complex function.
  i=0
  REPEATS=30

  echo
  echo "And now the fun really begins."
  echo

  sleep $JUST_A_SECOND    # Hey, wait a second!
  while [ $i -lt $REPEATS ]
  do
    echo "----------FUNCTIONS---------->"
    echo "<------------ARE-------------"
    echo "<------------FUN------------>"
    echo
    let "i+=1"
  done
}
\end{verbatim}

\subsection{Conditional statements}

\subsection{-- CASE}

CASE
\begin{verbatim}
case  $variable-name  in
     pattern1)   command
          ...
          ..
          command;;
     pattern2)   command
          ...
          ..
          command;;
     patternN)   command
          ...
          ..
          command;;
     *)  command
          ...
          ..
          command;;
esac
\end{verbatim}

\subsection{-- IF}

It is important to understand the use of single bracket and double bracket
(Sect.\ref{sec:single-bracket}). 
 
EXAMPLE:  IF
\begin{verbatim}
if [ expression ]
then
   statement;
fi

if [ condition ] 
then
	statement;
else
	statement;
fi

if [ cond1 ]
then
	statement;
elif [ cond2 ]
then
	statement;
elif [ cond3 ]
	statement;
else
	statement;
fi
\end{verbatim}

For negative we use !
\begin{verbatim}
if ! expression
then
  statement
fi

if [ ! expression ]
then
  statement
fi

if test ! condition
then
   statement
fi

if [ ! condition ]
then
  statement
fi
\end{verbatim}


If we want to put on the same line we need to use semi-colon (;)
\begin{verbatim}
if [ expression ]; then
   statement;
fi
\end{verbatim}

To deal with complicated logical expressions, we need to use \verb!(  )! or
\verb!{ }!.
\begin{verbatim}

# we can use &&, || 
if [ expr1 ] && [ expr2 ]
then
   statement;
fi

//negative of two 
if ! ([ expr1 ] || [ expr2 ])
then
   statement;
fi

# more complex conditional evaluation
if [ $a -eq 1 ] || { [ $b -eq 5 ] && [ "$c" = c ];}; then ...


if [ "$a" = a ] || { [ "$b" = b ] && [ "$c" = c ];}; then ...

\end{verbatim}


E.g.:
\begin{verbatim}
if cat $1
then
echo -e "\n\nFile $1, found and successfully echoed"
fi
\end{verbatim}
the \verb!cat! command return 0 on exit if success. 
\begin{verbatim}
if rm $1
then
echo "$1 file deleted"
fi
\end{verbatim}

\begin{verbatim}
if test $1 -gt 0
then
echo "$1 number is positive"
fi

if [ $1 -gt 0 ]
then
echo "$1 number is positive"
fi
\end{verbatim}


\subsection{single bracket [] vs. double bracket [[ ]]}
\label{sec:single-bracket}

Bash supports either using single bracket 
\begin{verbatim}
[ expression ]
\end{verbatim}
 or double bracket 
 \begin{verbatim}
[[ expression(s) ]]
 \end{verbatim}
to represent an expression. NOTE: double bracket is an enhancement that is not
available to all shell scripts, e.g. \verb!sh!-compatible shell. 

We can also use the double brackets version [IMPORTANT: Even though [[ ]] is a
new, improved version of [], only a few shells (bash, ksh, zsh) support. POSIX
doesn't specify what it does]. 
So make sure add shebang like at the beginning of the script 
\verb-#!/bin/bash-.

There are 2 main enhancements
\begin{enumerate}
  
  \item we can use logical operator (\verb!&& (AND), ||! (OR) and \verb.!.
  (NOT)) for boolean tests, and \verb!<! or \verb!>! for string comparison, such as 
\begin{verbatim}

[[ expression ]]

[[ expression1 && expression2 ]]

[[ expression1 || expression2 ]]

\end{verbatim}

NOTE: single bracket only accept operator such as \verb!-o! (OR), 
\verb!-a! (AND).
% Other changes are conditional evaluation (NOT recommend to use conditional
% evaluation with [[]])
\begin{verbatim}
   [ ]            [[ ]]
   &&            -a(**)
   ||            -o(**)
\end{verbatim}

  \item  we don't have to quote the variable in double quote, i.e.
  \verb!''$FILE''! in double brackets.
The reason for putting the variable name inside double quote, to ensure bash can
handle empty string, i.e. there is no value in the variable \verb!$FILE!.
\begin{verbatim}
// old syntax
if [ -f "$FILE" ] ; then
  //something
fi

// new syntax
if [[ -f $FILE ]] ; then
  //something
fi

if [[ -d ${DIRECTORY} && ! -L ${DIRECTORY} ]] ; then
    echo "It's a bona-fide directory"
fi

\end{verbatim}

  \item support regular expression using \verb!=~! operator, along with operators in
C-language.\footnote{\url{http://mywiki.wooledge.org/BashFAQ/031}} 
  
\begin{verbatim}
if [[ $ANSWER =~ ^y(es)?$ ]]


// old syntax
if [ "$ANSWER" = y -o "$ANSWER" = yes ]

\end{verbatim}

It has \verb!BASH_REMATCH! , which if we type \verb!yes!, then
\begin{verbatim}
${BASH_REMATCH[1]} would be "es" if you typed a full "yes" above
\end{verbatim}

Example: more relaxed criteria for 'yes'
\begin{verbatim}
if [[ $ANSWER = y* ]]



[[ $name = a* ]]  // a $name starts with 'a'

[[ $(date) =~ ^Fri\ ...\ 13 ]] // string start with 'Fri' and 
\end{verbatim}

\end{enumerate}


% The main difference is that with [ ] we have to quote
% \verb!$b!, i.e. \verb!''$b''!, while we don't have to with [[ ]]. 
% \begin{verbatim}
% if [[ -d "${DIRECTORY}" && ! -L "${DIRECTORY}" ]] ; then
%     echo "It's a bona-fide directory"
% fi
% 
% if [[ -e "$file" ]]; then
%    do_something
% fi
% \end{verbatim}
% [[ ]] also has some nice features, e.g. regular expression matching by using
% \verb!=~!, along with operators in
% C-language.\footnote{\url{http://mywiki.wooledge.org/BashFAQ/031}} Pattern
% matching is available with [[ ]].
% \begin{verbatim}
% [[ $name = a* ]]  // a $name starts with 'a'
% 
% [[ $(date) =~ ^Fri\ ...\ 13 ]] // string start with 'Fri' and 
% \end{verbatim}


\subsection{parentheses () vs. curly braces (\{ \})}


\begin{verbatim}
()

( list )
Placing a list of commands between parentheses causes a subshell environment to be created, 
and each of the commands in list to be executed in that subshell. 
Since the list is executed in a subshell, variable assignments do not remain in effect after the subshell completes.


{}

{ list; }
Placing a list of commands between curly braces causes the list to be executed in the current shell context. 
No subshell is created. The semicolon (or newline) following list is required.
\end{verbatim}

So, code in \verb!'{}'! is executed in the current thread/process/environment
and changes are preserved, so any modification of the variable will affect the
global context of the same variable name.

We often use curly brace to define the funciton's body; but again, we can also
use the parentheses as well.
\begin{verbatim}
// context 1
count_tmp() { cd /tmp; files=(*); echo "${#files[@]}"; }

// context 2 (though un-usual, as any change disappears after the completion of
//           function)
count_tmp() (cd /tmp; files=(*); echo "${#files[@]}")
\end{verbatim}


\subsection{Ternary operator}

Suppose you want something like
\begin{verbatim}
int a = (b == 5) ? c : d;
\end{verbatim}

Just for numeric comparison and assignment, you can do
\begin{verbatim}
(( a = b==5 ? c : d )) # string + numeric
\end{verbatim}
If will gives unpredictable result if we use it for string comparison, as the 
\verb!(( ))! treats any/all strings as zero.

In bash, you can do
\begin{verbatim}
a=$([ "$b" == 5 ] && echo "$c" || echo "$d")

// or  
  // yet watch out this construct, as the part after ||
  // also runs when part between && and || fails.
[[ $b = 5 ]] && a="$c" || a="$d"
\end{verbatim}


\url{http://stackoverflow.com/questions/3953645/ternary-operator-in-bash}

\subsection{Comparisons}

\subsection{-- Integer comparison}

NOTE: white-space is important before and after the operator, and for variable
substition we need to put them between double quotes (``'').

\begin{verbatim}
if [ "$a" -eq "$b" ]

-ne
-gt
-ge
-lt
-le
\end{verbatim}
or use within double parentheses
\begin{verbatim}
(("$a" < "$b"))

<
<=
>
>=
\end{verbatim}

\subsection{-- compare 2 string (igore trailing blank spaces)}

\begin{verbatim}
[[ $list =~ (^|[[:space:]])"$x"($|[[:space:]]) ]] && echo 'yes' || echo 'no' 
\end{verbatim}

\url{http://stackoverflow.com/questions/8063228/how-do-i-check-if-a-variable-exists-in-a-list-in-bash}

\subsection{String comparison}

We use ==, \verb.!=., <, > to compare two string. 
Read Sect.\ref{sec:single-bracket} to understand if single bracket or double
bracket is used, which may affect the way we do string comparison.

IMPORTANT: In single bracket, we need to put the user-defined variable within
double quotes (`` '').

\begin{verbatim}
[[ $a == z* ]]   # True if $a starts with an "z" (pattern matching).
[[ $a == "z*" ]] # True if $a is equal to z* (literal matching).

[ $a == z* ]     # File globbing and word splitting take place.
[ "$a" == "z*" ] # True if $a is equal to z* (literal matching).
\end{verbatim}


NOTE: white-space is important before and after the operator in single bracket
\begin{verbatim}
if [ "$a" == "$b" ]

== 
!=
\end{verbatim}
and check for null string (zero-length) with \verb!-z!. 
\begin{verbatim}
String=''   # Zero-length ("null") string variable.

if [ -z "$String" ]
then
  echo "\$String is null."
else
  echo "\$String is NOT null."
fi     # $String is null.
\end{verbatim}
and not null with \verb!-n!
\begin{verbatim}

\end{verbatim}

Special cases: for < and >
\begin{verbatim}
<   (using ASCII alphabetical order)
if [[ "$a" < "$b" ]]
if [ "$a" \< "$b" ]

if [[ "$a" > "$b" ]]

if [ "$a" \> "$b" ]
\end{verbatim}




\subsection{Exit a script}


We use \verb!exit! command, with a return-code.
\begin{verbatim}
# success
exit 0

# failure -> return a non-zero value
status_code = 3
exit status_code

# [default] exit using status code of the last command
exit $?
\end{verbatim}

NOTE: \verb!$?! contains the error code of the last command, including the
called script.

\subsection{Output}

The default output is to the terminal. To check the output later, we can
redirect it to a file. There are two options: (1) append to the end of the file,
(2) clean the file and add data to it. 
\begin{enumerate}
  \item Use \verb!>!
  \item Use \verb!>>!
\end{enumerate}

We can also pipeline the output to another command using \verb!|!
\begin{verbatim}
command_1  |   command_2
\end{verbatim}

If we don't want the output to be printed out, we can redirect it to the NULL
device.
\begin{verbatim}
command > /dev/null
\end{verbatim}

\section{-- Function}
\label{sec:bash_function}

\begin{verbatim}
function function_name { 
command... 
} 
\end{verbatim}
or
\begin{verbatim}
function_name () { 
command... 
} 
\end{verbatim}

To write a single-line function, we need to end each statement, including
the last one, with a semi-colon.
\begin{verbatim}
fun () { echo "This is a function"; echo; }
\end{verbatim}

NOTE: The function definition must precede the first call to it. A function
cannot be empty, even if it has some comments.

\subsection{Function inside function}

A function can be defined inside another one. We can also call it from outside,
after it has been evoked. However, using function-enclosing is not a good way.
\begin{verbatim}
f1 ()
{

  f2 () # nested
  {
    echo "Function \"f2\", inside \"f1\"."
  }

}  

f2  #  Gives an error message.
    #  Even a preceding "declare -f f2" wouldn't help.

echo    

f1  #  Does nothing, since calling "f1" does not automatically call "f2".
f2  #  Now, it's all right to call "f2",
    #+ since its definition has been made visible by calling "f1".

    # Thanks, S.C.
\end{verbatim}

\subsection{Parameters}

The parameters passed to a function is detected as \verb.$1., \verb.$2.

When calling a function, we pass argument without using parentheses
\begin{verbatim}
//wrong
func_name("arg1")


//correct
func_name "arg1"
\end{verbatim}

\subsection{Zero length parameters}

\begin{verbatim}
#!/bin/bash
# Functions and parameters

DEFAULT=default                             # Default param value.

func2 () {
   if [ -z "$1" ]                           # Is parameter #1 zero length?
   then
     echo "-Parameter #1 is zero length.-"  # Or no parameter passed.
   else
     echo "-Param #1 is \"$1\".-"
   fi

   variable=${1-$DEFAULT}                   #  What does
   echo "variable = $variable"              #+ parameter substitution show?
                                            #  ---------------------------
                                            #  It distinguishes between
                                            #+ no param and a null param.

   if [ "$2" ]
   then
     echo "-Parameter #2 is \"$2\".-"
   fi

   return 0
}

echo
   
echo "Nothing passed."   
func2                          # Called with no params
echo


echo "Zero-length parameter passed."
func2 ""                       # Called with zero-length param
echo

echo "Null parameter passed."
func2 "$uninitialized_param"   # Called with uninitialized param
echo

echo "One parameter passed."   
func2 first           # Called with one param
echo

echo "Two parameters passed."   
func2 first second    # Called with two params
echo

echo "\"\" \"second\" passed."
func2 "" second       # Called with zero-length first parameter
echo                  # and ASCII string as a second one.

exit 0
\end{verbatim}

\section{-- Script power}

\subsection{Wait 4 User-input}
\label{sec:User_input}

We can
read-in one
or a few
``word" input
using\url{http://tldp.org/LDP/Bash-Beginners-Guide/html/sect_08_02.html}
\begin{verbatim}
read [options] NAME1 NAME2 ... NAMEN
\end{verbatim}
Typically, the delimiter for input data is a new line. However, we can change to
using comma, semi-colon, or whatever it is using -d DELIM option


\begin{verbatim}
#!/bin/bash
# This script will test if you have given a leap year or not.

echo "Type the year that you want to check (4 digits), followed by [ENTER]:"

read year

if (( ("$year" % 400) == "0" )) || (( ("$year" % 4 == "0") && ("$year" % 100 !=
"0") )); then
  echo "$year is a leap year."
else
  echo "This is not a leap year."
fi

michel ~/test> leaptest.sh
Type the year that you want to check (4 digits), followed by [ENTER]:
2000
2000 is a leap year.
\end{verbatim}




\subsection{Using wildcard in script argument}

We cannot use wildcard (*) in passing to script as argument, e.g.
\begin{verbatim}
myscript path/to/data*
\end{verbatim}
The reason is that when the script is launched, the bash shell has already
expanded the wildcard, i.e. using the first item in the list of data returned by
expanding the wildcard.

To avoid expansion of characters (\verb.*, ?,. and \verb.[.), we can run bash
with the -f switch
\begin{verbatim}
bash -f -c 'ls *'
\end{verbatim}
However, this is inconvenient. A work-around solution is to quote parameters
with ' and '. 
\begin{verbatim}

\end{verbatim}



\section{-- String - Numbers}

\subsection{Read a number}

To make sure an input string is digits
\begin{verbatim}
while [[ $(echo -n ${inputNo} | wc -c) -lt 2 ]] ; do inputNo="0${inputNo}"; done
\end{verbatim}

\subsection{Numerical string of given length}

We can make a string of number with a given length, by adding 0 in the front.
\begin{verbatim}
export inputNo=5   ; inputNo=$(printf "%02d" $inputNo) ; echo $inputNo
05
\end{verbatim}

\url{http://stackoverflow.com/questions/1898712/make-sure-int-variable-is-2-digits-long-else-add-0-in-front-to-make-it-2-digits}


\subsection{Trim a string}

We can use the command \verb!sed -e 's/^ *//g' -e 's/!
\begin{verbatim}
machinename=`hostname`; machiname=`echo $machinename | sed -e 's/^ *//g' -e 's/
          *$//g'` ; echo $machinename ; IF [ $machiname == 'leak' ];; then 
          echo "machine detected; else echo "machine not detected"; fi
\end{verbatim}

\section{-- File/Folder}

Each file (folder) has three information about time
\begin{verbatim}
    atime - the time the file was last accessed (e.g. read)
    mtime - the time the file contents were last modified (e.g. appended to)
    ctime - time of last change to the file contents or the file meta-data (e.g. permissions change).
\end{verbatim}
The \verb!ls -l! command gives the mtime information. 


\subsection{Check file/folder existence}

IMPORTANT: The variable name must be wrapped by double quotes 

To check the existence of a folder
\begin{verbatim}
//folder
if [ -d "$DIRNAME" ]
then
	//do something
fi


//file
if [ -f $FILENAME ]
then
	//do something
fi
\end{verbatim}

If you want to check for the present of a binary file, e.g. executable file in a
given PATH, use one of this
\begin{verbatim}
$ command -v foo >/dev/null 2>&1 || { echo >&2 "I require foo but it's not installed.  Aborting."; exit 1; }
$ type foo >/dev/null 2>&1 || { echo >&2 "I require foo but it's not installed.  Aborting."; exit 1; }
$ hash foo 2>/dev/null || { echo >&2 "I require foo but it's not installed.  Aborting."; exit 1; }


Example:
    if hash gdate 2>/dev/null; then
        gdate "$@"
    else
        date "$@"
    fi
\end{verbatim}
IMPORTANT: DO NOT use \verb!which! command, as many operating systems have a
which that doesn't even set an exit status.
\url{http://stackoverflow.com/questions/592620/check-if-a-program-exists-from-a-bash-script}

But if it's a symbolic link, then you may consider
\begin{verbatim}
if [ -d "$LINK_OR_DIR" ]; then 
  if [ -L "$LINK_OR_DIR" ]; then
    # It is a symlink!
    # Symbolic link specific commands go here.
    rm "$LINK_OR_DIR"
  else
    # It's a directory!
    # Directory command goes here.
    rmdir "$LINK_OR_DIR"
  fi
fi
\end{verbatim}

\subsection{Get the creation time}

We use \verb!stat! command
\begin{verbatim}
stat -c %y filename
\end{verbatim}

\subsection{Get the folder with latest update}

Suppose that you have multiple folders, and you want to return the folder that
has the lastest change from its content, then you should do

\begin{verbatim}
NOTE: Here we only check folder 
ls Atrial_GPU_rogueRyR_0.* -d | xargs -I{} find {} -printf "%T+ %p\n" | sort -nr | head -n 1



NOTE: This is better as we check all the files

find Atrial_GPU_rogueRyR_0.* -type f 
    | xargs -I{} find {} -printf "%T+ %p\n" | sort -nr | head -n 1
    
    
    
NOTE: If we want full filepath, we use

find Atrial_GPU_rogueRyR_0.* -type f | xargs -I{} readlink -f {} |
    | xargs -I{} find {} -printf "%T+ %p\n" | sort -nr | head -n 1
          
\end{verbatim}
The first part list all folders you want to check, and then using 
\verb!xargs! to pass each line (containing the folder name) to the next
commands. The command \verb!find {}! find all the files/folders inside this
folder, and then extract the associated properties of the files/folders. 
The command \verb!-printf! can extract many useful information from
the file (Sect.\ref{sec:extract_information_printf}). What we need here is the
date of latest change (\verb!%T+!) and the full path (\verb!%p!).

\subsection{Get file/folder associated properties (find -printf)}
\label{sec:extract_information_printf}


There are many informations associated with a file/folder (e.g. creation date,
date of latest change, fullpath, \ldots), then we need to use the \verb!-printf!
option of the \verb!find! command

\begin{verbatim}
find <filename> -printf "%T+ %p %f \n"
\end{verbatim}
We use \verb!-printf! to print out the information we extract (NOTE: It
interpret \verb!\! as escape and \verb!%! as directives). The precision is used
the same  as \verb!printf! C function.
Below are some
directives and escape characters:

\begin{itemize}
  \item \verb!%c!: the creation time
  \item \verb!%T+!: the time of last modified (\verb!%Tk! with $k$ is the
   format)
  \item \verb!%f!: the filename
  \item \verb!%p!: the fullpath
  \item \verb!%F!: display filesystem (e.g. nfs, fat32)
  \item \verb!%s!: display filesize
  \item \verb!\n!: add NEWLINE to the end
\end{itemize}
More information: \url{http://linux.about.com/od/commands/l/blcmdl1_find.htm}

\subsection{Rename multiple files}

We use \verb!rename! command.
\begin{verbatim}
rename 's/ABC/XYZ/' *.dat
\end{verbatim}

Pad a number in filename to a fixed length
\begin{verbatim}
file.1.png
file.2.png
...
file.10.png
...
file.1000.png
file.1001.png
\end{verbatim}
to
\begin{verbatim}
file.0001.png
file.0002.png
...
file.0010.png
...
file.1000.png
file.1001.png
\end{verbatim}
We do
\begin{enumerate}
  
  \item The basic thing
\begin{verbatim}
rename 'file.' 'file.000' file.?.png
rename 'file.' 'file.00' file.??.png
rename 'file.' 'file.0' file.???.png
\end{verbatim}

  \item Check for existence before rename file as well 
\begin{verbatim}
for f in *.png; do
    int=`basename $f .png | cut -d '.' -f 2`
    new_name=`printf "file.%0.4i.png\n" $int`
    [ ! -f $new_name ] && mv $f $new_name
done
\end{verbatim}

  \item No check for file existence
\begin{verbatim}
for f in *.png
do
   foo='0000'${f#file.} #remove "file." and prepend 0000
   foo=${foo:(-8)} #last eight characters
   echo mv $f file.$foo   
done
\end{verbatim}

  \item A simple trick is start counting at 10001 
\begin{verbatim}
for x in file.*.png; do
  n=${x%.*}; n=${n##*.}
  n=$((n + 10000))
  mv "$x" "${x%.*.*}.${n#1}.${x##*.}"
done
\end{verbatim}

 \item In zsh, we can use \verb!zmv!
\begin{verbatim}
setopt extended_glob
autoload zmv
zmv '(file).([0-9]##).(png)' '$1.${(l:4::0:)2}.$3'
\end{verbatim}

  \item User Perl
\begin{verbatim}
ls -1 file.*.png | perl -ne
'chomp;($n=$_)=~s/(\d+)/sprintf"%04d",$1/ge;rename$_,$n'
\end{verbatim}

More complicate is to avoid renaming file if there is existing one
\begin{verbatim}
ls -1 | perl -ne 'chomp;($n=$_)=~s/(\d+)/sprintf"%04d",$1/ge;
  rename$_,$n if$_% ne$n&&!-f$n'
\end{verbatim}

Or we can avoid using loop, and substitute with \verb!sed! command
\begin{verbatim}
ls -1 file.*.png | sed -r
'/[0-9]{4}/d;h;:p;/[0-9]{4}/!s/([0-9]{1,3})/0\1/;tp;G;s/(.+)\n(.+)/mv \2 \1/' | sh
\end{verbatim}
\end{enumerate}
\url{http://unix.stackexchange.com/questions/21425/padding-a-number-in-a-filename-to-a-fixed-length}
\subsection{Extract column from a file}

Here we work with individual characters
\begin{enumerate}
  \item Extract a single character at a given column, at each line
\begin{verbatim}
cut -c2 file.txt //column 2nd
\end{verbatim}
   \item Extract a number of characters in a column-range, at each line
\begin{verbatim}
cut -c2-5 file.txt
\end{verbatim}   

If column 5-th to end of line
\begin{verbatim}
cut -c5- file.txt
\end{verbatim}

If begining of line to column 5-th
\begin{verbatim}
cut -c-5 file.txt
\end{verbatim} 
\end{enumerate}

Here, we work with fields, we combine option \verb!-f! (field index) and
\verb!-d! (field delimiter).
\begin{enumerate}
   \item Extract a specific field
\begin{verbatim}
cut -d':' -f1 file.txt
\end{verbatim}   

   Extract fields (1st-6th) from lines containing ``/bin/bash''
\begin{verbatim}
grep "/bin/bash" /etc/passwd | cut -d':' -f1,6
\end{verbatim}   

The problem is that if the line doesn't contain the delimiter, then it just
print out the whole line. To avoid that, we need to pass \verb!-s! option (to
filter out only lines containing the delimiter)
\begin{verbatim}
grep "/bin/bash" /etc/passwd | cut -d'|' -s -f1
\end{verbatim}

   \item Extract the compliment field, e.g. extract all fields except some of
   given fields, we use \verb!--complement! option
\begin{verbatim}
grep "/bin/bash" /etc/passwd | cut -d':' --complement -s -f7
\end{verbatim}   


   \item If we extract multiple fields, the delimiter is still the same as the
   input. However, if we want to use a different delimiter for the output, we
   use \verb!--output-delimiter! option
\begin{verbatim}
grep "/bin/bash" /etc/passwd | cut -d':'  -s -f1,6,7 --output-delimiter='#'

//use newline
grep bala /etc/passwd | cut -d':' -f1,6,7 --output-delimiter=$'\n'
\end{verbatim}   

\end{enumerate}

Example:
\begin{verbatim}
ps axu | grep python | sed 's/\s\+/ /g' | cut -d' ' -f2,11-
\end{verbatim}


NOTE: If you want to use space as delimiter, and there can be multiple
whitespaces between field, then we can squeeze them by removing duplicated
whitespace first using \verb!tr! command
\begin{verbatim}
ps | egrep 11383 | tr -s ' ' | cut -d ' ' -f 4
\end{verbatim}


Other than using \verb!cut! command, we can use \verb!awk! command
\begin{verbatim}
echo "11383 pts/1    00:00:00 bash" | awk '{ print $4; }'
//print 'bash' out

var=`awk '{getline;print $1;exit}' file`

vari=`awk -F"," 'NR==2{print $1}' filename` 
\end{verbatim}

If we check \verb!ps! command
\begin{verbatim}
  PID TTY          TIME CMD
11383 pts/1    00:00:00 bash
11771 pts/1    00:00:00 ps
\end{verbatim}

We want to extract the command line of a given PID
\begin{verbatim}
ps | awk "\$1==$PID{print\$4}"

ps | awk -v"PID=$PID" '$1=PID{print$4}'

readlink /proc/$PID/exe

xargs -0n1 </proc/$PID/cmdline | head -n1

set $(ps | egrep "^11383 "); echo $4

A=( $(ps | egrep "^11383 ") ) ; echo ${A[3]}

ps |&
while read -p first second third fourth etc ; do
   if [[ $first == '11383' ]]
   then
       echo got: $fourth
   fi       
done
\end{verbatim}

\subsection{Extract a line}
\label{sec:extract_line}

We can extract the first few lines using \verb!head -n!

\begin{verbatim}
line1=`head -n 1`
section=`head -n 10` #10 lines
\end{verbatim}

We can extract the last few lines using \verb!tail -n!

\begin{verbatim}
line1=`tail -n 1`
\end{verbatim}

Extract lines within a range of row indices, e.g. from line number 14 to 25.
Here 1-indexed based system
\begin{verbatim}
;; it scans still the end of file, which is not good for VERY large file
sed -n 14,25p filename > output

;; work BETTER for VERY large file: add a quit command on the next line
sed -n '14,25p;26q' filename > output
\end{verbatim}
\verb!-n! option to suppresses echoing the input as output. The command \verb!p!
prints out the relevant lines. We can also use
\begin{verbatim}
awk 'NR>=14&&NR<=25' filename > output
\end{verbatim}
with \verb!NR! means row index.

Extract lines matching a pattern, e.g. pattern is 'GAIN'
\begin{verbatim}
grep 'GAIN' filename > output
\end{verbatim}

\subsection{Extract columns}

After extracting lines (Sect.\ref{sec:extract_line}), you can extract the column
(1st, 2nd, etc.). The default delimiter is space \verb!' '!
\begin{verbatim}
;; <tab> is the delimiter
awk -F"\t" '{if ($1) print $1}' input.txt

;; extract the 
ls -l | awk '{ print $9; }' 
\end{verbatim}

\subsection{Size of a string/file}
\label{sec:size_in_bytes}


To know the size in bytes of a given string, or a few lines in a file, we can
use  \verb!wc -c!
\begin{verbatim}
tail -n 2 file1 | wc -c

\end{verbatim}

NOTE: To use word count, \verb!wc -w!. To use line count \verb!wc -l!. To detect
the length of the longest line among the lines \verb!wc -L!.

\subsection{Sort based on values in columns}

The \verb!sort! command can sort any field (white-space is the delimiter by
default, we can use a different one by specifying \verb!-t! option). By default
it use string comparison, but we can specify number comparison with \verb!-n!
option. Example: a file, each line with 8 fields, with semi-colon (;) as
delimiter, we want to sort based on first 4 fields (using number comparison) in
the order first-field then second then third then fourth
\begin{verbatim}
sort -t\; -k 1,1n -k 2,2n -k 3,3n -k 4,4n test.txt
\end{verbatim}
The option \verb!-k=POS1[,POS2]! defines the starting and end position. NOTE:
The semi-colon need to be escaped, i.e. \verb!\;! to avoid end-of-line
interpration by the shell.

Example: sort the first field, then sort on the fourth field (use
\verb!-s! (or --stable) to tell that ties on the fourth field are determined by
the initial position)
\begin{verbatim}
sort -k1,1 bigfile | sort --stable -n -k4,4
\end{verbatim}

\begin{framed}
NOTE: \verb!sort! doesn't need to put everything into memory, but Perl needs to
put the data into memory. So, using \verb!sort! is faster for very large input
files.
\end{framed}

Sort the last field
\begin{enumerate}
  \item  We can use Perl, just pipe the list of lines into this.
\begin{verbatim}
perl -e "print sort {(split '/', $a)[-1] <=> (split '/', $b)[-1]} <>"
\end{verbatim}
  
  \item AWK
\begin{verbatim}
awk '{print $NF"|"$0}' file | sort -t"|" -k1 | awk -F"|" '{print $NF }'
\end{verbatim}

  \item We can reverse the order of the last and first field, sort the 'new'
  first field, and then reverse the first and last field again
\begin{verbatim}
//reverse first and last field
perl -lne 'print join " ", reverse split / /'
\end{verbatim}
We can use \verb!/ +/! instead of \verb!/ /! so it squeezes spaces

  \item copy the last field to the beginging, separated with whitespace
  (default), sort with the new first field, and then remove this first field
\begin{verbatim}
awk '{print $NF,$0}' file | sort | cut -f2- -d' '
\end{verbatim}
or
\begin{verbatim}
find "$@" -maxdepth 1 -name "[0-9]*" -printf '%f/%p\n' | sort -t / | 
       cut -f2-% -d/
\end{verbatim}

  \item Use Ruby
\begin{verbatim}
#!/usr/bin/ruby

f = ARGF.read
lines = f.lines

broken = lines.map {|l| l.split(/:/) }

sorted = broken.sort {|a, b|
    a[-1] <=> b[-1]
}

fixed = sorted.map {|s| s.join(":") }

puts fixed
\end{verbatim}
\end{enumerate}

\subsection{String with special characters}

We can print out the string that contain special characters, and can recognize
them
\begin{verbatim}
LINE1="abc"
LINE2="cdf"
LINE3="fdg"

MYSTRING = ${LINE1}'\n'${LINE2}'\n'${LINE3}

echo -e ${MYSTRING}
\end{verbatim}
The special characters that can be recognized are
\begin{itemize}
  \item \verb!\ONNN! the character whose ASCII code is NNN (octal)
  \item \verb!\\!  backslash
  \item \verb!\a! alert
  \item \verb!\b! backspace
  \item \verb!\c! suppress trailing newline
  \item \verb!\n! newline  (move the cursor down - LF)
  \item \verb!\r! carriage return  (move the cursor to the begining of the line
  - CR)
  
NOTE: move the cursor down/to the left is the mechanism of printer in text mode.
Most printers (PCL or Postscripts) still respect CR and LF. 
In Java, best practice, to use native line ending, it is suggested to call
\verb!System.getProperty("line.separator")! which is OS-independent.

  \item \verb!\t! horizontal tab
  \item \verb!\v! vertical tab
\end{itemize}

\subsection{Symbolic link}

\begin{enumerate}
  \item \verb!readlink <arg>! return the names of the symbolic link linking to <arg>
  \item \verb!-L! to test if a file is symbolic link 
\end{enumerate}

\begin{verbatim}
if [[ -L a.c ]] && [[ "$(readlink a.c)" = "b.c" ]] ; then 
 echo "a.c is a link to b.c";
fi
\end{verbatim}

\subsection{Delete}

If you want to delete all, except one or a few files, you do
\begin{verbatim}
ls * | grep -v file_not_delete | xargs rm -rf

// or enable external feature (bash)
shopt -s extglob
rm !(f1)

// in ksh
rm !(f1)

// in zsh
rm ^f1
\end{verbatim}
with \verb!f1! is the name of file NOT to delete.

\subsection{Replace/substitute characters in a file/string}

To do character substitution, we can either use \verb!tr! or \verb!sed!
(Sect.\ref{sec:sed}) utilitites.
\begin{verbatim}
// replace '+' with white-space
echo This+is+test+for+tr+and+sed |tr '+' ' '

echo This+is+test+for+tr+and+sed |sed 's/\+/ /g'
\end{verbatim}

\verb!tr! utility translate/squeeze/delete characters from input, i.e.
character-based transformation.
\verb!sed! is a stream editor that performa basic text transformations on the
input stream, i.e. string-based transformation.

\begin{verbatim}
// if you put 2 strings (of the same length)
// tr will do one-by-one substitution
// e.g. g -> t, o -> e, o->s, d->t

echo I am a good boy | tr 'good' 'test'

I am a tsst bsy
\end{verbatim}



\subsection{Remove the last few lines in an EXTREMELY LARGE file}

SIMPLEST: We just extract the first (N-2) lines with N is the
total number of lines and redirect to another file
\begin{verbatim}
head -n -2 file1 > file2
\end{verbatim}
However, if the file size is EXTREMELY LARGE, e.g. can be 400 GB, the entire
file to be fed through a pipe and rewriting all data to a new location.

FATEST AND NEED ASSUMPTION: You need to know how many bytes to be removed
(Sect.\ref{sec:size_in_bytes}), given the number of characters in the last 2
lines, for example 160 bytes, then this is the fastest as it modifies the file in-place
\begin{verbatim}
truncate --size=-160 file1
\end{verbatim}
NOTE: \verb!truncate! shrinks or extends the file size to the specified size. By
default (unit is byte), but you can use suffixes: KB (1000), K (1024), MB
(1000*1000), M (1024*1024), and similarly for G, T, P, E, Z, Y. The prefixes:
(-) reduce by, (+) extend by, (<) at most, (>) at least, (/) round down to
multiple of, (\%) round up to multiple of. 

BETTER: A Python code can do it
\begin{verbatim}
./shorten.py 2 large_file.txt
\end{verbatim}
\url{http://superuser.com/questions/127786/how-to-remove-the-last-2-lines-of-a-very-large-file}

The code:\textcolor{red}{Python 2.4}
\begin{verbatim}
#!/usr/bin/env python2.4

import sys

if len(sys.argv) != 3:
    print sys.argv[0] + ": Invalid number of arguments."
    print "Usage: " + sys.argv[0] + " linecount filename"
    print "to remove linecount lines from the end of the file"
    sys.exit(2)

number = int(sys.argv[1])
file = sys.argv[2]
count = 0
SEEK_CUR = 1
SEEK_END = 2

f = open(file,'r+b')
f.seek(0, SEEK_END)
end = f.tell()

while f.tell() > 0:
    f.seek(-1, SEEK_CUR)
    char = f.read(1)
    if char != '\n' and f.tell() == end:
        print "No change: file does not end with a newline"
        f.close()
        sys.exit(1)
    if char == '\n':
        count += 1
    if count == number + 1:
        f.truncate()
        print "Removed " + str(number) + " lines from end of file"
        f.close()
        sys.exit(0)
    f.seek(-1, SEEK_CUR)

if count < number + 1:
    print "No change: requested removal would leave empty file"
    f.close()
    sys.exit(3)
\end{verbatim}


The code:\textcolor{red}{Python 2.5/2.6}
\begin{verbatim}
#!/usr/bin/env python2.5
from __future__ import with_statement
# also tested with Python 2.6

import os, sys

if len(sys.argv) != 3:
    print sys.argv[0] + ": Invalid number of arguments."
    print "Usage: " + sys.argv[0] + " linecount filename"
    print "to remove linecount lines from the end of the file"
    exit(2)

number = int(sys.argv[1])
file = sys.argv[2]
count = 0

with open(file,'r+b') as f:
    f.seek(0, os.SEEK_END)
    end = f.tell()
    while f.tell() > 0:
        f.seek(-1, os.SEEK_CUR)
        char = f.read(1)
        if char != '\n' and f.tell() == end:
            print "No change: file does not end with a newline"
            exit(1)
        if char == '\n':
            count += 1
        if count == number + 1:
            f.truncate()
            print "Removed " + str(number) + " lines from end of file"
            exit(0)
        f.seek(-1, os.SEEK_CUR)

if count < number + 1:
    print "No change: requested removal would leave empty file"
    exit(3)
\end{verbatim}

\textcolor{red}{Python 3.0}
\begin{verbatim}
#!/usr/bin/env python3.0

import os, sys

if len(sys.argv) != 3:
    print(sys.argv[0] + ": Invalid number of arguments.")
    print ("Usage: " + sys.argv[0] + " linecount filename")
    print ("to remove linecount lines from the end of the file")
    exit(2)

number = int(sys.argv[1])
file = sys.argv[2]
count = 0

with open(file,'r+b', buffering=0) as f:
    f.seek(0, os.SEEK_END)
    end = f.tell()
    while f.tell() > 0:
        f.seek(-1, os.SEEK_CUR)
        print(f.tell())
        char = f.read(1)
        if char != b'\n' and f.tell() == end:
            print ("No change: file does not end with a newline")
            exit(1)
        if char == b'\n':
            count += 1
        if count == number + 1:
            f.truncate()
            print ("Removed " + str(number) + " lines from end of file")
            exit(0)
        f.seek(-1, os.SEEK_CUR)

if count < number + 1:
    print("No change: requested removal would leave empty file")
    exit(3)
\end{verbatim}

\subsection{Look for 'n' largest files}

Here we look for the first 200 largest files, with size over 1,000,000 bytes
\begin{verbatim}
find /usr/pcapps/ -mount -type f -size +1000000c | perl -lpe ' s{ }{\\ }g ' |
   xargs ls -l | sort +4nr | head -200
\end{verbatim}
NOTE: \verb!head -200! is first 200 files, and \verb!-size +1000000c! is the
minimum size.

\subsection{List folders only}

\begin{verbatim}
ls -al | grep '^d'
\end{verbatim}
The above method display in the following format and includes both the
\verb!.! and \verb!..! folder
\begin{verbatim}
drwxr-xr-x 24 hadoop hadoop   4096 Dec  5 13:19 .
drwxr-xr-x 31 hadoop hadoop   4096 Dec  5 13:16 ..
drwxr-xr-x  4 hadoop hadoop   4096 Nov 14 15:58 bin
drwxr-xr-x  2 hadoop hadoop   4096 Nov 14 15:58 conf
drwxr-xr-x  4 hadoop hadoop   4096 Nov 14 15:58 dev-support
drwxr-xr-x  4 hadoop hadoop   4096 Dec  3 16:52 hbase-annotations
drwxr-xr-x  3 hadoop hadoop   4096 Dec  5 13:19 hbase-assembly
drwxr-xr-x  4 hadoop hadoop   4096 Dec  3 16:52 hbase-checkstyle
drwxr-xr-x  4 hadoop hadoop   4096 Dec  3 16:52 hbase-client
drwxr-xr-x  4 hadoop hadoop   4096 Dec  3 16:52 hbase-common
drwxr-xr-x  4 hadoop hadoop   4096 Dec  3 16:41 hbase-examples
drwxr-xr-x  3 hadoop hadoop   4096 Nov 14 19:00 hbase-hadoop1-compat
drwxr-xr-x  4 hadoop hadoop   4096 Dec  3 16:52 hbase-hadoop2-compat
drwxr-xr-x  4 hadoop hadoop   4096 Dec  3 16:52 hbase-hadoop-compat
drwxr-xr-x  4 hadoop hadoop   4096 Dec  4 14:23 hbase-it
drwxr-xr-x  4 hadoop hadoop   4096 Dec  3 16:52 hbase-prefix-tree
drwxr-xr-x  4 hadoop hadoop   4096 Dec  3 16:52 hbase-protocol
drwxr-xr-x  4 hadoop hadoop   4096 Dec  3 16:41 hbase-rest
drwxr-xr-x  4 hadoop hadoop   4096 Dec  5 10:43 hbase-server
drwxr-xr-x  4 hadoop hadoop   4096 Dec  3 16:52 hbase-shell
drwxr-xr-x  3 hadoop hadoop   4096 Dec  3 16:52 hbase-testing-util
drwxr-xr-x  4 hadoop hadoop   4096 Dec  3 16:41 hbase-thrift
drwxr-xr-x  3 hadoop hadoop   4096 Dec  2 20:05 src
drwxr-xr-x  3 root   root     4096 Dec  3 16:54 target
\end{verbatim}
To extract only the folder name, use column-extraction. 

\begin{verbatim}
ls -d */
\end{verbatim}
The above method displays only folders (no \verb!.! and \verb!..!) but ends with
a slash
\begin{verbatim}
bin/
conf/
dev-support/
hbase-annotations/
hbase-assembly/
hbase-checkstyle/
hbase-client/
hbase-common/
hbase-examples/
hbase-hadoop1-compat/
hbase-hadoop2-compat/
hbase-hadoop-compat/
hbase-it/
hbase-prefix-tree/
hbase-protocol/
hbase-rest/
hbase-server/
hbase-shell/
hbase-testing-util/
hbase-thrift/
src/
target/
\end{verbatim}


\begin{verbatim}
 find . -maxdepth 0 -type d
\end{verbatim}
The above method returns include the \verb!.! folder and the folder has
\verb!./! prefix
\begin{verbatim}
bin
conf
dev-support
hbase-annotations
hbase-assembly
hbase-checkstyle
hbase-client
hbase-common
hbase-examples
hbase-hadoop1-compat
hbase-hadoop2-compat
hbase-hadoop-compat
hbase-it
hbase-prefix-tree
hbase-protocol
hbase-rest
hbase-server
hbase-shell
hbase-testing-util
hbase-thrift
src
target
\end{verbatim}

Another option
\begin{verbatim}
find . -maxdepth 1 -type d -name [^\.]\* | sed 's:^\./::'
\end{verbatim}
which also only returns the folder name.

\url{http://stackoverflow.com/questions/3667329/listing-only-directories-in-unix}

\section{Common statements}
\label{sec:common-statements}

\begin{itemize}
\item \verb!clear! : clear the screen
\item \verb!echo "This string"!: print out the given string, we can
  use C format style in which we can get the result from a shell statement
\begin{verbatim}
echo "Today is \c"; date
echo "Number of login users: \c"; who | wc -l
echo "Calendar"
cal
\end{verbatim}
We can also print out system variable
\begin{verbatim}
echo $HOME
\end{verbatim}

There are some options we can use with \verb!echo!\footnote{\url{http://www.freeos.com/guides/lsst/ch02sec06.html}}
\begin{verbatim}
-n Do not output the trailing new line.
-e Enable interpretation of the following backslash escaped characters in the strings:
\a alert (bell)
\b backspace
\c suppress trailing new line
\n new line
\r carriage return
\t horizontal tab
\\ backslash
\end{verbatim}

E.g.:
\begin{verbatim}
echo -e "An apple a day keeps away \a\t\tdoctor\n"
\end{verbatim}



\item \verb!exit 0!: End of the script


\end{itemize}

\section{Colors in text output}


\begin{verbatim}
#!/bin/bash

# Escape code
esc=`echo -en "\033"`

# Set colors
cc_red="${esc}[0;31m"
cc_green="${esc}[0;32m"
cc_yellow="${esc}[0;33m"
cc_blue="${esc}[0;34m"
cc_normal=`echo -en "${esc}[m\017"`

echo "Here's ${cc_green}some green text${cc_normal} for you."
echo "Here's ${cc_blue}some green text${cc_normal} for you."
\end{verbatim}

The color codes (0;31 etc) are like those used in \verb!dir_colors!:

\begin{verbatim}
# Attribute codes:
# 00=none 01=bold 04=underscore 05=blink 07=reverse 08=concealed
# Text color codes:
# 30=black 31=red 32=green 33=yellow 34=blue 35=magenta 36=cyan 37=white
# Background color codes:
# 40=black 41=red 42=green 43=yellow 44=blue 45=magenta 46=cyan 47=white

\end{verbatim}


\section{xargs}
\label{sec:xargs}

What \verb!xargs! does is to extract each line from the input and pass it to the
command. By default, it's assumed that each line has one field.
\begin{verbatim}
ls | xargs rm
\end{verbatim}

If you want to explicitly control the field name because you want to wrap it
into a double quotes or you may have to deal with more than one field, then you
use \verb!-I! and field name, e.g. \verb!{}! as the field name
\begin{verbatim}
ls | xargs -I{} cp "{}" ~/foo/bar
ls xargs -0 -I{} cp "{}" ~/foo/bar
\end{verbatim}
NOTE: In BSD, by default, quotes are special to xargs unless you use \verb!-0!

If we have \verb!find! command, we can combine with \verb!exec!
\begin{verbatim}
find . -iname "*foobar*" -exec cp "{}" ~/foo/bar \;
\end{verbatim}
NOTE: The solution doesn't work if a filename begin with \verb!-! (a hyphen).


\section{XDG}
\label{sec:XDG}

freedesktop.org was created to help  designing cross-platform APIs to ease the
portability of application software between desktop environments and kernels. It
was first called {\bf Portland Project} (as hosted by Portland State
University).  The project goal is to let software developers worry less about
the desktop environment a distribution is using.
\begin{itemize}
  \item Portland 1.0 - Sect.\ref{sec:xdg-utils} in 2006
\end{itemize}

freedesktop.org was formerly known as X Desktop Group (XDG).

\subsection{XDG environment variables: XDG Base Directory Specification }
\label{sec:XDG-environment-variables}

{\bf XDG\_DATA\_HOME}: it gives the location relative to which user-specific
data files should be written. E.g.
\begin{verbatim}
      //nvim plugins installed by user
$XDG_DATA_HOME/nvim/site
\end{verbatim}


{\bf XDG\_CONFIG\_HOME}: it gives the location relative to which user-specific
configuration files should be written (default: \verb!$HOME/.config!).

{\bf XDG\_CONFIG\_DIRS[1]}: contain configuration files according to XDG,
including preferences from system administrator (default: \verb!/etc/xdg!)
{\bf XDG\_CONFIG\_DIRS[2]}

{\bf XDG\_DATA\_DIRS}: and we can access one particular folder using
{\bf XDG\_DATA\_DIRS[1]}, {\bf XDG\_DATA\_DIRS[2]} 
\begin{verbatim}
// default: 
//   XDG_DATA_DIRS = /usr/local/share/:/usr/share/
\end{verbatim}


\subsection{xdg-utils (released 2006, Portland 1.0)}
\label{sec:xdg-utils}
\label{sec:Portland-1.0}



xdg-utils is a set of tools (a set of common interfaces for desktop
environments) that allows applications to easily integrate with the desktop
environment of the user, regardless of the specific desktop environment that the
user runs.

\begin{itemize}
  \item xdg-desktop-menu 
  
  \item 
\end{itemize}



%%% Local Variables: 
%%% mode: latex
%%% TeX-master: "fortran_manual"
%%% End: 


