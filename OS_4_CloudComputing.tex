\chapter{Operating Systems for the Cloud Environment}


\section{Ubuntu Cloud}

Ubuntu Enterprise Cloud (UEC) is a software product that you can install on your
own servers to create your own cloud computing environment.

\section{IBM Cloud Private}

IBM Cloud Private is an application platform for developing and managing
on-premises, containerized applications.
It is an integrated environment for managing containers that includes the
container orchestrator Kubernetes, a private image repository, a management
console, and monitoring frameworks.

3.1.1-ce 
\begin{verbatim}
Linux on POWR 64-bit (ppc64le):  https://hub.docker.com/r/ibmcom/icp-inception-ppc64le

Linux 64-bit: https://hub.docker.com/r/ibmcom/icp-inception
\end{verbatim} 

2.1.0.3 ce edition
\begin{verbatim}
Ubuntu 16.04 LTS


Linux on POWER 64-bit Little Endian (LE) - https://hub.docker.com/r/ibmcom/icp-inception-ppc64le

Linux 64-bit - https://hub.docker.com/r/ibmcom/icp-inception/   
\end{verbatim}
\url{https://www.ibm.com/developerworks/community/wikis/home?lang=en#!/wiki/W1559b1be149d_43b0_881e_9783f38faaff/page/Overview of IBM Cloud Private}

\section{Google Container-Optimized OS}

Container-Optimized OS is maintained by Google and is based on the open source
Chromium OS project. Container-Optimized OS is an operating system image for
your Compute Engine VMs that is optimized for running Docker containers.

\url{https://cloud.google.com/container-optimized-os/docs/concepts/features-and-benefits}

Container-Optimized OS is the default node OS Image in Kubernetes Engine and other Kubernetes deployments on Google Cloud Platform


\section{Amazon Linux}
\label{sec:Amazon-Linux}

\url{https://docs.amazonaws.cn/en_us/AWSEC2/latest/UserGuide/amazon-linux-ami-basics.html}


Amazon Linux is a customized Linux provided by Amazon, it comes with 
\begin{itemize}
  \item  a number of packages that enable easy integration with AWS, including launch configuration tools and many popular AWS libraries and tools
  \item  ongoing security and maintenance updates
  
  \item Amazon Linux does not allow remote root SSH by default

By default, the only account that can log in remotely using SSH is ec2-user;
this account also has sudo privileges.

To enable SSH logins to an Amazon Linux instance, you must provide your key pair
to the instance at launch. You must also set the security group used to launch
your instance to allow SSH access.

   \item it uses \verb!/etc/image-id! file
   
\begin{verbatim}
image_name="amzn2-ami-hvm"
image_version="2"
image_arch="x86_64"
image_file="amzn2-ami-hvm-2.0.20180810-x86_64.xfs.gpt"
image_stamp="8008-2abd"
image_date="20180811020321"
recipe_name="amzn2 ami"
recipe_id="c652686a-2415-9819-65fb-4dee-9792-289d-1e2846bd"
\end{verbatim}
\end{itemize}

\begin{verbatim}
>> cat /etc/system-release
Amazon Linux 2

>> cat /etc/os-release
NAME="Amazon Linux"
VERSION="2"
ID="amzn"
ID_LIKE="centos rhel fedora"
VERSION_ID="2"
PRETTY_NAME="Amazon Linux 2"
ANSI_COLOR="0;33"
CPE_NAME="cpe:2.3:o:amazon:amazon_linux:2"
HOME_URL="https://amazonlinux.com/"
\end{verbatim}



Amazon Linux 2 and the Amazon Linux AMI. 

There are also  Amazon Linux Docker container images:
\url{https://hub.docker.com/_/amazonlinux/}

\subsection{Amazon Linux}

Amazon Linux 2018.03 is the last release for the current generation of Amazon
Linux and will be supported until June 30, 2020.

\begin{verbatim}
yum update 
\end{verbatim}
always moves your system to the latest Amazon Linux version.
There were no versions of Amazon Linux available, only snapshots.


Amazon Linux uses SysVinit to bootstrap the Linux user space and to manage
system processes after booting. This procedure is usually called init
(Sect.\ref{sec:init}).


\subsection{Amazon Linux 2 (from Dec, 2017)}

In late December 2017, AWS announced the successor of Amazon Linux: Amazon Linux 2.
It  adds some new capabilities:
\begin{verbatim}

long-term support: Amazon Linux 2 supports each LTS release for five years

on-premises support: virtual machine images for on-premises development and testing are available

systemd: replacing SystemVinit

extras library: provides up-to-date versions of software bundles such as nginx
\end{verbatim}
\url{https://cloudonaut.io/migrating-to-amazon-linux-2/}

The cloud-init package is an open-source application built by Canonical that is
used to bootstrap Linux images in a cloud computing environment, such as Amazon
EC2. Amazon Linux contains a customized version of cloud-init.

Amazon Linux uses the cloud-init actions found in /etc/cloud/cloud.cfg.d and
/etc/cloud/cloud.cfg. You can create your own cloud-init action files in
/etc/cloud/cloud.cfg.d. All files in this directory are read by cloud-init. They
are read in lexical order, and later files overwrite values in earlier files.



Extras Library to install application and software updates on your instances.
These software updates are known as topics.
 
\begin{verbatim}
>> amazon-linux-extras list


>> sudo amazon-linux-extras install <topic>

>> sudo amazon-linux-extras install <topic>=version <topic>=version

// to install source for a package
yumdownloader --source bash


\end{verbatim}

\subsection{Amazon Linux AMI}

Amazon Linux AMI is a supported and maintained Linux image provided by Amazon
Web Services for use on Amazon Elastic Compute Cloud (Amazon EC2). It is
designed to provide a stable, secure, and high performance execution environment
for applications running on Amazon EC2.


The image includes packages that enable easy integration with AWS services, such
as the AWS CLI, Amazon EC2 API and AMI tools, the Boto library for Python, and
the Elastic Load Balancing tools.

\url{https://docs.aws.amazon.com/AWSEC2/latest/UserGuide/AMIs.html#amazon-linux}


\begin{verbatim}
cat /etc/image-id
image_name="amzn-ami-hvm"
image_version="2018.03"
image_arch="x86_64"
image_file="amzn-ami-hvm-2018.03.0.20180811-x86_64.ext4.gpt"
image_stamp="cc81-f2f3"
image_date="20180811012746"
recipe_name="amzn ami"
recipe_id="5b283820-dc60-a7ea-d436-39fa-439f-02ea-5c802dbd"

cat /etc/system-release
Amazon Linux AMI release 2018.03
\end{verbatim}