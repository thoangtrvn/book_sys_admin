\chapter{Security O/S}

\section{Time-sharing}

In early days of computer, {\bf batch processing} was the only method.

In 1960s, {\bf time-sharing processing} was a novel feature that enable (1)
multiple programs to use the CPU, (2) multiple users to use one machine at the
same time. The time-sharing system was first designed for mainframe computer,
and implemented by SDC (System Development Corporation, the world's first
computer software company) for AN/FSQ-32 mainframe computer.
\url{http://en.wikipedia.org/wiki/System_Development_Corporation}

The "state" of each user and their programs would have to be kept in the
machine, and then switched between quickly. The first project to implement this
system took place on a modified IBM 704 (in 1957), and then on a modified IBM
7090 computer. The name of the time-sharing system was called {\bf CTSS}
(Compatible Time-Sharing System) - remained in use until 1973. Another system
developed around the same time was PLATO II at University of Illinois
\url{http://en.wikipedia.org/wiki/PLATO_(computer_system)}.
The first commercially successful (large-scale) time-sharing system was the
Dartmouth Time Sharing System in 1963 - remained in use until 1999.
\url{http://en.wikipedia.org/wiki/Dartmouth_Time_Sharing_System}

Different computer terminals were {\it multiplexed} onto a single large
mainframe computer via techniques:
\begin{itemize}
  \item mainframe computer sequentially polled the terminals to see whether any
  additional data was available or action was requested before switching to the
  next one.
  
  \item {\bf interrupt} driven
  \item parallel data transfer technologies: such as IEEE 488 standard in
  1975.
  
  The (GPIB (General Purpose Interface Bus)) bus was relatively easy to
  implement using the technology at the time, using a simple parallel bus and
  several individual control lines.
  \url{http://en.wikipedia.org/wiki/IEEE-488}
  
\end{itemize}

In the 1960s, several companies started providing time-sharing services as
service bureaus.

\section{Time-sharing vs. Security}

Time-sharing was the first time that multiple processes, owned by different
users, were running on a single machine, and these processes could interfere
with one another.

This raised new concerns about system security, when a system's resource is made
available to multiple users via the remote connection. The phrase {\bf
"penetration"} was used (at Spring 1967 Joint Computer Conference) to describe
an attack against a computer system.

{\it Tiger team} - a group backed by government - was created to test and
detect a system's security holes to patch it. The leading computer penetration
experts: James P. Anderson, \ldots

The skillset of a penetratio tester is certified by Information Assurance
Certification Review Board (IACRB) which issues Certified Penetration Tester
(CPT).
The CPT requires that the exam candidate pass a traditional multiple choice
 exam, as well as pass a practical exam that requires the candidate to perform a
 penetration test against servers in a virtual machine environment.

\section{O/S for penetration testing}


{\bf BackTrack} provided users with easy access to a comprehensive and large
collection of security-related tools ranging from port scanners to Security Audit
It was released from Feb-2006 to Aug-2012. BackTrack was replaced by {\bf Kali
Linux}.
\url{http://en.wikipedia.org/wiki/BackTrack}

{\bf Kali Linux} is Debian (Wheezy)-derived Linux distribution, a rewrite of
BackTrack O/S by its creators. Kali Linux is developed in a secure location with only a
small number of trusted people that are allowed to commit packages, with each package being
signed by the developer. Kali also has a custom built kernel that is patched for
injection.
\url{http://en.wikipedia.org/wiki/Kali_Linux}


{\bf Pentoo} based on Gentoo Linux and 

{\bf WHAX} based on Slackware Linux

here are many other specialized operating systems for penetration testing, each
more or less dedicated to a specific field of penetration testing.

\url{http://en.wikipedia.org/wiki/Penetration_test}


\section{Kali Linux}

POSTGRESQL is used to store te database. 
{\bf MetaSploit Framework} (penetration testing software) is integrated into the
ditro.
  
  \url{http://lifeofpentester.blogspot.com/2013/03/kali-linux-complete-review-by-pen.html}
  
\subsection{msfconfole}