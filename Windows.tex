\chapter{Windows Administration}

\section{Domain, workgroup, homegroup}


They are used to describe different ways of organizing computers in networks. 
\begin{itemize}
  \item homegroup: machines on home networks
  \item workgroup: machines on home networks (each machine has a set of local
  user accounts, to be able to login that machine, you must have access to the
  local account).  IMPORTANTLY: (1)   all machines are peer, i.e. no computer
  has control over another one; (2) the total machines is typically no more than
  12; (3) all machines in the same workgrup must be on the same subnet; (4) no
  password protected.
  
  \item domain: machines on workplace networks. IMPORTANT: (1) one or more
  computers function as a server (run by network administrator), (2) the user
  account is controlled on a server thus can be used to login in any machine in
  the domain without needing a local account on that machine, (3) thousands of
  machines in a domain, (4) the computers can be on different local networks.
\end{itemize}
NOTE: homegroup are not available in Windows Server 2008 R2.

\section{Wireless}

\subsection{WPA Enterprise}

\begin{enumerate}
  \item Control Panel, Network and Internet, Manage Wireless Network
  \item Add: type in the wireless network SSID (case-sensitive)
  \item Security: WPA2 Entperise
  \item Encryption: AES or TKIP (most likely AES)
  \item Check: Start this connection automatically
  \item Change Connection Setting
  \begin{itemize}
    \item Click Security tab
    \item Network authentication method: Microsoft Protected EAP (PEAP)
    \item Setting: uncheck Validate Server certificate
    \item Select authentication method: Secure Password (EAP-MSCHAP
    v2), click Configure \ldots (uncheck Automatically use my Windows logon name
    and password)
    \item Check/Uncheck my credentials for this connection each time I'm logged
    on
    \item 
  \end{itemize}
\end{enumerate}


\section{VC2008}	


A default CRT/MFC applications compiled with Windows Visual C/C++ 2008
(MSVC2008) on one machine may not be able to run on other machines. The error
can be
\begin{verbatim}
"This application has failed to start because the application configuration is
incorrect".
\end{verbatim}

The reason is that, by default, CRT/MFC applications use DLL files which are not
pre-installed by Windows. There are three solutions
\begin{enumerate}
  \item Statistically links to CRT/MFC
  \item Use vcredits.exe or \verb!vcredist_x64.exe! to install the DLL files
  \item Deploy these DLL with your applications (in the same directory with the
  subfolder \verb!Microsoft.VC90.CRT!)
\end{enumerate}
\url{http://blog.kalmbach-software.de/2009/05/27/deployment-of-vc2008-apps-without-installing-anything/}

%Microsoft.VC90.CRT 9.0.21022.8
%Microsoft.VC90.CRT 9.0.30729.17
%Microsoft.VC90.CRT 9.0.30729.6161
%Microsoft.VC90.CRT 9.0.21022.218
%Microsoft.VC90.CRT 9.0.30729.6161


\section{Editors}

\url{http://stackoverflow.com/questions/159521/text-editor-to-open-big-giant-huge-large-text-files}

\begin{enumerate}
  \item gVIM
  \item 010Editor: read binary
\end{enumerate}

\section[DEP]{DEP (Data Execution Protection)}
\label{sec:DEP}


DEP was added to Windows since Windows XP SP2. DEP can be software-based and
hardware-based. DEP was developed to prevent {\bf buffer-overflow} attacks.	

\begin{framed}
Buffer-overflow attacks happen when a program write N bytes to a buffer of size
n bytes ($N>n$). This violates memory safety, i.e. it corrupts data in the
adjacent memory addresses due to insufficient bounds checking. Languages like
C/C++ doesn't have this built-in protection. 
\end{framed}


\section{ASLR}
\label{sec:ASLR}

ASLR (address space layout randomization) was developed for 64-bit OS to
implement memory space protection. For 32-bit systems, ASLR is not effective as
there are only 16 bits available for randomization, which can be defeated by
brute-force in a few minutes.


\section{Remote Desktop Connection (Client) or Remote Desktop Services (Server)}
\label{sec:remote-dekstop-connection}

From RDP 6.0, Network Level Authentication was introduced (initially Windows
Vista). 


\subsection{Windows 7 to Ubuntu 12.04}
\label{sec:RDP-Windows-to-Ubuntu}

It was almost impossible in earlier version of Windows and Ubuntu. Since Ubuntu
12.04, it was made possible with \verb!xRDP! implementation of RDP protocol.

Thanks to xRDP server, we can use Microsoft RDB to connect to Ubuntu without any
configuration. 

\begin{enumerate}
  \item On the Ubuntu machine, install xrdp package
  \begin{verbatim}
  sudo apt-get install xrdp
  \end{verbatim}
  \item Create a new user only for RDP connectivity
  \item Restart Ubuntu
  \item Run Remote Desktop Connect from Windows, connect to Ubuntu with the IP
  or hostname
\end{enumerate}

Troubleshooting:
\begin{itemize}
  \item To disable remote connection to Ubuntu machine, on Ubuntu run
  \begin{verbatim}
  sudo service xrdp stop
  \end{verbatim}
  
  \item Login to the user for RDP connectivity, run this and restart Ubuntu
  \begin{verbatim}
  echo "gnome-session --session=ubuntu-2d" > ~/.xsession
  \end{verbatim}
  or run
  \begin{verbatim}
  echo "gnome-session --session=gnome-fallback" > ~/.xsession
  \end{verbatim}
  
  \item Or install and restart Ubuntu
  \begin{verbatim}
  sudo apt-get install gnome-session-fallback 
  \end{verbatim}
\end{itemize}

\subsection{Ubuntu to Windows}
\label{sec:RDP-Ubuntu-to-Windows}

On Windows, we need to disable firewall rule about RDP
\begin{enumerate}
  \item open Windows Firewalls with Advanced Security
  
  \item On both Inbound Rules and Outbound Rules, enable these
\begin{verbatim}
Remote Desktop (TCP-in)
Remote Desktop (UDP-in)
Remote Desktop - RemoteFX (TCP-in)
Remote Desktop - RemoteFX  (UDP-in) 
\end{verbatim}
\end{enumerate}

\section{Softwares}

\begin{enumerate}
  \item Update graphics driver
  \item Install CUDA toolkit (different versions) + Nsight
  \item Install Visual Studio (different versions)
  \item Install TexWorks
  \item Install Eclipse
  \item Install Cygwin (Sect.\ref{sec:Cygwin})
  \item Install Kdiff3 (file content comparisons)
  \item Install Unlocker (to detect process using the file/folder):
  \url{http://www.uniblue.com/cm/emptyloop/registrybooster/website/download/?aff=10080}
  \item Install Explorer++ \url{https://explorerplusplus.com/}
  \item Install ProcMon:
  \url{http://technet.microsoft.com/en-us/sysinternals/bb896645}
  
  
 \item Install Dependency Walker: to
 detect errors relating to loading library
 \url{http://www.dependencywalker.com/}
 
 \item Install Windows Powershell (if you don't use Cygwin)
 \url{http://support.microsoft.com/kb/968930/en-us},
 \url{http://www.microsoft.com/en-us/download/details.aspx?id=42554}
 
 \item Install NetDrive (map Google Drive, FTP link, \ldots) as a local drive
 \url{http://www.netdrive.net/}

\item Install 7zip 
 \item Install Find and Run Robot
 \item Install DeltaCopy : (rsync backup in Windows) - Sect.\ref{sec:DeltaCopy}
 \item Install WinHTTrack: website copy/download
  
 \item Merge multiple photos as a single one: Picture Merger Genius software
 \item Install NetDrive (map Google Drive, FTP link, \ldots) as a local drive
 \url{http://www.netdrive.net/}
\end{enumerate}

\subsection{DeltaCopy}
\label{sec:DeltaCopy}

There are two components: server component running on the machine where data will be copied, client component running on 
the machine where data need to be backed up.

On DeltaCopy Server: You configure the virtual alias name that maps to the physical folder.

On DeltaCopy Client: Configure the IP + port of the DeltaCopy Server, select the name of alias folder, and select one or more files/folders to be copied to this alias name.
\begin{itemize}
  \item if both on the same machine, use \verb!localhost! as IP.
\end{itemize}

You may get the errors
\begin{verbatim}
rsync: failed: File name too long
\end{verbatim} 
The solution is to upgrade the Cygwin libraries that are used by DeltaCopy. Cygwin 1.7.7 solves this problem.
So
\begin{enumerate}
  \item turn off DeltaCopy server
  
  \item install cygwin 32 bit with the right version, install Net/openssh, and Net/rsync
  
  \item jump to the folder installing cygwin (./bin): and copy the following files to DeltaCopy
\begin{verbatim}
chmod.exe
cygcrypto-0.9.8.dll
cyggcc_s-1.dll
cygiconv-2.dll
cygintl-8.dll
cygpopt-0.dll
cygwin1.dll
cygz.dll
rsync.exe
ssh.exe
\end{verbatim}
cygminires.dll (no longer required- (obsolete support package for openssh))

  \item edit \verb!C:\DeltaCopy\deltacd.conf!, and add two lines at the beginning of the file
\begin{verbatim}
uid = 0
gid = 0
\end{verbatim}
  
  \item restart DeltaCopy server
  
\end{enumerate}
\url{http://benevolentmachine.blogspot.com/2010/10/eliminating-deltacopys-file-name-too.html}

\subsection{Syncrify}
\label{sec:Syncrify}

NOTE: On Windows, DeltaCopy uses CYGWIN libraries to run rsync on Windows.
Syncrify does not need Cygwin as it uses a custom implementation. 
\url{http://web.synametrics.com/SyncrifyVsDeltaCopy.htm}


\section{Cygwin}
\label{sec:Cygwin}

Cygwin aims to create a complete UNIX/POSIX environment for running a Linux
application on Windows. To do this, Cygwin needs to use various DLLs which may
impose limitations on forking these DLLs.

\verb!clear! command is not available (which you can press Ctrl-L or Alt-F8 to
skip page (clean the screen)) by default. You need to install \verb!ncurses!
package - Sect.\ref{sec:ncurses}).
Another option is adding \verb!~/.bashrc! file the line
\begin{verbatim}
alias clear='printf "\033c"'
\end{verbatim}
\url{http://stackoverflow.com/questions/11249070/how-do-i-get-the-clear-command-in-cygwin}

To run with X-server, i.e. enable running remote GUI applications, we run either
\begin{verbatim}
xinit
startx
startxwin
\end{verbatim}

\subsection{List of to-installed packages}


\begin{verbatim}
xinit
vim
ncurses
screen
tmux
g++
gdb      (Devel)
cgdb
automake  (Devel)
make   (Devel)
patch   (Devel)
git
curl
wget
bzr   (Python-based distributed version-control system)
\end{verbatim}

See how to use \verb!screen! (Sect.\ref{sec:screen}), \verb!tmux!
(Sect.\ref{sec:tmux}). 


\subsection{byobu in Cygwin}

Check the Open-Source book with Chapter on Cygwin.


\section{MobaXterm}
\label{sec:MobaXterm}


IMPORTANT: the \verb!cmake! should be called in \verb!/usr/bin/cmake.exe!, so
modify \verb!PATH! in Windows or use the path explicitly to avoid the below
error

\begin{verbatim}
CMake Error: Could not find CMAKE_ROOT !!!
CMake has most likely not been installed correctly.
Modules directory not found in
//share/cmake-3.6.2
cmake version 3.6.2
\end{verbatim}


IMPORTANT: You need to instal \verb!make! 
\begin{verbatim}
CMake Error: CMake was unable to find a build program corresponding to "Unix Makefiles".  
CMAKE_MAKE_PROGRAM is not set.  You probably need to select a different build tool.
CMake Error: CMAKE_C_COMPILER not set, after EnableLanguage
CMake Error: CMAKE_CXX_COMPILER not set, after EnableLanguage
-- Configuring incomplete, errors occurred!
\end{verbatim}

To install \verb!make!
\begin{verbatim}
apt-get --legacy install libguile17 
apt-get install make
\end{verbatim}


To install \verb!cmake!
\begin{verbatim}
apt-get install cmake 
\end{verbatim}

To install dev versions, e.g. use \verb!lib<libname>-devel!
\begin{verbatim}
apt-get install libcurl-devel
\end{verbatim}

To install \verb!xrandr!
\begin{verbatim}
apt-get install xrandr	
\end{verbatim}

XKBlib.h is 
\begin{verbatim}
/usr/include/X11/XKBlib.h
\end{verbatim}

X11
\begin{verbatim}
apt-get install libxkbfile1 libxkbfile-devel libxkbcommon0 libxkbcommon-devel
\end{verbatim}

autoconf, automake, libtool
\begin{verbatim}
apt-get install autoconf automake libtool
\end{verbatim}
\url{http://www.delorie.com/gnu/docs/autoconf/autoconf_13.html}

RPC header files
\begin{verbatim}
apt-get install libtirpc-common libtirpc-devel libtirpc1
apt-get install rpcgen onc-rpc-devel -y
apt-get install rpcbind
\end{verbatim}
\url{https://cygwin.com/ml/cygwin/2014-08/msg00359.html}

liblxi: offers a simple API for communicating with LXI compatible instruments, 
e.g.  discover instruments on your network, send SCPI commands, and receive
responses via VXI-11/TCP and RAW/TCP connections.
NOTE: it needs rpc header files. 
\label{sec:liblxi}
\begin{verbatim}
git clone https://github.com/lxi-tools/liblxi.git

//configure PATH (in mobaxterm) so that /usr/bin is used first
export PATH=/usr/bin:$PATH

cd liblxi
autoreconf -f -v -i
# autoconf -f -v -i (use autoreconf)
./configure
make -j4
make install
\end{verbatim}


\url{https://bugzilla.redhat.com/show_bug.cgi?id=901333}

List of many plugins:
\url{https://mobaxterm.mobatek.net/plugins.html}


\section{MSYS2 (Minimal SYStem 2)}
\label{sec:MSYS2}

MSYS2 is a rewritten of MSYS (Sect.\ref{sec:MSYS}) using modern Cygwin (POSIS
compatability layer) and MinGW-w64 to support better with native Windows
software.

MSYS2 makes Windows be less of a special case for C/C++ software development.
You can use all your favorite build tools from Linux, and there is a large and
growing library of software you can install with the package manager

MSYS2 has: \url{www.msys2.org}

\begin{itemize}
  \item bash shell
  
  \item Autotools
  
  \item revision control systems
  
  \item Besides MinGW, it has MinGW-w64 tool chains (Sect.\ref{sec:MinGW-w64})
  to build 64-bit Windows-compatible softwares

\begin{verbatim}
mingw
mingw64
\end{verbatim}  
  
  \item package management tools (name pacman - Sect.\ref{sec:pacman}): to
  manage software installation and can resolve package dependency automatically.
  
\end{itemize}
\url{https://github.com/Alexpux/MSYS2-packages}

MSYS2 resyncs with the Cygwin project regularly. 

COMPARE with Cygwin (Sect.\ref{sec:Cygwin}): MSYS2 doesn't have cygwin.dll but
it does have msys-2.0.dll which does the same thing (and some more besides).

HOW TO
\begin{verbatim}
// update package database + core system packages
pacman -Syu

// close/restart MSYS2
package -Su

// 
pacman -S git

pacman -S openssh
pacman -S base-devel mingw-w64-x86_64-toolchain
//pacman -S openmpi python-mpi4py python2-mpi4py
\end{verbatim}

NOTE: MSYS2 provides 2 tool chains
\begin{enumerate}

  \item The MSYS2 toolchain which lives in /usr makes programs that depend on
  the msys-2.0.dll POSIX emulation layer.
  
  \item The MinGW toolchain which lives in /mingw32 or /mingw64 creates native
  Windows programs that basically only have access to the features provided by
  Microsoft's headers and lirbaries, and not much else. These programs are
  easier to redistribute to other users and generally run faster. If you really
  need a lot of features of POSIX, you'll probably not be able to use the MinGW
  toolchain.
\end{enumerate}
\url{https://stackoverflow.com/questions/44504216/why-is-unistd-h-different-in-msys2-and-mingw-w64-x86-64-toolchain}


To install MPI in Windows, follow the link
\url{http://www.math.ucla.edu/~wotaoyin/windows_coding.html}

use Microsoft MPI (MS-MPI) 
\begin{itemize}
  \item 
v7  which include msmpisdk  and msmpisetup

\url{https://www.microsoft.com/en-us/download/details.aspx?id=49926&751be11f-ede8-5a0c-058c-2ee190a24fa6=True}

  \item v8.1
  
\url{https://www.microsoft.com/en-us/download/details.aspx?id=55494} 
\end{itemize}

\begin{verbatim}
pacman -Syuu

pacman -S --needed base-devel mingw-w64-i686-toolchain mingw-w64-x86_64-toolchain \
                   git subversion mercurial \
                   mingw-w64-i686-cmake mingw-w64-x86_64-cmake
                   
Add C:\dev\msys64\mingw64\bin and C:\dev\msys64\mingw32\bin, in that order, to
                   your PATH. Note that MSYS2 also puts a lot of other tools in
                   this directory, most notably Python. So put these entries
                   below any other tools you might have installed in your PATH.   
\end{verbatim}
\url{https://github.com/orlp/dev-on-windows/wiki/Installing-GCC--&-MSYS2}

To make 32-bit binaries, use i686-w64-mingw32-g++
 
\subsection{MSYS}
\label{sec:MSYS}

MinGW provides no Unix tools on Windows.
To provide this, MSYS (Minimal SYSteem) is a collection of GNU utilities such as
bash, make, gawk and grep to allow building of C/C++ applications and programs
which depend on traditionally UNIX tools to be present. It is intended to
supplement MinGW and the deficiencies of the cmd shell.

MSYS was used as an ancilliary to MinGW. 

MSYS forked from Cygwin version 1.3.3 and never re-synced, 


\subsection{MinGW-w64}
\label{sec:MinGW-w64}

Unlike Cygwin (Sect.\ref{sec:Cygwin}), MinGW is a C/C++ compiler tool chain,
based on GCC, to help building native Windows application using a minimal list
of DLLs. It uses MSVC runtimes, which are part of any normal Microsoft Windows
installation.

MinGW-w64 is an advancement of the original mingw.org project;
with the aim to support 64-bit and new APIs

REMEMBER: MinGW is the project to create GCC compiler (a toolchain) on Windows,
which helps to build native-Windows application using code with Makefile build
scripts.


\url{https://sourceforge.net/projects/mingw-w64/files/Toolchains targetting Win64/Personal Builds/}

\subsection{TDM-GCC}

TDM-GCC is a compiler suite for Windows.
It uses the GCC toolset, a few patches for Windows-friendliness, and the free
and open-source MinGW or MinGW-w64 runtime APIs to create an open-source
alternative to Microsoft's compiler and platform SDK.


\section{Windows Powershell}
\label{sec:windows_powershell}

When DOS runs, it uses \verb!autoexec.bat! or \verb!config.sys! to control the
startup environment. 

When Windows boot up, it uses \verb!boot.ini! and boot configuration data
(BCD) stored in registry. To modify BCD, we can use \verb!BCDEdit! command-line
tool
\url{https://technet.microsoft.com/en-us/library/cc709667(v=ws.10).aspx}
or visual BCD editor \url{http://www.boyans.net/}

\subsection{version}
\label{sec:version_PowerShell}

\begin{verbatim}
$PSVersionTable.PSVersion
\end{verbatim}
which output can be
\begin{verbatim}

Major  Minor  Build  Revision
-----  -----  -----  --------
2      0      -1     -1

\end{verbatim}

 To update to 
 \begin{itemize}
   \item  version 3, you must download the Windows Management Framework 3: 
 \url{http://www.microsoft.com/en-us/download/details.aspx?id=34595}
   \item version 4: \url{http://www.microsoft.com/en-us/download/details.aspx?id=40855}
   
   \item version 5 Preview: \url{http://www.microsoft.com/en-us/download/details.aspx?id=44987}
 \end{itemize}
then choose either the x86 or the x64 files depending on your system.

A list of equivalent commands between Windows PowerShell and Unix shell is given in 
\url{http://en.wikipedia.org/wiki/Windows_PowerShell}


\subsection{user's Profile}
\label{sec:profile_PowerShell}

When Windows PowerShell runs, it reads the PowerShell profile which is a script
file that runs at session startup. The scripting language is discussed in
Sect.\ref{sec:scripting_language_PowerShell}.
The name of the file is \verb!Microsoft.PowerShell_profile.ps1! configuration
file.

To check the location of this file, run the command
\begin{verbatim}
$Profile
\end{verbatim}

The existence of this file can be tested using
\begin{verbatim}
Test-Path $profile
\end{verbatim}
command. If 'False' is returned, you need to create a new profile.
\begin{verbatim}
New-Item -path $profile -type file -force
\end{verbatim}
then a file named 
\begin{verbatim}


    Directory: \\tieserver1\users$\thoangtr\My Documents\WindowsPowerShell


Mode                LastWriteTime     Length Name
----                -------------     ------ ----
-a---         2/10/2015  11:25 AM          0 Microsoft.PowerShell_profile.ps1
\end{verbatim}

However, by default, running script is
disabled due to the execution policy is \verb!Restricted! (Sect.\ref{sec:execution_policy_PowerShell})
\begin{verbatim}
File \\tieserver1\users$\thoangtr\My Documents\WindowsPowerShell\Microsoft.PowerShell_profile.ps1 cannot be loaded beca
use the execution of scripts is disabled on this system. Please see "get-help about_signing" for more details.
\end{verbatim}

\textcolor{red}{Example: simple Profile's content}
\begin{verbatim}
$env:Path = $env:Path + ";C:\Program Files (x86)\Vim\vim74" + ";C:\Python64bit_2_77\Scripts" + ";C:\Program Files\Oracle\VirtualBox"
$env:WORKON_HOME = "C:\Users\thoangtr\my_virt_env"

$pshost = get-host
$pswindow = $pshost.ui.rawui

$newsize = $pswindow.buffersize
$newsize.height = 3000
$newsize.width = 250
$pswindow.buffersize = $newsize

Set-Alias vi gvim
New-Alias which get-command

set-location D:\OpenSource\NASA
\end{verbatim}


\subsection{execution policy}
\label{sec:execution_policy_PowerShell}

To check different execution policies
\begin{verbatim}
Get-Help Set-ExecutionPolicy -full
\end{verbatim}
\begin{itemize}
  \item \verb!Restricted!: no loading scripts/configuration file
  \item \verb!AllSigned! : require all scripts/configuration file be signed by a
  trusted publisher (including scripts write on the local computer)
  
  \item \verb!RemoteSigned!: require all scripts/configuration file downloaded
  from the Internet be signed by a trusted publisher
  
  \item \verb!Unrestricted! : run alls, but for those downloaded from the
  Internet, there is prompt for permission before it runs.
  
  \item \verb!Bypass!: nothing is blocked or prompted 
  
\end{itemize}

To modify it, run the PowerShell as administrator, and type
\begin{verbatim}
Set-ExecutionPolicy RemoteSigned
\end{verbatim}

To enable a third-party program, to use PowerShell as the terminal, and can execute script, we need to enable the current user with this privilege
as the third-party program evoke the Windows PowerShell as the current user.
\begin{verbatim}
Set-ExecutionPolicy Unrestricted -Scope CurrentUser
\end{verbatim}

Now we discuss how to modify this file. We can use PowerShell ISE editor
(Integrated Scripting Environment)
\url{https://technet.microsoft.com/en-us/library/dd315244.aspx}

To edit other file contents from the PowerShell, we can use
\begin{itemize}
  \item nano: \url{http://www.nano-editor.org/download.php}
  
  \item kinesics: (Sect.\ref{sec:kinesics})
  
  \item Vim (GVim) support: Sect.\ref{sec:gvim}
  
  \item from the PowerShell command line:
  
\begin{verbatim}
"`nNew-Alias which get-command" | add-content $profile
\end{verbatim}
the character \verb!`n! is to ensure it starts as a new line
\end{itemize} 

\subsection{scripting language}
\label{sec:scripting_language_PowerShell}

Windows PowerShell consists of a command-line shell and a scripting language built on the .NET Framework.
From the command-line shell, we can have full access to COM and WMI, enabling administrator to perform administrative
tasks on both local and remote Windows

\subsection{access remote-machine}

PowerShell provides the capability to connect to a remote Windows machine via
the command-line shell just like SSH on other O/S.
As PowerShell Remoting is locked-down by default, you need to enable it first.
This setup process is a bit more complex if you're using a workgroup - for example, on a home network - instead of a domain.
\url{http://www.howtogeek.com/117192/how-to-run-powershell-commands-on-remote-computers/}
Suppose there are two machine A (local) + B (remote).

On the machine B that you want to be remotely connected, 
\begin{enumerate}
  \item right click on Windows Powershell, select run as Administrator
  
  \item Run
\begin{verbatim}
Enable-PSRemoting -Force
\end{verbatim}
which runs WinRM service, i.e. a WinRM listener on HTTP to accept WS-Man request to any WinRM firewall exception enabled.
\end{enumerate}

On the machine A that you want to use to connect
\begin{itemize}
  \item If machine A is configured in a workgroup (e.g on a home network):
  
  Launch PowerShell as administrator and run
\begin{verbatim}
Enable-PSRemoting -Force
\end{verbatim}
  On both machine A and machine B, configure the \verb!TrustedHosts! settings so the computers will trust each other.s
  Example: (not recommended to use) trust all machine (*)
\begin{verbatim}
Set-Item wsman:\localhost\client\trustedhosts *
\end{verbatim}
or replace asterisk (*) with a comma-separated list of IP addresses or computer-names.
  
  On both machine A and machine B, restart WinRM service
\begin{verbatim}
Restart-Service WinRM
\end{verbatim}

  Test the connection, on machine A
\begin{verbatim}
Test-WsMan <REMOTE-COMPUTER-NAME>
\end{verbatim}


  \item If machine A is configured in a domain
\end{itemize}

Finally, on machine A, you can do
\begin{enumerate}
  \item Execute a remote command
\begin{verbatim}
Invoke-Command -ComputerName COMPUTER -ScriptBlock { COMMAND } -credential USERNAME
\end{verbatim}
with COMPUTER = machine B, COMMAND = command to run on machine B, USERNAME = user account on machine B.

Example:
\begin{verbatim}
Invoke-Command -ComputerName Monolith -ScriptBlock { Get-ChildItem C:\ } -credential chris
\end{verbatim}


  \item Start a remote session, to run multiple commands
\begin{verbatim}
Enter-PSSession -ComputerName COMPUTER -Credential USER
\end{verbatim}

Example:s  
\begin{verbatim}
PS C:\> Enter-PSSession -ComputerName dc1
[dc1]: PS C:\Users\Administrator\Documents> sl c:\
[dc1]: PS C:\> gwmi win32_bios

\end{verbatim}

  \item To enable multiple connections to a remote system using the same account, run \verb!New-PSSession!

\verb!New-PSSession! permits you to store the remote session in a variable, and it
provides you with the ability to enter and leave the remote session as often as
required, without the additional overhead of creating and destroying remote
sessions.
  
Example:
\begin{verbatim}
PS C:\> $dc1 = New-PSSession -ComputerName dc1 -Credential iammred\administrator

PS C:\> Enter-PSSession $dc1
[dc1]: PS C:\Users\Administrator\Documents> hostname
dc1
[dc1]: PS C:\Users\Administrator\Documents> exit


PS C:\> Enter-PSSession $dc1
[dc1]: PS C:\Users\Administrator\Documents> gps | select -Last 1

Handles  NPM(K)    PM(K)      WS(K) VM(M)   CPU(s)     Id ProcessName
-------  ------    -----      ----- -----   ------     -- -----------
    292       9    39536      50412   158     1.97   2332 wsmprovhost

[dc1]: PS C:\Users\Administrator\Documents> exit
PS C:\> Get-PSSession

 Id Name            ComputerName    State         ConfigurationName     Availability
 -- ----            ------------    -----         -----------------     ------------
  8 Session8        dc1             Opened        Microsoft.PowerShell     Available

PS C:\> Get-PSSession | Remove-PSSession
PS C:\>
\end{verbatim}  
  
  \item PowerShell has \verb!Start-Transcript!, the transcript tool captures output from the remote Windows PowerShell session and output from the local session.
  
\begin{verbatim}
PS C:\> Start-Transcript
Transcript started, output file is C:\Users\administrator.IAMMRED\Documents\PowerShe
ll_transcript.20120701124414.txt
PS C:\> Enter-PSSession -ComputerName dc1
[dc1]: PS C:\Users\Administrator\Documents> gwmi win32_bios

SMBIOSBIOSVersion : A01
Manufacturer      : Dell Computer Corporation
Name              : Default System BIOS
SerialNumber      : 9HQ1S21
Version           : DELL   - 6

[dc1]: PS C:\Users\Administrator\Documents> exit
PS C:\> Stop-Transcript
Transcript stopped, output file is C:\Users\administrator.IAMMRED\Documents\PowerShe
ll_transcript.20120701124414.txt
\end{verbatim}
  
\end{enumerate}
 
%Finally, on machine A, start the connection
\url{http://blogs.technet.com/b/heyscriptingguy/archive/2013/12/11/use-powershell-to-create-remote-session.aspx}
 
\subsection{setting startup locaion}

The default location is \verb!<user-document-path>!
\begin{verbatim}
set-location c:
\end{verbatim}


\subsection{change PowerShell window's title}

The default title is \verb!Windows PowerShell!
\begin{verbatim}
$Shell = $Host.UI.RawUI
$Shell.WindowTitle="SysadminGeek"
\end{verbatim}

\subsection{change window size and scrollback}

\begin{verbatim}
$Shell = $Host.UI.RawUI
$size = $Shell.WindowSize
$size.width=70
$size.height=25
$Shell.WindowSize = $size
$size = $Shell.BufferSize
$size.width=70
$size.height=5000
$Shell.BufferSize = $size
\end{verbatim}

\url{http://blogs.technet.com/b/heyscriptingguy/archive/2006/12/04/how-can-i-expand-the-width-of-the-windows-powershell-console.aspx}


\subsection{background and text color}

The default background color is blue, and text color is white
\begin{verbatim}
$Shell = $Host.UI.RawUI
$shell.BackgroundColor = "Gray"
$shell.ForegroundColor = "Black"
\end{verbatim}


\subsection{create alias}

To check the built-in aliases, we run
\begin{verbatim}
Get-Alias
\end{verbatim}

Set a single user-defined alias
\begin{verbatim}
Set-Alias gs Get-Service <enter>
\end{verbatim}

If you want to export the current aliases into a single file
\begin{verbatim}
Export-Alias -Path Aliases.txt <enter>
\end{verbatim}

If you have multiple aliases, and you save in the file, we can read-in
user-defined alias, from file
\begin{verbatim}
Import-Alias -Path Alias.txt <enter>
\end{verbatim}
The structure of this file should be 4 columns
\begin{verbatim}
# this is comment
"gs", "Get-Service", "", "None"
\end{verbatim}


Example: \verb!np! means running Notepad.exe
\begin{verbatim}
new-item alias:np -value C:WindowsSystem32notepad.exe
\end{verbatim}

\url{http://www.powershellpro.com/powershell-tutorial-introduction/tutorial-powershell-aliases/}

\subsection{environment variable}

\begin{verbatim}
$env:Path = $env:Path + ";C:\Program Files\GnuWin32\bin"

$env:Path += ";C:\Program Files\GnuWin32\bin"
\end{verbatim}

\url{http://stackoverflow.com/questions/714877/setting-windows-powershell-path-variable}


\url{http://www.computerperformance.co.uk/powershell/powershell_env_path.htm}

\subsection{Persistent history}

You want the command-line to save the history in the previous sessions. First
check PowerShell version (Sect.\ref{sec:version_PowerShell}). If \verb!Major<3!,
then we need to update PowerShell.

Then run
\begin{verbatim}
set-executionpolicy remotesigned

  ## install PsGet
(new-object Net.WebClient).DownloadString("http://psget.net/GetPsGet.ps1") | iex

import-module PsGet
install-module PsUrl
install-module PSReadline
\end{verbatim}
NOTE: We use
\begin{itemize}
  \item PSReadLine: to map a key to a particular command, e.g. Up key to browse history
\end{itemize}

Modify profile file (Sect.\ref{sec:profile_PowerShell})

\begin{verbatim}
$MaximumHistoryCount = 200  
$HistoryFilePath = Join-Path ([Environment]::GetFolderPath('UserProfile')) .ps_history

Register-EngineEvent PowerShell.Exiting -Action { Get-History -Count $MaximumHistoryCount | Export-Clixml $HistoryFilePath } | out-null

# Load previous history, if it exists
if (Test-path $HistoryFilePath) { 
   Import-Clixml $HistoryFilePath | ? {$count++;$true} | Add-History
   
   Write-Host -Fore Green "`nLoaded $count history item(s).`n" 
}

# if you don't already have this configured...
Set-PSReadlineKeyHandler -Key UpArrow -Function HistorySearchBackward
Set-PSReadlineKeyHandler -Key DownArrow -Function HistorySearchForward

New-Alias -Name i -Value Invoke-History -Description "Invoke history alias"
##backup previous 'h' command meaning to 'original_h'
Rename-Item Alias:\h original_h -Force

## use 'h' for history
function h { Get-History -c  $MaximumHistoryCount }
function hg($arg) { Get-History -c $MaximumHistoryCount | out-string -stream | select-string $arg }
\end{verbatim}
\url{https://software.intel.com/en-us/blogs/2014/06/17/giving-powershell-a-persistent-history-of-commands}
\url{http://stackoverflow.com/questions/16474973/windows-powershell-persistent-history}


Save and exit the editor. From now on, to check the history run
\begin{verbatim}
get-history
   ## or simply
h
\end{verbatim}
The last session's commands as well as the current session's commands are there

\subsection{commands: clear-host, new-item, get-command}

Clear the workspace
\begin{verbatim}
Clear-Host
\end{verbatim}

Create new file
\begin{verbatim}
new-item -type file NAME/__init__.py
\end{verbatim}

Check the location of a particular command
\begin{verbatim}
Get-command <command-to-search>

gcm <command-to-search>

Get-Command <your command> | Select-Object -ExpandProperty Definition
\end{verbatim}
with \verb!gcm! is the alias of \verb!Get-Command!. 

\url{http://stackoverflow.com/questions/63805/equivalent-of-nix-which-command-in-powershell}

\url{http://www.howtogeek.com/50236/customizing-your-powershell-profile/}

\url{http://www.computerperformance.co.uk/powershell/powershell_profile_ps1.htm}

%\subsection{Simple profile content}


\section{Editors}

\subsection{Kinesics editor}
\label{sec:kinesics}

The Kinesics Text Editor is a small, fast, efficient, portable text editor.

\url{http://www.turtlewar.org/projects/editor/}

Type \verb!k! to run the editor

\begin{itemize}
  \item F1 (or Contrl-]) : menu
\end{itemize}

\subsection{Vim Windows}
\label{sec:gvim}

\verb!gvim! includes GUI and console versions (32-bit and 64-bit).


  \url{http://www.vim.org/download.php\#pc}
  \url{https://github.com/PProvost/vim-ps1}
  
  \url{http://www.vim.org/scripts/script.php?script_id=1327}

\subsection{MathPad}
\label{sec:MathPad}

\url{http://mathpad.software.informer.com/}

\section{Windows 7: short-cut keys}

\url{http://www.scoroncocolo.com/go-mouseless-use-a -pc-with-nothing-but-a-keyboard.html}
{\bf Show the Desktop Shortcut}
\begin{verbatim}
Win-key + Spacebar      Peak at the Desktop without actually going there.
Win-key + d      Toggle back and forth from your Desktop to your open windows.
\end{verbatim}

{\bf Windows Keyboard Shortcuts Using the Arrow Keys}
\begin{verbatim}
Win-key plus the down Arrow-key      Makes a window less than full-screen.
Win-key pluse the up Arrow-key      Makes the window full-size again.
Win+Shift+Right/Left arrow      If you have more than one monitor, moves the current window left or right.
Win-key pluse the left Arrow-key      Docks a less than full-size window to the left side of the screen.
Win-key pluse the right Arrow-key      Docks a less than full-size window to the right side of the screen.
Win-key pluse t      Allows you to use your Arrow keys to scroll through the taskbar.
Win-key pluse b      Allows you to use your Arrow keys to scroll through your Systems Trey.
\end{verbatim}

{\bf More Cool Windows Keyboard Shortcuts}
\begin{verbatim}
Win-key plus (1 through 9)      Opens the corresponding program on your taskbar.
Alt plus Win-key plus (1 through 9)      Opens the Jump List of the corresponding program on your taskbar.
Win-key plus Tab      Activates Flip 3d.
Ctrl plus Win-key plus Tab      Freezes Flip 3d. Use Esc key or click a window to deactivate Flip 3d.

\end{verbatim}


\section{Windows special folders}

\url{http://windows.microsoft.com/en-us/windows-8/what-appdata-folder}
\subsection{Roaming: \%APPDATA\%}

\verb!%APPDATA%!
contains data that can move with your user profile from PC to PC.
This is a typical place to put a user-specific application's data.
Because this data has the ability to sync with a server. For example, if you
sign in to a different PC on a domain, your web browser favorites or bookmarks
will be available.


\subsection{Local \%localappdata\%}

\verb!%LOCALAPPDATA%!
contains data that can't move with your user profile. This data is typically
specific to a PC or too large to sync with a server.

\subsection{LocalLow: \%appdata\%/../locallow}

\verb!%APPDATA%/../locallow!
contains data that can't move, but also has a lower level of access




\section{Windows feature transition}


\subsection{Windows XP to Windows Vista}

\url{https://en.wikipedia.org/wiki/List_of_features_removed_in_Windows_Vista}
A number of capabilities and certain programs
parts of Windows XP are no longer available in Windows Vista.
NOTE: \textcolor{red}{Some of which were later re-introduced in Windows 7.} 

\subsection{Windows Vista to Windows 7}

\url{https://en.wikipedia.org/wiki/List_of_features_removed_in_Windows_7}
Windows 7 is an incremental update to Windows Vista.
 

\subsection{Windows 7 to Windows 8}

Windows 8 introduced major changes to the operating system's platform and user
interface.
\url{https://en.wikipedia.org/wiki/List_of_features_removed_in_Windows_8}

New features and functionality in Windows 8 include a faster startup through
UEFI integration and the new "Hybrid Boot" mode (which hibernates the Windows
kernel on shutdown to speed up the subsequent boot,
native support for USB 3.0 devices.
Windows Explorer, which has been renamed File Explorer,
uses a ribbon in place of the command bar.

Task Manager has been redesigned, including a new processes tab with the option
to display fewer or more details of running applications and background
processes, \ldots

Login: two new authentication methods tailored towards touchscreens (PINs and
picture passwords), besides the traditional password.

\url{https://en.wikipedia.org/wiki/Windows_8}

\subsection{windows 8 to Windows 10}

The goal of Windows 10 is to unify the operating systems of Microsoft's PC,
Windows Phone, Windows Embedded and Xbox One product families, along with new
products (Surface Hub, HoloLens) around a common internal core.

A new default web browser, Microsoft Edge, will be available.

Windows 10 adds a revision of the desktop start menu, similar to that of Windows
7, and a virtual desktop system, and allows Windows Store apps to run within
windows on the desktop as well as in full-screen mode.

Login: new frameworks for biometric authentication.

{\bf App-V} system (Sect.\ref{sec:App-V}): to pack web apps and desktop apps in
Win32 or .NET framework to be able to run in Windows 10 via sandboxing.

\section{Virtual screen}
\label{sec:virtual-screen}

\url{http://www.howtogeek.com/130650/the-best-free-programs-for-using-virtual-desktops-in-windows/}


\subsection{Sysinternals Desktops v2.0 }

\url{http://www.howtogeek.com/195962/unlock-virtual-desktops-on-windows-7-or-8-with-this-microsoft-tool/}
\section{Synergy: share keyboard and mouse}
\label{sec:synergy}

Synergy combines your desktop devices together in to one cohesive experience.
It's software for sharing your mouse and keyboard between multiple computers on
your desk. 

You can get 64-bit build for Windows, Linux from here:
\url{https://sourceforge.net/projects/synergy-stable-builds/}


Configure the machine with keyboard/mouse available as server, and you may not
to enable firewall on that machine to open port 24800
\begin{verbatim}
sudo iptables -I INPUT 1 -p tcp --dport 24800 -j ACCEPT
\end{verbatim}

