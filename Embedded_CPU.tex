\chapter{Embedded CPU architectures}


\section{Motorola 68k series}

The development of Motorola 68000 series was ceased in 1994, and switched to
PowerPC architecture (Sect.\ref{sec:PowerPC}).

Other chips compatible with the 68000 instruction set
\begin{enumerate}
  \item Freescale ColdFire
  \item Freescale DragonBall
\end{enumerate}
\url{https://en.wikipedia.org/wiki/Motorola_68000_family}

\section{PowerPC}
\label{sec:PowerPC}

PowerPC is the product from Motorola, IBM and Apple (i.e. AIM alliance).
\begin{itemize}
  \item Early generation of PowerPC-based chip lack a supported O/S
  \begin{itemize}
    \item PowerPC 601: first gen
    \item PowerPC 603 (second gen, the first native PowerPC instruction set),
    and its high-end PowerPC 604
  \end{itemize}
  
  \item PowerPC 615: support Intel x86 instructio set directly by the CPU
  
  \item PowerPC 620: first 64-bit PowerPC
  \item PowerPC 970: smaller 64-bit PowerPC using POWER3 design
  
  \item embedded PowerPC: 4xx lines (401, 403, 405, 440, 460), which was sold to
  AMCC company in 2004.
  
  
\end{itemize}
\url{https://en.wikipedia.org/wiki/PowerPC}

The core PowerPC Instruction Set has 107 instructions.
\url{http://pds.twi.tudelft.nl/vakken/in101/labcourse/instruction-set/}

PowerPC
vs
Intel:
\url{http://www.quora.com/Why-was-the-PowerPC-architecture-unable-to-keep-up-with-Intel}

\section{embedded Linux vs. Desktop Linux}


\section{embedded Linux vs. real-time OS}

In Real-time OS, events are handled within a time constraint, i.e. deterministic
behavior. An RTOS provides scheduling guarantees to ensure deterministic
behaviour and timely response events and interrupts.  In most cases this is
through a priority based pre-emptive scheduling algorithm, where the highest
priority task ready to run always runs - immediately - pre-empting any lower
priority task without a specific yield or relinquishing of the CPU, or
completion of a time-slice.

Embedded Linux is not an RTOS as events/interrupts are typically handled by
deferred procedures or 'bottom half' code which can not guarantee latency. Linux
has a number of scheduling options, including a real-time scheduler, but this is
at best "soft" real-time. If your application has no need of "hard" real-time,
that's fine, but typical latencies in real-time Linux will be in the order of
tens or hundreds of microseconds, whereas a typical RTOS real-time kernel can
achieve from latencies from zero to a few microseconds.


Linux is a general-purpose OS (GPOS); its application to embedded systems is
usually motivated by the availability of device support, file-systems, network
connectivity, and UI support.

Linux is soft real time and vxWorks is hard real time.
In soft real time you can expect a real time performance but not 100\%
deterministic.
Soft-real time is good enough for range of application (telecom/consumer device
etc); However when you need a product like robotics, Medical equipment you can't
go for an option which is not deterministic.
Realtime performance is also an issue as mentioned. Measure whether
you need uS or mS performance.

Many RTOS are not full OS in the sense that Linux is, in that RTOS  comprise of
a static link library providing only task scheduling, IPC, synchronisation timing
and interrupt services and little more - essentially the scheduling kernel only.
Such a library is linked with your application code to produce a single
executable that your system boots directly (or via a bootloader).
Most RTOS do not directly support the loading and unloading of code dynamically
from a file system as you would with Linux - it is all there at start-up and runs until power down.

Another issue with embedded Linux is that it needs significant CPU resources,
perhaps >200MIPS, 32bit processor, ideally with an MMU, 4Mb of ROM and 16MB or
RAM to just about boot (which may take several seconds). An RTOS on the other
hand can be up in milliseconds, run in less than 10Kb, on microcontrollers from
8-bit up. This can have a significant impact on system cost for volume
production despite being ostensibly "free".


There are larger RTOS products that exhibit some of the features of a GPOS such
dynamic loading, filesystems, networking, GUI (for example as QNX), and many
RTOS provide a {\bf POSIX API} (usually secondary to their native real-time API)
for example VxWorks and agian QNX, so that a great deal of code developed for Linux
and Unix can be ported relatively easily. These larger more comprehensive RTOS
products remain scalable, so that functionality not required is not included.
Linux in comparison has far more limited scalability.

\url{http://stackoverflow.com/questions/25871579/what-is-the-difference-between-rtos-and-embedded-linux}

\section{Real-time OS (RTOS)}

You could also look at OSes like $\mu$C/OS-II, eCos, RTEMS, FreeRTOS and others
...

\subsection{vxWorks}
\label{sec:vxWorks}

VxWorks is a proprietary, real-time OS (RTOS). 
Linux is royalty free; however with VxWorks you need to pay for
production license for each device you sell running VxWorks.

VxWorks support many CPUs which include those of the x86 family, MIPS, PowerPC,
and the families of ARM, StrongARM, and xScale.
\url{http://www.differencebetween.net/technology/difference-between-vxworks-and-linux/}


VxWorks is a RTOS designed for use in embedded systems requiring real-time,
deterministic performance and, in many cases, safety and security certification,
for industries, such as aerospace and defense, medical devices, industrial
equipment, robotics, energy, transportation, network infrastructure, automotive,
and consumer electronics.
\textcolor{red}{Some one measured ISR
latencies of about 1us for a test ISR on vxWorks, while rtai linux
latencies were as high as 30us}.
\url{http://fixunix.com/vxworks/331795-linux-vs-vxworks.html}

VxWorks supports
\begin{itemize}
  \item multitasking kernel: pre-emptive scheduling, round-robin scheduling,
  fast interupt response
  
  \item  memory protection: isolate user application from kernel memory space
  
  \item SMP support
  
  \item error handling framework
  
  \item filesystem
  
  \item local + distributed message queues
\end{itemize}

The development of applications targeted for VxWorks is done on a host machine
which runs Linux, Unix, or Windows. It cross compiles target software so that it
is capable of running on various target CPU architectures.


\textcolor{red}{VxWorks as a real time operating system handles threads somewhat
different from what "normal" OSes do}, regarding their scheduling! This could
lead to a scenario called "priority inversion" when using semaphores, see
\url{http://en.wikipedia.org/wiki/Priority_inversion}

\begin{verbatim}
VxWorks is not a genuine POSIX-OS in itself, rather it is using a
// kind of compatibility layer (sort of a wrapper) to emulate the
// POSIX-functionality by using its own resources and functions.
// At the time a task (thread) calls it's first POSIX-function during
// runtime it is being transformed by the OS into a POSIX-thread.
// This transformation does include a call to malloc() to allocate the
// memory required for the housekeeping of POSIX-threads. In a high
// priority RTP this malloc() call may be highly undesirable, as its
// timing is more or less unpredictable (depending on what your actual
// heap looks like). You can circumvent this problem by calling the
// function thread_self() at a well defined point in the code of the
// task, e.g. shortly after the task spawns up. Thereby you are able
// to define the time when the task-transformation will take place and
// you could shift it to an uncritical point where a malloc() call is
// tolerable. So, if this could pose a problem for your code, remember
// to call thread_self() from the affected task at an early stage.
//
\end{verbatim}
\url{https://searchcode.com/codesearch/view/85130696/}


\subsection{Embox}

Embox is an open source real-time operating system designed for resource
constrained hardware.

\url{https://code.google.com/p/embox/}



\section{cFE (core Flight Executive)}

\subsection{Macro vxWorks-PPC750}

Check \url{http://www.vxdev.com/docs/vx55man/vxworks/ppc/powerpc.html}

Some are specific to \verb!ccppc! compiler: 
\begin{enumerate}
  \item \verb!_VXWORKS_OS_! : if defined, use the proper include file in 
  
 \begin{verbatim}
 <project>/cfe/fsw/cfe-core/src/inc/network_includes.h
 \end{verbatim}
 \url{https://github.com/billvb/cfe/blob/master/cfe/fsw/cfe-core/src/inc/network_includes.h}
 
 \textcolor{red}{Use a different macro for other O/S}
 
 \verb!TARGET_DEFS! 
 \begin{itemize}
   \item RTEMS O/S : \verb!_RTEMS_OS_!
   
   \item MAC O/S: \verb!_MAC_OS_!
   
   \item Linux O/S: \verb!_LINUX_OS_!
 \end{itemize}
  
  \item \verb!_PPC_!,  \verb!__PPC__!: if defined, it indicates the PowerPC
  architecture (Sect.\ref{sec:macro-GNU-to-detect-CPU} in C/C++ Manual)
  
  However, no one use \verb!_PPC_!, it is better replaced by \verb!__ppc__!.
  
  \url{http://nadeausoftware.com/articles/2012/02/c_c_tip_how_detect_processor_type_using_compiler_predefined_macros}
  
  \item \verb!$(CFE_SB_NET)!: not being used anywhere
  
  \item \verb!$(OS)! : use on every configuration, the value is defined in
  another place to indicate which target O/S is
  
  \item \verb!MCP750!: if defined, it means the board is MCP750
  (Sect.\ref{sec:MCP750})
  
  if defined, it checks the files
\begin{verbatim}
psp/fsw/mip405t-vxworks6.4/src/mcpx750.h:#ifdef MCP750
psp/fsw/mcp750-vxworks6.4/src/mcpx750.h:#ifdef MCP750
psp/fsw/mip405t-linux/src/mcpx750.h:#ifdef MCP750
\end{verbatim}
The header is included in these files
\begin{verbatim}
psp/fsw/mcp750-vxworks6.4/src/cfe_psp_timer.c:#include "mcpx750.h"
psp/fsw/mcp750-vxworks6.4/src/cfe_psp_watchdog.c:#include "mcpx750.h"
psp/fsw/mcp750-vxworks6.4/src/cfe_psp_start.c:#include "mcpx750.h"
\end{verbatim}
  
  
  \item \verb!_EMBED_!
  
  \item \verb!TOOL_FAMILY=gnu!, \verb!TOOL=gnu! (can be \verb!gnu! or
  \verb!diab! or \verb!gnuv8!)
  
Only used by \verb!vxWorks!: as it is used by some program to detect compiler information, and compiler option.
\begin{verbatim}
ifeq ($(TOOL_FAMILY),diab)
CC := dcc
LD := dld
AR := dar
VXWORKS_VER := vxworks62
else
CC := cc$(VX_TOOL_SUFFIX)
LD := $(CC)
AR := ar$(VX_TOOL_SUFFIX)
endif

ifeq ($(TOOL_FAMILY),diab)
# compiler and linker flags using Diab compiler
  CFLAGS += -w -Xdialect-ansi -Xno-common -D__vxworks -D_DIAB_TOOL -D__XSCALE__
  LDFLAGS := -r -W:as:,-x,-X -Ws
  MAKE_DEP_FLAG := -Xmake-dependency

else
# compiler and linker flags using GNU compiler
  CFLAGS += -Wall -ansi -pedantic -fno-common -mno-sched-prolog -mcpu=xscale
  LDFLAGS := -nostdlib -r -Wl,-X
  MAKE_DEP_FLAG := -M
endif
\end{verbatim}
\url{https://github.com/mirror/dd-wrt/blob/master/src/linux/sl2312/linux-2.6.23/drivers/ixp400/ixp_osal/os/vxworks/make/macros.mk}

  
  \item \verb!_WRS_KERNEL!: if defined, use Wind River System's Kernel
  
  This is only for vxWorks's OSs
  \url{https://developer.windriver.com/thread/1171}
  
  \item \verb!CPU=PPC604! (other values: PPC403, PPC405, PPC440, PPC603,
  PPC604, PPC860)
  
  Motorola PowerPC MPC74xx CPUs are treated as a variation of the PowerPC 604
  CPU type.
  
\end{enumerate}

\verb!ARCH_OPTS! 
\begin{itemize}
  \item \verb!-mcpu=750!: Set architecture type, register usage, and instruction
  scheduling parameters for machine type for CPU \verb!750!. Other values
\begin{verbatim}
'401', '403', '405', '405fp', '440', '440fp', '464', '464fp', '476', '476fp',
'505', '601', '602', '603', '603e', '604', '604e', '620', '630', '740', 
'7400', '7450', '750', '801', '821', '823', '860', '970', '8540', 'a2', 
'e300c2', 'e300c3', 'e500mc', 'e500mc64', 'e5500', 'e6500', 'ec603e', 
'G3', 'G4', 'G5', 'titan', 'power3', 'power4', 'power5', 
'power5+', 'power6', 'power6x', 'power7', 'power8', 'powerpc', 'powerpc64', 'powerpc64le', 'rs64'.
\end{verbatim}
  \url{https://gcc.gnu.org/onlinedocs/gcc/RS_002f6000-and-PowerPC-Options.html}
  
 We can use \verb!-march=! option
 Example: ARM architecture
\begin{verbatim}
'armv2', 'armv2a', 'armv3', 'armv3m', 'armv4', 'armv4t', 'armv5', 'armv5t',
'armv5e', 'armv5te', 'armv6', 'armv6j', 'armv6t2', 'armv6z', 'armv6zk', 'armv6-m', 
'armv7', 'armv7-a', 'armv7-r', 'armv7-m', 'armv7e-m', 'armv7ve', 
'armv8-a', 'armv8-a+crc', 'iwmmxt', 'iwmmxt2', 'ep9312'.
\end{verbatim}
 
 \url{https://gcc.gnu.org/onlinedocs/gcc/ARM-Options.html}
 
 
 Example: Motorolla 68000 series or ColdFire V2
\begin{verbatim}

\end{verbatim}
\url{https://gcc.gnu.org/onlinedocs/gcc/M680x0-Options.html}
\end{itemize}

\verb!VXINCDIR! only for vxWorks.


\verb!X86PC! is not being used anywhere, so can be removed

\verb!BUILD=$(BUILD)! where \verb!$(BUILD)! is not being defined anywhere, so
can be removed.

\verb!_REENTRANT! 
\begin{itemize}
  \item  (up to GCC 3.3.x) could be used on x86 CPU Linux to determine
at compile time whether the compiler was called with \verb!-pthread! option.
Some libraries, e.g. the Boost library, relie on this macro to decide if
\verb!boost::smart_ptr! needs to be make thread-safe.

  \item (from GCC 3.4.0) is automatically defined if C++ standard header file is
  included (NOTE: other headers may cause the symbol to be defined too).
  
NOTE: \verb!_POSIX_THREADS! is defined only if certain headers are included.
It does not matter at all if -pthread was on the compilers command line.

 
\end{itemize}
\url{https://gcc.gnu.org/bugzilla/show_bug.cgi?id=15415}

\subsection{Macro Linux-PPC405}


\begin{enumerate}
  \item \verb!_LINUX_OS_!: if defined, use the proper include file in 
  
 \begin{verbatim}
 <project>/cfe/fsw/cfe-core/src/inc/network_includes.h
 \end{verbatim}
 
   \item \verb!__ppc__!,  \verb!__PPC__!: if defined, it indicates the PowerPC
  architecture (Sect.\ref{sec:macro-GNU-to-detect-CPU} in C/C++ Manual)
  
  \item \verb!$(OS)!: 
  
  \item \verb!_POSIX_C_SOURCE=199309!
  
  \item \verb!__USE_MISC!: The reason is that 
  
\verb!<sys/stat.h>! define \verb!S_IFDIR! only when
\begin{verbatim}
#if defined __USE_BSD || defined __USE_MISC || defined __USE_XOPEN
    # define S_IFDIR        __S_IFDIR
#endif
\end{verbatim}
To get rid of 
\begin{verbatim}
ls2.c:91: `S_IFMT' undeclared (first use in this function)
ls2.c:91: (Each undeclared identifier is reported only once
ls2.c:91: for each function it appears in.)
ls2.c:92: `S_IFREG' undeclared (first use in this function)
ls2.c:93: `S_IFDIR' undeclared (first use in this function)
ls2.c:94: `S_IFCHR' undeclared (first use in this function)
ls2.c:95: `S_IFBLK' undeclared (first use in this function)
\end{verbatim}
try to add \verb!-D_XOPEN_SOURCE=600! to the compiler option.


\begin{itemize}
  \item \verb!__USE_BSD!: uses APIs based on BSD (Sect.\ref{sec:BSD})
  
  \item \verb!__USE_XOPEN!: use APIs based on XPG (Sect.\ref{sec:XPG})
  
  \item \verb!__USE_MISC!: use APIs from both \verb!__USE_BSD! and
  \verb!__USE_SVID!
\end{itemize}
  
    \item \verb!MCP750!: if defined, it means the board is MCP750
  (Sect.\ref{sec:MCP750})
  
  \textcolor{red}{HOW TO remove this and use a different header/source file?}
  
  if defined, it checks the files
\begin{verbatim}
psp/fsw/mip405t-vxworks6.4/src/mcpx750.h:#ifdef MCP750
psp/fsw/mcp750-vxworks6.4/src/mcpx750.h:#ifdef MCP750
psp/fsw/mip405t-linux/src/mcpx750.h:#ifdef MCP750
\end{verbatim}
The header is included in these files
\begin{verbatim}
psp/fsw/mcp750-vxworks6.4/src/cfe_psp_timer.c:#include "mcpx750.h"
psp/fsw/mcp750-vxworks6.4/src/cfe_psp_watchdog.c:#include "mcpx750.h"
psp/fsw/mcp750-vxworks6.4/src/cfe_psp_start.c:#include "mcpx750.h"
\end{verbatim}
  
  
\end{enumerate}

\verb!SYSINCS! is empty
\begin{verbatim}
SYSINCS=
\end{verbatim}
and we remove \verb!VXINCDIR! macro.

As PowerPC CPU, \verb!ENDIAN_DEFS! is the same as that in PPC750
\begin{verbatim}
ENDIAN_DEFS=-D_EB -DENDIAN=_EB -DSOFTWARE_BIG_BIT_ORDER
\end{verbatim}
Except, modify \verb!ARCH_OPTS! to use \verb!-mcpu=405!
\begin{verbatim}
ARCH_OPTS = -mcpu=405 -mstrict-align -fno-builtin
APP_EXT = so
\end{verbatim}



Cross-compiler:
\begin{verbatim}
COMPILER   = $(CROSS_COMPILE)cc
ASSEMBLER  = $(CROSS_COMPILE)cc
LINKER    = $(CROSS_COMPILE)ld
AR              = $(CROSS_COMPILE)ar
NM         = $(CROSS_COMPILE)nm
OBJCPY     = $(CROSS_COMPILE)objcopy
OBJDUMP    = $(CROSS_COMPILE)objdump
\end{verbatim}


When you compile it may asks for these files
\begin{verbatim}
vxworks.h
sysLib.h
taskLib.h
ramDrv.h
dosFsLib.h
xdbBlkDev.h
errnoLib.h
usrLib.h
cacheLib.h
drv/hdisk/ataDrv.h
\end{verbatim}
\url{http://opensource.apple.com/source/gdb/gdb-282/src/gdb/vx-share/vxWorks.h}
\url{ftp://ftp.desy.de/pub/EPICS/vx_GPFC/h/vxWorks.h}
\url{http://svn.gna.org/svn/xenomai/trunk/include/vxworks/vxworks.h}

\section{Embedded Boards}
\label{sec:embedded-boards}


An embedded board includes
\begin{enumerate}
  \item ROM (can be PROM, EPROM or Masked): persistent storage
  
  writing to EPROM is possible, but require rewrite the entire chip, but
  also with a limited number of read/write cycles 
  
  
  \item RAM (DRAM or SRAM): temporarily, volatile storage
  
  \item Hybrid:
  \begin{itemize}
    \item EEPROM:
    
   more expensive than Flash, write per byte, also with a limited number of
   read/write cycles
       
    \item Flash: combine the best of ROM and RAM
    
   less expensive the EEPROM, write per sector, also with a limited number of
   read/write cycles
       
    \item NVRAM:
    
   expensive, write per byte, and unlimited number of read/write cycles 
  \end{itemize}
  \url{http://www.barrgroup.com/Embedded-Systems/How-To/Memory-Types-RAM-ROM-Flash}
  
   
  \item embedded CPU
  \item I/O port
  \item Ethernet port
\end{enumerate}

\subsection{MCP750}
\label{sec:MCP750}

  \begin{verbatim}
  CPU: Motorola MPC750 233MHz, 1MB L2 cache, 4-8MB FLASH ROM, RAM 16-256MB, 
       64-bit  CompactPCI interface 
  
  Instruction set (architecture): Freescale/IBM 7xx
  \end{verbatim}
  \url{https://bsp.windriver.com/index.php?bsp&on=details&bsp=138}
  \url{http://www.artisantg.com/TestMeasurement/63140/Emerson_Motorola_MCP750_MCP750HA_CompactPCI_Single_Board_Computer}

\subsection{MCF525x}


\begin{verbatim}
CPU: Freescale ColdFire v2 32-bit, 140MHz, 


Instruction set: compatible Motorola 68000 series 
\end{verbatim}

\url{http://www.freescale.com/webapp/sps/site/prod_summary.jsp?code=MCF525X}

\subsection{Arduino}
\label{sec:Arduino}

An Arduino is a microcontroller motherboard. A microcontroller is a simple
computer that can run one program at a time, over and over again.
Tasks: opening and closing a garage door, reading the outside temperature and
reporting it to Twitter, driving a simple robot.

The  Arduino Uno differs from all preceding boards in that it does not use the
FTDI USB-to-serial driver chip. Instead, it features the Atmega16U2 (Atmega8U2
up to version R2) programmed as a USB-to-serial converter.

\url{https://www.arrow.com/en/products/a000066/arduino-corporation}




\subsection{Beaglebone}
\label{sec:BeagleBone}

BeagleBone has its roots in Texas Instruments.
These boards have always been popular as a high-end alternative to an Arduino (Sect.\ref{sec:Arduino}).

They are great for those situations where you need to attach a screen or need
some serious processing power.

\begin{verbatim}
CPU:  AM335x 1GHz ARM Cortex-A8 32-bit, 4GB 8-bit FLASH ROM,

Instruction set: ARMv7-A architecture 
(A=application profile, R=realtime, M=microcontroller)
\end{verbatim}
\url{http://beagleboard.org/BLACK}

\subsection{MPL MIP405T}
\label{sec:MPL_MIP405T}

Chap.\ref{chap:embedded_MPL_MIP405-MIP405T}

\subsection{Adapteva Epiphany Parallella}

This affordable platform is designed for developing and implementing high
performance, parallel processing.

The Epiphany 16 or 64 core chips consists of a scalable array of simple RISC
processors programmable in C/C++ or a parallel programming framework like OpenCL.
The mesh of independent cores are connected together with a fast on chip network
within a single shared memory architecture.

As Parallella has no GPU, you can connect to RasperiPi
\url{http://forums.parallella.org/viewtopic.php?f=34&t=708}

\subsection{Raspberry Pi}
\label{sec:Raspberry-Pi}

Raspberry Pi is a general-purpose computer, usually with a Linux operating
system, and the ability to run multiple programs. It is more complicated to use
than an Arduino (Sect.\ref{sec:Arduino}).

\begin{mdframed}

The Arduino Uno and the Raspberry Pi 3 are popular choices when it comes to DIY,
IoT, or just fun engineering projects. They can be used for prototyping and
real-world engineering solutions


\end{mdframed}

\begin{enumerate}
  \item Raspberry Pi 3 Model B+
  
\begin{verbatim}
1.4GHz 64-bit quad-core processor, 

dual-band wireless LAN, 

Bluetooth 4.2/BLE, faster Ethernet, and 

Power-over-Ethernet support (with separate PoE HAT)
\end{verbatim}

\url{https://www.raspberrypi.org/products/raspberry-pi-3-model-b-plus/}


  \item Raspberry Pi Zero W - (RPZW) (released 2018): as an upgrade to the
  original Raspberry Pi Zero (RPZ) and a “smaller brother” to the Raspberry Pi
  3, i.e.
  20\% slower than the RP3. Single-core can handle easier applications, like
  surfing the web, watching videos, or checking emails.
  
\begin{verbatim}
CPU: Broadcom BCM2835 SoC 
[1 GHz 32-bit ARM1176JZF-S core] (single-core)

GPU: Broadcom VideoCore IV.
(similar to RP3]


RAM: 512 MB memory 

1 micro-USB port
[simply adding the Zero4U USB hub to make more USB]

a height of 1.18 inches and width of 2.55 inches. 
weight 9g

integrate: 2.4GHz 802.11n wireless LAN and Bluetooth Classic 4.1
no Ethernet

display via mini-HDMI
[1080p HD video and stereo audio.]

analog video: accessed by an unpopulated pin on the board.

no support LCD panel


~ 10$
\end{verbatim}

  The RPZ and RPZW have very similar specifications, the only difference being
  that the RPZW has built-in Wi-Fi while the RPZ has no Wi-Fi connectivity
  unless you add a dongle.
  
  \ite Raspberry Pi 3 model B - RP3 (released Feb, 2016): A quad core can handle more
  processing load better, like video editing, GPS systems, or audio/video
  chatting.
  
 The first 64-bit product; and to intergrate wireless connectivity of Raspberry

\begin{verbatim}
CPU: Broadcom BCM2837 SoC
[1.2 GHz 64-bit ARM Cortex-A53 core] (quad-core)

GPU: Broadcom VideoCore IV.
(similar to RPZW]

RAM: 1 GB memory 
(shared with the GPU)

4 USB 2.0 ports

a height of 2.22 inches and width of 3.37 inches.
weight 45g


integrate: 2.4GHz 802.11n wireless LAN and Bluetooth Classic 4.1
one 10/100 Mbit/s Ethernet connectivity

display via HDMI rev 1.3
[1080p HD video and stereo audio.]

analog video: sharing the analog video output with its audio jack

support LCD panel
[touchscreen LCD easily]

~ 35$


\end{verbatim}  
  
  \item Raspberry Pi Model B: (released Oct, 2014) uses quad-core 900 MHz ARM chip and
512 GB RAM.
\url{http://www.digitaltrends.com/computing/raspberry-pi-2-hands-on-more-than-a-hobby/}

Beside enhanced processing, the Raspberry Pi 1 Model B+ added two additional USB ports.  


  \item Raspberry Pi 2: (\$ 75, released 2014) uses quad-core 900 MHz ARM Cortex-A7
chip and 1 GB LPDDR2 RAM. The cost to start with Pi 2: a power adapter for \$9,
a 16GB microSD card for \$8, and a mouse and keyboard combo for \$15. A case is another \$8, though we went without. All
told, then, the final price (without monitor) is around \$75. Not bad, for sure,
but more than double the \$35 MSRP that makes headlines.


  \item Raspberry Pi: (\$ 35, released early 2012)
its broad base of support and ability to run variants of common Linux
distributions, yet it uses single-core 700 MHz ARM chip and 256 MB RAM (slow).
There's no power adapter, no microSD card, not even a case. All of that must be purchased individually.


\end{enumerate}


\url{http://www.adapteva.com/parallella-board/}

\subsection{-- using Raspberry Pi}


The Raspberry Pi models do NOT come with an operating system. You must either
buy a blank SD card (4 GB or more) and install your own disk image onto it using
one of many tools available to do this from your PC, Mac or Linux box.

The O/S for Raspberry Pi: the 'standard' and most used distribution for
Raspberry Pi is 'Raspbian' which is based on Debian.

\begin{mdframed}

BeagleBone Black has on-board flash storage, with  Ångström Linux pre-installed,
making it fully ready to go, straight from the box.
Ångström is fast and small and will allow the BeagleBone Black to boot and
establish a network connection in about 10 seconds, compared with 20 or 30
seconds for a Raspberry Pi running Raspbian.


Ångström Linux is not a particularly popular distribution, and the makers of the
BeagleBone Black also provide a micro-SD card slot on the underside of the
board, which can be used either to boot directly onto an operating system
installed on a micro SD card, or with the press of a button, to upload a disk
image from the micro SD card directly into the 2 GB on-board storage.
\end{mdframed}


The Raspberry Pi was intended to be used as a tool for kids to learn
programming. The Raspbian distribution includes Python (2 and 3) as well as
Scratch, the visual programming environment and of course C / C++. Indeed the name 'Pi' is a reference to the language 'Python'. 


\subsection{Tinker Board (ASUS)}
\label{sec:TinkerBoard-Asus}

Asus recently launched the Tinker Board, a mini PC similar to the Raspberry Pi.

It features a similar layout to the Raspberry Pi 3 Model B, with four USB 2.0
ports and an array of GPIO pins. The Asus board boasts more power than the
Raspberry Pi 3, thanks to its faster processor and 2GB of RAM.
\begin{verbatim}

Cortex-A17 Quad-core 1.8GHz	
 	vs. (Raspberry Pi 3) Cortex-A53 Quad-core 1.2GHz

2GB LPDDR3	
	vs. 1GB LPDDR2
	
Wi-Fi 802.11b/g/n, Bluetooth 4.0, Gigabit LAN
	vs. Wi-Fi 802.11n, Bluetooth 4.1, 10/100Mbit/s Ethernet
\end{verbatim}

Tinker Board uses maximum 5W compared to the Raspberry Pi 3’s maximum of 4W.

Asus Tinker Board provides DIY enthusiasts with a powerful SoC and has several
advantages over the Raspberry Pi, including 4K support.



\subsection{Stamp9G20 }
\label{sec:Stamp9G20}

 one of the smallest ARM9 Computer-on-Module (CoM) available on the market.
 \begin{verbatim}
CPU: Atmel AT91SAM9G20 396 MHz, 32KB Inst-Cache, 32KB Data-Cache
     3.3V
Flash: 128MB NAND flash memory
RAM: 64 MB SDRAM
     2x16 KB SRAM
     128bytes EEPROM       
 \end{verbatim}
\url{http://www.taskit.de/tl_files/cust/dl/stamp9g20technicalreference_EN.pdf}

\section{Architecture-specific exception handling}


\url{http://www.vxdev.com/docs/vx55man/vxworks/ref/excArchLib.html}
Exception Stack Frame (ESF) \verb!ESFPPC! is a structure designed for PowerPC
used to save the registers is defined in \verb!h/arch/ppc/esfPpc.h! header file

\begin{verbatim}
 void excHandler (ESFPPC *);
\end{verbatim}

\section{Flash an image to flash memory on embedded board}

If you have a NOR-Flash based system use \verb!flashcp! instead of
\verb!nandwrite!.

Example: (board Stamp9G20 or PortuxG20)
Suppose the IP of the embedded system is \verb!192.168.4.238!
\begin{verbatim}
mount -o nolock,tcp 192.168.4.238:/develop /mnt/net
flash_eraseall /dev/mtd5
   Erasing 128 Kibyte @ 1fd60000 -- 100 % complete.

nandwrite /dev/mtd5 /mnt/net/jffs2-root.img
   Writing data to block 137 at offset 0x1120000
\end{verbatim}

Now reboot, interrupt booting with u-boot, type run flashboot to boot your
newly created image on Stamp9G20 or PortuxG20.
\url{http://armbedded.eu/node/175}

\section{FPGA (field-programmable gate arrays)}
\label{sec:FPGA}

Programming on FPGA mainly uses VHDL or Verilog. A Python-based system - PyCPU
converts very, very simple Python code (.py) into either VHDL or Verilog (.vhd
or .v file).
\url{https://pycpu.wordpress.com}

The current support is minimal, with only ints being the only built-in data type
supported. Of course ifs and whiles are still included along with all the
assignments and operators. A new addition is a way to get digital IO access with
Python, and obvious requirement if you’re going to be programming Silicon.


\subsection{MyHDL}

MyHDL is an open source, pure Python package. 
\url{http://www.myhdl.org/docs/examples/}