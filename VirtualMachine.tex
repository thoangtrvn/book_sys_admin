\chapter{Virtual Machines}
\label{chap:virtual-machine}

A virtual machine instance is the realization of an O/S running ontop of another
O/S using a hypervisor (Sect.\ref{sec:hypervisor}) running on a VPS
(Sect.\ref{sec:VPS}).

\section{VPS (Virtual Private Server) or a virtual dedicated server (VDS)}
\label{sec:VPS}
\label{sec:VDS}

A virtual private server (VPS) or a virtual dedicated server (VDS) is a {\it
virtual machine} that appears to user as a dedicated server, though the server
is shared by more than one user. A single physical computer can have several
VPSs, each one with its own operating system (OS) that runs the hosting software
for a particular user. This is different from Cloud, which is also a virtual
environment, yet a cloud environment allows you to more easily add more
resources to your server, such as RAM, processors or even cloned copies of your
server for data back-up (Sect.\ref{sec:Cloud-hosting}).

A singler customer has full control to a VPS, i.e. with superuser-level access
to that operating system instance, so they can install almost any software that
runs on that OS. VPS is {\bf sold as a service} by an Internet hosting service
(Sect.\ref{sec:VPS-web-hosting}). 

So, as a user, you don't need a dedicated physical server, but you can share
that with other users; while it gives the user a full control of a separate
machine. Because of that, performance may be lower, depending on the workload of
any other executing virtual machines.

Partitioning a single server to appear as multiple servers has been increasingly
common on microcomputers since the launch of VMware ESX Server in 2001. 
This physical server needs to run a {\bf hypervisor}
(Sect.\ref{sec:hypervisor}).

VPSs are usually based on server virtualization, which should be able to keep
VPSs adequately isolated from one another. Even so, those with significant
security requirements commonly avoid multi-tenant environments as a best
practice.  

\subsection{VPS hosting}
\label{sec:VPS-web-hosting}

VPS hosting puts your website on a server (Sect.\ref{sec:VPS}) that also has
other sites running on it, except that there are fewer sites per server.
You can expect to pay between \$20 per month and \$100 per month, depending on
the configuration. This is different from
\begin{enumerate}
  \item dedicated web hosting
  Dedicated web hosting will typically set you back \$100 per month or more.
  
  \item shared web hosting:
  Shared web hosting, on the other hand, is incredibly cheap hosting; you can
  often set up shop for less than \$10 per month.
  
  
\end{enumerate}

The sites share the cost of running on the server, which results in a monthly or
yearly charge that's less than the relatively high price tag of dedicated hosting.

Most VPS hosts only offer servers running Linux-based operating systems.
Popular companies: GoDaddy, Liquid Web and HostGator.

A solid web host should offer at least 4GB of RAM, 100GB of storage, and an
ample volume of monthly data transfers.
\url{https://www.pcmag.com/article2/0,2817,2455706,00.asp}




\subsection{Cloud hosting}
\label{sec:Cloud-hosting}


In a Cloud hosting, we rent a small portion of a Big network of machines
connected together. A giant setup of machines that offer resources are connected
together and then they are further rented to clients.
You can always scale up your resources as per your requirements and it will be
done in a few minutes of clicks.

In a Cloud environment, you are totally separated from others, i.e. you get
exclusively access to a hardware; unlike in VPS.
You will be getting dedicated resources of what you have been told. But In a
VPS, you get a virtual environment with portioned disk space \& certain
bandwidth, however the CPU \& the memory of the parent machine is equally
distributed among all VPS slices.
\url{https://www.quora.com/What-is-the-difference-between-cloud-hosting-and-VPS-hosting}


In Cloud hosting, the downtime, if happens, is minimal. 
In a Cloud, things are pretty fast. Every server slot is available and is always
ready to be served in seconds. But in a VPS, it takes time to virtualized a
dedicated server and then allot VPS slices to the client. 



\section{Virtual Hard Drive}


\begin{itemize}
  \item VDI (VirtualBox Disk Image): native format of VirtualBox
  
  \item VMDK (Virtual Machine Disk): developed by VMWare, but Sun xVM, QEMU
  (Sect.\ref{sec:QEMU}), VirtualBox (Sect.\ref{sec:VirtualBox}), SUSE Studio and
  .NET DiscUtils support.
  
  \item VHD (Virtual Hard Disk): native format of Microsoft Virtual PC (popular
  in Windows)
  
  \item HDD (Parallels Hard Disk 2): initially designed for Mac O/S
  \item QED (QEMU enhanced disk)
  \item QCOW (QEMU copy-on-write)
\end{itemize}

Features
\begin{itemize}
  \item dynamically allocated sizing: VDI, VMDK, VHD
  
  VMDK has an additional capability of splitting the storage file into files
  less than 2GB each, i.e. useful if your system has a small file size limit.
  
  \item have snapshots: VDI, VMDK, VHD, HDD
  
  \item able to move the virtual machine to another O/S, with minimal effort:
\end{itemize}
  
\section{Windows host}

\url{http://en.wikipedia.org/wiki/Comparison_of_platform_virtualization_software}
\begin{itemize}
  \item AlphaVM
  \item Bochs
  \item CHARON
  \item DOSBox
  \item Hercules
  \item Hyper-V (2008): Windows Server 2008 with Hyper-V role, Microsoft Hyper-V
  server
  \item Hyper-V (2012): Windows 8/8.1, Windows Server 2012 \ldots
  
  Hyper-V products support hardware-assisted virtualization (Intel VT-x or
  AMD-V) that runs virtual O/S faster

  \item QEMU: Sect.\ref{sec:QEMU}
    
  \item SIMH
  \item Simics
  \item VirtualBox: Sect.\ref{sec:VirtualBox_howto}
  
  VirtualBox also support hardware-assisted virtualization (Intel VT-x
  and AMD-V)
  
  \item Virtual PC 2007: Windows Vista, XP Pro, XP Tablet PC Edition
  
  \item Windows Virtual PC: Windows 7
  \item Virtual Server 2005 R2: Windows Server 2003, 2008
  
  \item Synopsys (CoWare) Virtual Platform:
  
  \item VMWare server
  \item VMWare workstation 10.0.2
  \item VMWare Player 6.0.2
\end{itemize}

\section{Linux host}

\url{http://en.wikipedia.org/wiki/Comparison_of_platform_virtualization_software}

\section{Mac OS host}

\url{http://en.wikipedia.org/wiki/Comparison_of_platform_virtualization_software}

\section{Virtual Networking Hardware}
\label{sec:virtual_networking}

To enable network access from the guest O/S, we need to enable
\begin{verbatim}
Enable Network Adapters
\end{verbatim}
in the Settings/Networks section of the guest O/S instance.

VirtualBox (Sect.\ref{sec:VirtualBox_howto}) supports 6 virtual network
card type
\begin{enumerate}
  \item AMD PCNet PCI II (Am79C970A);

  \item AMD PCNet FAST III (Am79C973, the default): work with many O/S and GNU
  GRUB boot manager.

  \item Intel PRO/1000 MT Desktop (82540EM): works with Windows Vista and later
  versions

  \item Intel PRO/1000 T Server (82543GC): recognized by Windows XP without
  additional driver installation.

  \item Intel PRO/1000 MT Server (82545EM): facilitates OVF imports from other
  platforms

  \item Paravirtualized network adapter (virtio-net): available for Linux kernel
  2.6.25+. In other guest O/S (Windows 2000, XP and Vista), we need to download
  virtio driver from KVM project.
  
  Starting with version 3.1, VirtualBox provides support for the
  industry-standard "virtio" networking drivers, which are part of the
  open-source KVM project. 
\end{enumerate}
which can be attached to one of the seven networking modes
\begin{enumerate}
  \item Not attached: VB reports to the guest O/S that a network card
  (selected above) is present, but not connected
  
  
  \item NAT (Network Address Translation): this mode is good enough for Web
  browsing, download files and view e-mail inside guest O/S
  
  LIMITATIONS:
  \begin{itemize}
    \item some tools that rely on ICMP protocol may not working in guest O/S
    \item not reliable to receive UDP broadcast message
    \item tools that rely on GRE protocols or VPN products (e.g. PPTP) are not
    supported
    \item port forwarding to host port < 1024 is not possible for applications
    that are not run by \verb!root!.
  \end{itemize}
  \url{https://www.virtualbox.org/manual/ch06.html#nat-limitations}
  
  \item NAT network:  the experimental version of NAT
  
  \item \textcolor{red}{Bridge Networking}: more advanced networkings to be used
  in network simulation and running servers in a guest.
  
  Bridging to a wireless interface is done differently from bridging to a wired
  interface, because most wireless adapters do not support promiscuous mode.
  
  \item Internal Network: create a software-based network which is visible to
  selected virtual machines, but not to applications running on the host or to
  the outside world
  
  \item Host-only Adapter: create a network containing the host and a set of 
  virtual machines, without using the host's physical network interface.
  
  Here, a virtual network interface (similar to loopback interface) is created
  on the host, providing connectivity among virtual machines and the host.
  
  \item Generic driver: rarely used mode
\end{enumerate}

\section{VMWare Player}
\label{sec:VMwarePlayer}

If you get error when mounting a shared folder
\begin{verbatim}
sudo apt-get install build-essential linux-headers-$(uname -r)
sudo apt-get install build-essential linux-headers-$(uname -r)

// reinstall the VMWare tool
sudo ./vmware-install.real.pl
\end{verbatim}

To resize the partition, you need
\begin{itemize}
  \item resize from the VMWare Player: to make the image file bigger.
  This can be done either (1) command-line, or (2) from VMWare Player in the later version

\begin{verbatim}
vmware-vdiskmanager -x 100Gb vm.vmdk
\end{verbatim}  
NOTE: Use \verb!Kb! or \verb!Mb! or \verb!Gb!.
\url{http://kb.vmware.com/selfservice/microsites/search.do?language=en_US&cmd=displayKC&externalId=1004047}
  
  
  \item resize the partition in the guest O/S: this can be done using
  GParted LiveCD \url{http://gparted.org/download.php}
  
  NOTE: You may need to run GParted from the guest O/S first to move some partitions to
  make enough space for the primary one, then boot from the GParted LiveCD
  to resize the primary partition.
  
  NOTE: Once you reboot the guest O/S, press F2 to change the boot order to CD ROM
  before Physical Disk, and then quickly connect the CD ISO File
  from the Menu option of VMWare Player.
  \url{http://xmodulo.com/how-to-change-the-boot-order-of-guest-vm-on-vmware-player.html}
\end{itemize}

\section{VirtualBox} 
\label{sec:VirtualBox_howto}

VirtualBox's 64-bit guest support (added with version 2.0) and 
multiprocessing (SMP, added with version 3.0) both require hardware 
virtualization to be enabled. (This is not much of a limitation since the 
vast majority of today's 64-bit and multicore CPUs ship with hardware 
virtualization anyway; the exceptions to this rule are e.g. older Intel 
Celeron and AMD Opteron CPUs.)".

VT is required to install 64-bit guest O/S. To enable VT, we need to restart and
adjust BIOS (UEFI) setting. Check if VT (Virtualization Technology) is enabled ?
\url{http://yoyoclouds.wordpress.com/2012/04/25/how-to-check-whether-your-pc-has-virtualization-enabled-or-not/}


Once the guest O/S is installed, we need to install {\bf Guest Additions}
(mount the VBoxGuestAdditions.iso file) which is important  as it consist of
device drivers and system applications that optimize the guest operating system for better performance and usability
\url{https://www.virtualbox.org/manual/ch04.html\#idp48295392}
\begin{itemize}
  \item share folder with host O/S
  
  \item mouse pointer integration: move out of or into the guest O/S
  transparently
  
  \item better video support: 
  
  NOTE: We do not need to install Nvidia proprietary driver, once Guest
  Additions is installed, it should install the correct video driver from
  there.
  
\end{itemize}

\url{http://trivialproof.blogspot.com/2011/01/resizing-virtualbox-virtual-hard-disk.html}

\subsection{OpenGL and graphics}

Run the command
\begin{verbatim}
cd /media/VBOXADDITIONS*
sudo sh ./VBoxLinuxAdditions.run
\end{verbatim}

Run the command and restart the machine
\begin{verbatim}
sudo apt-get install virtualbox-guest-dkms \
           virtualbox-guest-utils virtualbox-guest-x11
\end{verbatim}
\url{http://askubuntu.com/questions/451805/screen-resolution-problem-with-ubuntu-14-04-and-virtualbox}

The /etc/xorg.conf file
{\tiny
\begin{verbatim}
#VirtualBox Linux Guest xorg.conf

Section "Device"
   Identifier   "Configured Video Device"
   Driver      "vboxvideo"
EndSection

Section "Monitor"
   Identifier   "Configured Monitor"
EndSection

Section "Screen"
   Identifier   "Default Screen"
   Monitor      "Configured Monitor"
   Device      "Configured Video Device"
EndSection

Section "InputDevice"
        Identifier  "vboxmouse"
        Driver          "vboxmouse"
        Option          "CorePointer"
        Option          "Device"        "/dev/input/mice"
EndSection

Section "ServerLayout"
        Identifier      "Default Layout"
        Screen          "Default Screen"        0 0
        InputDevice     "vboxmouse"
EndSection
\end{verbatim}
}

To add VMGL (3D Acceleration support), we add another part
\begin{verbatim}
#VirtualBox Linux Guest xorg.conf with 3D Acceleration Support

Section "Module"
    Load "vmglext"
EndSection
\end{verbatim}

\url{http://forum.notebookreview.com/threads/190-42-nvidia-driver-ubuntu-virtualbox.436427/}


\url{https://www.virtualbox.org/manual/ch04.html\#guestadd-3d}


\subsection{VirtualBox Python}

\url{http://stackoverflow.com/questions/2652146/what-is-the-advantage-of-using-python-virtualbox-api}

\subsection{Resize harddisk}

Currently, resize VDI is supported.

Virtualbox has a command line \verb!VBoxManage.exe!
\begin{itemize}
  \item Configure the path to PowerShell script 
\begin{verbatim}
C:\Program Files\Oracle\VirtualBox
\end{verbatim}
  \item Suppose the image file is in 
\begin{verbatim}
C:\Users\thoangtr\VirtualBox VMs\UDesktop
\end{verbatim}

  \item Open Powershell, jump to that location, and type
\begin{verbatim}
VBoxManage clonehd "source.vmdk" "cloned.vdi" --format vdi
VBoxManage modifyhd "cloned.vdi" --resize 51200
VBoxManage clonehd "cloned.vdi" "resized.vmdk" --format vmdk
\end{verbatim}

  \item Remount the disk
  
If you get UUID already registered error, then you need to delete the VM
instance, restart and recreate a new Ubuntu guest O/S instance, but use an
existing VM file instead of creating a new file.

  \item Next-step is to resize the partition, which can be done using either (1)
  gparted booted form the ISO, (2) when the guest O/S is running
  
  \url{http://askubuntu.com/questions/24027/how-can-i-resize-an-ext-root-partition-at-runtime}
\end{itemize}
\url{http://stackoverflow.com/questions/11659005/how-to-resize-a-virtualbox-vmdk-file}

\url{http://derekmolloy.ie/resize-a-virtualbox-disk/}


\subsection{Mount shared folder with host O/S}

\url{https://www.virtualbox.org/manual/ch04.html\#sharedfolders}

When VM is running, click on VirtualBox menu 
\begin{verbatim}
Devices
  Shared Folder settings...
\end{verbatim}

The shared folder is mounted in
\begin{verbatim}
/media/sf_<name>
\end{verbatim}
We need to reboot the machine for the shared folder to be seen.

However, the owner is \verb!root:vboxsf! and thus it cannot be accessible by the
current owner. The solution is to add the current user to the \verb!vboxsf!
group
\begin{verbatim}
sudo gpasswd -a <username> vboxsf
\end{verbatim}
and reboot the system or re-login. 

\subsection{Troubleshoot}

\begin{verbatim}
kernel panic - not syncing fatal exception
\end{verbatim}
Solution:
\begin{verbatim}
Go to your virtual machine -> Settings -> System
Then change Chipset from PIIX3 to ICH9 and enable "Enable IO APIC"
\end{verbatim}
\url{https://forums.virtualbox.org/viewtopic.php?f=6&t=54926}

Before VirtualBox 4.0, PIIX3 is the only supported chipset. For newer guest O/S,
it is recommended to use ICH9 chipset, which supports PCI express, three PCI
buses, PCI-to-PCI bridges and Message Signaled Interrupts (MSI)


\section{CUDA-support}

\url{http://www.acceleware.com/blog/state-gpu-virtualization-cuda-applications-2014}


\section{QEMU}
\label{sec:QEMU}

Unlike other virtualization programs (e.g. VirtualBox),
QEMU does not provide a GUI. You need to run from the command-line, and specify
with O/S image file, what configuration you need (e.g. enable NIC, CD-ROM,
\ldots or not) at command-line each time you run, unless you have created a
custom script to start your virtual machine(s). 

\begin{itemize} 
  \item    QEMU w/ kqemu module: is open-source and uses hardware virtualization
  (not as powerful as hardware-assisted virtualization) that emulates CPUs
  through dynamic binary translation.
  
  \item KQEMU was a Linux kernel module to speed up x86 or x86-64 guest on
  platforms of the same CPU architecture.
  
  \item QVM86 was a GNU GPLv2 licensed drop-in replacement for the then
  closed-source KQEMU. The developers of QVM86 ceased development in January, 2007.
  
\end{itemize}

Purchasing ARM development hardware can be an expensive proposition. Thankfully,
the QEMU developers have added the functionality of emulating the ARM processor
to QEMU. You can use QEMU for two purposes in this arena - to run an ARM
program, and to boot and run the ARM kernel.

QEMU runs on x86 systems running Linux, Microsoft Windows, and some UNIX
platforms, and can host target systems from a range of different
microprocessors, i.e.  QEMU can run OSes and programs made for one machine (e.g.
an ARM board) on a different machine (e.g. your own PC). By using dynamic
translation, it achieves very good performance.

\subsection{Build QEMU on host machine to run on a different machine}

\url{http://www.linux-kvm.org/page/PowerPC_Host_Userspace#Building_Qemu_for_PowerPC_KVM}


\subsection{Build QEMU on host machine to run there}


\begin{enumerate}
  \item Download QEMU:
    
  \item Build QEMU as a virtual machine for a target machine:

\begin{verbatim}
./configure --target-list=ppc-softmmu --prefix=/home/devel/bin \
         --disable-vnc   --enable-sdl
\end{verbatim}


NOTE: To build in on MacOS, add option
\begin{verbatim}
./configure --target-list=ppc-softmmu --prefix=/home/devel/bin \
         --disable-vnc   --enable-cocoa 
\end{verbatim}
In Mac OS, to install a package run \verb!port! command
\begin{verbatim}
sudo port install glib-2.12
sudo port selfupdate
sudo port upgrade outdated
\end{verbatim}
\url{https://theintobooks.wordpress.com/2012/10/30/installing-qemu/}
\end{enumerate}



Eventually, you have three types of binary files
\begin{itemize}
  \item \verb!qemu-! (qemu-arm): to execute the binary file on target machine
  
  \item \verb!qemu-system-! (qemu-system-arm): to boot the target O/S

\verb!qemu-system-*! are used to run the virtualized guest
NOTE: \verb!qemu-kvm! is qemu built to take advantage of hardware virtualization
features. \verb!qemu-system-x86_64! is just the emulator without hardware
virtualization features.

\end{itemize}

{\bf EXPLAIN}: If you run \verb!./configure!, is shows all supported target CPU
{\footnotesize
\begin{verbatim}
target list        aarch64-softmmu alpha-softmmu arm-softmmu cris-softmmu i386-softmmu 
                lm32-softmmu m68k-softmmu microblaze-softmmu
               microblazeel-softmmu mips-softmmu mips64-softmmu mips64el-softmmu 
               mipsel-softmmu moxie-softmmu or32-softmmu 
               ppc-softmmu ppc64-softmmu ppcemb-softmmu 
               ppc-linux-user ppc64-linux-user ppc64abi32-linux-user
               ppc64le-linux-user
               s390x-softmmu sh4-softmmu sh4eb-softmmu 
               sparc-softmmu sparc64-softmmu tricore-softmmu unicore32-softmmu 
               x86_64-softmmu xtensa-softmmu xtensaeb-softmmu 
               aarch64-linux-user alpha-linux-user arm-linux-user armeb-linux-user cris-linux-user 
               i386-linux-user m68k-linux-user microblaze-linux-user microblazeel-linux-user 
               mips-linux-user mips64-linux-user mips64el-linux-user mipsel-linux-user 
               mipsn32-linux-user mipsn32el-linux-user or32-linux-user  
               s390x-linux-user sh4-linux-user sh4eb-linux-user 
               sparc-linux-user sparc32plus-linux-user 
               sparc64-linux-user unicore32-linux-user x86_64-linux-user
\end{verbatim}
}
  
E.g.: MIP405T is the target machine (which uses PowerPC cpu)
\begin{verbatim}
./configure --target-list=ppc-softmmu

 # PowerPC 64-bit
./configure --target-list=ppc64-softmmu
\end{verbatim}
which build a binary at \verb!./ppc64-softmmu/qemu-system-ppc64.!

\begin{verbatim}
 # x86 CPU
./configure --target-list=i386-softmmu

 # ARM CPU
./configure --target-list=arm-softmmu
\end{verbatim}

Example: you can build for a few targets
\begin{verbatim}
./configure --target-list="arm-softmmu arm-linux-user"
\end{verbatim}
  
{\bf IMPORTANT}: When you compile, make sure you have support for \verb!sdl!
   or \verb!gtk!
\begin{verbatim}
./configure --disable-vnc --enable-sdl
\end{verbatim}
You may want to install first, run
\begin{verbatim}
sudo aptitude
\end{verbatim}
and (in Ubuntu Hardy 8.x) select \verb!Libsdl1.2-dev!,
\verb!libsdl-image1.2-dev!; or in newer version of Ubuntu
%and nothing happens, then you need to install \verb!sdl-devel!
\begin{verbatim}
sudo apt-get install libsdl-console
sudo apt-get install libsdl-console-dev
\end{verbatim}


%   \item 
%    \verb!qemu-arm! and \verb!qemu-system-arm!, in the source code directory. The
%    first is used to execute ARM binary files, and the second to boot the ARM OS.
  

\subsection{QEMU command-line}
\label{sec:QEMU-command-line-options}

Use \verb!qemu-system-*! for booting the O/S kernel
\begin{verbatim}
Linux/Multiboot boot specific:
-kernel bzImage use 'bzImage' as kernel image
-append cmdline use 'cmdline' as kernel command line
-initrd file    use 'file' as initial ram disk
-dtb    file    use 'file' as device tree image
-cpu <cpu name>
-M <board>       
-bios <output/ppc405_rom.bin>  the location of the BIOS .bin file
                               (choose the right one depending on the target
                               machine)
\end{verbatim}
\begin{itemize}
  \item To find out the list of board supported, type
  either 
\begin{verbatim}
qemu-system-ppc -M ?
qemu-system-ppc -M help
\end{verbatim}

and you use the first column to run
\begin{verbatim}
qemu-system-ppc -M prep
\end{verbatim}

  \item To find out the list of CPU supported, type either 
\begin{verbatim}
qemu-system-ppc  -cpu ?
qemu-system-ppc  -cpu help
\end{verbatim}

 OUTPUT:
\begin{verbatim}
$ qemu-system-ppc -cpu ? | grep 405
PowerPC 405D2            PVR 20010000
PowerPC 405GPa           PVR 40110000
PowerPC 405GPb           PVR 40110040
PowerPC 405CRa           PVR 40110041
PowerPC 405GPc           PVR 40110082
PowerPC 405GPd           PVR 401100c4
PowerPC 405GP            (alias for 405GPd)
PowerPC 405CRb           PVR 401100c5
PowerPC 405CRc           PVR 40110145
PowerPC 405CR            (alias for 405CRc)
PowerPC 405GPe           (alias for 405CRc)
PowerPC Npe405H          PVR 414100c0
PowerPC Npe405H2         PVR 41410140
PowerPC 405EZ            PVR 41511460
PowerPC Npe405L          PVR 416100c0
PowerPC 405D4            PVR 41810000
PowerPC 405              (alias for 405D4)
PowerPC 405LP            PVR 41f10000
PowerPC 405GPR           PVR 50910951
PowerPC 405EP            PVR 51210950
\end{verbatim}  
and you use the second column to run
\begin{verbatim}
qemu-system-ppc -cpu 405GPd
\end{verbatim}
\end{itemize}


Use \verb!qemu-*! for running a program as emulated on the target machine 
\begin{verbatim}
$ qemu -cpu ?
x86           [n270]
x86         [athlon]
x86       [pentium3]
x86       [pentium2]
x86        [pentium]
x86            [486]
x86        [coreduo]
x86          [kvm32]
x86         [qemu32]
x86          [kvm64]
x86       [core2duo]
x86         [phenom]
x86         [qemu64]
x86           [host]
\end{verbatim}  
and we use the second column to specify the CPU type
\begin{verbatim}
qemu -cpu coreduo 
\end{verbatim}

\url{http://wiki.qemu.org/download/qemu-doc.html}

\subsection{Run an embedded program}

Develop the program that runs on that target machine:
\begin{itemize}
   \item Write the code
\begin{verbatim}
#include<stdio.h>
int main(){
    printf("Welcome to Open World\n");
}
\end{verbatim}
   
   \item Compile the code: Configure the proper cross-compiler toolchain:

If you want the code to run directly (without the target O/S) with
\verb!qemu-arm!, then you should compile with \verb!-static!
\begin{verbatim}
arm-none-linux-gnueabi-gcc test.c -o test -static
\end{verbatim}  

  E.g. if you want to build qemu to target PowerPC machine, use the
  cross-compiler that support
   
   \item Run the code
 
 E.g.: on ARM machine  
 \begin{verbatim}
 #qemu-arm -L /your-path/arm-2010.09/arm-none-linux-gnueabi/libc ./test 
 \end{verbatim}
\end{itemize}

\subsection{Create a new virtualized system}
\label{sec:qemu-create-OS-image}

To run QEMU you will need a hard disk image, unless you are booting a live
system from CD-ROM or the network
\url{http://www.aurel32.net/info/debian_arm_qemu.php}
\url{http://opensourceforu.efytimes.com/2011/05/quick-quide-to-qemu-setup/}
\url{http://opensourceforu.efytimes.com/2011/06/qemu-for-embedded-systems-development-part-1/}
\url{http://opensourceforu.efytimes.com/2011/07/qemu-for-embedded-systems-development-part-2/}
\url{http://opensourceforu.efytimes.com/2011/08/qemu-for-embedded-systems-development-part-3/}


\subsubsection{Create a hard-disk image file to host the guest O/S to
use with QEMU}

  QEMU supports several image types
  \begin{itemize}
    \item raw:  
    least I/O overhead, but can waste a lot of space, as not-used space on the
    guest cannot be used on the host.
    
    
    \item cloop
    
    \item cow: copy-on-write format, support for historical reason only
    
    \item qcow: old QEMU copy-on-write, replaced by qcow2
    
    \item qcow2: 
    
    \item vmdk: VMWare 3,4 or 6 image file format
    
    \item vdi: VirtualBox 1.1. compatible file format
  \end{itemize}


You can use the prebuilt for QEMU:
  Aurelien Jarno of Debian has prepared a number of pre-packaged Debian qemu
  images for several architectures, including ARM, Sparc, PowerPC, x86-64, and
  i386: \url{http://people.debian.org/~aurel32/qemu/}
  

If you have an image file, you can test it
\begin{verbatim}
qemu -m 512M -cdrom <isoname>.iso
\end{verbatim}

You can build your own
\begin{enumerate}
  \item First, create a blanc disc image (default: \verb!raw! format)

Create 3GB blanc image with \verb!qcow2! format
\begin{verbatim}
qemu-img create -f qcow2 winxp.img 3G
\end{verbatim}  
It just means that your new system is limited up to 5GB - if the new sytem
takes only 1,2 GB also the image will only be at 1,2GB.

  
  \item Second, we can install the O/S to that blanc image from either (1)
  image file, (2) real CD-ROM

(1) install from ISO file, and use 256MB RAM (\verb!-m 256!)  
\begin{verbatim}
qemu -m 256 -hda winxp.img -cdrom winxpsp2.iso -boot d
\end{verbatim}
NOTE: Default RAM i 128MB, here we use \verb!-m 256! (256MB).
You use \verb!-boot d! to means that you want to boot from the second device
(CD-ROM).

(2) install from a real CD-ROM/DVD (\verb!-cdrom! option and the location)
\begin{verbatim}
qemu -m 256 -hda winxp.img -cdrom /dev/cdrom -boot d 
     -net nic -net user -localtime
\end{verbatim}
\verb!"-user -net"! is important to have internet access within your new system.


The install may take some time. After the install, qemu will try to boot the new
OS itself. If the boot was failed, it's okay. If
that happens: just close the qemu window and type the following command into
your konsole to launch your new OS:
\begin{verbatim}
qemu -m 256 -hda winxp.img -boot c 
     -net nic -net user -localtime
\end{verbatim}
Now, you don't need the CD-ROM.
You use \verb!-boot c! to means that you want to boot from the first device
(\verb!-hda!).


\end{enumerate}

NOTE: \verb!kqemu! improves the speed of guest O/S
\begin{verbatim}
# 32-bit
qemu -m 256 -hda winxp.img -cdrom winxpsp2.iso -kernel-kqemu

# 64-bit
qemu-system-x86_64 -m 256 -hda winxp.img -cdrom winxpsp2.iso -kernel-kqemu
\end{verbatim}

You can run upto 4 image files, with one for the O/S, the other ones for the
storage
\begin{verbatim}
qemu -m 256 -hda winxp.img -hdb pagefile.img -hdc testdata.img 
     -hdd tempfiles.img -kernel-kqemu
\end{verbatim}
This gives additional space to a QEMU guest without reconfiguring the primary
image.

\url{https://en.wikibooks.org/wiki/QEMU/Images}

\subsection{Load O/S from image file}
\label{sec:qemu-load-OS}

You can either (1) build the kernel for the target machine
(Sect.\ref{sec:build_Linux-kernel}), or (2) create an image and install the O/S
on it (Sect.\ref{sec:qemu-create-OS-image}).

You must make sure that the kernel is able to mount the root filesystem. It must
have drivers for the filesystem type and for all the layers involved in the
block device (disk controller (SCSI/SATA/IDE/USB/... adapter), partition type,
etc.).

\begin{itemize}  
  \item Use the final \verb!uImage! file
\begin{verbatim}
# file arch/arm/boot /uImage 
uImage: u-boot legacy uImage, Linux-2.6.37, Linux/ARM, OS Kernel Image (Not compressed), 1575492 bytes, Thu May  5 17:11:30 2011, Load Address: 0x00008000, Entry Point: 0x00008000, Header CRC: 0xFC6898D9, Data CRC: 0x5D0E1B70
\end{verbatim}

  \item Create a  little hard-disk 
\begin{verbatim}
qemu-img create disk.img 512M
mkfs.ext2 -F disk.img
\end{verbatim}

  \item Now we can load the kernel: 3 supported types (uImage, ELF, and raw
  binary blob)

Using \verb!-initrd! is optional which specify the initramfs
(Sect.\ref{sec:initramfs}) or initrd (Sect.\ref{sec:initrd}). If you integrate
the initramfs into the kernel, then you don't need to pass this information.
\begin{verbatim}
qemu-system-i386 -hda disk.img -kernel ../linux-3.3/arch/x86/boot/bzImage
              -initrd my-initramfs.cpio

qemu-system-ppc -m 258M -kernel uImage -nographic 
            -append "console=ttyS0 -serial file:/tmp/qemu-output.log"

\end{verbatim}

NOTE: You can get initrd image 
\url{http://www.aurel32.net/info/debian_arm_qemu.php}


NOTE: We use \verb!"-serial file:/tmp/qemu-output.log"! to get the kernel
dumped to a file outside the virtual system.
This output is particularly helpful if you are having trouble booting the
system, in which case you may also wish to remove "rhgb" and "quiet" from the kernel boot parameters.

% Use QEMU to load the kernel built for target machine, e.g. running ReactOS
% inside QEMU on Linux
% \begin{verbatim}
% $>qemu -hda c.img -m 256
% \end{verbatim}
% Replace linux-0.2.img with the name of your guest OS image file.


 Example:
\begin{verbatim}
 qemu-system-arm -M versatilepb -m 128M -kernel
    /home/manoj/Downloads/linux-2.6.37/arch/arm/boot/uImage
\end{verbatim}

If you get 
\begin{verbatim}
VNC Server running
\end{verbatim}
then make sure you build QEMU with SDL support, and then rerun with options
either
\begin{verbatim}
-gtk
-sdl
-curses	
\end{verbatim}
NOTE:  if system that you're starting has menus in it, so that cant be displayed
in ncurses mode and you need to use SDL.


The kernel will crash at the point where it searches for a root filesystem
(RFS), which you didn't specify in the above command. Potential error messages
\begin{verbatim}
 "No NFS Server available giving up"
 "Root Floppy"
  fix the root option
\end{verbatim}

   \item Create a RFS: Sect.\ref{sec:create-RFS-BusyBox}

When the kernel boots, it mounts rootfs as its filesystem, and starts the hello
   program as init for ARM Versatile Platform Baseboard (versatilepb)
\begin{verbatim}
# qemu-system-arm -M versatilepb -m 128M 
       -kernel /home/manoj/Downloads/linux-2.6.37/arch/arm/boot/uImage 
       -initrd  rootfs -append "root=/dev/ram rdinit=/hello"
\end{verbatim}   
Here, we use \verb!-append! to add addition kernel
command line options, e.g. \verb!"root=/dev/ram"!.

   \url{http://opensourceforu.efytimes.com/2011/06/qemu-for-embedded-systems-development-part-1/}
   
   
   
\end{itemize}

\subsection{Load O/S with GUI}

Once you load an O/S (Sect.\ref{sec:qemu-load-OS}), if it has a GUI and you want
to use your mouse with it, double-click on the window and QEMU will grab your
mouse. To make QEMU release your mouse again, hold down the Control and Alt keys
simultaneously, then let go.


\subsection{emulate ARM}

There are two popular platforms: Integrator or a Versatile platform
\begin{itemize}
  \item Versatile: has a hard disk SCSI controller, an Ethernet card and a
  graphical display.
  
\end{itemize}


\subsection{emulate PowerPC}
\label{sec:QEMU-emulate-PowerPC}

The default machine when not passing \verb!-M! for \verb!qemu-system-ppc! is
g3beige (PowerMac G3) and default cpu is PowerPC 750. To see the list of
supported machine
\begin{verbatim}
qemu-system-ppc -M ?
\end{verbatim}
Output is
\begin{verbatim}
Supported machines are:
bamboo               bamboo
g3beige              Heathrow based PowerMAC (default)
mac99                Mac99 based PowerMAC
mpc8544ds            mpc8544ds
none                 empty machine
ppce500              generic paravirt e500 platform
prep                 PowerPC PREP platform
ref405ep             ref405ep
taihu                taihu
virtex-ml507         Xilinx Virtex ML507 reference design
\end{verbatim}


The default machine when not passing \verb!-M! for \verb!qemu-system-ppc64! is
\begin{itemize}
  \item mac99 (PowerMac G3 ): which is not being actively maintained, and represents a
bizarre hybrid of components that never actually existed as a real system.
 
  \item pseries: which is actively maintained and works well with most modern
  ppc64 Linux distributions as a guest
  
  \url{https://lists.gnu.org/archive/html/qemu-ppc/2013-06/msg00560.html}

{\footnotesize  
\begin{verbatim}
---
 hw/ppc/mac_newworld.c | 3 ---
 hw/ppc/spapr.c        | 1 +
 2 files changed, 1 insertion(+), 3 deletions(-)

diff --git a/hw/ppc/mac_newworld.c b/hw/ppc/mac_newworld.c
index 61c25a4..d8e4db3 100644
--- a/hw/ppc/mac_newworld.c
+++ b/hw/ppc/mac_newworld.c
@@ -458,9 +458,6 @@ static QEMUMachine core99_machine = {
     .desc = "Mac99 based PowerMAC",
     .init = ppc_core99_init,
     .max_cpus = MAX_CPUS,
-#ifdef TARGET_PPC64
-    .is_default = 1,
-#endif
     DEFAULT_MACHINE_OPTIONS,
 };
 
diff --git a/hw/ppc/spapr.c b/hw/ppc/spapr.c
index 218ea23..5363c3f 100644
--- a/hw/ppc/spapr.c
+++ b/hw/ppc/spapr.c
@@ -971,6 +971,7 @@ static void ppc_spapr_init(QEMUMachineInitArgs *args)
 static QEMUMachine spapr_machine = {
     .name = "pseries",
     .desc = "pSeries Logical Partition (PAPR compliant)",
+    .is_default = 1,
     .init = ppc_spapr_init,
     .reset = ppc_spapr_reset,
     .block_default_type = IF_SCSI,
-- 
1.8.1.4
\end{verbatim}
}
\end{itemize}


\url{http://ports.ubuntu.com/ubuntu-ports/dists/precise/main/installer-powerpc/current/images/powerpc/netboot/}
download: mini.iso, initrd.gz, vmlinux

and run
\begin{verbatim}
qemu-system-ppc -m 1024 -kernel vmlinux -cdrom mini.iso -boot d
\end{verbatim}
We need to use \verb!-kernel! option (with the kernel image file), and 
\verb!-initrd! 


Emulate netbsd/P2020 platform
\begin{verbatim}
qemu-system-ppc -cpu e500v2 -m mpc8544ds -cdrom <my_bootable.iso) -boot d,
\end{verbatim}
mpc8544ds is the board, \verb!e500v2! is the Book E FreeScale MPC85xx (e500
core) CPU on board.

\textcolor{red}{PPC64}:
Start the blank image to install the guest O/S
\begin{verbatim}
qemu-system-ppc64 -nographic -hda debian-sid-ppc64.qcow2 -kernel vmlinux -initrd
initrd.gz -append "console=ttyPZ0 libata.dma=0 debian-installer/allow_unauthenticated=true"
\end{verbatim}
\begin{itemize}
  \item \verb!console=ttyPZ0!: needed to get the console working when using
  \verb!-nographic!
  
  \item \verb!libdata.dma=0!: disable DMA on ATA controller (making the
  controller more stables)
  
  \item \verb!debian-installer/allow_unauthenticated=true!: disable security
\end{itemize}

Start the image with installed guest O/S
\begin{verbatim}
qemu-system-ppc64 -nographic -hda debian-sid-ppc64.qcow2 -kernel
vmlinux-3.2.0-4-powerpc64 -initrd initrd.img-3.2.0-4-powerpc64 -append "console=ttyPZ0 libata.dma=0 root=/dev/sda3"
\end{verbatim}
\url{http://www.themccallums.org/nathaniel/2013/03/30/qemu-with-powerpc64-guests/}

\url{http://jk.ozlabs.org/blog/post/157/powerpc-testing-without-powerpc-hardware/}

\subsection{Networking}

\url{http://www.nuke24.net/docs/2012/QEMUNetworking.txt}

\begin{verbatim}
qemu-system-ppc
-netdev socket,id=mynetdev123,udp=localhost:6666,localaddr=localhost:6667 \
	  -device e1000,netdev=mynetdev123 \
	  debian_squeeze_i386_standard.bin
\end{verbatim}

\subsection{Troubleshooting}
\label{sec:QEMU-troubleshooting}

\begin{verbatim}
Disabling PIE due to missing toolchain support
\end{verbatim}
When you try to compile QEMU. 



\begin{verbatim}
Unable to find PowerPC CPU definition
\end{verbatim}
SOLUTION: make sure you use the name from the second column of the list of
supported CPU (the list can be displayed using \verb!-cpu ?! option)


\begin{verbatim}
qemu: hardware error: Bus model not supported on OldWorld Mac machine
\end{verbatim}
The default mache model for qemu-system-ppc is g3beige.

\begin{verbatim}
qemu: hardware error: Only 6xx bus is supported on PREP machine
\end{verbatim}
\url{http://git.greensocs.com/fkonrad/mttcg/commit/dd37a5e4d7ebc4e698f4c69ad2a5ee922824703f}


\begin{verbatim}
qemu-system-ppc: Could not load PowerPC BIOS 'ppc405_rom.bin'
\end{verbatim}
SOLUTION: Download the file
\url{https://code.google.com/p/cellos/source/browse/trunk/output/?r=2}
and save to \verb!<qemu-source>/pc-bios/ppc405_rom.bin!, modify \verb!Makefile!
file (add \verb!ppc405_rom.bin! to BLOBS macro) and recompile QEMU.

EXPLAIN: If you go to the source of QEMU and type
\begin{verbatim}
find . -iname "*.bin"
\end{verbatim}
the file \verb!ppc405_rom.bin! is not available there
\begin{verbatim}
./roms/openbios/forth/device/romfont.bin
./pc-bios/vgabios-vmware.bin
./pc-bios/vgabios-cirrus.bin
./pc-bios/sgabios.bin
./pc-bios/vgabios-stdvga.bin
./pc-bios/linuxboot.bin
./pc-bios/ppc_rom.bin
./pc-bios/vgabios.bin
./pc-bios/bios.bin
./pc-bios/slof.bin
./pc-bios/multiboot.bin
./pc-bios/vgabios-qxl.bin
./pc-bios/optionrom/linuxboot.bin
./pc-bios/optionrom/multiboot.bin
./pc-bios/optionrom/kvmvapic.bin
./pc-bios/QEMU,cgthree.bin
./pc-bios/QEMU,tcx.bin
./pc-bios/spapr-rtas.bin
./pc-bios/kvmvapic.bin
./pc-bios/bios-256k.bin
./dtc/tests/incbin.bin
\end{verbatim}


\begin{verbatim}
qemu: hardware error: PowerPC 601 / 620 / 970 need a 1MB BIOS
\end{verbatim}


\section{PearPC (host: x86, target: PowerPC)}
\label{sec:PearPC}

PearPC is another new emulator that can use JIT dynamic translation, but only on
an x86 host with a PowerPC target.

\url{http://www.ibm.com/developerworks/library/pa-emulation/}

\section{Migrate a virtual machine to a physical machine}

We need a Ubuntu LiveCD, sticky-back-plastic, and an external USB disk (if you
don't have more than one internal disks).

\begin{enumerate}
  \item Convert the disk formats used by these VM-tools (VMWare, VirtualBox)
  to a standard imaging format
\begin{verbatim}
cd /media/wherever-the-image-is/

# for VMWare
sudo apt-get install qemu-kvm
qemu-img convert your-vmware-disk.vmdk -O raw disk.img

# For VirtualBox
VBoxManage internalcommands converttoraw your-virtualbox-disk.vdi /dev/sdX
\end{verbatim}

  \item Move the new image file to a new location, i.e. write the image to
  overwrite the disk on which will be used as the boot disk.
\end{enumerate}


\url{http://askubuntu.com/questions/32499/migrate-from-a-virtual-machine-vm-to-a-physical-system}

