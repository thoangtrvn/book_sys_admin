\chapter{Ubuntu bugs}


\section{Backports}

For a given release of Ubuntu, it freezes all other packages, i.e. these
packages only get bug fixes and minor updates; but not major updates.

Backports is a repository with a collection of such packages with up-to-date
versions that can be used to installed on these older Ubuntu version.

\subsection{Trusty (14.04) and Xenial (16.04)}	

\begin{verbatim}
sudo add-apt-repository  ppa:jonathanf/backports

sudo apt-get update
\end{verbatim}




\section{stopping early crypto disks}

Ubuntu 14.04 Desktop, Ubuntu 12.04

After installation on a virtual machine (Hyper V), It says to restart. IT will stop at:

\begin{verbatim}
stopping early crypto disks ok
\end{verbatim}
Then it will stop here and it seems I have to shutdown the virtual machine to restart Ubuntu.

\url{http://askubuntu.com/questions/474346/installation-14-04-machine-does-not-restart-stopping-early-crypto-disks-ok}

\url{https://bugs.launchpad.net/ubuntu/+source/casper/+bug/966480}

\url{http://askubuntu.com/questions/501722/what-is-ubuntu-14-04-doing-while-showing-stopping-early-crypto-disks}


\section{Failed to fetch }

\begin{verbatim}
W: Failed to fetch http://us.archive.ubuntu.com/ubuntu/dists/trusty-updates/universe/i18n/Translation-en  Hash Sum mismatch

E: Some index files failed to download. They have been ignored, or old ones used instead.

Failed to fetch http://dl.google.com/linux/chrome/deb/dists/stable/Release  
Unable to find expected entry 'main/binary-i386/Packages' in Release file (Wrong sources.list entry or malformed file)
\end{verbatim}

SOLUTION:
\begin{verbatim}
sudo rm /var/lib/apt/lists/* -vf

sudo apt-get update 
\end{verbatim}
\url{http://askubuntu.com/questions/553765/failed-to-fetch-update-on-ubuntu-14-04-lts-trusty-tahr}



CHROME: edit 2 files, search for 'deb' and add after that [arch=amd64]
\begin{verbatim}
sudo gedit /etc/apt/sources.list.d/google-chrome.list
sudo gedit  /opt/google/chrome/cron/google-chrome

deb [arch=amd64] http://dl.google.com/linux/chrome/deb/ stable main

\end{verbatim}
\url{http://askubuntu.com/questions/743814/unable-to-find-expected-entry-main-binary-i386-packages-chrome}

\section{NO\_PUBKEY}

\begin{verbatim}
Reading package lists... Done  
 W: GPG error: http://ppa.launchpad.net precise 
 Release: The following signatures couldn't be verified because the public key is not available: 
 NO_PUBKEY 2EA8F35793D8809A
\end{verbatim}


SOLUTION: put the key and ask Ubuntu to update
\begin{verbatim}
sudo apt-key adv --keyserver keyserver.ubuntu.com --recv-keys 2EA8F35793D8809A
\end{verbatim}
\url{http://askubuntu.com/questions/127326/how-to-fix-missing-gpg-keys}



\section{etc/apt/sources.list}
\label{sec:sources.list}

The file \verb!/etc/apt/sources.list! 
\begin{enumerate}
  \item column 1 (deb, deb-src): indicate the type of the archives
  
  NOTE: deb = link contains pre-compiled binary packages, deb-src = contain
  sources
  
  \item column 2 = URL to the link
  
  \item column 3: can be either release code name (e.g. xenial,
  xenial-updates,), or release class (e.g. unstable, oldstable, stable, testing)
  
  \item column 4, 5, 6 : a component can be either \verb!main! (i.e. refers to
  officially supported software) or \verb!contrib! or \verb!non-free!,
  \verb!restricted! (supported software that is not available under a completely
  free license), \verb!universe! (community maintained software, i.e. not
  officially supported software.), \verb!multiverse! (software that is not
  free.)
  

\end{enumerate}
\begin{verbatim}
deb http://site.example.com/debian distribution component1 component2 component3
deb-src http://site.example.com/debian distribution component1 component2 component3
\end{verbatim}

and in any file with the suffix .list under the directory

/etc/apt/sources.list.d/

\url{https://help.ubuntu.com/community/Repositories/CommandLine}

\subsection{Create local repository}

You may get
\begin{verbatim}
: The repository 'http://10.1.1.21/install/ubuntu16.04.1/ppc64el xenial-updates
Release' does not have a Release file.
\end{verbatim}

\begin{enumerate}
  \item (on the xcat machine for example) Create a directory where you will keep
  your packages. For this example, we'll use /usr/local/mydebs.

\verb!$repository_dir! is
\begin{verbatim}
sudo mkdir -p /usr/local/mydebs
\end{verbatim}  
or create the repository in the /var/www

  \item createa symbolic link  (if the folder is outside /var/www)
  
\begin{verbatim}
sudo ln -s ~/repository_dir /var/www/install/ubuntu16.04.1
\end{verbatim}

  \item move your packages (*.deb files) into the directory you've just created.

\begin{verbatim}
sudo mkdir -p repository_dir/dists/stable/main/binary

sudo mv location_of_package/package_name.deb
\end{verbatim}  


  \item (on any computer that want to access the local/shared repository) add to
  \verb!/etc/apt/sources.list! file
  
(local machine only)  
\begin{verbatim}
deb file:/usr/local/mydebs ./
\end{verbatim}

or if you use web server
\begin{verbatim}
deb http://10.1.1.21/install/ubuntu16.04.1/ppc64el xenial main restricted
deb http://10.1.1.21/install/ubuntu16.04.1/ppc64el xenial-updates main restricted

\end{verbatim}


  \item authenticate Repository and Packages

\begin{verbatim}
gpg --gen-key

//list keys
gpg --list-keys

// suppose your generated key is
  gpg: key 041DA354 marked as ultimately trusted
  public and secret key created and signed.
  gpg: checking the trustdb
  gpg: 3 marginal(s) needed, 1 complete(s) needed, PGP trust model
  gpg: depth: 0  valid:   1  signed:   0  trust: 0-, 0q, 0n, 0m, 0f, 1u
  pub   4096R/041DA354 2012-06-01
        Key fingerprint = 2253 4C89 DE74 CF68 39D7  2A2E DB3E 384F 041D A354
  uid                  Repository
  
  
// export to a text file and store it in the root of the repository:
sudo gpg --output keyFile --armor --export 041DA354


\end{verbatim}
\end{enumerate}

\section{Update Ubuntu}


\subsection{14.04 to higher}

When you upgrade, it has a default behavior, to upgrade to an LTS version only, or to any newer version,
check the file (with possible values: \verb!lts!, \verb!normal!, \verb!never!)
\begin{verbatim}
/etc/update-manager/release-upgrades


[DEFAULT]
Prompt=lts
\end{verbatim}

TO: Ubuntu 16.04.3 LTS (Xenial Xerus)
\url{https://wiki.ubuntu.com/XenialXerus/ReleaseNotes}

\begin{verbatim}
// change to 'lts' (if current Ubuntu is 14.04)
sudo vi /etc/update-manager/release-upgrades

$ sudo apt update 
$ sudo apt upgrade
$ sudo apt dist-upgrade

// above mnimize the difference

// then remove no-longer needed
$  sudo apt autoremove


sudo apt-get install update-manager-core && sudo do-release-upgrade

sudo apt update && sudo apt -y dist-upgrade

sudo reboot
\end{verbatim}

\url{https://linuxconfig.org/how-to-upgrade-to-ubuntu-18-04-lts-bionic-beaver}

\section{decrypt disk}
\label{sec:decrypt-LVM-disk}

Boot into USB disk, choose Try Ubuntu

Check which one is encrypted
\begin{verbatim}
sudo blkld | grep crypt

/dev/sda5: UUID="....." TYPE="crypto_LUKS" PARTUUID="..."
\end{verbatim}
or
\begin{verbatim}
sudo  lvdisplay

-- logical volume --
LV path /dev/ubuntu-vg/root
LV name          root
LV Name          ubuntu-vg
...
\end{verbatim}


To unlock the partition
\begin{verbatim}

udisksctl unlock -b /dev/sda5

 // to mount (check the path from lvdisplay result)
sudo mount /dev/ubuntu-vg/root  /mnt
\end{verbatim}

UNLOCK: 
\begin{verbatim}
sudo apt-get installcryptsetup
//decrypt
sudo cryptsetup luksOpen /dev/sda5
\end{verbatim}

MOUNT TO a device:
\begin{verbatim}
sudo mkdir /media/my_device
sudo mount /dev/mapper/my_encrypted_volume /media/my_device
\end{verbatim}

LOCK again:
to lock again
\begin{verbatim}
sudo umount /media/my_device
sudo cryptsetup luksClose my_encrypted_volume
\end{verbatim}

\section{no ethernet working}


One way to do is booting from USB, 

if the root disk is encrypted, try to decrypt it
(Sect.\ref{sec:decrypt-LVM-disk})

Install boot-repair
\begin{verbatim}
sudo add-apt-repository ppa:yannubuntu/boot-repair

sudo apt-get update

sudo apt-get install -y boot-repair && boot-repair 

\end{verbatim}

Wait until the screen shows up with instructions
\begin{verbatim}

sudo chroot "/mnt/boot-sav/mapper/ubuntu-vg-root" dpkg --configure -a
sudo chroot "/mnt/boot-sav/mapper/ubuntu-vg-root" apt-get install -fy

// just in case you use RAID disks
sudo chroot "/mnt/boot-sav/mapper/ubuntu-vg-root" apt-get install -y dmraid
sudo chroot "/mnt/boot-sav/mapper/ubuntu-vg-root" dmraid -ay

// just in case you use LVM disk
sudo chroot "/mnt/boot-sav/mapper/ubuntu-vg-root" apt-get install -y lvm2

// reinstall grub
sudo chroot "/mnt/boot-sav/mapper/ubuntu-vg-root" apt-get purge -y grub*-common
    grub-common:i386 shim-signed


// update kernel 
sudo chroot "/mnt/boot-sav/mapper/ubuntu-vg-root" sudo apt-get dist-upgrade


// update intel-microcode
sudo chroot "/mnt/boot-sav/mapper/ubuntu-vg-root"  sudo apt-get install intel-microcode
\end{verbatim}



\section{weird characters on Linux terminal}


\begin{verbatim}
reset
\end{verbatim}



\section{Chrome on VNC session and X11 session}


A user-data-dir is used for a single display only, if you want to launch chrome
on multiple display, uses a differetn user-data-dir

\begin{verbatim}
// the first chrome
google-chrome

// the second chrome
cp ~/.config/chrome{,_local}
google-chrome --user-data-dir=~/.config/chrome_local
\end{verbatim}
