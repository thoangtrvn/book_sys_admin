\chapter{Build a Linux kernel and/or modules from sources}
\label{chap:build-Linux-kernel}

To build a Linux kernel from source, there are 3 important concepts you need to
understand (Sect.\ref{sec:build-host-target}).
You need to build a \verb!built platform! (a tool-chain) that can run on a
\verb!host! machine, and can compile the source code designed to be used on a
(different) \verb!target! machine. If the target machine and the host machine
are not the same (i.e.
different CPU architecture), then the compiler to be built is called {\bf
cross-compiler}.

IMPORTANT information
\begin{itemize}
  \item select the host machine: O/S kernel version?, compiler version?
  
  
  \item build the (cross-platform) tool-chain: which include many tools
  (Sect.\ref{sec:cross_compiler})
  
  You can choose to build yourself (Sect.\ref{sec:cross_compiler_build}) or
  using crosstool-ng (Sect.\ref{sec:crosstool-ng})
  
  \item buithe target machine: O/S kernel version?, root file system (folder
  structure), kernel modules and other modules
  
  This requires proper understanding of the target machine system to have
  proper tweak to the kernel source and settings, e.g.
  MIP405T embedded system (Sect.\ref{sec:Linux_kernel_PowerPC}).
  Then, you configure the host machine to use the built tool-chain for compiling
  \begin{itemize}
    \item Linux kernel
    \item kernel module
    \item other modules
  \end{itemize}
  
  You don't need the tool-chain to create the root file system, but you need to
  know the target Linux kernel version to create the right folder structure
  (Sect.\ref{sec:root-file-system}) which then can be compress into a single
  image file.
\end{itemize}

\section{Introduction}

\subsection{Headless system}
\label{sec:headless_system}

A {\bf headless system} is the one operating without a monitor, GUI or
peripheral devices (e.g.
keyboards, mouse). In other words, it also refer to a motherboard without a
graphics card and graphic output port.

A headless system is usually an embedded system in various devices (automobiles,
medical equipments, cameras, houshould appliances, airplanes, vending machines,
\ldots).
\url{http://whatis.techtarget.com/definition/headless-system}

% \subsection{root file system (RFS) vs. Linux kernel}
% 
% One of the important kernel boot parameters is "root=", which tells the kernel where to find the root filesystem.
% \begin{verbatim}
% root=/dev/hda1
% \end{verbatim}
% 
% First of all, at kernel initialization time, there is an absolutely minimal
% filesystem registered, called "rootfs" (Sect.\ref{sec:rootfs}). 
% 

% 
% \subsection{RAM Disk with initrd vs. initramfs}
% \label{sec:RAM_Disk}


% There are two implementation of \verb!initrd!
% \begin{itemize}
%   \item old \verb!initrd! (ramdisk) : acts as a block device (called \verb!ramdev!), so requires a filesystem driver (e.g. ext2)
%   It means the kernel must have at least one built-in module for detecting filesystem.
%   Also as it is a block device, it is cached, i.e. waste memory and memory bus bandwidth.  
%   
%   \item new \verb!initrd! (\verb!ramfs! and \verb!tmpfs!): 
%   
%   \verb!ramfs!  is to replace ramdisk. The downside of \verb!ramfs! is it does not prevent
%   memory fill-up. \verb!tmpfs! is a derivative of ramfs, with adding size limits, and the 
%   ability to write data to swap space. \verb!rootfs! is a special instance
%   of \verb!ramfs! (or tmpfs, if it is enabled), which is always present since Linux
%   kernel 2.6.x. NOTE: If \verb!CONFIG_TMPFS! is enabled, then \verb!rootfs! uses
%   \verb!tmpfs! instead of \verb!ramfs! by default. In such case, to force using ramfs,
%   add \verb!rootfstype=ramfs! to the kernel command-line.
%   
% %   use a dummy
% %   filesystem (called \verb!tmpfs!) and there is no extra copy of data in memory.
%   
% NOTE: The image is a compressed \verb!cpio! file and is compressed into the \verb!/tmpfs! 
% \begin{verbatim}
% zcat initramfs | cpio -i
% \end{verbatim}
% \url{https://www.kernel.org/doc/Documentation/filesystems/ramfs-rootfs-initramfs.txt}
% %  unpacked via \verb!cpio! into the memory
% \end{itemize}
% \url{http://stackoverflow.com/questions/10603104/the-difference-between-initrd-and-initramfs}





\subsection{Concept: Build/Host/Target platform}
\label{sec:build-host-target}

A \verb!build! platform is the platform that you want to build the compiler
and it can be configured to run on some \verb!host! platform.
When it runs on this \verb!host! platform, it compiles the code
for running on some \verb!target! platform. 
Building compilers that will run on some other linux system is called a Canadian Cross.


The format of a GNU target specification is CPU-PLATFORM-OS. For example,
Solaris 8 for x86 is \verb!i386-pc-solaris2.8!, while Macintosh OS X is
\verb!powerpc-apple-darwin7.6.0!.

Most Linux targets can be specified according to their CPU; hence, PowerPC/Linux
is \verb!ppc-linux!. 



\section{Cross-compiler toolchain}
\label{sec:cross_compiler}


Cross-compilation is to build the code on one platform but the compiled code is
expected to run on another platform. We pass two different
options: \verb!--build=! and \verb!--host=! to the tool \verb!./configure!
\begin{itemize}
  \item \verb!--host=!: the system where the built program/library will run
  \item \verb!--build=!: the system on which the package is built
  \item \verb!--target=!: use only if the built package is a cross-compiler,
  which tells the system for which the built package will create output.
\end{itemize}

\begin{verbatim}
./configure --build i686-pc-linux-gnu --host i586-mingw32msvc
\end{verbatim}
before we call
\begin{verbatim}
make
\end{verbatim}

This is not a single tool, but a series of tools, thus giving them the
collective name cross-compiler {\bf toolchain} (Sect.\ref{sec:toolchain}).  
A cross-compiler is capable of building compiled code for a platform
other than the one the compiler is running
(Sect.\ref{sec:cross_compiler_build}).
\url{http://landley.net/writing/docs/cross-compiling.html}


You can either get a pre-built cross-compilers for your host machine if it is
running an O/S that has the pre-built versions.
There are several groups that provide the pre-built cross-compilers, e.g. ELDK
(Sect.\ref{sec:ELDK}) or Buildroot (Sect.\ref{sec:Buildroot}).


You can build your cross-compiler either (1) using \verb!crosstool-ng! package
to suppor the compilation (Sect.\ref{sec:crosstool-ng} -
\textcolor{red}{RECOMMENDED METHOD}), (2) do it yourself by setting up
everything (Sect.\ref{sec:cross_compiler_build}).


\subsection{Toolchain}
\label{sec:toolchain}


Toolchains are made of different pieces of
software, each being quite complex and requiring specially crafted options to
build and work seamlessly.
A toolchain is a very sensitive piece of software, as any bug in one of the
components, or a poorly configured component, can lead to execution problems,
ranging from poor performance, to applications ending unexpectedly, to
mis-behaving software (which more than often is hard to detect), to hardware
damage, or even to human risks (which is more than regrettable).

IMPORTANT concepts in cross-compilation is {\bf host, target}
(Sect.\ref{sec:build-host-target}).

You'll need the GNU toolchain to compile the kernel, particularly gcc and binutils.
\url{http://elinux.org/Toolchains}
Depending on your host O/S to compile the kernel, you can choose the right pre-built toolchain
\url{http://www.denx.de/wiki/PPCEmbedded/CompilerToolset}
\begin{itemize}
  \item Debian-based O/S: \url{http://www.emdebian.org/}
  \item DENX ELDK: \url{http://www.denx.de/wiki/DULG/ELDK}
\end{itemize}

or compile the toolchain from source using \verb!crosstool-ng!
\url{http://crosstool-ng.org/download/crosstool-ng/}
crosstool-NG can build from generic, general purpose toolchains, to very
specific and dedicated toolchains.


\subsection{make-kpkg}
\label{sec:make-kpkg}

If the target platform and the host platform is the same, you do not need to
build a cross-compiler. Instead, the installed compiler on the host platform is
good enough
\url{https://wiki.debian.org/BuildADebianKernelPackage}

\url{http://sukhanov.net/cross-compile-with-make-kpkg.html}

Example: build a different version of the Debian kernel on the same platform
\begin{verbatim}
// install necessary packages
// NOTE: kernel-package provides 'make-pkg' tool
apt-get install build-essential linux-source kernel-package libncurses5-dev

//  update information for
// 	maintainer
//  email
emacs -nw /etc/kernel-pkg.conf

// extract kernel source to compile
// linux-source-x.x.tar.bz2
cd /usr/src
tar xjvf linux-source-3.2.tar.bz2

// create 'defconfig' file
cd linux-source-3.2
make ARCH=i386 defconfig
\end{verbatim}
\url{http://www.makelinux.net/books/lkd2/ch02lev1sec3}

\begin{verbatim}
make config

make <some settings> menuconfig

make <some settings> xconfig

make oldconfig
    update the kernel configuratio using
    current .config file, and prompt for any new
    options that have been added to the kernel
    
make <some settings> defconfig
    generate new kernel configuration using
    default values from
    arch/$ARCH/defconfig file
        
\end{verbatim}
\url{http://www.linuxdevcenter.com/pub/a/linux/excerpts/9780596100797/kernel-build-command-line-reference.html}

% NOTE: \verb!defconfig! file contains the configuration based on the defaults for
% your architecture.
If you use \verb!crosstool-ng! it can help to generate a more detail of this
file using
\begin{verbatim}
./cng-nt menuconfig
\end{verbatim}


Finally, build the kernel based upon the previous configuration
\begin{verbatim}
make-dpkg
\end{verbatim}
a script which automates and replaces the sequence 
\begin{verbatim}
make dep; make clean; make bzImage; make modules
\end{verbatim}
Check Sect.\ref{sec:compressed-kernel-file-format} for information about \verb!bzImage!.

SYNTAX:
\begin{verbatim}
make-kpkg <options> <target>
\end{verbatim}
\url{http://newbiedoc.sourceforge.net/system/kernel-pkg.html#KPKG-KERNEL-PKG}

Example: build PowerPC kernel
\begin{verbatim}
make-kpkg --cross-compile powerpc-linux-gnu- --arch ppc --subarch powerpc
\end{verbatim}



\subsection{Build a native-compiler for target platform}
\label{sec:configure-environment-target-platform}

Once you have the cross-compiler, you need to setup the environment 
on the host platform for building the binary for this target platform.

\begin{verbatim}
// make folder
mkdir crossbuild
cd crossbuild

// obtain latest version of gcc, binutils, gdb, newlib
// and extract them in the above folder
$ bunzip2 -c gcc-3.3.2.tar.bz2|tar xf -
$ bunzip2 -c binutils-2.14.tar.bz2 |tar xf -
$ bunzip2 -c linux-2.6.9.tar.bz2 |tar xf -
$ bunzip2 -c gdb-6.3.tar.bz2|tar xf -
$ bunzip2 -c glibc-2.3.tar.bz2|tar xf -

// create the folder with target name
mkdir powerpc-405
cd powerpc-405

// create the subfolders
mkdir build-binutils build-gcc build-glibc build-gdb

// configure the right environment variable
export TARGET=i386pc
export PREFIX=/usr/local/crossgcc
export TARGET_PREFIX=$PREFIX/$TARGET
export PATH=$PATH:$PREFIX/bin


// obtain Linux headers for the version you 
// want to run on that target platform
// NOTE: use generic kernel
//       you may need some patches
cd linux-2.6.9
make ARCH=i386 CROSS_COMPILE=i386-linux- menuconfig
\end{verbatim}


How to build a compiler, from the host platform, and this compiler
only run on that target platform.

\subsection{Configure a complete environment for compiling target platform}

\subsection{Buildroot}
\label{sec:Buildroot}

Buildroot is able to generate a cross-compilation toolchain, a root filesystem,
a Linux kernel image and a bootloader for your target.

Buildroot is useful mainly for people working with embedded systems. Embedded
systems often use processors that are not the regular x86 processors everyone is
used to having in his PC. They can be PowerPC processors, MIPS processors, ARM
processors, etc.

\url{http://buildroot.uclibc.org/downloads/manual/manual.chunked/ch01.html}

\subsection{DENX ELDK}
\label{sec:ELDK}

DENX ELDK provides (1) some pre-built toolchains (SDK), (2) some pre-built 
target Linux kernel (RFS - Root FileSystem) that users can download and use. 

The pre-built toolchain was compiled to run on 32-bit host O/S, maintained by
Wolfgang Denk. Thus, if you use 64-bit host O/S, you need to install a compatibility layer
called \verb!ia32-libs! or \verb!gcc-multilib!, and \verb!g++-multilib! (if C++
is used). On Debian or Ubuntu
\begin{verbatim}
sudo apt-get install ia32-libs
\end{verbatim}
On 64-bit Fedora 11
\begin{verbatim}
sudo yum -y install glibc.i686 zlib.i686
\end{verbatim}
\url{http://www.denx.de/wiki/view/DULG/ELDKUsageIn64BitEnvironment}

Stable versions of the ELDK are distributed in the form of an ISO image, which
can be either burned onto a DVD or mounted directly, using the loopback Linux
device driver (Linux host only).

\begin{itemize}
  \item ELDK v5.5.3 can be installed and run on any recent Linux distribution
  running on x86 (i686) and \verb!x86_64! systems.
 
\end{itemize}


NOTE: MIP405/MIP405T uses target as \verb!powerpc-4xx-softfloat! (or
\verb!ppc_4xx! in older ELDK versions)
\begin{verbatim}
powerpc-4xx-softfloat = AMCC 4xx processors without FPU
(replace for ppc_4xx since ELDK 5.3)
 
powerpc-4xx = AMCC 4xx processors with FPU
(replace for ppc_4xxFP since ELDK 5.3)

ppc_6xx = PowerPC processors based on 60x cores 
(This includes support for MPC5xxx, 7xx, 82xx and 83xx processors).
ppc_74xx = 74xx processors
ppc_8xx = MPC8xx processors
ppc_85xx = MPC85xx processors
ppc64 = PPC64 processors with FPU (PA6T)
\end{verbatim}

Rebuild ELDK from scratch (so that it can run directly on 64-bit host O/S:
\url{http://www.denx.de/wiki/view/DULG/ELDKDownload}

\subsubsection{Download binary files}

Making sure you keep the same folder structure, as the \verb!install.sh!
script only recognizes files under the given structure.

Example: you want to download the chain tool with target is \verb!armv7a!
\begin{verbatim}
$ mkdir eldk-download
$ cd eldk-download
$ mkdir -p targets/armv7a
$ wget ftp://ftp.denx.de/pub/eldk/5.5.3/install.sh
$ cd targets/armv7a
$ wget ftp://ftp.denx.de/pub/eldk/5.5.3/targets/armv7a/target.conf
$ wget ftp://ftp.denx.de/pub/eldk/5.5.3/targets/armv7a/eldk-eglibc-i686-arm-toolchain-gmae-5.5.3.sh
$ wget ftp://ftp.denx.de/pub/eldk/5.5.3/targets/armv7a/core-image-minimal-generic-armv7a.tar.gz
$ wget ftp://ftp.denx.de/pub/eldk/5.5.3/targets/armv7a/core-image-sato-sdk-generic-armv7a.tar.gz
\end{verbatim}
\url{http://www.denx.de/wiki/ELDK-5/WebHome}

NOTE: from the  link
\begin{itemize}
  \item \verb!qte! = Qt-embedded based image, 
  
  \item \verb!gmae! = GNOME Mobile and
Embedded Initiative software stack, 
 
   \item  \verb!xenomai! = Xenomai is a real-time
development framework cooperating with the Linux kernel, to provide a pervasive,
interface-agnostic, hard real-time support to user space applications,
seamlessly integrated into the Linux environment. The Xenomai project was
launched in August 2001,

   \item \verb!sato! = Sato is a new visual style for Poky Linux, the embedded
   Linux distribution.
  
\end{itemize}

\url{http://www.denx.de/wiki/ELDK-5/WebHome}
\begin{itemize}
  \item name: minimal, base, lsb, qte, sato 

  \item  \verb!lsb! = 'basic' image suitable for implementations that conform Linux Standard Base (LSB - Sect.\ref{sec:LSB}),

  \item name-\verb!dev!: with tools suitable for development work
  
  \item name-\verb!sdk!: include everythings in meta-toolchain and development headers + libraries
  to form a complete standalone SDK
\end{itemize}

NOTE: In the past, ELDK provided a tool chain, but to set up your software build
environment, you were mostly on your own.
With ELDK 5.x, ELDK brings you not only the cross tools and the target
environment, but also a complete, sophisticated software production environment
that can easily be extended and customized for your specific project needs.

NOTE: Target is PowerPC 4xx: \url{ftp://ftp.denx.de/pub/eldk/5.5.3/targets/powerpc-4xx-softfloat}

To install ELDK 
\begin{itemize}
  \item targetting a particular CPU, e.g. \verb!powerpc-4xx-softfloat!
  
  \item and specify the proper kernel having a list of associated tools (i.e.
  target configuration) we choose the value for \verb!-s! option

\verb!-s! specify the SDK folder to be created in \verb!/opt/eldk-5.5.3/!
\begin{verbatim}
-s '-' 
    no folder created
    
    
\end{verbatim}
% NOTE: The script \verb!install.sh! misses \verb!sato! SDK image, i.e. 
% add one line with \verb!sato! so that we can use \verb!-s 'sato'!
Available values for \verb!-s! options
\begin{verbatim}
: ${SDK_KNOWN_IMAGES:="
    gmae
    qte
    qte-xenomai
 "}
\end{verbatim}

\verb!-r! specify the RFS (Root FileSystem) folder to be created in
\verb!/opt/eldk-5.5.3/!
\begin{verbatim}
-r 'lsb-dev'
\end{verbatim}
% NOTE: Use either \verb!-s! option (SDK image) or \verb!-r! option (Root FileSystem).
% It seems that only \verb!-r! works.

\end{itemize}

Example: Install the cross-compiler ELDK running on the 64-bit host platform to
build the kernel for powerpc-4xx with no floating support 
\begin{verbatim}
// NOTE: use this for MPL MIP405T
// core-....tar.gz files refers to -r option
./install.sh -D -a 'x86_64' -s '-' -r 'lsb-dev' powerpc-4xx-softfloat

./install.sh -D -a 'x86_64' -s 'gmae' powerpc-4xx-softfloat

./install.sh -D -a 'x86_64' -r 'sato-sdk' powerpc-4xx-softfloat
\end{verbatim}
NOTE: If the host O/S is 64-bit, use \verb!-a 'x86_64'!, the last argument is
the target platform. \verb!-D! is for dry-run (testing), and then
we can remove to do the real install of ELDK.


While ELDK v5.3 (and newer) can be installed in any directory
(modify \verb!install.sh! script), for ELDK v5.2 and
older, the cross tool chain packages are not relocatable and can only be
installed into their default location, i. e. into the "/opt/eldk-5.5.3/"
directory.
\begin{verbatim}
sudo chmod 01777 /opt/eldk-5.5.3/
\end{verbatim}

The two folders under \verb!/opt/eldk-5.5.3/powerpc-405-softfloat/!
\begin{itemize}
  \item \verb!sysroots! : the root filesystem content,
  which has a complete Linux filesystem to run on MPL MIP405 board's CPU.
  
    \item \verb!rootfs-qte-sdk! : the SDK content (when \verb!-s 'qte'! is used)
\end{itemize}

Setting ups environment variables
\begin{itemize}
  \item ELDK 5.2.1:
\begin{verbatim}
CC="powerpc-linux-gcc  -m32 -mhard-float --sysroot=/opt/eldk-5.3/powerpc/sysroots/powerpc-linux"
CFLAGS=" -O2 -pipe -g -feliminate-unused-debug-types"
\end{verbatim}  

  \item ELDK 5.3:
\begin{verbatim}
CC=powerpc-linux-gcc
CFLAGS=" -m32 -mhard-float --sysroot=/opt/eldk-5.2.1/powerpc/sysroots/powerpc-linux"
\end{verbatim}
\end{itemize}
MAKE sure you use the correct \verb!--sysroot! settings.


\begin{verbatim}
 export PATH=/opt/eldk-5.5.3/powerpc/sysroots/i686-eldk-linux/usr/bin:/opt/eldk-5.5.3/powerpc/sysroots/i686-eldk-linux/usr/bin/powerpc-linux:$PATH
 
 export ARCH=powerpc
 
 export CROSS_COMPILE=powerpc-linux-
 
source /opt/eldk-5.5.3/powerpc/environment-setup-*

 export CROSS_COMPILE=${TARGET_PREFIX}
 export ARCH=${TARGET_PREFIX%%-*}
\end{verbatim}


Once the environment variables are setup, we can jump to 
\begin{itemize}
  \item U-Boot
  \item Source of Linux kernel
\end{itemize}
and compile each of them, using (see the same commands in Sect.\ref{sec:make-kpkg})
\begin{verbatim}
make <some setting> ARCH=... CROSS_COMPILE=...

make install 
\end{verbatim}

\subsection{armv5}
\label{sec:armv5}

To compile the code for \verb!ARM! architecture, we need to install
build-essential and \verb!gcc-arm-linux-gnueabi!.

\begin{verbatim}
arm-linux-gnueabi-gcc -static -march=armv5 hi.c -o hi
\end{verbatim}

The new compiler accepts several \verb!-m! options (defined for ARM
architectures).
\begin{verbatim}
ARCH=arm CROSS_COMPILE=arm-linux-gnueabi-
\end{verbatim}

The toolchain \verb!CodeSourcery! is suggested. In Ubuntu, it uses Linaro
optimizations in the toolchain, so may not work properly when compiling code for
ARM.

\section{Marvell ARM CPUs}
\label{sec:Marvell}

Marvell designs architecture for CPUs that implements ARM instruction set
designed by ARM holdings.

\url{https://wiki.openwrt.org/doc/hardware/soc/soc.marvell}

\subsection{Marvell Feroceon}
\label{sec:Marvell-Feroceon-}

Marvell Feroceon has a variable-length processing pipeline that allows
out-of-order instruction execution. The Feroceon made some significant changes
to the standard ARM fixed pipeline, with a variable-stage pipeline that ranges
from six stages to eight if the writeback stage is included.



\section{Build cross-compiler}
\label{sec:cross_compiler_build}

A typical build process is 
\begin{verbatim}
./configure
make
make install
\end{verbatim}

SCENARIO: When you call \verb!./configure!, it needs to detect the information
of the target platform, not the host platform. However, most of the time,
it uses the information from the host platform.

SOLUTION: We need to have a cross-compiler toolchains.
The cross-compiler toolchain tends to prefix the utilities, e.g. \verb!gcc!
becomes \verb!armv51-linux-gcc!. The prefix is specifed in the environment
variable \verb!CROSS_COMPILE!.
\begin{verbatim}
export CROSS_COMPILE=armv51-linux-!
\end{verbatim}
NOTICE the ending hyphen (-) character of the value in the \verb!CROSS_COMPILE!
environment variable.

\begin{mdframed}
The core components of any compiler
\begin{enumerate}
  \item parser: convert source code to assembly code
  
  \item assembler: convert assembly code to bytecode that CPU can execute.
  
  \item linker: combine one or more object files generated by the assembler
  into an executable file.
  
  The linker needs to know the target platform to link the files.
  
  \item standard C library (glibc, $\mu$libc, \ldots): the library of core C
  functions, to be used with the linker and source code to produce the final
  executable files, any calls to the underlying target platform
  will use the C functions from this library.
  
\end{enumerate}

\end{mdframed}

The question is how to set up an environment (with the right tool chains and the
right information of the target platform) that allows you to compile an
application for a variety of different targets. This is explained in
Sect.\ref{sec:configure-environment-target-platform}.
\url{http://www.sourceware.org/ml/crossgcc/2005-08/msg00114/l-cross-ltr.pdf}



PREREQUISITE:
\begin{itemize}
  \item working \verb!gcc! on host platform
  \item source code of the various tools used to build the cross-compiler
  \item copy of header files for the target platform

  \item OPTION: Build also the \verb!gdb! for target platform.
\end{itemize}

\begin{verbatim}
sudo apt-get install gawk bison flex texinfo automake libtool \
   cvs curl aria2 patch lzma libncurses5-dev g++ \
   openjdk-7-jre-headless build-essential \
   libcloog-ppl-dev gettext
\end{verbatim}
%sudo apt-get install gawk bison flex texinfo automake libtool cvs curl aria2 patch lzma libncurses5-dev g++ openjdk-7-jre-headless build-essential libcloog-ppl-dev gettext

To build a cross-compiler, we need 3 components
\begin{enumerate}
  \item \verb!binutils! : contains basic binary utilities (assembler, linker,
  and associated tools - e.g. size, strip)
  
  \item \verb!gcc! : the main component of the compilation process
  
  \item \verb!newlib/glibc! : standard C library (\verb!newlib! is developed
  from RedHat and can be more user-friendly in cross-compilers designed to be
  used for embedded targets)
  
\end{enumerate}
and the {\it header files} of the target platform. The cross-compiler needs to
be able to compile the code on that platform so it needs to know the content of
the header files for glibC codes. 


Make sure you have the right combinations of versions:
\url{http://pmc.polytechnique.fr/pagesperso/dc/arm-en.html}


To be able to build 32-bit package on 64-bit platform, we need
\begin{verbatim}
sudo apt-get install libc6-dev-i386 g++-multilib
\end{verbatim}
\url{http://stackoverflow.com/questions/7412548/error-gnu-stubs-32-h-no-such-file-or-directory-while-compiling-nachos-source}

\subsection{cross-compiler 3.2}
\label{sec:cross-compiler-GNU_3.2}

Due to the change in gcc 4.2.x, it cannot be used to compile gcc 3.2. The
solution is \textcolor{red}{compiled GCC v3.4 with GCC v4.2.*, and then used
v3.4 to compile v3.2...}.

NOTICE: Choose gcc 3.2 (is too old, and you may get errors. However, it can be
built using gcc-3.4.6). Most old applictions can be buit using
cross-compiler gcc-3.3.3 (Sect.\ref{sec:cross-compiler-GNU_3.3.3}).
\url{http://www.trevorpounds.com/blog/?p=111}
\url{https://ftp.gnu.org/gnu/gcc/gcc-3.2.3/gcc-3.2.3.tar.gz}

%To download the gcc 3.2 version:   \url{http://ftp.gnu.org/gnu/gcc/gcc-3.2/}
  
\begin{verbatim}
wget https://ftp.gnu.org/gnu/gcc/gcc-3.2.3/gcc-3.2.3.tar.gz
tar xzf gcc-3.2.3.tar.gz
cd gcc-3.2.3
patch -p0 < gcc-multilib-fix-v3.2.x.debian.x86_64.diff
mkdir ../gcc-3.2.3-objdir
cd ../gcc-3.2.3-objdir
../gcc-3.2.3/configure --prefix=/opt/x86_64/gcc/gcc-3.2.3 --enable-languages=c,c++
make
make install
\end{verbatim}
Error:
\begin{verbatim}
../../gcc-3.2/gcc/read-rtl.c: In function 'read_rtx':
../../gcc-3.2/gcc/read-rtl.c:662: error: lvalue required as increment operand
\end{verbatim}
\url{http://www.linuxquestions.org/questions/linux-newbie-8/installing-gcc-3-2-make-errors-884121/}

\subsection{cross-compiler 3.3.3}
\label{sec:cross-compiler-GNU_3.3.3}

Build gcc 3.3.3 as cross-compiler
\begin{verbatim}
wget https://ftp.gnu.org/gnu/gcc/gcc-3.3.3/
\end{verbatim}
or download Debian-compatible (Ubuntu Hardy 8) version: \url{https://launchpad.net/ubuntu/+source/gcc-3.3/1:3.3.6-15ubuntu6}

If you get error
\begin{verbatim}
> >  > f/com.c:11080: Error conflicting types for: >ffecom_gfrt_basictype<
> >  > f/com.h:236: Error existing declaration of >ffecom_gfrt_basictype< was here
\end{verbatim}
check if you have \verb!--enable-languages=c,c++!.
\url{https://gcc.gnu.org/ml/gcc-help/2007-09/msg00096.html}

If you get error
\begin{verbatim}
fatal error: gnu/stubs-32.h: No such file or directory
\end{verbatim}
gnu/stubs-32.h is not directed included in programms. It's a back-end type
header file of gnu/stubs.h, just like gnu/stubs-64.h. You can install the
\verb!multilib! package to add both.

If Ubuntu 12.04, there is a known problem as the files reside in non-standard location
\begin{verbatim}
export LIBRARY_PATH=/usr/lib/$(gcc -print-multiarch)
export C_INCLUDE_PATH=/usr/include/$(gcc -print-multiarch)
export CPLUS_INCLUDE_PATH=/usr/include/$(gcc -print-multiarch)
\end{verbatim}

\subsection{cross-compiler 3.4.6}
\label{sec:cross-compiler-GNU_3.4.6}

Choose gcc 3.4.6

\section{Build cross-compiler using crosstool-ng}



%\section{Build kernel from source (target=host platform)}
\subsection{crosstool}
\label{sec:crosstool}

\verb!crosstool! is the tool to support building a cross-compiler, based on
\verb!crossgcc! script
\url{http://kegel.com/crosstool/}

Crosstool was originally developed for embedded system developers, but is also
useful for mainstream developers.
It includes minimal patches for gcc and glibc needed to build a few combinations
of (alpha, arm, i686, ia64, mips, powerpc, powerpc64, sh4, sparc, sparc64, s390,
\verb!x86_64!) x (gcc-2.95.3 ... gcc-4.0.0) x (glibc-2.1.3 ... glibc-2.3.5).



\subsection{crosstool-ng}
\label{sec:crosstool-ng}

\verb!crosstool-ng! is a set of scripts that assist in building a cross-compiler
from scratch, which is a better version of \verb!crosstool! (Sect.\ref{sec:crosstool}).
These scripts can also produce a matching cross-glibc, which means you can
cross-compile userspace applications as well as kernels.

Make sure you download the latest version, otherwise some old links for
necessary packages may not be available.
\begin{itemize}
  \item crosstool-ng 1.6.0
  \item crosstool-ng 1.6.2
  \item crosstool-ng 1.9.3
\end{itemize}

Fix: \url{http://crosstool-ng.org/download/crosstool-ng/01-fixes/}

Example: use \verb!crosstool-ng! to build the toolchain
\begin{verbatim}
// download and untar

// install prequisites
sudo apt-get install automake cvs ncurses-dev

// choose
//  (1) install to the subfolder of the current directory
./configure --local
//  (2) install to the specified directory
./configure --prefix=/opt/crosstool
//or ./configure --prefix=~/x-tools/
//or if you want to keep track whick kernel is used to compile
//   for which version of crosstool-ng
// ./configure --prefix=~/x-tools_1.6.2_kernel_2.6.12

// generate pre-processed files
// This is extremely useful when you suspect something could be wrong with your
//  macros.
make -j4 2>&1 | tee m.txt
make install | tee mi.txt

\end{verbatim}

\textcolor{red}{This is the setting for cross-compilation target PowerPC 32-bit CPU}
\begin{verbatim}
// you can adjust the above settings with
//  e.g. select the desired kernel
//       select the output location
./ct-ng menuconfig
\end{verbatim}

\textcolor{red}{If you build cross-compiler target MIP405/MIP405T
(Sect.\ref{sec:MIP405_MIP405T})}, you should disable floating-point
instruction (as they are not used in MIP405/MIP405T and is not used in the
kernel). \verb!glibc! (Sect.\ref{sec:GLIBC}) is not used in the kernel too.
  

\begin{itemize}
  \item \verb!Path and misc options!: 

RECOMMEND: for first time crosstool-NG build, in 
\verb!MenuConfig / Paths and Misc options! enable
\begin{verbatim}
[x]Debug crosstool-ng
[x]Save intermediary steps
\end{verbatim}
  
  \begin{enumerate}
    \item keep the new tarballs (compared to files in \verb!$HOME/sources!)
    
    put new tarballs in \verb!$CT_TOP_DIR/targets!
    
    \item prefix of directory: 
 \begin{verbatim}
 /home/daglitch/v2plin/crosstool-ng-1.6.1/${CT_TARGET}
 \end{verbatim}
 
   \item build behavior: 3 or 4 parallel jobs (assuming quad-core CPU), nice
   level = 10
   
   \item logging: set EXTRA (maximum level)
   
  \end{enumerate}

   \item Target options: choose \verb!powerpc!, emit assembly for CPU
   (e.g. 405), tune assembly for (e.g. 405), floating-point (e.g. software in
   the case of PowerPC 405Gr)
   
   \item Toolchain options: vendor string = 405
   
   \item O/S: target OS = linux, \textcolor{red}{linux kernel version} = 2.6.12
  
  \item C Compiler:
  
  \url{http://ftp.gnu.org/gnu/gcc/}
\end{itemize}
\url{http://blog.droidzone.in/2012/01/17/building-a-toolchain-crosstool-ng/}

Finally, run
\begin{verbatim}
// specify the configurations
// for the target processor architecture
// e.g. powerpc-405-linux-gnu is the ${CT_TARGET}
//   NOTE: check the link for supported target names
// http://crosstool-ng.org/

./ct-ng powerpc-405-linux-gnu
\end{verbatim}

\textcolor{red}{Troubleshoot}:
NOTE: Sometimes the version you want is not available, so you need to download them separately
\begin{verbatim}
http://ftp.gnu.org/gnu/gcc/  for gcc
http://ftp.gnu.org/gnu/binutils/
http://ftp.gnu.org/gnu/gmp/
http://ftp.gnu.org/gnu/glibc/
http://ftp.gnu.org/gnu/gmp/
http://ftp.gnu.org/gnu/mpfr/
http://www.mr511.de/software/ for Libelf 
\end{verbatim}
and put them into the \verb!.build/tarballs! directory,
\begin{verbatim}
cd .build/tarballs

wget http://ftp.gnu.org/gnu/glibc/glibc-2.1.3.tar.gz
wget http://ftp.gnu.org/gnu/gcc/gcc-4.6.1/gcc-4.6.1.tar.bz2
wget http://ftp.gnu.org/gnu/binutils/binutils-2.20.1a.tar.bz2
wget http://ftp.gnu.org/gnu/glibc/glibc-2.1.3.tar.gz
wget http://www.mr511.de/software/libelf-0.8.13.tar.gz
\end{verbatim}
Finally, manually add them to the list
\begin{verbatim}
./scripts/addToolVersion.sh --binutils -s -c 2.20.1a
./scripts/addToolVersion.sh --gcc -s -c 4.6.1
./scripts/addToolVersion.sh --glibc -s -c 2.1.3
\end{verbatim}



Now we can do the real build of crosstool-NG
\begin{verbatim}
// do the real work: building the toolchain
./ct-ng build

// finally, put it's in the PATH
export PATH=$PATH:~/x-tool/powerpc-405-linux-gnu
\end{verbatim}

\textcolor{red}{TROUBLESHOOT DURING BUILD}:  You may need to modify the scripts 
in \verb!./script/build/gcc/gcc.sh!
\begin{verbatim}
## if sstrip not found
## 

## if ecj-latest not found, 
## add this path
https://www.mirrorservice.org/sites/sourceware.org/pub/java/
\end{verbatim}


NOTE:
\begin{verbatim}
CT_TARGET = powerpc-405-linux-gnu
CT_TARBALLS_DIR = 
CT_CC_VERSION = 
\end{verbatim}

If you find the location, and you can download it yourself, 
you just need to put the right file into
\verb!.build/tarballs/! (NOTE: older versions of crosstool-NG uses
\verb!target/tarballs/!), such as
\begin{verbatim}
./build/tarballs/ecj-latest.jar
./build/tarballs/gmp-4.3.2.tar.bz2
\end{verbatim}
and copy
\begin{verbatim}
cp -v "${CT_TARBALLS_DIR}/ecj-latest.jar" "${CT_SRC_DIR}/gcc-${CT_CC_VERSION}/ecj.jar"
\end{verbatim}
it means
\begin{verbatim}
cp <somewhere>/ecj-latest.jar .build/tarballs/
cp .build/tarballs/ecj-latest.jar .build/src/ecj.jar
\end{verbatim}

TROUBLESHOOT: \verb!duma_2_5_15!
Modify the script to change the download link from
\begin{verbatim}
http://mesh.dl.sourceforge.net/sourceforge/duma/
\end{verbatim}
to
\begin{verbatim}
http://kent.dl.sourceforge.net/sourceforge/duma/
\end{verbatim}
\url{https://gitorious.org/oselas/crosstool-ng/commit/c790487094a61afcf80a48f2db5bcb7a36c69235}
or
\begin{verbatim}
http://downloads.sourceforge.net/project/duma/duma
\end{verbatim}
\url{https://gitorious.org/pibox/pibox/commit/5507049fe6deb3c2d02735c46375043069c6cb00?format=patch}
NOTE: We need to modify multiple files 
\begin{verbatim}
scripts/build/debug/200-duma.sh
scripts/build/debug/300-gdb.sh
scripts/build/debug/500-strace.sh
\end{verbatim} 
or we can download the corresponding file and put into \verb!.build/tarballs! folder
\begin{verbatim}
http://dfn.dl.sourceforge.net/project/duma/duma/2.5.15/duma_2_5_15.tar.gz
http://sourceforge.net/projects/expat/files/expat/2.0.1/expat-2.0.1.tar.gz/download
ftp://invisible-island.net/ncurses
\end{verbatim}

TROUBLESHOOT: \verb!sstrip! should be disabled as it becomes
obsolete and is removed from croosstool-NG
\url{https://sourceware.org/ml/crossgcc/2012-11/msg00081.html}

TROUBLESHOOT: if you get the error
\begin{verbatim}
error: 'IFCVT_ALLOW_MODIFY_TEST_IN_INSN' undeclared (first use in this
function)
\end{verbatim}
It means you should choose C compiler 4.2.4 (not 4.2.2), i.e. rerun
\begin{verbatim}
./ct-ng menuconfig
\end{verbatim}


TROUBLESHOOT: if you choose custom header files for Linux kernel, it
may not be able to detect the version
\begin{verbatim}
Linux version is unavailable in installed headers files
\end{verbatim}
the reason is that \verb!<linux-kernel-dir>/include/linux/version.h!
file does not exist. Suppose the kernel is 2.6.12, you can create the file with the content
\begin{verbatim}
#define LINUX_VERSION_CODE 132620
#define KERNEL_VERSION(a,b,c) (((a) << 16) + ((b) << 8) + (c))
\end{verbatim}
NOTE: To get the value, just write a simple code
\begin{verbatim}
#include <stdio.h>
int main()
{
  printf("%d", KERNEL_VERSION(2,6,12));
}
\end{verbatim}

TROUBLESHOOT: use \verb!uClibc!, then you get error
\begin{verbatim}
You did not provide a uClibc config file!
\end{verbatim}
If you elect to build a uClibc-based toolchain, you will have to prepare a config file for uClibc.
\url{http://www.uclibc.org/downloads/INSTALL}
 
TROUBLESHOOT: if you choose to use custom header files from an old version of Linux kernel, error
\begin{verbatim}
error: redefinition of 'struct termios'
\end{verbatim} 
the reason is that glibc and the linux kernel both define the same structure \verb!termios!
\begin{verbatim}
//file .build/src/glibc/


// file <linux-kernel>/include/asm/termbits.h

\end{verbatim}
The solution is to add a guard in the \verb!bits/termbits.h! of glibc.
\begin{verbatim}
#ifndef NCCS
#define NCCS 32
struct termios {

}
#endif
\end{verbatim}
and to \verb!include/asm/termbits.h! as well.

TROUBLESHOOT: if you choose to use custom header files from an old version of Linux kernel, error
\begin{verbatim}
sys-root/usr/include/asm/stat.h:26
  error: expected specifier-qualifier-list before '__kernel_into_t'
\end{verbatim}
if you look into the file \verb!include/asm/stat.h! line 26
\begin{verbatim}
#include <linuxx/types.h>

     ino_t    st_ino;   // line 26
\end{verbatim}
and if we check the file \verb!linux/types.h!
\begin{verbatim}
#include <linux/posix_types.h>
#include <asm/types.h>

typedef __kernel_ino_t     ino_t;
\end{verbatim}
and the true type is (defined in \verb!include/asm/posix_types.h!)
\begin{verbatim}
typedef unsigned long  __kernel_ino_t;
\end{verbatim}
the reason is that the compiler doesn't know that \verb!__kernel_ino_t! is a type,
but at the end of \verb!linux/posix_types.h! it already include
\begin{verbatim}
#include <asm/posix_types.h>
\end{verbatim}
So the true problem is that when we do symbolic link \verb!include/asm! to
\verb!include/asm-ppc!, the final output of 
\begin{verbatim}
~/x-tools/powerpc-405-linux-gnu/powerpc-405-linux-gnu/sys-root/usr/include/
\end{verbatim}
does not have the \verb!asm! folder. To resolve the problem, we need to make a copy of
\verb!asm-ppc! to \verb!asm! in the folder of Linux kernel source header files. However, you may still
get the same problem when compiling \verb!strace! package. The reason is that the file
\begin{verbatim}
.build/src/strace/file1.c
\end{verbatim}
does not include \verb!<asm/posix_types.h!, so add that at the beginning of \verb!file1.c! 
and recompile the crosstool-NG.
% , as 
% we should use
% \begin{verbatim}
% #include <asm/posix_types.h>  // instead of <asm/types.h>
% \end{verbatim}
% so that the compiler knows to map to \verb!unsigned long!.
 
% the reason is that \verb!<linux-kernel>/include/asm/termios.h!
% also include \verb!<asm/termbits.h>!.
% Both files define \verb!struct termios!. RESOLUTION:
% block this line from the file \verb!termios.h!
% \begin{verbatim}
% /*#include <asm/termbits.h>*/
% \end{verbatim}
 
Example:
\begin{verbatim}
Paths and misc options
Paths
Local tarballs directory: ${HOME}/sources
Save new tarballs: Yes
Prefix directory: /home/daglitch/v2plin/crosstool-ng-1.6.1/${CT_TARGET}
Build behavior
Number of parallel jobs: 3 or 4 (if you have a quad-core computer)
Nice level: 10
Logging
Maximum log level to see: EXTRA
Target options
Target Architecture: powerpc
Emit assembly for CPU: 405
Tune assembly for CPU: 405
Floating point: software
Toolchain options
Vendor string: 405
Operating System
Target OS: linux
Linux kernel version: 2.6.32.7
binutils
binutils version: 2.20
binutils libraries for the target: Yes (libiberty and libbfd)
C compiler
gcc version: 4.4.3
C++: Yes
Fortran: Yes (optional)
Java: Yes (optional)
(Untested: enable any other languages you may want)
C-library
C library: glibc
glibc version: 2.9
Threading implementation to use: nptl
Use the ports addon: Yes
Tools (Optional, Untested!)
libelf: Yes
sstrip: No (because I couldn't get it to build)
Debug facilities (Optional, Untested!)
dmalloc: Yes (5.5.2)
duma: Yes (2_5_15)
gdb: Yes
Build a static cross gdb: Yes
Native gdb: Yes
Build a static native gdb: Yes
Use GMP and MPFR: yes
Version: 6.8
ltrace: No (couldn't get it to build)
strace: No (because I couldn't get it to build)
Companion Libraries
GMP and MPFR
Build libraries for target: Yes
GMP version: 4.3.1
MPFR version: 2.4.2
\end{verbatim}


\section{Root file system (RFS image or folder structure)}
\label{sec:root-file-system}
\label{sec:RFS}

A Linux kernel is not very useful without a root file system (RFS), e.g. initrd
(Sect.\ref{sec:initrd}). It is a folder structure (compressed into a image file,
and will be loaded into RAM as a virtual filesystem) containing
\begin{itemize}
  \item a set of files/subfolders necessary for the system to run

This depends on which Linux kernel versions you want to use
\begin{itemize}
  \item Linux kernel 2.4 and earlier: Sect.\ref{sec:RFS-Linux-kernel-2.4}
  \item Linux kernel 2.6 and earlier: Sect.\ref{sec:RFS-Linux-kernel-2.6}
\end{itemize}
NOTE: There are significant changes from Linux 2.4 to Linux 2.6, so the folder
structures of the RFS are different (Sect.\ref{sec:kernel-change-2.4-to-2.6}).
 %Consult Sect.\ref{sec:rootfs} for the content of this files/folders structure
  
  \item a set of applications and utilities

The Busybox is a collection of utilities that can be put in the RFS
(Sect.\ref{sec:busybox}).

\end{itemize}
\url{https://www.tecmint.com/linux-directory-structure-and-important-files-paths-explained/}

The whole folder structure can be compressed into an image file:
initrd or initramfs. Sect.\ref{sec:boot-process-Linux-kernel} how a root
filesystem image is used. Also, you can keep a regular RFS
folder that is mountable over the network (NFS) - Sect.\ref{sec:rootfs}).
  
  
A RFS image can be created in several formats:
\begin{enumerate}
  \item (Linux kernel 2.4) create a RAMDISK image file in ext2 format (e.g.
  initrd  image - Sect.\ref{sec:initrd}).

% It is similar to creating a initial RAM disk (Sect.\ref{sec:initrd}) except that
% you don't have to create an image file.
  
  \item (LInux kernel 2.6) create a cpio archive (e.g. a initramfs image).
    
  \item Create RAMDISK in ext3 format
  
\begin{verbatim}

\end{verbatim}
  
  \item Create image in JFFS2 format (jffs2-image):
  
  Make sure the complete RFS is available in \verb!./jffs2-test! folder, and we
  build a jffs2-image  to store into flash of embedded device to boot from it.

\begin{verbatim}
su
mkfs.jffs2 --little-endian --eraseblock=0x20000 -n --pad -d ./jffs2-test -o jffs2-root.img
\end{verbatim}
NOTE: add \verb!--squash! to enable root squashing (i.e. chown to root for all
files/folders - but do not use it if they already have the proper ownership).
    
  \item Create a U-bootable image to stored in embedded flash ROM
  
From the RADMSIK image file above, run \verb!mkimage! (Sect.\ref{sec:mkimage})  
\end{enumerate} 
\url{http://processors.wiki.ti.com/index.php/Creating_a_Root_File_System_for_Linux_on_OMAP35x}



\subsection{Kernel 2.4 RFS}
\label{sec:RFS-Linux-kernel-2.4}

This section describes the minimal required folder structure for a RFS in Linux
kernel 2.4 and before. A corrupted root filesystem will
generally mean that the system becomes unbootable.
\begin{verbatim}
/bin/
/dev/
/sbin/
/etc/
/root/
/lib/
/lib/modules/
/dev/
/tmp/
/mnt/
/proc/
/usr/
/var/
/home/
\end{verbatim}

\url{http://www.tldp.org/HOWTO/Bootdisk-HOWTO/buildroot.html}

\subsection{Kernel 2.6 RFS}
\label{sec:RFS-Linux-kernel-2.6}

This section describes the minimal required folder structure for a RFS in Linux
kernel 2.6 and above. 

Suppose the folder holding the RFS structure is name
\verb!rootfs!
\begin{verbatim}
mkdir rootfs
cd rootfs
mkdir bin dev etc lib mnt proc root sbin sys tmp usr var
\end{verbatim}
totally 11 subfolders.

Create a main folder, which serves as the root filesystem. Inside that folder,
we create other files/fodlers, e.g. /dev, /bin, /etc, and /sbin
\begin{verbatim}
su
mkdir rootfs 
cd rootfs
mkdir -p {boot,proc,sys,mnt,sbin,dev,lib,usr/bin}
 # add required device files
 # if you want to output to the console
mknod -m 644 dev/console c 5 1
 #use a loop device, if booting from CD and rootfs is in compressed filesystem
 # Using a loopback device you can create a three megabyte file on your hard
 #disk and build the filesystem on it.
mknod -m 644 myinitfs/dev/loop0 b 7 0   	
\end{verbatim}
\url{http://www.tldp.org/HOWTO/Bootdisk-HOWTO/buildroot.html}

Then, we need to put shell script and utilities at least from BusyBox.
An example to create the RFS for Linux kernel 2.6 with utilities using Busybox
is given in Sect.\ref{sec:create-RFS-BusyBox}.

\url{http://www.tldp.org/LDP/intro-linux/html/sect_03_01.html}

% 
%  When the system boots up, the system requires some essential device
% nodes

\url{http://www.linux.org/threads/specfs-devfs-tmpfs-and-others.4989/}



The first process on a normal unix system is called init (Sect.\ref{sec:init}).
Its job is to start other services. BusyBox contains an \verb!init! binary,
which we can use. In addition to the init binary (usually located in /sbin),
you'll need its configuration files (usually called /etc/inittab -
Sect.\ref{sec:inittab}) - some modern init replacement do away with that file
but you won't find them on a small embedded system) that indicate what services
to start and when.
  
\begin{enumerate}
  \item  Simply pass \verb!init=/bin/sh! to the kernel command line!

  \item Come with a set of utilities, e.g. busybox
  (Sect.\ref{sec:create-RFS-BusyBox})
  
Using \verb!/etc/inittab! is optional; if it's missing, you get a root shell on
the console and the script /etc/init.d/rcS (default location) is executed at
boot time.

\end{enumerate}	

Unless all your executables are statically linked, you will need the
  dynamic loader (ld.so, which may be called by different names depending on the
  choice of libc and on the processor architectures) and all the dynamic
  libraries (/lib/lib*.so, perhaps some of these in /usr/lib) required by these
  executables.
  
\subsection{Create a dummy RFS}
\label{sec:create-dummy-RFS}

Create a dummy root filesystem, which contains only a single executable file;
and we put it in \verb!cpio! archive like initramfs image
(Sect.\ref{sec:initramfs}).

This dummy root filesystem basically wait for user to press a key, and print a
message. IMPORTANT: The programs must runs infinitely, if it returns then kernel
panic.

\begin{verbatim}
#include<stdio.h>
int main(){
    while(1){
        printf("Hello Open World\n");
        getchar();
    }
\end{verbatim}
The endless loop (while(1)) will print a message when the user presses a key. 

Build the following code as static, i.e. we use \verb!-static! for the target
machine, e.g. ARM
\begin{verbatim}
# arm-none-linux-gnueabi-gcc hello.c -static -o hello
\end{verbatim}

Copy the output binary executable file to a cpio archive which will be
used as a root filesystem.
\begin{verbatim}
# echo hello | cpio -o --format=newc > rootfs
1269 blocks

# (to check the file)
# file rootfs
rootfs: ASCII cpio archive (SVR4 with no CRC)
\end{verbatim}
%The command cpio is used for this purpose.

\subsection{Create RFS + Busybox}
\label{sec:create-RFS-BusyBox}

The example above (Sect.\ref{sec:create-dummy-RFS}) is not practicle, as the
RFS does not have a complete folder structure.
Here, we create a complete folder structure, put some essential utilities
(Busybox), and (optionally put some Linux kernel modules - given in
Sect.\ref{sec:add-kernel-modules-to-RFS}).

\textcolor{red}{CRITICAL: Whenever you create a root filesystem, make sure
you are under root privilege. If you want to copy the root filesystem folder
structure, use the following command}
\begin{verbatim}
su
tar -cvf backup_rfs.tar <folder-of-rfs>
\end{verbatim}
and you can freely copy the tar file to another place, then at the new
machine/location, also under root privilege, type
\begin{verbatim}
su
tar -xvf backup_rfs.tar
\end{verbatim}


\begin{enumerate}
  
  \item Setup skeleton of folder for the RFS: Depending on Linux kernel 2.4
  (Sect.\ref{sec:RFS-Linux-kernel-2.4}) or Linux kernel 2.6+
  (Sect.\ref{sec:RFS-Linux-kernel-2.6})
    
Suppose the name of the folder holding the RFS structure is \verb!rootfs!.
    
  \item Download, Expand and Jump to the busybox source:

  Compile busybox (using the correct cross-compiler) to the target machine
  where the Linux kernel image 
  will runs (Sect.\ref{sec:busybox-compile-target-machine}), 
  and specify the location where the build will be save which is
  the name of the folder holding the RFS via 
  \verb!CONFIG_PREFIX! argument. 
\begin{verbatim}
// untar 
// then
cd <busybox-folder> 
// make necessary patch so that busybox work best on target machine
// e.g. target machine is mip405
cp -iv <path>/busybox-mip405.config .config

make oldconfig
make

make install CONFIG_PREFIX=</absolute/path/to/rootfs>
\end{verbatim}
\end{enumerate}  

\textcolor{red}{EXAMPLE}: Linux 2.6.12 and Busybox with target machine is MPL
MIP405T (Chap.\ref{chap:embedded_MPL_MIP405-MIP405T}). First create the folder
where you want to store the RFS structure
\begin{verbatim}
cd <Linux-kernel-2.6-source-dir>
mkdir ./rfs_image
\end{verbatim}

Next, install busybox (Sect.\ref{sec:busybox}) into this folder
\begin{verbatim}
cd <busybox-src>
export CROSS_COMPILE=powerpc-405-linux-gnu-
make menuconfig
make install CONFIG_PREFIX='<path-to-rfs_image>'
\end{verbatim}

Among the commands that Busybox supports, it provides \verb!init!-like
capabilities, which is well adapted to embedded environment as it provides most
of the \verb!init! functionality that an embedded system needs without dragging
the weight of using System V \verb!init! (Sect.\ref{sec:init}). So make a
symbolic link
\begin{verbatim}
cd <rfs_image>
ln -s /bin/busybox init
cd <rfs_image>/sbin
ln -s /bin/busybox init
\end{verbatim}
At this point, there are only these folders and files
\begin{verbatim}
drwxr-xr-x  2 root root 4096 Nov 21 10:20 bin
lrwxrwxrwx  1 root root   11 Nov 21 10:20 linuxrc -> bin/busybox
drwxr-xr-x  2 root root 4096 Nov 21 10:20 sbin
drwxr-xr-x  4 root root 4096 Nov 21 10:20 usr
\end{verbatim}

NOTE: The 1.11.1 version of BusyBox does not create a link for /bin/sh so change
any /bin/sh to /bin/busysbox.
\begin{verbatim}
cd bin
sudo ln -s sh busybox
\end{verbatim}

NOTE: we need these folders:
\verb!dev, dev/pts, etc, etc/init.d, lib!, 
\verb!mnt, opt, proc, root, sys, tmp, var, var/log!, and device node for initial
console.
\begin{verbatim}
cd rfs_image
sudo mkdir dev  dev/pts  etc  etc/init.d  lib  mnt  opt  proc  root  sys  tmp 
sudo mkdir var  var/log 
sudo mknod dev/console c 5 1
sudo mkdir debug
\end{verbatim}
Since 2.6.11 the debugfs has been introduced and so you may optionally create a
directory called debug for mounting the debugfs (Sect.\ref{sec:debugfs}).

To have the /proc and /dev/pts file systems mounted at boot time, the file
\verb!etc/fstab! must be created in the etc directory. \verb!vi etc/fstab! and
add
\begin{verbatim}
proc            /proc           proc    defaults        0 0
none            /dev/pts        devpts  mode=0622       0 0
/dev/nfs       /               nfs     defaults        0 0
\end{verbatim}

Create the minimum \verb!/etc/group!, \verb!/etc/passwd!, and \verb!/etc/hosts!
with the contents
\begin{verbatim}
[root@localhost etc]# vi group
root:x:0:root

[root@localhost etc]# vi passwd
root::0:0:root:/root:/bin/ash 

[root@localhost etc]# vi hosts
127.0.0.1       localhost
\end{verbatim}
\textcolor{red}{NOTE:} if you upgrade the kernel, you can copy the three files
from existing kernel settings.

\textcolor{red}{IMPORTANT}: The kernel starts \verb!/sbin/init! after it boots
(actually the kernel attempts to execute several known programs until one
succeeds). Init reads the etc/inittab file to determine what to do at start up,
shutdown, or when a user logs in (Sect.\ref{sec:inittab}). 

NOTE: 
\begin{itemize}
  \item Please note that in more recent kernels (2.6.29 and newer) the
\verb!sysfs! entries shown below to disable power management and control the LCD
inactivity timeout shown above have been removed.
\url{http://elinux.org/OMAP_Power_Management}
  
\end{itemize}


\begin{verbatim}
#!/bin/sh
#   ---------------------------------------------
#   Common settings
#   ---------------------------------------------
HOSTNAME=OMAP3EVM
VERSION=1.0.0

hostname $HOSTNAME

#   ---------------------------------------------
#   Prints execution status.
#
#   arg1 : Execution status
#   arg2 : Continue (0) or Abort (1) on error
#   ---------------------------------------------
status ()
{
       if [ $1 -eq 0 ] ; then
               echo "[SUCCESS]"
       else
               echo "[FAILED]"

               if [ $2 -eq 1 ] ; then
                       echo "... System init aborted."
                       exit 1
               fi
       fi

}

#   ---------------------------------------------
#   Get verbose
#   ---------------------------------------------
echo ""
echo "    System initialization..."
echo ""
echo "    Hostname       : $HOSTNAME"
echo "    Filesystem     : v$VERSION"
echo ""
echo ""
echo "    Kernel release : `uname -s` `uname -r`"
echo "    Kernel version : `uname -v`"
echo ""


#   ---------------------------------------------
#   MDEV Support
#   (Requires sysfs support in the kernel)
#   ---------------------------------------------
echo -n " Mounting /proc             : "
mount -n -t proc /proc /proc
status $? 1

echo -n " Mounting /sys              : "
mount -n -t sysfs sysfs /sys
status $? 1

echo -n " Mounting /dev              : "
mount -n -t tmpfs mdev /dev
status $? 1

echo -n " Mounting /dev/pts          : "
mkdir /dev/pts
mount -t devpts devpts /dev/pts
status $? 1

echo -n " Enabling hot-plug          : "
echo "/sbin/mdev" > /proc/sys/kernel/hotplug
status $? 0

echo -n " Populating /dev            : "
mkdir /dev/input
mkdir /dev/snd

mdev -s
status $? 0

#   ---------------------------------------------
#   Disable power management
#   (Requires sysfs support in the kernel)
#   ---------------------------------------------
# echo -n " Disabling Power mgmt       : "
# echo -n "1" > /sys/power/cpuidle_deepest_state
# status $? 1

#   ---------------------------------------------
#   Turn off LCD after 1 hour of inactivity
#   (Requires sysfs support in the kernel)
#   ---------------------------------------------
# echo -n " Turn off LCD after 1 hour  : "
# echo -n "3600" > /sys/power/fb_timeout_value
# status $? 1


#   ---------------------------------------------
#   Mount the default file systems
#   ---------------------------------------------
echo -n " Mounting other filesystems : "
mount -a
status $? 0


#   ---------------------------------------------
#   Set PATH
#   ---------------------------------------------
export PATH=/bin:/sbin:/usr/bin:/usr/sbin:/usr/local/bin


#   ---------------------------------------------
#   Start other daemons
#   ---------------------------------------------
echo -n " Starting syslogd           : "
/sbin/syslogd
status $? 0

echo -n " Starting telnetd           : "
/usr/sbin/telnetd
status $? 0


#   ---------------------------------------------
#   Done!
#   ---------------------------------------------
echo ""
echo "System initialization complete."


#   ---------------------------------------------
#   Start demo app
#   ---------------------------------------------
#if [[ -x /etc/init.d/demo_start ]]; then
#       echo " Starting Demo Application..."
#       /etc/init.d/demo_start &
#       sleep 5
#fi
\end{verbatim}

Next, copy the binary files from the crosstool-chain to the \verb!bin! folder
\begin{verbatim}
sudo cp ~/Tuan/crosstool-ng-1.9.3/.build/tools/bin <rfs_image>/bin/
\end{verbatim}
which provides these binary
\begin{verbatim}
bash     grep     install  libtool  make     objcopy  objdump  patch    readelf 
sed 
\end{verbatim}

Next, copy the glibc library from the cross-compiler toolchain, e.g.
crosstool-ng put it in \verb!.build! folder. Make sure copy the minimum first, 
to avoid oversizing the kernel libraries.
\begin{verbatim}
strings .build/powerpc-405-linux-gnu/build/build-libc/libc.so.6  | grep 'GLIBC'

GLIBC_2.0
GLIBC_2.1
GLIBC_2.1.1
GLIBC_2.1.2
GLIBC_2.1.3
GLIBC_2.2
GLIBC_2.2.1
GLIBC_2.2.2
GLIBC_2.2.3
GLIBC_2.2.4
GLIBC_2.2.6
GLIBC_2.3
GLIBC_2.3.2
GLIBC_2.3.3
GLIBC_2.3.4
GLIBC_2.4
GLIBC_2.5
GLIBC_2.6
GLIBC_2.7
GLIBC_2.8
GLIBC_2.9
GLIBC_PRIVATE

\end{verbatim}

At this step, you boot it into a prompt without login request. 
Next steps
\begin{itemize}
  \item add kernel modules (Sect.\ref{sec:kernel-modules}) to the RFS -
  Sect.\ref{sec:add-kernel-modules-to-RFS}
  
  \item Configure the login prompt (Sect.\ref{sec:login-prompt})
\end{itemize}

\subsection{Enhance the root filesystem: add kernel modules}
\label{sec:add-kernel-modules-to-RFS}

The root file system built contains only the bare minimum files to operate.
Kernel driver modules have not been added yet. Also, as applications are built
and added, the shared libraries used by the applications will need to be added.
The file system will get larger when these files are added.

Next, put the Linux kernel modules into this RFS folder (the folder that
contains \verb!lib/modules!, i.e. so that the modules will be saved there)

\begin{verbatim}
cd <Linux-kernel-2.6-source-dir>
sudo make modules_install INSTALL_MOD_PATH='<path-to-rfs_image>'
\end{verbatim}
to reduce RFS memory foot print of kernel modules, we can strip some debug information from the
driver modules by adding one of these options to the command
\begin{verbatim}
INSTALL_MOD_STRIP=1
INSTALL_MOD_STRIP=--strip-unneeded
\end{verbatim}

The busybox executable and all symbolic links to it must have user and group
ownership set to \verb!root!. This is required because the kernel boots as
root. If the generated root file system does not have user/group set to root,
the kernel will not boot.
\begin{verbatim}
sudo chgrp root rfs_image/* -R
ssudo chown root rfs_image/* -R
\end{verbatim}

In Linux root file systems the /dev directory contains device nodes - special
files that give application access to devices on the system.
The problem when creating a root file system is which device nodes are needed
and which are not? (Sect.\ref{sec:devices}).
\begin{itemize}
  \item early on Linux RFS, it need to statically build the nodes with
  \verb!mknod! command
\begin{verbatim}
 mknod target/dev/ttyS0 c 4 64
\end{verbatim}
This led to missing nodes, unnecessary nodes, or nodes with the wrong name

  \item Linux 2.4.x : it uses \verb!devfs! filesystem (sect.\ref{sec:devfs})
  
  When device drivers loaded they would register with the devfs file system and
  the device node would be created. 

NOTE: \verb!devfs! is completely removed since Linux 2.6.16.
  
  \item Linux 2.6.11:  There were problems with \verb!devfs! this so that starting at 2.6.11 kernels,
  devfs was removed and replaced. The mechanism to pupulate \verb!/dev! is done
  via \verb!sysfs! (Sect.\ref{sec:sysfs}) mounted on /sys and \verb!tmpfs! file
  system (Sect.\ref{sec:tmpfs}).
  
%   SOLUTION: During  initialization, when file systems are getting mounted, the
%   \verb!sysfs! file system (Sect.\ref{sec:sysfs}) is mounted on /sys.
%   
%   \verb!udev! file system (Sect.\ref{sec:udev}) is mounted on /dev.
  
  That way all device nodes are in virtual memory and are not saved between
  reboots of the board (Sect.\ref{sec:hotplug_udev}).
  
 
  
\end{itemize}



In Linux kernel 2.6.11, the kernel source file init/main.c includes these lines:
\begin{verbatim}
    /*
     * We try each of these until one succeeds.
     *
     * The Bourne shell can be used instead of init if we are
     * trying to recover a really broken machine.
     */

    if (execute_command)
        run_init_process(execute_command);

    run_init_process("/sbin/init");
    run_init_process("/etc/init");
    run_init_process("/bin/init");
    run_init_process("/bin/sh");

    panic("No init found.  Try passing init= option to kernel.");
\end{verbatim}



The 2.6 kernel build process always creates a gzipped cpio format \verb!initramfs!
archive and links it into the resulting kernel binary. 
  
\begin{verbatim}
// create RFS only gzipped \verb!cpio! format  
// NOTE: <location of RFS root directory>
cd <path-to-rfs_image>
find . | cpio -o -v -H newc | gzip -9 -n > ~/mip405t_rfs.gz

// or use the script come with the Linux kernel source
cd <Linux-kernel-2.6-source-folder>
sh scripts/gen_initramfs_list.sh -o ~/mip405t_rfs.gz -u squash -g squash \
       <path-to-rfs_image>
\end{verbatim}



\subsection{Create U-boot Flash image of RFS}
\label{sec:Create-U-boot-image}

First, you need to create a ramdisk image (Sect.\ref{sec:RAMDISK-for-RFS}),
and then we build \verb!ramdisk! file using \verb!mkimage!
(Sect.\ref{sec:mkimage}).

\begin{verbatim}
$ mkimage -T ramdisk -C gzip -n 'Test Ramdisk Image' -d ramdisk.img.gz uRamdisk

Image Name:   Test Ramdisk Image
Created:      Sun Jun 12 16:58:06 2005
Image Type:   PowerPC Linux RAMDisk Image (gzip compressed)
Data Size:    1618547 Bytes = 1580.61 kB = 1.54 MB
Load Address: 0x00000000
Entry Point:  0x00000000
\end{verbatim}
\url{http://www.denx.de/wiki/view/DULG/RootFileSystemOnARamdisk}
 
\subsection{Create U-boot Flash image of RFS + Linux kernel}
\label{sec:Create-Flash-image}

Example: we put 3 binary files (\verb!u-boot.bin! file, \verb!zImage! linux
kernel image, \verb!ramdisk! of RFS) into the
single image file \verb!flash.bin!, at a distance of 2MiB. 
\begin{itemize}
  \item we first create two U-boot images: zImage.uimg, rootfs.uimg
  
  \item create 6MiB empty file \verb!flash.bin!
  
  \item copy the content of \verb!u-boot.bin!
  \item copy the content of \verb!zImage.uimg! at 2MiB from the beginning of
  flash.bin file
  \item copy the content of \verb!rootfs.uimg! at 4MiB from the beginning of
  flash.bin file
\end{itemize}
Here, we build for arm architecture
\begin{verbatim}
mkimage -A arm -C none -O linux -T kernel -d zImage -a 0x00010000 -e 0x00010000 zImage.uimg
mkimage -A arm -C none -O linux -T ramdisk -d rootfs.img.gz -a 0x00800000 -e 0x00800000 rootfs.uimg
dd if=/dev/zero of=flash.bin bs=1 count=6M
dd if=u-boot.bin of=flash.bin conv=notrunc bs=1
dd if=zImage.uimg of=flash.bin conv=notrunc bs=1 seek=2M
dd if=rootfs.uimg of=flash.bin conv=notrunc bs=1 seek=4M
\end{verbatim}


\section{Linux kernel modules}
\label{sec:kernel-modules}

Linux Kernel modules (LKMs) are pieces of code that can be loaded and unloaded
into the kernel upon demand. They extend the functionality of the kernel without the need
to reboot the system, nor recompile the kernel.
\url{http://www.tldp.org/HOWTO/Module-HOWTO/x73.html}

LKMs did not exist in Linux in the beginning.
LKMs have been around at least since Linux 1.2 (1995).
So, the kernel you buid is indeed a {\bf base kernel}
(Chap.\ref{chap:build-Linux-kernel}), and LKMs (when loaded) are very much part
of the kernel, and communicate with the base kernel.

However, a LKM is compiled for a specific ABI version of the kernel. So, if
there is a kernel update that changes ABI (Sect.\ref{sec:ABI}), the module needs to be
recompiled.

To build a kernel module, read Sect.\ref{sec:build-Linux-kernel-module}.


As the Ubuntu updates appear to break the ABI quite frequently and require
kernel modules to be recompiled. As this is a very much boring and repetitive
task, DKMS was invented to overcome this (Sect.\ref{sec:DKMS}). 
It inserts hooks in APT to trigger compiling and installing the new version.

\subsection{DKMS}
\label{sec:DKMS}

The mechanism of DKMS is specifically created for automatic recompiling
kernel modules against the new kernel installed when there is an ABI change for
that new Linux kernel.

Example: \verb!nvidia-XXX! (XXX=version) driver is a kernel module

To check current kernel modules installed using DKMS
\begin{verbatim}
dkms status
\end{verbatim}
\url{https://askubuntu.com/questions/249509/how-does-updating-the-kernel-affect-custom-modules}


To rebuild all DKMS modules for all installed kernels
\begin{verbatim}
ls /var/lib/initramfs-tools | \
    sudo xargs -n1 /usr/lib/dkms/dkms_autoinstaller start
\end{verbatim}
\url{https://askubuntu.com/questions/53364/command-to-rebuild-all-dkms-modules-for-all-installed-kernels}


\subsection{extension: .ko or .o ?}

The loadable module has the extension \verb!.ko! (kernel object) since Linux
kernel 2.6, e.g. nvidia.ko (before 2.6, the extension is .o). 

\subsection{load/unload a modules}
\label{sec:rmmod}
\label{sec:load-module}
\label{sec:unload-module}
\label{sec:insmod}
\label{sec:modprobe}


\textcolor{red}{To load a KLM}, we use \verb!insmod! or \verb!modprobe!.
\begin{verbatim}
insmod mymodule.ko
\end{verbatim}

NOTE: \verb!modprobe! load the modules from
\verb!/lib/modules/$(uname -r)/modules.dep.bin! (or without the .bin suffix if
the other file is not available)

NOTE: \verb!insmod! on the other hand accepts paths to files.
The module does not have to reside in \verb!/lib/modules/$(uname -r)!, but
dependencies are not automatically loaded. 

\textcolor{red}{To unload a KLM}, we use \verb!rmmod!
\begin{verbatim}
rmmod mymodule.ko
\end{verbatim}
\verb!rmmod! removes a kernel name based on the name from /proc/modules.
his name does not necessarily have to be the same as the one passed to
\verb!modprobe!.

Once loaded, they are in \verb!/lib/modules!. To list all the loaded modules, we
use
\begin{verbatim}
lsmod
\end{verbatim}
command.  Similar commands in FreeBSD
\begin{verbatim}
kldload
kldunload
kldstat
\end{verbatim}
Similar commands in OS/X (extension is .kext)
\begin{verbatim}
kextload
kextstat
\end{verbatim}
The modules distributed by 
\begin{itemize}
  \item  the Linux OS is in \verb!/boot/kernel! or
\verb!/boot/modules!. 

  \item the Mac OS/X are in \verb!/System/Library/Extensions!.
\end{itemize}

If one module is broken which can affect the boot up processes, we can disable
it or replace with an alternative one by modifying the kernel boot parameters in
the bootloader, e.g. GRUB (press 'e' in the GRUB boot menu to edit). There are
free as well as proprietary modules.

References:
\begin{itemize}
  \item
  \url{http://askubuntu.com/questions/20070/whats-the-difference-between-insmod-and-modprobe}
\end{itemize}

\subsection{module information}
\label{sec:modinfo}

\verb!modinfo! accepts a filename, or the filename without .ko suffix in
\verb!/lib/modules/$(uname -r)!.

\begin{verbatim}
modinfo hello-4.ko 
license:        GPL
author:         Peter Jay Salzman <p@dirac.org>
description:    A sample driver
vermagic:       2.6.5-1.358 686 REGPARM 4KSTACKS gcc-3.3
depends:        

\end{verbatim}

\subsection{Troubleshoot}

\textcolor{red}{\bf Error 'wrong magic number'}:
When you load a module, you get the error
\begin{verbatim}
version magic '3.3.0-xxx SMP preempt mod_unload ARMv7 ' should be '3.3.0-xxx+
    SMP preempt mod_unload ARMv7 '
          invalid module format    
\end{verbatim}
or
\begin{verbatim}
version magic '2.6.12-mip405t preempt gcc-4.2' should be
          '2.6.12-mip405t preempt gcc-4.3'          
          invalid module format
\end{verbatim}

The reason is due to the mismatch in the version magic number, which can be
checked using \verb!modinfo! (Sect.\ref{sec:modinfo})

\begin{verbatim}
modinfo -F vermagic myModule.ko
 # or
objdump myModule.ko --full-contents --section=.modinfo
  
Result:  
  2.6.23.1-42.fc8 SMP mod_unload 
     means kernel was build with CONFIG_MODULE_UNLOAD set.
     most distros enable module unloading; otherwise
     when you run "rmmod myModule" will cause this:
     - FATAL: Kernel does not have unload support
     
modules built against 2.6.x.y.* (e.g. 2.6.25.9-76.fc9.x86_64)
will load into 2.6.x.y kernels.          
\end{verbatim}
 
Here are different options
\begin{enumerate}
  \item force it to load (make sure the kernel was compiled with
  \verb!MODULE_FORCE_LOAD! for Linux kernel 2.6.26+)
   
\begin{verbatim}
insmod -f mymodule.ko
\end{verbatim}
-f or --force options is the same as --force-vermagic
and -force-modversion.

Behind the scene: insmod/modprobe call a system call,
\verb!sys_init_module()! or \verb!check_version()! (kernel/module.c). See the
code snipet
\begin{verbatim}
modmagic = get_modinfo(sechdrs, infoindex, "vermagic");
/* This is allowed: modprobe --force will invalidate it. */
if (!modmagic) {
tainted |= TAINT_FORCED_MODULE;
printk(KERN_WARNING "%s: no version magic, tainting kernel.\n",
        mod->name);
} else if (!same_magic(modmagic, vermagic)) {
        printk(KERN_ERR "%s: version magic '%s' should be '%s'\n",
               mod->name, modmagic, vermagic);
        err = -ENOEXEC;
        goto free_hdr;
}
\end{verbatim}

  \item 
  Or the quick fix is to modify the \verb!version magic number! in the source file
of the target Linux kernel, so that all modules compiled against the kernel will
load with no issues.
DETAILS explanation: \verb!mymodule.mod.o! get the vergion magic number from
\verb!include/linux/vermagic.h! 
\begin{verbatim}
#include <linux/version.h>
#include <linux/module.h>

#define VERMAGIC_STRING                                                 \
        UTS_RELEASE " "                                                 \
        MODULE_VERMAGIC_SMP MODULE_VERMAGIC_PREEMPT                     \
        MODULE_ARCH_VERMAGIC                                            \
        "gcc-" __stringify(__GNUC__) "." __stringify(__GNUC_MINOR__)                                                                            
\end{verbatim}
which returns 
\begin{verbatim}
2.6.12-mip405t preempt gcc-4.3
\end{verbatim}
or if we break down
\begin{verbatim}
UTS_RELEASE = 2.6.12-mip405t
MODULE_VERMAGIC_PREEMPT = preempt
__GNUC__    =  4
__GNUC_MINOR__ = 3
\end{verbatim}

The quick fix for compiler version mismatch is to modify to (gcc-4.3) 
\begin{verbatim}
        "gcc-" __stringify(__GNUC__) ".3"                                                                             
\end{verbatim}
%which in turn includes
%\verb!generated/utsrelease.h!, which supplies the version magic explicitly.

  
 \item You can choose to build the module for many Linux distribution:
\url{http://haifux.org/lectures/192/rrLight.pdf}
 
\end{enumerate}

\url{http://blog.umbrellaj.com/blog/2013/03/15/trick-on-the-verson-magic-number-of-linux-kernel/}
\url{http://billauer.co.il/blog/2013/10/version-magic-insmod-modprobe-force/}


\url{http://tldp.org/LDP/lkmpg/2.6/html/x380.html}

\subsection{compatibility}

COMPATIBILITY ISSUE: A loadable kernel module needs to be compatible in ABI/API
with the kernel version. Free BSD maintain compatibility within minor version,
e.g. a kernel module compiled with Linux kernel 6.1 can be loaded in Linux
kernel 6.4; but not with Linux kernel 7.0. To check for compatibility, read the
symbol versioning data in \verb!.modinfo! section in ELF modules
(Sect.\ref{sec:ELF}).
\begin{verbatim}
objdump myModule.ko --full-contents --section=.modinfo
\end{verbatim}



\section{Compile kernel modules}
\label{sec:build-Linux-kernel-module}
\label{sec:LKM}
\label{sec:build-Linux-kernel-modules}

Essentially, (Linux kernel) modules are to Linux as drivers are to Windows.

To compile, a "typical" 
\begin{verbatim}
./configure; make; sudo make install 
\end{verbatim}
will just install one specific kernel module and requires you to recompile it
every time.  If you fail to do so, the kernel module will simply not be found
after an update.

DKMS helps to automate this process (Sect.\ref{sec:DKMS})
\url{https://askubuntu.com/questions/249509/how-does-updating-the-kernel-affect-custom-modules}

\subsection{Command}

Suppose you are on the host machine
\begin{verbatim}
host: intel x86 32 bit, linux-3.0.0
\end{verbatim}
and try to compile a module for a target system
\begin{verbatim}
target: ARM machine (Parrot AR.Drone), linux-2.6.27.47
\end{verbatim}

You need to configure \verb!CROSS_COMPILE!, \verb!ARCH!, \verb!PATH! (to bin
folder of the cross-compiler).

Example:
\begin{verbatim}
export CROSS_COMPILE=arm-none-linux-gnueabi- 
export ARCH=arm  
\end{verbatim}

This is the Makefile
\begin{verbatim}
PWD := $(shell pwd)
obj-m := test.o
all:
    $(MAKE) -C /path/to/target-machine/kernel M=$(PWD) modules
\end{verbatim}
\url{http://www.gnu.org/software/make/manual/make.html#toc-Features-of-GNU-make}
or examples
\begin{verbatim}
obj-m := myModule.o
CC:= /work/tools/fc6/gcc/bin/gcc
KDIR:=/usr/src/kernels/2.6.18-1.2798.fc6-x86_64
PWD := $(shell pwd)
default:
   $(MAKE) CC=$(CC) -C $(KDIR) SUBDIRS=$(PWD) modules
\end{verbatim}
or custom build by adding the checking code
\begin{verbatim}
#if LINUX_VERSION_CODE >= KERNEL_VERSION(2,2,0)
  ....
#endif
\end{verbatim}

NOTE:
\begin{verbatim}
mymodule.c --> mymodule.mod.c -->   mymodule.mod.o
   
   mymodule.mod.o + mymodule.o  --[MODPOST]-> mymodule.ko
\end{verbatim} 
mymodule.mod.c that is responsible for supplying the version magic. It's
compiled into mymodule.mod.o, and then fused together with mymodule.o to become
mymodule.ko. This is what the famous "MODPOST" compilation stage does (in fact,
it's a \verb!make -f! on \verb!scripts/Makefile.modpost!, which in turn calls
\verb!scripts/mod/modpost!).
  
\subsection{Troubleshoot}

\textcolor{red}{\bf Error}:
You may get the error
\begin{verbatim}
falsely claims to have parameter debug 
\end{verbatim}
when you try to load a module
\begin{verbatim}
insmod module.ko
\end{verbatim}
when trying to load a module from Linux 2.6.*. The problem is due to some
changes (Sect.\ref{sec:module_Linux-2.6.x})

\textcolor{red}{\bf Error}: Another potential error is 
\begin{verbatim}
error: 'MOD_INC_USE_COUNT' undeclared
\end{verbatim}
This was previously use in Linux 2.5.x to manipulate the use count to protect
against module being removed
\footnote{\url{http://people.netfilter.org/rusty/unreliable-guides/kernel-hacking/routines-module-use-counters.html}}
Using \verb!MOD_INC_USE_COUNT! is obsolete and has been removed since Linux
2.5.x (as the kernel can take care of this automatically).



\subsection{*.o: Linux 2.5-}
\label{sec:module_Linux-2.5-}

The purpose of \verb!MODULE_PARM! is to enable passing the command-line
arguments to a module. Suppose you want to be able to pass \verb!myvar=5!
\begin{verbatim}
insmod mymodule.o myvar=5
\end{verbatim}

Then, you need to define the appropriate \verb!MODULE_PARM!
which takes 2 arguments: the name of the variable (which is \verb!'myvar'! in
this case) and its type (which is a single character representing the type).
\begin{itemize}
  \item 'b' = single byte
  \item 'h' = short int
  \item 'i' = integer (can be signed or unsigned)
  \item 'l' = long int
  \item 's' = string

 should be a string declared as \verb!char *!, with the content
 
  \item 'num1-num2<char-type>': represent an array of some minimum and/or
  maximum length (num1= minimum length,  num2=maximum length)
   
\begin{verbatim}
int myshortArray[4];
  # an array of 'int' with length in the range [3-9]
MODULE_PARM (myintArray, "3-9i");

#define SM518_MAX_NR_DEVICES 8

MODULE_PARM(i0, "0-" __MODULE_STRING(SM518_MAX_NR_DEVICES) "l")
\end{verbatim}
NOTE: The macro \verb!__MODULE_STRING! is define in
\verb!<linux-kernel-source>/include/module.h! file
\begin{verbatim}
#define __MODULE_STRING(x) __stringify(x)
\end{verbatim}
with \verb!__strinify(x)! convert the parameter value of a given macro to string

Example:
\begin{verbatim}
-DFOO=bar, __stringify(FOO)
\end{verbatim}
returns \verb!''bar''!.

\end{itemize}
Example:
\begin{verbatim}
char myvar;
char *mystr;

MODULE_PARM(myvar, "i");
MODULE_PARM(mystr, "s");

MODULE_PARM_DESC(myvar, "This integer really does nothing at all.");
MODULE_PARM_DESC(mystr, "Pass this string to do something.");
\end{verbatim}
You may also add the description to explain the parameter with
\verb!MODULE_PARM_DESC! macro.
\url{http://www.tldp.org/LDP/lkmpg/2.4/html/x354.htm}

If you have to use \verb!MOD_INC_USE_COUNT!, make sure add
\verb!MOD_DEC_USE_COUNT! follow this
\begin{verbatim}

/* THIS IS BAD */
foo_open (...)
{
        stuff..
        if (fail)
                return -EBUSY;
        sleep.. (might get unloaded here)
        stuff..
        MOD_INC_USE_COUNT;
        return 0;
}

/* THIS IS GOOD /
foo_open (...)
{
        MOD_INC_USE_COUNT;
        stuff..
        if (fail) {
                MOD_DEC_USE_COUNT;
                return -EBUSY;
        }
        sleep.. (safe now)
        stuff..
        return 0;
}
\end{verbatim}
\url{http://people.netfilter.org/rusty/unreliable-guides/kernel-hacking/routines-module-use-counters.html}



\subsection{*.ko: Linux 2.6+}
\label{sec:module_Linux-2.6.x}

Since the first 2.6.x releases, \verb!MODULE_PARM! has been deprecated, and need
to be replaced by \verb!module_param! macro which now \textcolor{red}{takes 3 arguments}
the name of the variable (similar to above), its type (different: use
full type name) and permissions for the corresponding file in sysfs.
\begin{itemize}
  \item type name: \verb!int!, \verb!charp! (string: char*), \verb!short!,
  \verb!long!
   
  \item permission bits (defined in sys/stat.h): can be a combination of any of
  the following using \verb!|! (or) operator (NOTE: We can use zero (0) as the permission bit)
  \url{http://www.gnu.org/software/libc/manual/html_node/Permission-Bits.html}
\begin{verbatim}
0  
   disable everything
S_IRUGO | S_IWUSR
      
\end{verbatim}
and some pre-defined combination (include/linux/stat.h)
\begin{verbatim}
#define S_IRWXUGO       (S_IRWXU|S_IRWXG|S_IRWXO)
#define S_IALLUGO       (S_ISUID|S_ISGID|S_ISVTX|S_IRWXUGO)
#define S_IRUGO         (S_IRUSR|S_IRGRP|S_IROTH)
#define S_IWUGO         (S_IWUSR|S_IWGRP|S_IWOTH)
#define S_IXUGO         (S_IXUSR|S_IXGRP|S_IXOTH)
\end{verbatim}  
  
  Example:
  \begin{verbatim}
  (S_IRUSR | S_IWUSR | S_IRGRP | S_IWGRP)
  \end{verbatim}
\end{itemize}

Example:
\begin{verbatim}
int myint = 3;
module_param(myint, int, 0);
\end{verbatim}

\textcolor{red}{To handle array, you need to use a new macro}
\verb!module_param_array! (for array of integer) and \verb!module_param_string! (for array of strings)
\begin{verbatim}
module_param_array(name, type, num, perm);
   name = name of array
   type = data type of elements in the array (int, charp, long, ...)
   num = pointer to the variable that will store the number of elements 
         in the array (i.e. array size) (use NULL if not care)
   perm = permission bits (see above, use zero (0) to disable)
\end{verbatim}

Example:
\begin{verbatim}
int myintarray[2];
module_param_array(myintarray, int, NULL, 0); /* not interested in count */

short myshortarray[4];
int count;
module_parm_array(myshortarray, short, &count, 0); /* put count into "count" variable */

static int fish[MAX_FISH];
static int nr_fish;
   # avoid to use the explicit permission number
   # but use the one from sys/stat.h file
   #   which may requires #include <sys/stat.h>
module_param_array(fish, int, &nr_fish, 0444);
\end{verbatim}

{\bf EXTRA}: 
You can name the internal array something different than the external parameter with 
\verb!module_param_array_named()!:

\begin{verbatim}
module_param_array_named(name, array, type, nump, perm);
\end{verbatim}


\section{Build Linux kernel}
\label{sec:build_Linux-kernel}

Before that you can build a kernel, you need to understand the difference
between a build machine vs. a host machine vs. target machine
(Sect.\ref{sec:build-host-target}). If the build machine and the target machine
is different, then read Sect.\ref{sec:cross-compile-Linux-kernel}

The target machine PowerPC is a big-endian, RISC, 64-bit architecture with
32-bit implementations. 32-bit PPC userland binaries run without modification on
64-bit processors. The Linux kernel has run on all POWER/PowerPC processors
since the year 2000. It is usually the primary software enablement target for
PowerPC semiconductor vendors.

Then, you need to know (1) what version of Linux you want to build, (2)
the kernel type
\begin{itemize}
  \item 'vanilla' kernel: officially released by Linus Torvald, or by a member
  of the community appointed by Linus.
  
Link: \url{http://ftp.kernel.org}
   
  \item distribution kernel - the 'vanilla' kernel + patches on top of it
  
  The distribution's kernel has gone through more orderly testing and could be
  more stable. Example:
  \begin{itemize}
    \item RedHat:
    \item Debian:
  \end{itemize}
\end{itemize}

\textcolor{red}{Overview:}
\begin{verbatim}
make mrproper        # remove existing object files and configuration
make menuconfig      # a full-screen text-mode program for you to select configuration
   make config      # a simple text-mode programs (not recommend)
   make xconfig     # a graphical program
make depend         # prepare the dependencies list
make clean
make bzImage        # build the kernel in bzImage
         # the result is stored in arch/<target-architecture>/boot/bzImage 
         # e.g. arch/i386/boot/bzImage 
         
    # the two belows for building kernel modules
make modules        
make modules_install INSTALL_MOD_PATH=<RFS-folder> 
\end{verbatim}

\subsection{compile a Linux kernel}

\textcolor{red}{Details}: This is the steps
\begin{enumerate}
  \item download the kernel source:

  \item make proper adjustment to the source configuration file

\verb!make ***config! create the configuration file, and make sure save them
into \verb!.config! file of the kernel folder to compile, e.g.
\begin{verbatim}
make menuconfig
make xconfig
\end{verbatim}
allows you to select the features you want to compile into the kernels (e.g. SCSI support,
sound support, \ldots). The available values for each feature
 \begin{verbatim}
'y'       = all required software to support that feature is installed as well
'm'       = the feature becomes a loadable module (i.e. can be loaded or
            unloaded at will (see the next methods)
'n'       = the feature is not enabled in the kernel and will not be available            
 \end{verbatim}

Make sure to enable some features for loading kernel modules as given in
Sect.\ref{sec:build-Linux-kernel-modules}, such as
\begin{verbatim}
MODULE_FORCE_UNLOAD=y  # to enable unload a module using 'rmmod -f <module>'
\end{verbatim}

% The same jobs can be done by using 
% \begin{verbatim}
% ./ct-ng menuconfig
% \end{verbatim}
% when trying to built the crosstool-NG command.
  
Once the config file is stored, build the dependencies of your chosen configuration
\begin{verbatim}
make dep
\end{verbatim}
  
  \item run the \verb!make! (with proper arguments) to build kernel source
  into a kernel image (\verb!vmlinuz!, \verb!uImage!, or \verb!bzImage! -
  Sect.\ref{sec:compressed-kernel-file-format})

NOTE: Use \verb!V=1! to turn verbosity on for debugging purpose
\begin{verbatim}
make uImage  V=1
make bzImage  
\end{verbatim}
which creates an image file in \verb!./arch/<architecture>/boot/images! folder.


  \item (optional) \verb!make install! to copy the compiled kernel:
  \verb!make install! install the built kernel into \verb!/vmlinuz! 
  
\textcolor{red}{IMPORTANT: It is strongly recommended to back-up your system}
\verb!/vmlinuz! and 
\verb!/lib/modules/<kernel-version>! before installing a new Linux kernel
module.

NOTE: On your machine, the currently kernel image is \verb!/boot/vmlinuz! (DO NOT delete it),
if the target machine is also the build machine, then you can move the new compress kernel image
to the \verb!/boot/! directory.
\begin{verbatim}
sudo cp ./arch/.../bzImage /boot/testImg
\end{verbatim}

  \item create the kernel modules: those features selected as 'm'
  
\begin{verbatim}
make modules
\end{verbatim}
This will compile individual files for each question you
answered 'm' during kernel config (i.e. use \verb!.config! file).
 
Typically, we want to put the modules to a particular folder being used as a
root filesystem (Sect.\ref{sec:RFS})
\begin{verbatim}
\end{verbatim}

\url{http://unix.stackexchange.com/questions/20864/what-happens-in-each-step-of-the-linux-kernel-building-process}

  \item copy the kernel modules to a particular location:
    
\verb!make modules_install! installs the built kernel into \verb!/lib/modules!
or \verb!/lib/modules/<version>!

You can install into a user-defined folder by adding
\verb!INSTALL_MOD_PATH=<folder>! option.

\begin{verbatim}
make modules_install INSTALL_MOD_PATH=<WhereTheModulesShallGo> 
\end{verbatim}
Typically \verb!INSTALL_MOD_PATH! is the path to the \verb!lib! subfolder
of the RFS (Sect.\ref{sec:RFS}).
\begin{verbatim}
make modules_install INSTALL_MOD_PATH=./my_rfs/lib/
\end{verbatim}

\end{enumerate}


% First we need to know the target machine, i.e. the machine on which the
% Linux kernel image runs.

% Then, we need to have a right compiler, e.g. cross-compiler to compile the Linux
% kernel if the target machine is not the same as the build machine
% (Sect.\ref{sec:cross_compiler}).
% 
% Next, we need to configure the Linux kernel, i.e. 
% \begin{verbatim}
% // choose one
% make menuconfig
% make xconfig
% \end{verbatim}

% Next, we compile the kernel (in the same directory of the Linux kernel source)
% \begin{verbatim}
% make
% \end{verbatim}
% 
% Next, choose to install the kernel (to \verb!/boot/! directory by default) 
% \begin{verbatim}
% make install
% \end{verbatim}
% \url{http://www.linux.org/threads/the-linux-kernel-compiling-and-installing.5208/}

% Next, we need to compile the modules, i.e. saved in a compressed kernel image
% (e.g. zImage or uImage or bzImage)
% \begin{verbatim}
% make uImage modules
% \end{verbatim}
% which creates an image file in \verb!./arch/<architecture>/boot/images! folder.

% Next, we need to install the kernel modules (if we want any parts of the kernel
% to be compiled as 'modules',
% \begin{verbatim}
% sudo make modules_install  INSTALL_MOD_PATH=<location-of-RFS-root-directory>
% \end{verbatim}
% which requires 'root' permission.


Once the new linux kernel is available, you can put the new entry into the
boot-loader, for it to know the location of the new kernel
image
\begin{itemize}
  \item use LILO as boot-loader - Sect.\ref{sec:LILO}
  \item use U-boot as boot-loader - Sect.\ref{sec:U-boot_loader}
  \item use GRUB as boot-loader - Sect.\ref{sec:GRUB1}
\end{itemize}

Example:
{\bf LILO}: append to the content of \verb!/etc/lilo.conf! file
\begin{verbatim}
image = /boot/testImg
root = /dev/hda5
label = test
append = 'idebus=66'
\end{verbatim}
then reinstall LILO
\begin{verbatim}
sudo lilo
sudo reboot
\end{verbatim}

then once the machine is reboot, type at the LILO prompt the label of the new
kernel, i.e. \verb!test! in this case (IMPORTANT: hit TAB to list available boot optionsq)
\begin{verbatim}
LILO boot: test
\end{verbatim}
\url{http://www.csee.umbc.edu/courses/undergraduate/CMSC421/fall02/burt/projects/howto_build_kernel.html}


\subsection{cross-compile a Linux kernel}
\label{sec:cross-compile-Linux-kernel}

To compile the kernel source for the target PowerPC platform, from a non-PowerPC
platform, it is better to use the SoC motherboard manufacturer website for their
distributions' source packages. Distributions apply many fixes and enhancements
to their kernels. However, you can also check the right Linux kernel source:
\url{http://penguinppc.org/dev/\#kernel}

Configure what modules is 'y', 'n' and 'm' as usual
\begin{verbatim}
make menuconfig
\end{verbatim}


Cross-compile a Linux kernel is a little bit different.
First, you need to have a cross-compiler for the target machine is ready
(Sect.\ref{sec:cross_compiler}).
A cross-compiler tool chain is used to build the code on one machine, and for
running on a machine with a different architecture.
The host and target architectures are not binary compatible, so you need to
install a cross-compiler tool chain on the host machine.
Example: The Linux kernel is compiled from source on the host machine (e.g. the
Laptop you are using) which is using Intel CPU architecture. However, the binary one is
expected to run on a target machine, e.g. MIP405/405T which is using ARM CPU
architecture.

Check if the host platform is 64-bit or 32-bit, and the guest platform is 32-bit
or 64-bit. Example:  
  \textcolor{red}{the MIP405/MIP405T use 32-bit CPU}, so we need to
  know how to compile the 32-bit kernel using 64-bit host platform
  (Sect.\ref{sec:compile_32-bit-kernel_from-64-bit-platform})


PARAMETERS: \verb!ARCH=! and \verb!CROSS_COMPILE=! are the two common options,
and others \verb!V=1! (verbose), \verb!-j4! (parallel 4 threads), and target \verb!all!.
  
\verb!make! compiles and links the kernel image (generating \verb!vmlinuz! file
by default). We can choose other format, by using \verb!uImage! or
\verb!bzImage!.
Make sure the cross-compiler is selected \verb!CROSS_COMPILE=! and \verb!ARCH=!
environment variables, before you run
\begin{verbatim}
make

make modules
make modules_install ...
\end{verbatim}
More details: Sect.\ref{sec:Linux_kernel_PowerPC}.

Example:
\begin{verbatim}
make CROSS_COMPILE=powerpc-linux-gnu- ARCH=powerpc V=1 -j4 all

make bzImage CROSS_COMPILE=powerpc-linux-gnu- ARCH=powerpc V=1 -j4 all
\end{verbatim}

For making mmodules
\begin{verbatim}
make uImage modules CROSS_COMPILE=powerpc-linux-gnu- ARCH=powerpc V=1 -j4 all
\end{verbatim}


\subsection{compile target 32-bit kernel on 64-bit host platform}
\label{sec:compile_32-bit-kernel_from-64-bit-platform}

First jump to the source folder of the kernel code to compile.

To compile a 32 bit kernel for x86 on a \verb!x86_64! host, run 
\begin{verbatim}
i386 
\end{verbatim}
(which is a link to /usr/bin/setarch), then proceed as normal (make config;
make).

Other options:
\begin{verbatim}
make ARCH=i386 CFLAGS=-m32 menuconfig

make ARCH=i386 CFLAGS=-m32
\end{verbatim}

To clean the previous compiled code
\begin{verbatim}
make clean
\end{verbatim}
or save your .config somewhere safe for the latter, and then run
\begin{verbatim}
make mrproper
\end{verbatim}

NOTE: 
\begin{verbatim}
make distclean
\end{verbatim}
does everything mrproper does and removes some editor backup and patch leftover
files.

NOTE: \verb!make clean! removes what you've compiled, while \verb!make mrpoper!
does the same but a removing a little bit more (the .config file and a few
others I think).
% When you get the kernel .tar.gz/.tar.bz2 from kernel.org and unpack it in
% /usr/src/ -- that is considered the 'original state'.
% We can all \begin{verbatim} make mrproper \end{verbatim}

\subsection{Build kernel from source (module-assistant)}
\label{sec:Linux_kernel_rebuild}

The kernel built for a given distro (e.g. Ubuntu) has a fixed numbers of
modules, suppose you want to add some new features to the kernel, you can decide
to change the build configuration and rebuild the kernel from source.

\verb!module-assistant! is a powerful, yet not very well documented tool to
build Debian kernel modules 

\url{http://wiki.debian.org/ModuleAssistant}



\subsection{Build Linux kernel 2.4 (minimal) with initrd}
\label{sec:build_Linux-kernel-with-initrd}

\begin{verbatim}
make menuconfig
   CONFIG_BLK_DEV_RAM=y  (RAM disk support)
   CONFIG_INITRAMFS_SOURCE=""
   CONFIG_BLK_DEV_INITRD=y (initrd support)
   CONFIG_BLK_DEV_LOOP=y  (loop device support)
   CONFIG_FAT_FS=y       (fast FS support)
   CONFIG_MSDOS_FS=y     (optional, MSDOS support)
   
   CONFIG_BLK_DEV_RAM=y
   CONFIG_BLK_DEV_RAM_COUNT=1
   CONFIG_BLK_DEV_RAM_SIZE=8192   (i.e. 8 MiB)
   CONFIG_BLK_DEV_RAM_BLOCKSIZE=1024
\end{verbatim}

\url{http://www.tldp.org/HOWTO/archived/Loopback-Root-FS/Loopback-Root-FS-3.html}

\subsection{Build Linux kernel 2.6+ (minimal) with initramfs}
\label{sec:build_Linux-kernel-with-initramfs}

From Linux kernel 2.6, it is a good idea to add the initramfs
(Sect.\ref{sec:initramfs}) into the Linux kernel image if the kernel is designed
to run on embedded system or for testing.

NOTE: By default, \verb!CONFIG_INITRAMFS_SOURCE=``''! (empty).
Suppose the location of of the folder containing the whole RFS structure is
\verb!/home/somewhere/initramfs/!, then we run

\begin{verbatim}
make allnoconfig
make menuconfig

   CONFIG_BLK_DEV_INITRD=y
   CONFIG_INITRAMFS_SOURCE=/home/somewhere/initramfs
   CONFIG_PCI=y
   CONFIG_BINFMT_ELF=y
   CONFIG_SERIAL_8250
   CONFIG_EXT2_FS=y
   CONFIG_IA32_EMULATION=y # might not need if on 32-bit host
   CONFIG_NET=y
   CONFIG_PACKET=y
   CONFIG_UNIX=y
   CONFIG_INET=y
   CONFIG_WIRELESS=n
   CONFIG_ATA=y
   CONFIG_NETDEVICES=y
   CONFIG_NET_VENDOR_REALTEK=y
   CONFIG_8139TOO=y (unchecked all other Ethernet drivers)
   CONFIG_WLAN=n
   CONFIG_DEVTMPFS=y

make
\end{verbatim}

References:
\begin{itemize}
  \item 
\url{http://mgalgs.github.io/2012/03/23/how-to-build-a-custom-linux-kernel-for-qemu.html}

   \item
   \url{http://linuxtips.pbworks.com/w/page/17653834/Howtos Creating Initrd and Initramfs}
\end{itemize}

\subsection{kernel file format}
\label{sec:compressed-kernel-file-format}

The kernel image isn't so much an executable kernel, but a compressed kernel
image.
\begin{enumerate}
  \item \verb!zImage! = old format for small kernels (compressed with zlib,
  below 512KB). At boot, this image gets loaded low in memory (the first 640KB
  of the RAM).

At the Linux kernel source, type
\begin{verbatim}
make zImage
\end{verbatim}  
to create a compressed kernel image (zImage.elf).

  \item \verb!bzImage! = The big zImage (this has nothing to do with bzip2), was
  created while the kernel grew and handles bigger images (compressed, over
  512KB). 
  
  The image gets loaded high in memory (above 1MB RAM). As today's
  kernels are way over 512KB, this is usually the preferred format.

\begin{verbatim}
make bzImage
\end{verbatim}  

\end{enumerate}

\begin{verbatim}
ls ~/Tuan/linux-2.6.12_crosstool_1.9.3./arch/ppc/boot/images/
   uImage
   vmlinux.bin
   vmlinux.gz
   zImage.elf
\end{verbatim}


NOTE: 
\begin{itemize}
  \item \verb!vmlinux! = This is the Linux kernel in an statically linked
  executable file format. Generally, you don't have to worry about this file,
  it's just a intermediate step in the boot procedure.
  
  The raw vmlinux file may be useful for debugging purposes.
  
  \item \verb!vmlinux.bin! = The same as vmlinux, but in a binary file format.
  
  \item \verb!vmlinuz! (=\verb!bzImage!) = the \verb!vmlinux! file compressed
  using \verb!zlib! format. 
\url{http://www.linfo.org/vmlinuz.html}
  
  %This is the same as \verb!bzImage! kernel image.
  
  Since Linux kernel 2.6.30 LZMA and bzip2 are also available.
  By adding further boot and decompression capabilities to vmlinuz, the image
  can be used to boot a system with the vmlinux kernel. 
  
  %The compression of  vmlinux can occur with zImage or bzImage.
  The function \verb!decompress_kernel()! handles the decompression of vmlinuz
  at bootup, a message indicates this:
  \begin{verbatim}
Decompressing Linux... done
Booting the kernel.
  \end{verbatim}
  
  \item \verb!vmliux.gz!
  
  
  \item \verb!make zdisk! creates a boot disk (without root filesystem or LILO)
  (make sure you insert the floppy in A: drive)
  

Example: 
\begin{verbatim}
ls -lh /boot/vmlinuz-$(uname -r)
-rw-r--r-- 1 root root 4.1M 2010-11-24 12:21 /boot/vmlinuz-2.6.35-23-generic

file /boot/vmlinuz-$(uname -r)
/boot/vmlinuz-2.6.35-23-generic: Linux kernel x86 boot executable bzImage, 
    version 2.6.35-23-generic (buildd@rosea, RO-rootFS, root_dev 0x6801,
   swap_dev 0x4, Normal VGA
\end{verbatim}  
  
  \item \verb!uImage! = 
  
  \verb!make uImage! creates a kernel image in uImage
  format (which is required by U-boot - Sect.\ref{sec:U-boot_loader})
  
  
\end{itemize}
\url{http://www.csee.umbc.edu/courses/undergraduate/CMSC421/fall02/burt/projects/howto_build_kernel.html}

\url{http://stackoverflow.com/questions/22322304/image-vs-zimage-vs-uimage}

\url{http://unix.stackexchange.com/questions/5518/what-is-the-difference-between-the-following-kernel-makefile-terms-vmlinux-vml}


\subsection{Tiny ppc cloud image}

\url{https://www.ibm.com/developerworks/community/wikis/home?lang=en#!/wiki/W51a7ffcf4dfd_4b40_9d82_446ebc23c550/page/Building a tiny ppc cloud image for CI testing}

\subsection{To enable USB support}

To enable USB support in Linux kernel, run
\begin{verbatim}
make menuconfig

#
# USB support
#
CONFIG_USB=y
# CONFIG_USB_DEBUG is not set

Also make sure you also compile at least one of the USB controllers also.

#
# USB Host Controller Drivers sample
#
CONFIG_USB_EHCI_HCD=y
CONFIG_USB_OHCI_HCD=y
CONFIG_USB_UHCI_HCD=y
\end{verbatim}

\subsection{login prompt}
\label{sec:login-prompt}

For the Linux kernel that use \verb!init! (Sect.\ref{sec:init}), we
need to create and modify \verb!/etc/inittab! to specify
(1) what user-space process to run at bootup (Sect.\ref{sec:inittab}) based on
the running level, (2) what is the prompt once the kernel start (this section).

% is often
% called by the kernel with an argument to specify the runlevel, in order to do different
% things according to different runlevel, and that runlevel can be found from
% /etc/inittab.


Now edit the content of \verb!/etc/init.d/rcS! file and \verb!/etc/mdev.conf!
file.

Make sure to change the permission of \verb!rcS! file to executable.
\begin{itemize}
  \item The rcS script will test for the existence of file systems and mount
  them accordingly.

  \item \verb!mdev! is used (Sect.\ref{sec:mdev}). The content of
  \verb!mdev.conf!
\begin{verbatim}
audio       0:5 0666
console     0:5 0600
control.*   0:0 0660 @/bin/mv /dev/$MDEV /dev/snd/
dsp         0:5 0666
event.*     0:0 0600 @/bin/mv /dev/$MDEV /dev/input/
fb          0:5 0666
nfs         0:5 0770
null        0:0 0777
pcm.*       0:0 0660 @/bin/mv /dev/$MDEV /dev/snd/
rtc         0:0 0666
tty         0:5 0660
tty0*       0:5 0660
tty1*       0:5 0660
tty2*       0:5 0660
tty3*       0:5 0660
tty4*       0:5 0660
tty5*       0:5 0660
tty6*       0:5 0660
ttyS*       0:5 0640
urandom     0:0 0444
zero        0:0 0666
\end{verbatim}
\end{itemize}

\section{Boot BSD kernel with U-boot}


\url{http://www.bsdcan.org/2008/schedule/attachments/49_2008_uboot_freebsd.pdf}

\section{Debug a Linux kernel}

Debugging the kernel of a running operating system has always been tricky, but
now the Qemu emulator supports cross-platform kernel and module debugging at the
programming language level.
\url{http://www.linux-magazine.com/Online/Features/Qemu-and-the-Kernel}

Try to download these Linux kernel, initrd for PowerPC:
\url{https://code.google.com/p/cellos/source/browse/trunk/output/?r=2}

\begin{verbatim}
qemu-system-ppcemb -S -s -bios output/ppc405_rom.bin -kernel output/cellos.bin
            -nographic -M taihu -cpu 0x51210950 -d in_asm,op,op_opt,int,exec,cpu
\end{verbatim}
\url{http://cellos.googlecode.com/svn-history/r21/trunk/Makefile}

\section{Linux From Scratch (LFS)}
\label{sec:LFS}

Linux From Scratch (LFS) is a way to install a working Linux system by building
all components of it manually. It also refers to the book written by  Gerard
Beekmans, and now is maintained by  Bruce Dubbs.
\url{https://en.wikipedia.org/wiki/Linux_From_Scratch}

In order to build Linux system, we need a cross-compiling systems
\begin{itemize}
  \item LFS 8.1: 
  
  \item LFS 
\end{itemize}

Later on, there are others
\begin{enumerate}
  \item  Beyond Linux From Scratch (BLFS) 
   instructions on how to further develop the basic Linux system that was
  created in LFS
  
  Since Release 5.0, the BLFS book version matches the LFS book version.
  
  \item Cross Linux from Scratch (CLFS) focuses on cross compiling, including
  compiling for headless or embedded systems that can run Linux, but lack the
  resources needed to compile Linux.
  
  \item Hardened Linux From Scratch (HLFS) focuses on security enhancements
  
  \item Automated Linux From Scratch (ALFS): automate the process of creating an
  LFS system
\end{enumerate}
