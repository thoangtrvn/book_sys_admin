\chapter{Vim}
\label{chap:Vim}

\section{To read}

\url{http://tech.myemma.com/textmates-features-vim/}



\section{History}
\label{sec:vim-history}
\label{sec:ed-text-editor}
\label{sec:ex-text-editor}
\label{sec:vi-text-editor}

TIMELINE:
\begin{verbatim}
ed  -->   ex   --> vi   -->  vim
1969                         1991
\end{verbatim}

NOTE: \verb!ed! is a line editor for Unix O/S, one of the first part of Unix O/S
developed in 1969.

NOTE: \verb!ex! is an editor, the name means \verb!extended ed!.

The name \verb!vi! means \verb!visual ex!, which is a GUI part of the historical
editor called \verb!ex!. 

\verb!vim! means \verb!Vi IMproved! (initially it means Vi IMitation), a clone
of \verb!vi! editor started since 1988, and was first released in 1991.
To add more features to VIM, plugins are used (Sect.\ref{sec:vimm-plugins}).


The major revision starts with vim 8.0 (Sect.\ref{sec:vim-8.0})
\begin{verbatim}

vim 7    ---->  vim 8.x
2007             09/12/2016
\end{verbatim}



\subsection{vim's vanilla plugins}
\label{sec:vim-plugins}
\label{sec:vim-autoload}
\label{sec:vim-colors}
\label{sec:vim-indent}


A plugin for \verb!vim! can be written Vim script (Sect.\ref{sec:vim-script}).
Vanilla vim (Sect.\ref{sec:vim-vanilla}), i.e. the original vim source, supports
splitting a single plugin into multiple files.
Vim has a more specific meaning of "plugin", which is "a file in
\verb!~/.vim/plugin/!".

In vim vanilla, there are a number of different directories you can create under
\verb!~/.vim! that mean various things. The logic of handling a file of a given
type is plit into 3 folders:
\verb!ftdetect!, \verb!ftplugin! and \verb!indect!.
\footnote{\url{http://learnvimscriptthehardway.stevelosh.com/chapters/42.html}}

\begin{enumerate}
  \item \verb!~/.vim/colors!: any file in this folder is treated as color scheme
  (Sect.\ref{sec:vim-colorscheme})
  
  \item \verb!~/.vim/plugin/!: any file in this folder is automatically loaded
  at vim starts
  
  \item \verb!~/.vim/ftdetect!: a single or two line that detect and set the
  filetype of a file. The file in this folder are automatically loaded
  at vim starts - Sect.\ref{sec:vim-ftdetect}.
    
  \item \verb!~/.vim/autoload!  - Sect.\ref{sec:vim-autoload}: 
  any file in this folder is loaded once a certain condition is met, e.g. for
  a given filetype. 
  It's important as it provides a way to delay the loading of your plugin's code
  until it's actually needed.
  
  \item \verb!~/.vim/ftplugin!: each subfolder represents the name of a given
  filetype.
  
If we run
\begin{verbatim}
set filetype=derp
\end{verbatim}  
then, vim will load all (*.vim) files in \verb!~/.vim/ftplugin/derp/!.
  
NOTEL These files are run every time a buffer's filetype is set they must only
set buffer-local options! If they set options globally they would overwrite them for
all open buffers!
  
  \item \verb!~/.vim/indent!:
  
The files here set options related to indentation for their filetypes, and those
options should be buffer-local.
   
  \item \verb!~/.vim/after!: 
  
  Files in this directory will be loaded every time Vim starts, but after the
  files in \verb!~/.vim/plugin/!.
  
  This allows you to override Vim's internal files. 
  
  \item \verb!~/.vim/doc! - Sect.\ref{sec:vim-doc}
  
  \item \verb!~/.vim/compiler! - Sect.\ref{sec:vim-compiler}: set
  compiler-related options if the current buffer is of a given filetype
  
  
\end{enumerate}

User needs to configure \verb!vim! to know the location of these plugins.
Vim's vanilla assumed those directories are inside \verb!~/.vim! which is part
of the \verb!runtimepath! variable. However, you can add any location to this so
that vim can search - Sect.\ref{sec:vim-runtimepath}.

There are also several problems with managing plugins using these folders.
\footnote{\url{http://learnvimscriptthehardway.stevelosh.com/chapters/43.html}}
 Thus, different plugin managers have been developed
\begin{enumerate}
  \item pathogen - Sect.\ref{sec:pathogen}
\end{enumerate}



\subsection{-- compiler}
\label{sec:vim-compiler}

The files in \verb!~/.vim/compiler/! work exactly like \verb!indent! files. They
should set compiler-related options in the current buffer based on their names.


\subsection{-- doc}
\label{sec:vim-doc}

In \verb!~/.vim/doc! folder, where you can add documentation for your plugin,
and thus you can call
\begin{verbatim}
:help <command,name,...>
\end{verbatim}


\subsection{-- colorscheme}
\label{sec:vim-colorscheme}  

A color scheme is a file that should contain all the Vimscript commands
necessary to generate your color scheme. The file is put inside
\begin{verbatim}
~/.vim/colors
\end{verbatim}

To enable a color scheme
\begin{verbatim}
:color mycolor
\end{verbatim}
it will use \verb!~/.vim/colors/mycolor.vim!. 

If you want to create your own you should copy an existing scheme and modify
it.



\subsection{-- plugin}

Any files inside \verb!~/.vim/plugin/! will each be run once every time Vim
starts. These files are meant to contain code that you always want loaded
whenever you start Vim.

To load on-demand, i.e. delay the loading, we should put these files into
\verb!~/.vim/autoload! - Sect.\ref{sec:vim-autoload}.

\subsection{-- autoload (\~/.vim/autoload)}
\label{sec:vim-autoload}

\verb!autoload! is a way to delay the loading of your plugin's code until it's
actually needed; by putting the plugin in this folder.
\begin{verbatim}
  //vim
~/.vim/autoload

  //nvim-spacevim
~/.SpaceVim/autoload
\end{verbatim} 
\url{http://learnvimscriptthehardway.stevelosh.com/chapters/42.html#vimautoload}


\subsection{-- ftdetect}
\label{sec:vim-ftdetect}
  
  \verb!ftdetect! stands for "filetype detection". The files in this directory
  should set up autocommands that detect and set the filetype of files, and
  nothing else. This means they should never be more than one or two lines long.  



\subsection{-- after}
\label{sec:vim-after}


\subsection{vim: get help with commands}
\label{sec:vim-help}

\verb!vim! provides \verb!:help! command to get help for commands and features.
To get help for a command, e.g. \verb!laststatus!, run
\begin{verbatim}
:help laststatus
\end{verbatim}

\subsection{vim version}
\label{sec:vim-version}

To check version and see what vim support
\begin{verbatim}
# inside vim
:version

# command-line
vim --version
\end{verbatim}

\subsection{Variants of vim}
\label{sec:vim_variants}

Besides the vim package, there appear to be at least six "vim-variants".
\verb!vim.tiny! is the trim-down version of \verb!vim.basic! editor.

A GUI version of vim is called \verb!gvim!. 

Ubuntu usually offers
\begin{itemize}
  \item \verb!vim-basic! (or vim): the classic vim (support Python) -
  Sect.\ref{sec:vim-vanilla}
  
  \item \verb!vim-tiny! - Sect.\ref{sec:vim-tiny}
  
  \item \verb!vim-nox! :  a classic Non-GUI vim with support for extra scripting
  languages (TCL, Ruby, Lua, Perl, Python) - Sect.\ref{sec:vim-nox}
    
  \item \verb!vim-athena!: enhanced vim (using Athena GUI; but not GTK) -
  Sect.\ref{sec:vim-athena}

  \item \verb!vim-gnome! : enhanced vim (using GNOME2 GUI, e.g. GNOME, Unity,
  Cinnamon)
   
  \item \verb!vim-gtk! : enhanced vim (using GTK2 GUI, i.e. non-GNOME such as
  XFCE, LXDE) - Sect.\ref{sec:vim-gtk}
  
  Vim-gtk is important only to people who use Kubuntu (or some lightweight
  desktop environment) and don't want to install GNOME libraries
  
  \item \verb!jvim-canna! : japanese version of vim (you can type Kanji from the console)
  
  
\end{itemize}
\url{http://askubuntu.com/questions/281886/what-is-the-difference-between-the-different-vim-packages-available-in-ubuntu}

NOTE: Enhanced vim supports Perl, Python, Ruby and TCL scriptting.

\url{http://vi.stackexchange.com/questions/4873/is-there-a-difference-between-sudo-apt-get-vim-and-configure-with-feature}

\subsection{-- vanilla vim}
\label{sec:vim-vanilla}

Vanilla vim refers to the original vim source, which use \verb!vim.basic!
configuration (Sect.\ref{sec:vim.basic}).

\subsection{-- vim.tiny vs. vim.basic vs. vim.full}
\label{sec:vim.basic}
\label{sec:vim.full}
\label{sec:vim.tiny}

In Debian distro, it uses \verb!vim.tiny! by default (by vanilla vim) -
Sect.\ref{sec:vim-tiny}.

You can change to a different configuration by running
\begin{verbatim}
sudo update-alternatives --config vi

// which shows
Selection   Path     Priority      Status
0      /usr/bin/vim.gtk  50    auto mode
1     /usr/bin/vim.gtk   50    manual mode
2     /usr/bin/vim.tiny  10    manual mode
3     /usr/bin/vim.basic 30    auto mode
\end{verbatim}

You can also configure the default editor
\begin{verbatim}
sudo update-alternatives --config editor

// which shows
Selection   Path     Priority      Status
0      /usr/bin/vim.gtk  50    auto mode
1     /bin/ed           -100   manual mode
2     /bin/nano          40    manual mode
3    /usr/bin/emacs24    0     manual model
\end{verbatim}

\url{https://2buntu.com/articles/1507/installing-vimbasic-making-vi-normal-again/}



\subsection{-- vim-tiny}
\label{sec:vim-tiny}


\verb!vim-tiny! is the trimmed-down version of vanilla vim
(Sect.\ref{sec:vim-vanilla}). Ubuntu has vim in 
\verb!vim.tiny! configuration installed by default.

In this configuration, in the \verb!insert! mode, when you press the arrow keys,
you end up with inserting characters - A, B, C and D.


On Ubuntu 11.10, vim.tiny has 12 features enabled
(Sect.\ref{sec:vim-build-from-source})
\begin{verbatim}
vim.tiny --version
\end{verbatim}
\url{http://askubuntu.com/questions/104138/what-features-does-vim-tiny-have}


Commands
\begin{verbatim}
// go between buffers
:n  
:N

// split screen
:split
// jump between buffers in split
Ctrl-W-W

:close 
\end{verbatim}




\subsection{-- vim-nox}
\label{sec:vim-nox}

If you'd like all the features that the packaging allows, but don't want GUI
(and the dependencies of a GUI, e.g. gtk package), pick vim-nox.


\subsection{-- vim-athena}
\label{sec:vim-athena}

If you'd like a GUI, and but don't want anything to do with GTK, pick
vim-athena.

\subsection{-- vim-gtk}
\label{sec:vim-gtk}

\verb!vim-gtk! is a separate source of \verb!vim! with many features selected as
default in the build system. To build form source -
Sect.\ref{sec:vim-gtk-build}.

Make sure you install vim-gtk which has many features compiled with it.
\begin{verbatim}
+clipboard
+python 
-python3
+windows
+folding
...
\end{verbatim}

\subsection{---- VIM-gtk Ubuntu 14.04}

{\tiny 
\begin{verbatim}
VIM - Vi IMproved 7.4 (2013 Aug 10, compiled Jan  2 2014 19:39:59)
Included patches: 1-52
Modified by pkg-vim-maintainers@lists.alioth.debian.org
Compiled by buildd@
Huge version with GTK2 GUI.  Features included (+) or not (-):
+acl +farsi   +mouse_netterm   +syntax
+arabic  +file_in_path+mouse_sgr   +tag_binary
+autocmd +find_in_path-mouse_sysmouse  +tag_old_static
+balloon_eval+float   +mouse_urxvt -tag_any_white
+browse  +folding +mouse_xterm +tcl
++builtin_terms  -footer  +multi_byte  +terminfo
+byte_offset +fork()  +multi_lang  +termresponse
+cindent +gettext -mzscheme+textobjects
+clientserver-hangul_input+netbeans_intg   +title
+clipboard   +iconv   +path_extra  +toolbar
+cmdline_compl   +insert_expand   +perl+user_commands
+cmdline_hist+jumplist+persistent_undo +vertsplit
+cmdline_info+keymap  +postscript  +virtualedit
+comments+langmap +printer +visual
+conceal +libcall +profile +visualextra
+cryptv  +linebreak   +python  +viminfo
+cscope  +lispindent  -python3 +vreplace
+cursorbind  +listcmds+quickfix+wildignore
+cursorshape +localmap+reltime +wildmenu
+dialog_con_gui  +lua +rightleft   +windows
+diff+menu+ruby+writebackup
+digraphs+mksession   +scrollbind  +X11
+dnd +modify_fname+signs   -xfontset
-ebcdic  +mouse   +smartindent +xim
+emacs_tags  +mouseshape  -sniff   +xsmp_interact
+eval+mouse_dec   +startuptime +xterm_clipboard
+ex_extra+mouse_gpm   +statusline  -xterm_save
+extra_search-mouse_jsbterm   -sun_workshop+xpm
   system vimrc file: "$VIM/vimrc"
 user vimrc file: "$HOME/.vimrc"
 2nd user vimrc file: "~/.vim/vimrc"
  user exrc file: "$HOME/.exrc"
  system gvimrc file: "$VIM/gvimrc"
user gvimrc file: "$HOME/.gvimrc"
2nd user gvimrc file: "~/.vim/gvimrc"
system menu file: "$VIMRUNTIME/menu.vim"
  fall-back for $VIM: "/usr/share/vim"
Compilation: gcc -c -I. -Iproto -DHAVE_CONFIG_H -DFEAT_GUI_GTK  -pthread -I/usr/include/gtk-2.0 -I/usr/lib/x86_64-linux-gnu/gtk-2.0/include -I/usr/include/atk-1.0 -I/usr/include/cairo -I/usr/include/gdk-pixbuf-2.0 -I/usr/include/pango-1.0 -I/usr/include/gio-unix-2.0/ -I/usr/include/freetype2 -I/usr/include/glib-2.0 -I/usr/lib/x86_64-linux-gnu/glib-2.0/include -I/usr/include/pixman-1 -I/usr/include/libpng12 -I/usr/include/harfbuzz -g -O2 -fstack-protector --param=ssp-buffer-size=4 -Wformat -Werror=format-security -U_FORTIFY_SOURCE -D_FORTIFY_SOURCE=1 -I/usr/include/tcl8.6  -D_REENTRANT=1  -D_THREAD_SAFE=1  -D_LARGEFILE64_SOURCE=1  
Linking: gcc   -L. -Wl,-Bsymbolic-functions -Wl,-z,relro -rdynamic -Wl,-export-dynamic -Wl,-E  -Wl,-Bsymbolic-functions -Wl,-z,relro -Wl,--as-needed -o vim   -lgtk-x11-2.0 -lgdk-x11-2.0 -latk-1.0 -lgio-2.0 -lpangoft2-1.0 -lpangocairo-1.0 -lgdk_pixbuf-2.0 -lcairo -lpango-1.0 -lfontconfig -lgobject-2.0 -lglib-2.0 -lfreetype   -lSM -lICE -lXpm -lXt -lX11 -lXdmcp -lSM -lICE  -lm -ltinfo -lnsl  -lselinux  -lacl -lattr -lgpm -ldl  -L/usr/lib -llua5.2 -Wl,-E  -fstack-protector -L/usr/local/lib  -L/usr/lib/perl/5.18/CORE -lperl -ldl -lm -lpthread -lcrypt -L/usr/lib/python2.7/config-x86_64-linux-gnu -lpython2.7 -lpthread -ldl -lutil -lm -Xlinker -export-dynamic -Wl,-O1 -Wl,-Bsymbolic-functions  -L/usr/lib/x86_64-linux-gnu -ltcl8.6 -ldl -lz -lpthread -lieee -lm -lruby-1.9.1 -lpthread -lrt -ldl -lcrypt -lm  -L/usr/lib   

\end{verbatim}}


\section{vim 8.0}
\label{sec:vim-8.0}

Vim 8.0 is released in Dec-2016, with many changes. Again, there are many variants of vim - Sect.\ref{sec:vim_variants}.

New features
\begin{enumerate}
  \item asynchronous I/O, with channels, JSON 
  
  plugins to interact without interfering the GUI.
  
  The communication protocol used for this supports JSON, which means you can
  write servers in practically any language as long as you can read and write JSON.
  The functions to use are \verb!json_encode()! and \verb|json_decode()|.
  
  \item supports packages. 
  
  Packages can contain multiple Vim plugins, and can be a version control
  repository, like Git. The package system scans packpath for plugins under
  \verb!pack/*/*start! during startup, and then adds those paths to runtimepath.
  In that sense it reminds me of how Pathogen works. You can also optionally load
  plugins with \verb!:packadd!
  
  \verb!:help package-create! with details on how to write packages.
  
  the manual recommends using version control instead of archives to distribute
  packages so it's easier for users to update them
  
  \item lambdas look nice for Vim script authors 
  
  \verb!:call filter(mylist, {idx, val -> val > 20})!, and there are some new
  Visual and Insert mode commands.
  
  \item jobs
  
  \item timers
  
  \item Partials, Lambdas and Closures
  
  \item New style testing 
  
  \item Viminfo merged by timestamp 
  
  \item GTK+ 3 support 
  
  \item MS-Windows DirectX support 
  
\end{enumerate}

\subsection{in Ubuntu}


For Ubuntu 16.04, Ubuntu 14.04, Linux Mint 18/17, and their derivatives, Vim 8.0
has been made into this backport PPA
\begin{verbatim}

sudo add-apt-repository ppa:jonathonf/vim

sudo apt update

# vim-nox or vim-nox-py2 for Python 3 or 2 support
sudo apt install vim-nox

# Downgrade to the stock version
sudo apt install ppa-purge && sudo ppa-purge ppa:jonathonf/vim


\end{verbatim}

\section{Introduction}

\subsection{Build vim from source}
\label{sec:vim-build-from-source}

It is recommended to learn about vim history (Sect.\ref{sec:vim-history}).
Before deciding to build vim yourself, check a number of pre-built options -
Sect.\ref{sec:vim_variants}. All vim packages except vim-tiny
(Sect.\ref{sec:vim-tiny}) gets built with \verb!--with-features=huge!.

\subsection{-- build automately}


\url{https://gist.github.com/shaypal5/4decf299db737dc66de0647a5d58f96a}
\begin{verbatim}
       # 1. install dependencies
       sudo apt install libncurses5-dev libgnome2-dev libgnomeui-dev \
               libgtk2.0-dev libatk1.0-dev libbonoboui2-dev \
               libcairo2-dev libx11-dev libxpm-dev libxt-dev python-dev \
               python3-dev ruby-dev lua5.1 lua5.1-dev libperl-dev git

       # 2. remove vim
       sudo apt remove vim vim-runtime gvim
       sudo apt remove vim-tiny vim-common vim-gui-common vim-nox


       # if pyenv is installed, set the python command to point at the system python with:
       # pyenv shell system

       # 3. configure
       cd ~
       git clone https://github.com/vim/vim.git
       cd vim
       ./configure --with-features=huge \
               --enable-multibyte \
               --with-x \
               --enable-python3interp=yes \
               --with-python3-config-dir=/usr/lib/python3.5/config-3.5m-x86_64-linux-gnu \
               --enable-rubyinterp \
               --enable-gui=gtk2 \
               --enable-cscope \
               --prefix=/usr/local
       make VIMRUNTIMEDIR=/usr/local/share/vim/vim81

       # 4. use make to install
       cd ~/vim
       sudo make install

       # 5. Set vim as your default editor with update-alternatives.
       sudo update-alternatives --install /usr/bin/editor editor /usr/local/bin/vim 1
       sudo update-alternatives --set editor /usr/local/bin/vim
       sudo update-alternatives --install /usr/local/bin/vi vi /usr/local/bin/vim 1
       sudo update-alternatives --set vi /usr/local/bin/vim

\end{verbatim}

\subsection{-- build vim-gtk}
\label{sec:vim-Ruby}
\label{sec:vim-gtk-build}
\label{sec:install-vim}

Example: \verb!vim-gtk! is built using HUGE features, Python (2.7+ and
3.x) and Ruby supports.
\begin{verbatim}
//pre-req.
sudo apt-get install ncurses-dev libx11-dev libxtst-dev python-dev python3-dev \ 
    ruby ruby-dev xorg-dev -y

//vim source
wget http://archive.ubuntu.com/ubuntu/pool/main/v/vim/vim_7.3.429.orig.tar.gz
tar -xvf vim_7.3.429.orig.tar.gz
cd vim_7.3.429
rm src/auto/config.cache
./configure --enable-pythoninterp --enable-rubyinterp --enable-python3interp \
 --with-features=huge --with-x --enable-cscope --enable-gui=auto \
 --enable-gtk2-check --enable-gnome-check \
 --with-python-config-dir=/usr/lib/python2.7/config-x86_64-linux-gnu/ \
 --with-python3-config-dir=/usr/lib/python3.4/config-3.4m-x86_64-linux-gnu 
 
make -j4 
sudo -E -H make install 
cd -
\end{verbatim}


\url{http://superuser.com/questions/708245/re-compile-vim-with-options}

If you install Python 2.7+ in a custom folder, then
\begin{verbatim}
// Suppose Python is installed in $HOME/root
LD_LIBRARY_PATH=$HOME/root/lib PATH=$HOME/root/bin:$PATH \
./configure --enable-pythoninterp \ 
--with-python-config-dir=$HOME/root/lib/python2.7/config \
--enable-rubyinterp \
--with-ruby-command=/path/to/ruby \
--prefix=$HOME/vim27 
% make -j4
% make install
% $HOME/bin/vim27
\end{verbatim}
\url{http://stackoverflow.com/questions/5872079/compiling-vim-with-specific-version-of-python}

NOTE: For some reason if Ruby header files are not found, modify 
\verb!./src/auto/configure! file the variable
\begin{verbatim}
rubyhdrdir=/gsa/yktgsa/home/t/m/tmhoangt/.rvm/rubies/ruby-2.2.1-dev/include/ruby-2.2.0

rubylibdir=/gsa/yktgsa/home/t/m/tmhoangt/.rvm/rubies/ruby-2.2.1-dev/lib/
\end{verbatim}
then, when you compile the vim make sure you modify \verb!RUBY_CFLAGS! option

\subsection{-- Build vim with static link to gcc}

Add LDFLAGS="-static" 

\begin{verbatim}
rm src/auto/config.cache
./configure LDFLAGS="-static -L/usr/lib/x86_64-linux-gnu/libncurses.a"
--enable-pythoninterp --enable-rubyinterp --enable-python3interp \ 
 --with-features=huge --with-x --enable-cscope --enable-gui=auto \
 --enable-gtk2-check --enable-gnome-check \
 --with-python-config-dir=/usr/lib/python2.7/config-x86_64-linux-gnu/ \
 --with-python3-config-dir=/usr/lib/python3.4/config-3.4m-x86_64-linux-gnu 

\end{verbatim}
\url{https://github.com/ericpruitt/static-vim}


\subsection{Status line:Display informations}

To know what information that you can display on the status line, run
\verb!help statusline! (Sect.\ref{sec:vim-statusline}).  

%\subsection{ * Show file path}

To show file path, press \verb!Ctrl-g!, or to set it permanently in
\verb!~/.vimrc! file.

\begin{verbatim}
set statusline+=%F
\end{verbatim}

Other options
\begin{verbatim}
:f
\end{verbatim}
shows the path, line count, modified state, current cursor position, and more
\url{http://stackoverflow.com/questions/10488717/how-can-i-permanently-display-the-path-of-the-current-file-in-vim}


\subsection{Vim leader key}
\label{sec:vim-leader-key}


\url{http://usevim.com/2012/07/20/vim101-leader/}




\subsection{Using Python as a scripting language for VIM}

Make sure Python support is available
\begin{verbatim}
vim --version  //look for +python

// or inside vim run
:python import sys; print(sys.version)

2.7.6 (default, Sep  9 2014, 15:04:36)
[GCC 4.2.1 Compatible Apple LLVM 6.0 (clang-600.0.39)]

\end{verbatim}

\subsection{Make new directory}

\begin{verbatim}
:!mkdir "<path-to-new-dir>"
\end{verbatim}
or browse the folder and use \verb!d! command
\begin{verbatim}
:Explorer
d 
<path-to-new-directory>
\end{verbatim}

\subsection{Make new folder on save file at the same time}

Suppose you edit the file \verb!~/myfolder/mypath/myfile.txt!, but you have not
created the folder \verb!myfolder! and \verb!mypath! subfolder yet. You can do
(1) create the folder first (using the above method), (2) create the folder
without retyping the path if you are editing the file and then save the file
\begin{verbatim}
:mkdir -p %:h
:w
\end{verbatim}
or (3) define a new command (say W) to do creating-saving at once. 
\url{http://stackoverflow.com/questions/4292733/vim-creating-parent-directories-on-save}


\subsection{Edit compressed files}

\subsection{ * Zip files}
\label{sec:zip_files}

\subsection{ * Jar files}
\label{sec:jar_files}

Jar files are really just zip files, and you can browse them from within Vim
simply by treating them the same as Vim treats zip files
(Sect.\ref{sec:zip_files}).

Modify \verb!~/.vim/ftdetect/jar.vim!
and put
\begin{verbatim}
au BufRead,BufNewFile *.jar,*.war,*.ear,*.sar,*.rar set filetype=zip
\end{verbatim}
\url{http://vim.wikia.com/wiki/Edit_Java_jar_and_other_files}

\subsection{Treat file of a given extension for another known filetype}

Example:
\begin{enumerate}
  \item treat .jar files as .zip file: Sect.\ref{sec:jar_files}
  
  \item treat .h file as file in pure C language:

By default: .h files are files in C++	
\begin{verbatim}
augroup project
autocmd!
autocmd BufRead,BufNewFile *.h,*.c set filetype=c.doxygen
augroup END
\end{verbatim}
\end{enumerate}

\section{Running vim}

\subsection{5 basic modes}
\label{sec:modes-in-vim}
\label{sec:vim-modes}

Vim has 5 BASIC modes
\begin{enumerate}
  \item {\bf normal mode} (command mode): where you can type in command
  
  
  \item {\bf visual mode}: the movement commands (e.g. left/right/.. keys)
  extend the highlighted area
  
  \item {\bf select mode}
  
  
  \item {\bf insert mode}
  
  \item {\bf command-line mode} (CmdLine mode): then you can enter the one line
  of text at the bottom of the window (and this is treated as a command)
  
  
  \item {\bf Ex mode} (i.e. a persistent CmdLine mode): like CmdLine mode, yet
  after entering the command, you remain in Ex mode
  
  To enter this mode, type \verb!Q! (upper-case). To exit, type 
  \verb!:vi! 
\end{enumerate}

There are also other modes
\begin{verbatim}
:h Ex-mode

\end{verbatim}

\subsection{Configuration file}
\label{sec:vim-configuration}

Depending on the build (tiny, basic, full: Sect.\ref{sec:vim.tiny}), vim comes
with a different set of features. If you want to extend its functionality,
consider using configuration file

If you run vim, and then type \verb!:version!, it will list the default location
that vim will search for these configuration files.
\begin{verbatim}
   system vimrc file: "$VIM/vimrc"              
     user vimrc file: "$HOME/.vimrc"                    
 2nd user vimrc file: "~/.vim/vimrc"
      user exrc file: "$HOME/.exrc"
  system gvimrc file: "$VIM/gvimrc"
    user gvimrc file: "$HOME/.gvimrc"
2nd user gvimrc file: "~/.vim/gvimrc"
       defaults file: "$VIMRUNTIME/defaults.vim"
    system menu file: "$VIMRUNTIME/menu.vim"
  fall-back for $VIM: "/packages/vim/8.0/share/vim"
\end{verbatim}

\begin{itemize}
  \item \verb!$MYVIMRC! environment variable: (if defined) it contains the
  location of the file which is used as the configuration file

Run vim and check
\begin{verbatim}
:echo $MYVIMRC
\end{verbatim}
  
  \item user configuration file  (Sect.\ref{sec:.vimrc}), by default (if
  \verb!$MYVIMRC! is NOT defined)
  
Once vim launches, it sets \verb!$MYVIMRC! to the above file. Inside vim, type
\begin{verbatim}
:h $MYVIMRC
\end{verbatim}

\end{itemize}

Name: if you want to load a different file, pass the filename to \verb!-u!
\begin{verbatim}
vim -u othervimrc
\end{verbatim}

To skip loading the file
\begin{verbatim}
vim -u NORC   # skip vimrc only

vim -u NONE   # skip both reading vimrc and loading plugins
\end{verbatim}
\url{http://vimdoc.sourceforge.net/htmldoc/starting.html}


\subsection{-- .vimrc vs. .exrc}
\label{sec:.vimrc}
\label{sec:.exrc}

\verb!$HOME/.vimrc! is the file that contains initialization commands, which
\verb!vim! reads when launched (Sect.\ref{sec:vim-configuration}).

\verb!.exrc! is similar to .vimrc, but is specific for \verb!vi! editor.

Location:
\begin{verbatim}
Unix		$HOME/.vimrc
OS/2		$HOME/.vimrc or $VIM/.vimrc (or _vimrc)
MS-DOS and Win32		$HOME/_vimrc or $VIM/_vimrc
Amigas:					$HOME/.vimrc or $VIM/.vimrc
\end{verbatim}

\begin{verbatim}
"$HOME/.vimrc"      (for Unix and OS/2) (*)
"$HOME/.vim/vimrc"      (for Unix and OS/2) (*)
"s:.vimrc"      (for Amiga) (*)
"home:.vimrc"      (for Amiga) (*)
"home:vimfiles:vimrc"      (for Amiga) (*)
"$VIM/.vimrc"      (for OS/2 and Amiga) (*)
"$HOME/_vimrc"      (for MS-DOS and Win32) (*)
"$HOME/vimfiles/vimrc"      (for MS-DOS and Win32) (*)
"$VIM/_vimrc"      (for MS-DOS and Win32) (*)
\end{verbatim}


NOTE: If you use neovim (nvim), the configuration file is
\verb!init.vim! and the location is different
(Sect.\ref{sec:nvim-configuration}). However, you can symlink to 
\verb!~/.vimrc! so that vim and neovim can share the same resource file. 

If you use \verb!dotfiles! (Sect.\ref{sec:dotfiles}), then \verb!~/.vimrc! is a
symbolic link to a base configuration file in \verb!~/dotfiles/vimrc! file. This
file also loads to files
\begin{verbatim}
~/.vimrc.bundles            //default set of vim plugins
~/.vimrc.local
~/.vimrc.bundles.local
\end{verbatim}
Then, you can have your own customization in \verb!~/.vimrc.local!
(Sect.\ref{sec:dotfiles-customized}), and for plugins use
\verb!~/.vimrc.bundles.local! (Sect.\ref{sec:dotvimrc.bundles}).

Example: \url{https://github.com/samlaudev/ConfigurationFiles/tree/master/vim}

\subsection{.vimrc.bundles}
\label{sec:dotvimrc.bundles}

The  default set of vim plugins in .vimrc.bundles


Customize with \verb!.vimrc.bundles.local! file

\begin{enumerate}
  \item If you don't wish to install a vim plugin from the default set 
  
\begin{verbatim}
" Don't install vim-scripts/tComment
UnPlug 'tComment'
\end{verbatim}

  \item 
\end{enumerate}


\subsection{configure vim's search path (runtime-path)}
\label{sec:vim-search-path}
\label{sec:vim-rtp}
\label{sec:vim-runtimepath}

% \begin{verbatim}
% let &path.="src/include,/usr/include/AL,"
% \end{verbatim}
Example: add more path (can combine environment variable) to project specific
\verb!.vimrc!
\begin{verbatim}
:set path+=$NTSROOT/mdl/include,$NTS/gsl/framework/dca/include,

 //or overwrite everything
:set runtimepath=/Users/sjl/Desktop
\end{verbatim}
\url{http://usevim.com/2013/01/04/vim101-jumping/}

\section{Scripting language in VIM}
\label{sec:vim-scripting-language}

\subsection{define a function}

To get support about using \verb!function! keyword
\begin{verbatim}
:h function
\end{verbatim}

The function name must start with an uppercase letter, to avoid confusion with
builtin functions. We can use poud sign to indicate that a function belong to a
given file

Example: \verb!overrides.vim! should in in the folder as part of RTP
(runtimepath)
\begin{verbatim}
"This should not work.
"But it does as long as the function is in a file called 'overrides.vim'.
function! overrides#name() abort
  echo 'Test overrides\name'
endfunction

"This should not work either.
"But it does as long as the file above is in a folder called 'plugin'.
function! plugin#overrides#name() abort 
  echo 'Test plugin\overrides\name'
endfunction

let stupid = {}
"This should not work.
"But it does aslong as the stupid Dictionary is defined.
function! stupid.name() abort
  echo 'Test stupidname'
endfunction


call overrides#name()
call plugin#overrides#name()
call stupid.name()
\end{verbatim}

If a function already exists with the same name as you want to 
create, the "!" will tell it to overwrite the previously-existing 
function
\begin{verbatim}

\end{verbatim}
\url{http://vim.1045645.n5.nabble.com/why-quot-function-quot-instead-of-just-quot-function-quot-td5723708.html}

Using exclaimation mark 
\begin{verbatim}
 :w! foo.txt 
\end{verbatim}
(overwrite an existing file).

\subsection{let vs. set}
\label{sec:vim-script-let}
\label{sec:vim-script-set}
\label{sec:vim-script-variable}
\label{sec:vim-script-option}

\begin{verbatim}
:set is for setting options, 
:let for assigning a value to a variable.

:set  tw=40
:let &tw=40

if &compatible
	set nocompatible
endif
\end{verbatim}

The value of an option is linked to the name by using \verb!&! (ampersand), i.e.
\verb!&option-name!.

A global variable is prefixed with \verb!g:!
\begin{verbatim}
let g:foo=50
\end{verbatim}
NOTE: g:foo is a variable; so we cannot use \verb!set! command.

Some options are boolean like. When setting these, no value is needed (as in
\verb!:set noic! and the opposite \verb!:set ic!)

\url{https://stackoverflow.com/questions/9990219/vim-whats-the-difference-between-let-and-set}

\subsection{check O/S}
\label{sec:vim-script-if-statement}
\label{sec:vim-script-check-OS}

\begin{verbatim}
if has('win16') || has('win32') || has('win64')
    let s:Psep = ';'
    let s:Fsep = '\'
else
    let s:Psep = ':'
    let s:Fsep = '/'
endif
\end{verbatim}

\subsection{comparison}
\label{sec:vim-script-comparison}

\begin{verbatim}
 if g:spacevim_plugin_manager ==# 'neobundle'
 
 elseif  g:spacevim_plugin_manager ==# 'dein'
 
 endif
\end{verbatim}

\subsection{expand, fnamemodify}
\label{sec:vim-script-expand()}
\label{sec:vim-script-fnamemodify()}

\begin{verbatim} 
  //display relative path of the current editing buffer
  // % = means current file
:echom expand('%')
  // :p = display full, absolute path 
:echom expand('%:p')
  // fnamemodify = return the absolute path to the file 'foo.txt'
  //     in the current directory, regardless of whether the file actually exist
:echom fnamemodify('foo.txt', ':p')

http://learnvimscriptthehardway.stevelosh.com/chapters/40.html

https://stackoverflow.com/questions/4976776/how-to-get-path-to-the-current-vimscript-being-executed

" Relative path of script file:
let s:path = expand('<sfile>')

" Absolute path of script file:
let s:path = expand('<sfile>:p')

" Absolute path of script file with symbolic links resolved:
let s:path = resolve(expand('<sfile>:p'))

" Folder in which script resides: (not safe for symlinks)
let s:path = expand('<sfile>:p:h')

" If you're using a symlink to your script, but your resources are in
" the same directory as the actual script, you'll need to do this:
"   1: Get the absolute path of the script
"   2: Resolve all symbolic links
"   3: Get the folder of the resolved absolute file
let s:path = fnamemodify(resolve(expand('<sfile>:p')), ':h')
\end{verbatim}
To deal with symbolic link, we use fnamemodify
\begin{verbatim}
~/.vimrc is a symbolic link to a script in a git repository.
\end{verbatim}

\section{Extension to vim (i.e. Plugins)}


Extensions in VIM are often referred to as bundles or plugins.

\subsection{Plugin manager}


Choose one plugin manager:
\begin{enumerate}

  \item Use pathogen to make it easier to install plugin - Sect.\ref{sec:pathogen}
  
  
  \item Use vundle: - Sect.\ref{sec:vundle} : automatically handle cloning
  plugins from github

  \item Use neobundle: like vundle, plus the capability to specify which
version of the plugin to use and handle plugins in Mercurial repos and sites
besides Github - Sect.\ref{sec:neobundle}

\url{https://github.com/Shougo/neobundle.vim}

 \item dein.vim  - Sect.\ref{sec:dein.vim}: the neobundle with vim-plug's speed.
 
 \item vim-plug - Sect.\ref{sec:vim-plug}: simple and speed
 
  \item vim-addon-manager: 
\url{http://www.vim.org/scripts/script.php?script_id=2905}

\end{enumerate}

\url{https://junegunn.kr/2013/09/writing-my-own-vim-plugin-manager/}
\url{http://tilvim.com/2013/12/28/pathogen-for-vundle.html}
\url{https://www.reddit.com/r/vim/comments/2tubi5/vundle_is_so_painless_compared_to_pathogen/}

\subsection{-- vim-pathogen}
\label{sec:pathogen}

Pathogen is simple (once it is installed). Essentially it just add path each
subfolder of \verb!~/.vim/bundle/! to vim's runtimepath
(Sect.\ref{sec:vim-runtimepath}) automatically. So, this subfolder needs to
contain other subfolders, which has some or all of the standard Vim plugin
directories, like colors/ and syntax/.

What pathogen does?
\begin{verbatim}
autoload plugins from a folder (e.g. ~/.vim/bundle)
generate help tags for these plugins

Pros:
minimalist
This makes it trivial to update plugins. You can simply blow away the old
plugin's directory entirely and replace it with the new version.

Cons:
everything else done manually (installing, updating, removing, etc.)
no lazy loading
\end{verbatim}
\url{https://github.com/tpope/vim-pathogen}

REQUIREMENT: User has to manually download the plugin, i.e. cloning from
git-repositories. This is relaxed in the new plugin manager - vundle
(Sect.\ref{sec:vundle}).

HOW TO: 
\begin{itemize}
  \item  put \verb!pathogen.vim! to \verb!~/.vim/autoload!
  (Sect.\ref{sec:vim-autoload}). Basically, it manages vim's \verb!runtimepath!  
  
  \item (In Linux) put any new plugins into \verb!~/.vim/bundle! (default
  location) and pathogen will automatically load them once vim is evoked.

In Windows, the default location is
\begin{verbatim}
%USERPROFILE%\.vim\bundle
\end{verbatim}

To choose a diffferent location, e.g. \verb!myfold! folder, put into
\verb!~/.vimrc! file.

\begin{verbatim}  
// 1 location called 'myfold' (relative path to ~/.vim/!)
execute pathogen#infect('myfold/{}')

// 2 different locations
//  one relative path; and one absolute path
execute pathogen#infect('bundle/{}', '~/src/vim/bundle/{}')
\end{verbatim}
NOTE: \verb!{}! indicate where the expansion should occur.

Any plugin put into as a subfolder inside \verb!~/.vim/bundle! will be added to
vim's \verb!runtimepath! automatically. 
\end{itemize}  


HOW TO Add a new plugin
\begin{enumerate}
  \item put the plugin folder to right location

\begin{verbatim}
cd ~/.vim/bundle

//git clone <path-to-plugin-git>
// which checkout code to a folder, e.g.
// ~/.vim/bundle/command-t

# later on, we can check out the latest version
  cd ~/.vim/bundle/command-t
  git pull

# or a specific version
  cd ~/.vim/bundle/command-t
  git checkout 1.10
\end{verbatim}

  \item re-run vim, and generate documents
  
\begin{verbatim}
:Helptags
\end{verbatim}
\end{enumerate}

There are two ways to disable one or many plugins if Pathogen is used as plugin
manager
\begin{enumerate}
  
  \item disable all plugins
\begin{verbatim}
// then vim only loads ~/.vimrc file.
vim --noplugin

// load nothing
vim -u NONE -N 
\end{verbatim}

To load one particular plugin, inside vim run \verb!source! command
\begin{verbatim}
source 'path-to-plugin'
\end{verbatim}

Or to load plugins for filetype
\begin{verbatim}
:filetype plugin on 
\end{verbatim}

  \item rename the plugin folder (you want to disable) with \verb!~! at the
  suffix
  
  \item Pathogen uses an array \verb!g:pathogen_disabled[]! in which
you can add the names of plugins that you don't want added to the runtimepath. 

\begin{verbatim}
" To disable a plugin, add it's bundle name to the following list
let g:pathogen_disabled = []

" for some reason the csscolor plugin is very slow when run on the terminal
" but not in GVim, so disable it if no GUI is running
if !has('gui_running')
    call add(g:pathogen_disabled, 'csscolor')
endif

" Gundo requires at least vim 7.3
if v:version < '703' || !has('python')
    call add(g:pathogen_disabled, 'gundo')
endif

if v:version < '702'
    call add(g:pathogen_disabled, 'autocomplpop')
    call add(g:pathogen_disabled, 'fuzzyfinder')
    call add(g:pathogen_disabled, 'l9')
endif
\end{verbatim}
  \item Since Pathogen 2.4, use:  
\begin{verbatim}
g:pathogen_blacklist
\end{verbatim}
\end{enumerate} 
\url{http://stackoverflow.com/questions/4261785/temporarily-disable-some-plugins-using-pathogen-in-vim}

\subsection{-- vim-plug}
\label{sec:vim-plug}

vim-plug is a nice alternative to Vundle, with some technical differences that
make it faster. It has most (or all?) of the features of Vundle.

\subsection{-- vundle}
\label{sec:vundle}

\verb!vundle! is \verb!v!im b\verb!undle!.
Better than pathogen (Sect.\ref{sec:pathogen}), vundle takes the process of
cloning repositories (from github only) and manages it all for you.

Think of it as pip for VIM (NOTE: Python use pip to install packages).

\url{https://github.com/VundleVim/Vundle.vim}

NOTE: Inside vim
\begin{verbatim}
 //install all plugins
:PluginInstall
 
 //update
:PluginInstall!

 //once a plugin is removed from your vimrc
:PluginClean
\end{verbatim}


\subsection{-- neobundle}
\label{sec:neobundle}

\verb!neobundle! is like vundle (Sect.\ref{sec:vundle}), plus the capability to
specify which version of the plugin to use.

NOTE: NeoBundle has stopped active development now and will be replaced by
dein.vim (Sect.\ref{sec:dein.vim}), which is Shougo's another brand new plugin
manager.


Neobundle is supposed to be better than vundle in
\begin{enumerate}
  \item not only Github, but also other repos
  
  \item  you can specify certain revisions or branches for used plugins.
\begin{verbatim}
NeoBundle 'tyru/open-browser.vim' , '30a558b', { 'name' : 'open-browser' }
NeoBundle 'vim-ruby/vim-ruby' , { 'name' : 'ruby' }
\end{verbatim}
\end{enumerate}

\subsection{-- dein.vim}
\label{sec:dein.vim}

{\bf dein.vim} is a replacement for neobundle (Sect.\ref{sec:neobundle}), i.e.
it is a neobundle with vim-plug's speed, or a vim-plug with neobunle's feature
(see vim-plug - Sect.\ref{sec:vim-plug}).
 
\begin{itemize}
  \item all features of neobundle.
  \item faster than neobundle (Sect.\ref{sec:neobundle}), as it uses dark power.
  
  \item simpler
%  \item No Vundle/NeoBundle compatibility
%  \item Non github plugins support
%  \item 
\end{itemize}

{\bf Install via SpaceVim}: NOTE: If you install dein.vim as part of SpaceVim
(Sect.\ref{sec:SpaceVim}), it will be installed to
\begin{verbatim}
$HOME/.cache/vimfiles/repos/github.com/Shougo/dein.vim
\end{verbatim}

{\bf Install manually}: If you want to install dein.vim separately, specify the
installation directory. IMPORTANT: DO NOT use
\begin{verbatim}
~/.vim/plugin
~/.config/nvim/plugin
\end{verbatim}
accepted locations: \verb!~/.vim/bundles!, \verb!~/.cache/dein!,
\verb!~/.local/share/dein!, or below. 
\begin{verbatim}
mkdir -p ~/.vim/dein/repos/github.com/Shougo/dein.vim #recommended path
git clone https://github.com/Shougo/dein.vim.git \
    ~/.vim/dein/repos/github.com/Shougo/dein.vim
    
    //or
$ curl https://raw.githubusercontent.com/Shougo/dein.vim/master/bin/installer.sh > installer.sh
$ sh ./installer.sh {specify the installation directory}
\end{verbatim}
\textcolor{red}{You notice that dein.vim's path so different. It is because dein
uses a new approach to manage plugin's source.}

\textcolor{red}{\bf Using}: You add a vim-plugin section to your \verb!~/.vimrc!
(for vim) or \verb!~/.config/nvim/init.vim! (for neovim).
\begin{verbatim}
if &compatible
	set nocompatible
endif

let dein_path="~/.cache/vimfiles/repos/github.com/Shougo/dein.vim"
set runtimepath+=dein_path  "path to dein.vim directory

if dein#load_state({path to plugin base path directory})
  call dein#begin({path to plugin base path directory})

  // list all plugins here
  call dein#add({path to dein.vim directory})
  call dein#add('Shougo/neocomplete.vim')
  ...

  call dein#end()
  call dein#save_state()
endif

filetype plugin indent on
syntax enable
//call plug#begin()

// finally, initialize the plugin system
call plug#end()
\end{verbatim}

Install plugins via dein.vim
\begin{verbatim}
:call dein#install()
\end{verbatim}
\url{https://herringtondarkholme.github.io/2016/02/26/dein/}

\subsection{-- zvim}
\label{sec:zvim}


\begin{verbatim}
$VIM/autoload/zvim/zvim.vim

$VIM/autoload/zvim/
                  ./util.vim
                  ./plugin.vim
                  ./mpv.vim
                  ./debug.vim

\end{verbatim}

The folder
\begin{itemize}
  \item \verb!./plugin.vim! handle installing the proper plugin manager for
  spacevim, e.g. Shougo/neobundle, 
\end{itemize}
\url{https://github.com/monological/zvim}

\subsection{-- unite}
\label{sec:unite-vim-plugin}
\label{sec:vim-plugin-unite}


\begin{verbatim}
:Unite file
:Unite file_rec/async:!
:Unite tag
:Unite buffer
\end{verbatim}

\url{https://devhints.io/vim-unite}

\url{http://bling.github.io/blog/2013/06/02/unite-dot-vim-the-plugin-you-didnt-know-you-need/}

\url{https://github.com/Shougo/unite.vim}

\subsection{-- write-your-own plugin manager}

\url{https://junegunn.kr/2013/09/writing-my-own-vim-plugin-manager/}

\subsection{Load packages depending on O/S}

Put in \verb!~/.vimrc! file:
\begin{verbatim}
if has("unix")
    Plugin 'Valloric/YouCompleteMe'
elseif has("win64")
    Plugin 'YouCompleteMe_win_64', {'pinned': 1}
elseif has("win32")
    Plugin 'YouCompleteMe_win_32', {'pinned': 1}
endif
\end{verbatim}


\subsection{Syntastic}
\label{sec:Syntastic}

NOTE: Syntastics is recommended to be replaced by plugins supporting
asycnrhnous: neomake (Sect.\ref{sec:neomake}), or ALE (Sect.\ref{sec:ALE}).
Syntastic might eventually support async checking (once Vim's API for job
control gets stable), and when / if that happens the async functionality will
use Vim's API rather than Neovim's.  

Syntastics helps to check your code, using a number of different checkers
available on your system depending upon the language
\begin{itemize}
  \item python: use pylint, pyflake, 
  
  \item C/C++ : 

lists of all checkers:
\url{https://github.com/vim-syntastic/syntastic/tree/master/syntax_checkers}

\end{itemize}

You specify what checkers to use in \verb!.vimrc! file
\begin{verbatim}
" regardless of language, you call it using name as 
" g:syntastic_<filetype>_checkers

let g:syntastic_c_checkers=['make','splint']
\end{verbatim}

Once you install syntastics, you can have these commands

\begin{enumerate}
  \item \verb!:SyntasticInfo! : 

{\tiny 
\begin{verbatim}
Syntastic version: 3.8.0-34 (Vim 800, Linux, GUI)
Info for filetype: python
Global mode: active
Filetype python is active
The current file will be checked automatically
Available checkers: pyflakes pylint python
Currently enabled checkers: python pylint
The current file will not be checked (file not readable / not local)
\end{verbatim}
}  
  
  
\end{enumerate}

\section{Making vim the best tool}


\subsection{for python}
\begin{verbatim}

" syntastic
let g:syntastic_python_checkers=['pyflakes']
let g:syntastic_python_checkers=['jshint']

let g:syntastic_c_check_header = 1
let g:syntastic_c_compiler = 'clang'

" ctrl-p
let g:ctrlp_working_path_mode=0
" let g:ctrlp_map = 

" pymode
let g:pymode_python = 'python2'
let g:pymode_lint_checkers = ['pyflakes', 'mccabe']
let g:pymode_lint_ignore = "C0301,W0105,R0902,C0103,W0403,C901,R0903,W0142,R0201,R0904,E1103,W0703,W0221,C0323,E1101,R0914,E0611,F0401,R0915,E501,E203,W0232,W1001,W0122,W0511,W0104,R0912,W0621,W0223,W0201,W0612,C0302,I0011,R0901,W1201,W0622,R0913,E0702"
let g:pymode_options_colorcolumn=0
let g:pymode_folding=0
let g:pymode_doc=0
let g:pymode_rope=0


" Delete whitespace for python and coffeescript
func! DeleteTrailing()
    exe "normal mz"
    %s/\s\+$//ge
    exe "normal `z"
endfunc
autocmd BufWrite *.py :call DeleteTrailing()
autocmd BufWrite *.coffee :call DeleteTrailing()

\end{verbatim}
\url{https://github.com/monological/zvim/blob/master/vimrc}


\subsection{Simple tweaks}


\begin{itemize}
  \item make vim show the current line
  \item connect vim clipboard with X11 system: this requires using vim-gtk or
  vim-gnome
  
  Also, in Ubuntu, make sure you add \verb!:set clipboard^=unnamedplus!
  \url{http://vim.wikia.com/wiki/Accessing_the_system_clipboard}

\end{itemize}


\begin{verbatim}
:set cursorline 
execute pathogen#infect()
syntax on
filetype plugin indent on
map <C-n>  :NERDTreeToggle<CR>
:set clipboard^=unnamedplus
\end{verbatim}

\subsection{Project-specific .vimrc}

One good feature of VI is allowing per-folder configuration, which requires user
to simply put the \verb!.exrc! file in that folder, and all configuration in
that file will be used for VI.
This feature is disabled by default in VIM, unfortunately.

To enable this feature, provide the \verb!.vimrc! in the folder, and then in the
global (user-scope) \verb!.vimrc! file, add  these two lines 
\begin{verbatim}
set exrc
set secure
\end{verbatim}
\url{http://www.ilker.de/specific-vim-settings-per-project.html}

Project-specific .vimrc file: suppose when you run vim, and there is a
\verb!.vimrc! file in the current folder, we want vim to use this file.
Also for security, we want to limit the commands to use in this project-specific
\verb!.vimrc! file, i.e. disable commands that write or execute shell commands.


Example: project's scope \verb!.vimrc!
\begin{verbatim}
" Indentation rule
set tabstop=4
set softtabstop=4
set shiftwidth=4
set noexpandtab

" Warning too long line
set colorcolumn=110
highlight ColorColumn ctermbg=darkgray

" Default: .h is mapped to C++ files
" Map .h to pure C (some projects)
augroup project
autocmd!
autocmd BufRead,BufNewFile *.h,*.c set filetype=c.doxygen
augroup END
\end{verbatim}

\subsection{Colorscheme: solarized}

The quite popular color scheme named solarized can be used not only form vim
(gvim, vimmac) but also other tools (e.g. terminator, xterm, Sublime 2).

To download solarized colorscheme for vim: 
\begin{itemize}
  \item for vim: \url{http://ethanschoonover.com/solarized}
  
  \item for shell terminal, e.g. terminator:
  
  modify \verb!~/.config/terminator/config! file
  
  \url{http://www.webupd8.org/2011/04/solarized-must-have-color-paletter-for.html}
\end{itemize}


If you already configure the terminal to use solarized, then PLEASE don't enable
it in the vim version, but only in gvim version
\begin{verbatim}
// so inside ~/.vimrc put

if has('gui_running')
   set background=dark
   colorscheme solarized
endif
\end{verbatim}
\url{http://stackoverflow.com/questions/9108713/solarized-theme-on-vim-terminator-ubuntu-11-10}

Other colorschemes:
\url{http://bytefluent.com/vivify/}

\subsection{Folder/file explorer: NERDtree}	
\label{sec:browse_folder}
\label{sec:NERDtree}

NERDtree is for browsing file/folder
\begin{itemize}
  \item in spacevim, we evoke with \verb!:NERDTreeToggle! 
  
  \item 
\end{itemize}

%\section{Browse the folder}
\begin{enumerate}
  \item NERDtree to enable opening folder tree inside Vim

\verb!Ctrl-n! short-cut in command-mode.
  
 
  
\url{http://www.vim.org/scripts/script.php?script_id=1658}  
\url{https://github.com/scrooloose/nerdtree}
  
If you see \verb!V! in front of the folder
\begin{verbatim}
V folder1
V folder2
\end{verbatim}
It means that you have a vim' version not compatible with the one you compiled
NERDtree. 

  \item Built-in feature:

\begin{verbatim}
:Explorer
\end{verbatim}
  
\end{enumerate}




\subsection{Folder/file explorer: vimfiler}	
\label{sec:vim-plugin-vimfiler}
\label{sec:vimfiler}


Show/Toggle file tree
\begin{verbatim}
F3   - may not work in Tmux

 // <LEADER> = Space key in spacevim
<LEADER> f t
<LEADER> f T  
\end{verbatim}
\url{https://spacevim.org/documentation/#file-tree}

Vimfiler advantages are:
\begin{verbatim}
Integration with unite.vim.
https://github.com/Shougo/unite.vim

Integration with vimshell.
https://github.com/Shougo/vimshell.vim

More customization.
External sources(for example, unite-ssh).
https://github.com/Shougo/unite-ssh

More options(see |vimfiler-options|)

Fast(if you use |if_lua| enabled Vim).

Column customization.

Double screen file explorer.
\end{verbatim}

\url{https://github.com/Shougo/vimfiler.vim/issues/79}

Ignore files
\begin{verbatim}
let g:vimfiler_ignore_filters = ['matcher_ignore_wildignore']
\end{verbatim}
\url{https://github.com/Shougo/vimfiler.vim/issues/261}

Config
\begin{verbatim}
NeoBundle 'Shougo/vimfiler'
    let s:hooks = neobundle#get_hooks("vimfiler")
    function! s:hooks.on_source(bundle) abort
        let g:vimfiler_as_default_explorer = 1
        let g:vimfiler_restore_alternate_file = 1
        let g:vimfiler_tree_indentation = 1
        let g:vimfiler_tree_leaf_icon = '¦'
        let g:vimfiler_tree_opened_icon = '▼'
        let g:vimfiler_tree_closed_icon = '▷'
        let g:vimfiler_file_icon = '-'
        let g:vimfiler_readonly_file_icon = '*'
        let g:vimfiler_marked_file_icon = '√'
        "let g:vimfiler_preview_action = 'auto_preview'
        let g:vimfiler_ignore_pattern =
                    \ '^\%(\.git\|\.idea\|\.DS_Store\|\.vagrant\|.stversions'
                    \ .'\|node_modules\|.*\.pyc\)$'

        if has('mac')
            let g:vimfiler_quick_look_command =
                        \ '/Applications//Sublime\ Text.app/Contents/MacOS/Sublime\ Text'
        else
            let g:vimfiler_quick_look_command = 'gloobus-preview'
        endif

        call vimfiler#custom#profile('default', 'context', {
                    \ 'explorer' : 1,
                    \ 'winwidth' : 30,
                    \ 'winminwidth' : 30,
                    \ 'toggle' : 1,
                    \ 'columns' : 'type',
                    \ 'auto_expand': 1,
                    \ 'direction' : 'rightbelow',
                    \ 'parent': 0,
                    \ 'explorer_columns' : 'type',
                    \ 'status' : 1,
                    \ 'safe' : 0,
                    \ 'split' : 1,
                    \ 'no_quit' : 1,
                    \ 'force_hide' : 0,
                    \ })
        autocmd FileType vimfiler call s:vimfilerinit()
        autocmd BufEnter * if (winnr('$') == 1 && &filetype ==# 'vimfiler') |
                    \ q | endif
        function! s:vimfilerinit()
            set nonumber
            set norelativenumber
        endf
\end{verbatim}


\subsection{Quick file search and open}
\label{sec:vim-file-search-and-open}

\begin{enumerate}
  \item Command-T: focus only on one thing (file navigation) -
  Sect.\ref{sec:Command-T}


  \item FuzzyFinder - Sect.\ref{sec:fzf}
  
  \item FuzzyFileFinder
  \url{http://github.com/jamis/fuzzy_file_finder}
  
  \item FuzzyFinder:TextMate: 
  \url{http://github.com/jamis/fuzzyfinder_textmate} 
  
  \item LustyExplorer: 
  
  \item Built-in feature:
  
  \verb!:find <filename>! which search for file under vim's search path
  (Sect.\ref{sec:vim-search-path}). Short form: \verb!:fin <filename>!
  
  \verb!:2find <filename>! open the second match filen in the result
  
  \verb!:tabfind <filename>! open file in a new tab.
  Short form:  \verb!:tabf <filename>!
  
\end{enumerate}

\subsection{ctrlp.vim}
\label{sec:ctrlp.vim}

\subsection{fzf}
\label{sec:fzf}


In spacevim, we launch using
\begin{verbatim}
:FZF
\end{verbatim}

\subsection{-- Command-T}
\label{sec:Command-T}

Press 
\begin{itemize}
  \item  \verb!<leader>-t!  (providing this has not been used)
  
\begin{verbatim}
nmap <silent><Leader>t  :CommandT<CR>
\end{verbatim}

  \item :CommandT
  
To rescan the path for newly added or deleted folders/files, run
\begin{verbatim}
:CommmandTFlush
\end{verbatim}

\end{itemize}
to open file search (a prompt appear at the bottom of the screen), 
\begin{itemize}
  \item , showing all files in the current directory -  the directory as in returned by \verb!:pwd! command

NOTE: Force to search file from a specific folder
\begin{verbatim}
:CommandT ../path/to/other/files
\end{verbatim}

  \item you can type in search string, and only files whose paths contains this string are shown.
  
NOTE: Letters do not need to appear consecutively in a path.

  \item select the desired file: first by moving the highlighter up/down using C-j (move down matching list) or
  C-k (move up matching list) keys, and open it using 
  
  \begin{verbatim}
CR (Enter) key
   open and show in the current buffer
   
C-CR (Ctrl-Enter or Command-Enter) 
    open and show in a new split window
    
C-s  the same C-CR

C-v   
    open and show in a vertical split window

C-t
    open and show in new tab
  
  \end{verbatim}
  
  
\end{itemize}

press Ctrl-c to cancel.

\url{https://wincent.com/products/command-t}

Command-T is downloaded by adding to .vimrc file

However, it needs to be compiled, as part of its tool is written in C, and it
needs ruby.
So, vim must support ruby, e.g. using vim-nox (Sect.\ref{sec:vim-nox}). Then,
ruby header file, e.g. ruby.h, must also be available for the compilation
\begin{verbatim}

sudo apt-get install ruby-dev
\end{verbatim}

To compile
\begin{verbatim}
cd ~/.vim_plugins/command-t/ruby/command-t/ext/command-t

ruby extconf.rb
# Makefile is created, then run

make
\end{verbatim}



\subsection{Open file under cursor}
\label{sec:vim-open-file-under-cursor}

In source code, suppose the cursor is at a given filename
\begin{verbatim}
#include "someheader.h"
\end{verbatim}
and you want to open that filename, e.g. someheader.h.

Built-in VIM command 
\begin{itemize}
  \item \verb!gf! which open a file whose name is under or after
the cursor

NOTE: To open a file with space in name, there are two options:
\begin{enumerate}
  \item permanently make space as a valid character for a filename
\begin{verbatim}
:set isfname+=32
\end{verbatim}

   \item visual select the
file name (press \verb!v! then move the cursor, using \verb!E! key, until you select the
wholename), then press \verb!gf!.
   
\end{enumerate}


  \item \verb!Ctrl-w!f : open \ldots and put in a new buffer, after a horizontal
  split
  
  For a vertical split, press \verb!Ctrl-w!L to move the window to the right.
  
\end{itemize}
\url{http://vim.wikia.com/wiki/Open_file_under_cursor}

By default, vim searches file in working directory. 
However, most projects have separated directory for include files. Thus, you
should set vim's path option to contain comma-separated list of directories to
look for the file (Sect.\ref{sec:vim-search-path}).

{\bf TIPS Java-code}: In Java, \verb!A.B.C! means A/B/C folder hierarchical
structure. Then, we can tell vim to convert \verb!.! to \verb!/! using
\begin{verbatim}
set includeexpr=substitute(v:fname,'\\.','/','g')
\end{verbatim}

\subsection{Buffer navigation}

Suppose you open multiple files (in multiple buffers). How to switch from one to
another quickly?

\begin{enumerate}
  \item \verb!bufExplorer!:
  
  \url{https://github.com/jlanzarotta/bufexplorer}
  
  
  
\end{enumerate}

\subsection{Text navigation}

In command-mode



In insert-mode
\url{http://stackoverflow.com/questions/1737163/traversing-text-in-insert-mode}

\begin{verbatim}
" provide hjkl movements in Insert mode via the <Alt> modifier key
inoremap <A-h> <C-o>h
inoremap <A-j> <C-o>j
inoremap <A-k> <C-o>k
inoremap <A-l> <C-o>l
\end{verbatim}


\subsection{Code complete}

Code complete
  \begin{enumerate}

\item YouCompleteMe (YCM): as you type, YCM combines output from several
sources (i.e. engines). Knowing about these engines will prevent confusion
(Sect.\ref{sec:YouCompleteMe}).

\item \verb!clang_complete!: Sect.\ref{sec:clang_complete}

\item \verb!deoplete-clang! : Sect.\ref{sec:deoplete-clang}

\item ctags: Sect.\ref{sec:ctags-vim} explains how to create an indexed file
by ctags, and how to jump to different tags using shortcut.
 
\item VIM built-in feature: (type a first few characters of the word you
want, then while in insert-mode, press) \verb!Ctrl-N!  
  \end{enumerate}


\subsection{Code folding/unfold}

Vim enables different choices for code folding
\begin{verbatim}
// indent, syntax, expr, manual, 
set foldmethod=indent
set foldlevel=99
\end{verbatim}

With Python code, use \verb!SimplyFold! package.

\subsection{Syntax hightlight}

C++: \url{http://www.vim.org/scripts/script.php?script_id=1640}


\subsection{Column-based alignment (CSV, JSON)}


\begin{enumerate}
  \item \verb!vim-easy-align!
  
  \url{https://github.com/junegunn/vim-easy-align}
  
\end{enumerate}


\subsection{Realign code (code format)}


\begin{enumerate}
  
  \item \verb!auto-format!: support using different tools
  (clang-format for C/C++, astyle for C/C++/C\#/Java, autopep8 for Python,
  js-beautiful for JSON/Javascript \ldots )
  
  \url{https://github.com/Chiel92/vim-autoformat}
  
  
  \item \verb!vim-clang-format!:
  \url{https://github.com/rhysd/vim-clang-format}
  
  \url{http://clang.llvm.org/docs/ClangFormat.html#vim-integration}
  
  \item \verb!vim-easy-align!
  
  \url{https://github.com/junegunn/vim-easy-align}
  
  \item Vim built-in feature:
  
\begin{verbatim}
:filetype indent on
\end{verbatim}
and make sure VIM knows the filetype you're editing.

Call \verb!={motion}!, and VIM will guess the indentation based on the filetype.
\begin{verbatim}
== align the current line
=i{ align the inner block
=% align to the matching parenthesis/bracket under the cursor
=14j or 14== align the next 14 lines
=G align to the end of the file
vG= same thing, align to the end of the file (but using visual mode)
vjjj= align four lines (using visual mode)
\end{verbatim}
and the indentation is based on the configuration of how many spaces per tab,
and if you want to expand tab into spaces or not
\begin{verbatim}
:set expandtab softtabstop=3 tabstop=3 shiftwidth=3
\end{verbatim}

If vim is not guessing the indentation level correctly, there are two ways to
change it:
\begin{itemize}
  \item In command-mode
\begin{verbatim}
<<  To shift the line left
>>  To shift the line right
\end{verbatim}
  \item In insert-mode
\begin{verbatim}
Ctrl-T  To shift to the right
Ctrl-D  To un-indent one tab
\end{verbatim}
\end{itemize}

  \item \verb!astyl! utility
\begin{verbatim}
apt-get install astyle
\end{verbatim}
and use inside VIM
\begin{verbatim}
:%!astyle   default mode C/C++

%!astyle --mode=c --style=ansi -s2  (ANSI C++ style: 2 spaces indentation)

:1,40!astyle --mode=c --style=ansi (apply from lines 1 to 40 only)
\end{verbatim}

\url{http://stackoverflow.com/questions/2506776/is-it-possible-to-format-c-code-with-vim}

\end{enumerate}
\url{http://stackoverflow.com/questions/2587435/how-does-one-align-code-braces-parens-etc-in-vi}

\subsection{Compile your project}


\begin{enumerate}
  \item Built-in feature:
  
\verb!:make! command which, by default, executes make in current directory and
parses output for errors.

NOTE: The actual command to execute is stored in \verb!makeprg! option.
So, you can change it to use a different build system
\begin{verbatim}
set makeprg=make\ -C\ ../build\ -j9
\end{verbatim}
 
TIPS: Bind to shortcut key F4
\begin{verbatim}
nnoremap <F4> :make!<cr>
\end{verbatim}
(\verb.!. mark prevents vim from jumping to location of first error found)

\subsection{Run your program}

Map to F5 key to run a program
\begin{verbatim}
nnoremap <F5> :!./my_great_program<cr>
\end{verbatim}
\end{enumerate}

\subsection{Version control with VIM}


\subsection{ * git}

Use \verb!fugitive! plugin.
\url{http://vimcasts.org/episodes/fugitive-vim---a-complement-to-command-line-git/}

You may want to install \verb!lighthttpd! if you want to use \verb!Gbrowse!
command.


Once installed fugitive, using \verb!:Git! command to access all git
commands.
You still need to switch to shell for git commands that generate a log of
output.

\begin{Verbatim}
:Git add % 	:Gwrite 	Stage the current file to the index
:Git checkout % 	:Gread 	Revert current file to last checked in version
:Git rm % 	:Gremove 	Delete the current file and the corresponding Vim buffer
:Git mv % 	:Gmove 	Rename the current file and the corresponding Vim buffer

		 :Gcommit command opens up a commit window in a split window.
		 :Gblame command opens a vertically split window containing annotations for
		 each line of the file: the last commit reference, with author and timestamp.
\end{Verbatim}

NOTE: fugitive provides autocomplete of branch name.


\section{Ruby-based plugins}
\label{sec:ruby-based-plugins-vim}


To add Ruby-binding to neovim (Sect.\ref{sec:neovim-Ruby})

To add Ruby to vim (Sect.\ref{sec:vim-Ruby})

These are Ruby-based plugins
\begin{enumerate}
  \item Command-T: Sect.\ref{sec:Command-T}
  
  \item 
\end{enumerate}

To write a Vim/Neovim's plugin in Ruby, read
\url{https://github.com/alexgenco/neovim-ruby}


\section{Text processing}

\subsection{change uppercase/lowercase}

\url{http://vim.wikia.com/wiki/Switching_case_of_characters}

\subsection{evaluate expression}
\label{sec:vim-evaluate-expression}
\label{sec:vim-evaluate-commands}

We use \verb!''=! expression register (Sect.\ref{sec:register-vim-nvim})

Example: If, in insert mode, you type Ctrl-r =, you will see a \verb!=! sign in
the command line. Then if you type \verb!2+2 <enter>!, 4 will be printed. This
    can be used to execute all sort of expressions, even calling external
    commands.


Example: get return of linux's command: if you type Ctrl-r = and then, in the
command line, \verb!system('ls') <enter>!, the output of the ls command will be
pasted in your buffer.

\subsection{text replacement}

\begin{verbatim}
:%s/test-source/test-new/gc
\end{verbatim}

To replace that text (that you just searched with \verb!/!) with something else
(e.g. mustache), you don't have to retype that text, just use \verb!/! register
(Sect.\ref{sec:register-vim-nvim})
\begin{verbatim}
:%s/<Ctrl-r />/mustache/g
\end{verbatim}



\section{Buffers in vim}
\label{sec:vim-buffer}

\subsection{switch buffer}
\label{sec:vim-buffer-switch}

Switch between the last two buffers
\begin{verbatim}
Ctrl-^
 // or
:e Ctrl-r #
\end{verbatim}
NOTE: \verb!''#! register hold the name of last open buffer
(Sect.\ref{sec:register-vim-nvim}).


\section{VimL: VIM native scripting language}
%\section{VIM script}
\label{sec:vim-script}

VIM script is mostly written in imperative programming style, i.e.
using statements to change the program's state
\begin{itemize}
\item a simple form of object-oriented programming
  
  \item basic form of functional programming using \verb!map()! and
  \verb!filter()! functions.
  
\end{itemize}

Most plugins for VIM are written using VimL, but plugins can also utilize other
interpreted languages like Perl, Python, Lua, or Ruby (if support for them is
compiled into the Vim binary).

Vim script files are stored in plain text format and the file name extension is
\verb!.vim!. 

\url{https://en.wikipedia.org/wiki/Vim_(text_editor)}

\url{https://en.wikibooks.org/wiki/Learning_the_vi_Editor/Vim/VimL_Script_language}

A library of useful functions written in VimL can be downloaded from here
\url{https://github.com/LucHermitte/lh-vim-lib}


\subsection{data type}

Since VIM 7.1, new supported data types
\begin{enumerate}
  \item \verb!list!
  
  \item \verb!dictionaries!
\end{enumerate}

\subsection{declare variable}

\begin{verbatim}
let i = 1
\end{verbatim}


\subsection{let vs. set}

\url{http://stackoverflow.com/questions/9990219/vim-whats-the-difference-between-let-and-set}

:set is for setting options, :let for assigning a value to a variable.



\subsection{comparison}

There are three options of operator: \verb!==!, \verb!==?!, and \verb!==#!.

Be careful when comparing strings. Use \verb!==?! or \verb!==#! only.
\begin{itemize}
  \item \verb!==?! : comparison based on "case-insensitive no matter what the
  user has set" 
  
  \item \verb!==#! : comparison "case-sensitive no matter what the user has
  set" 
\end{itemize}

\begin{verbatim}
:if "foo" == "bar"
:echom "one"
:elseif "foo" == "foo"
:echom "two"
:endif
\end{verbatim}

\begin{verbatim}
:set noignorecase
:if "foo" ==? "FOO"
:echom "first"
:elseif "foo" ==? "foo"
:echom "second"
:endif
\end{verbatim}

\subsection{case-sensitive or case-insensitive ?}

Use
\begin{verbatim}
:set ignorecase

:set noignorecase
\end{verbatim}
\url{http://learnvimscriptthehardway.stevelosh.com/chapters/22.html}

\begin{verbatim}
:set noignorecase
:if "foo" == "FOO"
:echom "vim is case insensitive"
:elseif "foo" == "foo"
:echom "vim is case sensitive"
:endif
\end{verbatim}

\subsection{condition}


\begin{verbatim}
:set ignorecase
:if "foo" == "FOO"
:echom "no, it couldn't be"
:elseif "foo" == "foo"
:echom "this must be the one"
:endif
\end{verbatim}

\section{Encoding for file/folder name}

If you need to use non-Unicode encoding scheme (which is not encouraged), then
you can do, e.g. with CP1251 encoding scheme.
\begin{verbatim}
:call mkdir(iconv("son-non-Unicode-character", "utf-8", "cp1251"))
\end{verbatim}
\url{http://stackoverflow.com/questions/2271135/create-directory-in-vim}

\section{VIM with Python 3.x support}
\label{sec:vim-Python-3.x}
\url{http://vi.stackexchange.com/questions/2403/vim-code-completion-for-python-3}


\section{Code complete}



\subsection{YouCompleteMe plugin}
\label{sec:YouCompleteMe}

YouCompleteMe enables code-completion for one or many programming langugages
(e.g. C/C++/Objective-C/Objective-C++).
It is written in C++ and glue to Vim via Python.

In neovim or SpaceVim, it is recommended to use deoplete.nvim
(Sect.\ref{sec:deoplete})
\begin{verbatim}
:help SpaceVim-faq
\end{verbatim}

However, certains people still like YCM; part of that because it has
Python-based configuration file \verb!.ycm_extra_conf.py!
(Sect.\ref{sec:.ycm_extra_conf.py}).

To get YCM to be activated in neovim/SpaceVim, 
\url{https://github.com/neovim/neovim/issues/1315}


\begin{verbatim}
cd ~/.vim/plugins/
# or cd ~/.vim_plugins/

cd YouCompleteMe

./install.py --clang-completer

./install.py --clang-completer --system-libclang
\end{verbatim}

TO ENABLE RECOMPILATION: in \verb!~/.vimrc!, 
introduce post git-update or git-clone action
\begin{verbatim}
Plug 'Valloric/YouCompleteMe', { 'do': './install.py --clang-completer --system-libclang' }
\end{verbatim}

TROUBLESHOOT:
\begin{verbatim}
ycm_core.so: undefined symbol: clang_getCompletionFixIt
\end{verbatim}
EXPLAIN: The system libclang is too old. We need at least version 7.0.0.


\subsection{-- installation}
\label{sec:YouCompleteMe-install-on-PowerPC-IBM}


As the built-in libclang.so.7 is for x86-64, we need to use the system libclang.
Make sure you also first install the latest libclang

\begin{verbatim}
sudo apt-get install libclang-6.0-dev
sudo apt-get install clang-tidy-6.0


cd YouCompleteMe

./install.py --all --system-libclang  --clang-tidy
\end{verbatim}

\subsection{-- installation}
\label{sec:YouCompleteMe-install}

Make sure you add the language support when compile YouCompleteMe.
% 
% HOST_GCC=$HOME/bin/gcc/4.8.2; 
% export EXTRA_CMAKE_ARGS="-DCMAKE_C_COMPILER=${HOST_GCC}/bin/gcc
%-DPYTHON_LIBRARY=`python-config --prefix`/lib/libpython2.7.so"
To install
\begin{verbatim}
# with custom Python
# with custom gcc version and custom libstdc++ location
# with location to libclang.so.3.8
export
EXTRA_CMAKE_ARGS="-DCMAKE_C_COMPILER=/gsa/yktgsa/home/t/m/tmhoangt/bin/gcc/4.8.2/bin/gcc
-DCMAKE_CXX_COMPILER=/gsa/yktgsa/home/t/m/tmhoangt/bin/gcc/4.8.2/bin/g++
-DCMAKE_CXX_LINK_FLAGS='-L/gsa/yktgsa/home/t/m/tmhoangt/bin/gcc/4.8.2/lib64
-Wl,-rpath,/gsa/yktgsa/home/t/m/tmhoangt/bin/gcc/4.8.2/lib64'
-DCMAKE_BUILD_TYPE=Release
-DPYTHON_LIBRARY=/home/tmhoangt/bin/lib/libpython2.7.so
-DPYTHON_EXECUTABLE=/home/tmhoangt/bin/bin/python
-DLIBCLANG_LIBRARY=/usr/lib/llvm-3.5/lib/libclang.so
-DLIBCLANG_INCLUDE_DIR=/usr/lib/llvm-3.5/include/"
 

# tell to use different libstdc++.so.6
export LD_LIBRARY_PATH=~/bin/gcc/4.8.2/lib64:$LD_LIBRARY_PATH

# then if you already have clang
#  to use clang for C-based language code completion (C/C++)
./install.py --clang-completer --system-libclang
\end{verbatim}

NOTE: For help debugging, add
\begin{verbatim}
-DCMAKE_VERBOSE_MAKEFILE=ON
\end{verbatim}

\subsection{-- code completion}

How does YouCompleteMe perform code completion.
In C++, even though Clang is the front-end, it doesn't know the location of the
header files to perform code completion until you configure the path.

You could use anything: Visual C++, Borland C++, GCC, MinGW, MinGW-w64,
LLVM/Clang, etc. Each of them has their own standard library and runtime
supplied. Thus, in each case you'd have to specify all the paths which your
current toolchain uses to find standard includes.
To know, run
\begin{verbatim}

//IF GCC, MinGW, MinGW-w64
  // POSIX shell
g++ -E -x c++ - -v < /dev/null

  // Windows
g++ -E -x c++ - -v < nul

//IF LLVM/Clang,
  // POSIX shell
clang++ -E -x c++ - -v < /dev/null

clang++ -E -x c++ - -v < nul
\end{verbatim}

Finally, add them to the \verb!flags! list with \verb!'-I'! for each path
{\tiny
\begin{verbatim}
'-I',
'D:/Toolchains/x64/MinGW-w64/4.8.1/POSIX/SEH/lib/gcc/x86_64-w64-mingw32/4.8.1/include/c++',
'-I',
'D:/Toolchains/x64/MinGW-w64/4.8.1/POSIX/SEH/lib/gcc/x86_64-w64-mingw32/4.8.1/include/c++/x86_64-w64-mingw32',
'-I',
'D:/Toolchains/x64/MinGW-w64/4.8.1/POSIX/SEH/lib/gcc/x86_64-w64-mingw32/4.8.1/include/c++/backward',
'-I',
'D:/Toolchains/x64/MinGW-w64/4.8.1/POSIX/SEH/lib/gcc/x86_64-w64-mingw32/4.8.1/include',
'-I',
'D:/Toolchains/x64/MinGW-w64/4.8.1/POSIX/SEH/lib/gcc/x86_64-w64-mingw32/4.8.1/include-fixed',
'-I',
'D:/Toolchains/x64/MinGW-w64/4.8.1/POSIX/SEH/x86_64-w64-mingw32/include',
\end{verbatim}
}
\url{http://stackoverflow.com/questions/18801354/c-family-semantic-autocompletion-plugins-for-vim-using-clang-clang-complete-yo}


There are one important file: 
\verb!.ycm_extra.conf.py! (Sect.\ref{sec:.ycm_extra_conf.py})

Engines: 
% an identifier-based engine that works with every
% programming language
\begin{itemize}
  \item  C/C++/Objective-C/Objective-C++ (from now on referred to as
"the C-family languages"): , a semantic, Clang-based engine that provides native semantic code
completion.

NOTE: You need to add \verb!include! folders for searching header files to
the file above by adding \verb!-I! options.
  \begin{verbatim}
  '-I', '/to/this/include',
  '-I', '$NTSROOT/to/another/include',
  \end{verbatim}
  
  TIPS: if you have project's specific \verb!.vimrc! and you already populate
  \verb!path! with all include folders (Sect.\ref{sec:vim-search-path}), then
  you only need
  \begin{verbatim}
  "='-I'.substitute(&path, ',', '\n-I', 'g')<cr>p
  \end{verbatim}

 
  \item Python: a Jedi-based completion engine for Python, 

NOTE: YCM currently does not support Python 3.x. To get Python 3.x support
in vim, read Sect.\ref{sec:vim-Python-3.x}

  \item C\#: an
OmniSharp-based completion engine for 

  \item many other languages (Ruby, PHP etc): 
and an omnifunc-based completer
that uses data from Vim's omnicomplete system 
\end{itemize}
YCM prioritize them with a complex algorithm.


There is no need to press any shortcut to invoke the completion menu.
\url{http://www.alexeyshmalko.com/2014/youcompleteme-ultimate-autocomplete-plugin-for-vim/}

YCM has a client-server architecture; the Vim part of YCM is only a thin client
that talks to the ycmd HTTP+JSON server that has the vast majority of YCM logic
and functionality. The server is started and stopped automatically as you start
and stop Vim.


Final steps: make sure you have the file 
\verb!.ycm_extra_conf.py! at the global level, or for specific project
configuration, add to each project folder, and then in \verb!~/.vimrc! 
add
\begin{verbatim}
let g:ycm_global_ycm_extra_conf = '~/.ycm_extra_conf.py'
let g:ycm_confirm_extra_conf=0
\end{verbatim}

DEBUG purpose: in Vim, run
\begin{verbatim}
:YcmDebugInfo
\end{verbatim}

ERROR:
\begin{verbatim}
YouCompleteMe unavailable:  no module named ycmd
\end{verbatim}
SOLUTION:
\begin{verbatim}
// check :h SpaceVim-faq

cd ~/.cache/vimfiles/repos/github.com/Valloric/YouCompleteme/
git submodule update --init --recursive
./install.py --clang-completer --system-libclang

\end{verbatim}

 \subsection{.ycm\_extra\_conf.py}
 \label{sec:.ycm_extra_conf.py}




IMPORTANT: When you open a source file, the engine
looks for the file \verb!.ycm_extra.conf.py! file in the directory of the opened file or in any
directory above it in the hierarchy (recursively); when the file is found,
it is loaded (only once!) as a Python module. This file is important as YCM
uses the information from it to compile the source code.


\begin{enumerate}
  \item You can copy and modify the sample file:
\url{https://github.com/Valloric/ycmd/blob/master/cpp/ycm/.ycm_extra_conf.py}
 
  \item Another good example:
  \url{https://jonasdevlieghere.com/a-better-youcompleteme-config/}
  
   \item You can  use the YCM-Generator tool to help generate this file,
   if your project is designed to be compiled with CMake, make, qmake, autotools: 
\url{https://github.com/rdnetto/YCM-Generator}
\end{enumerate}

In order to let clang know about your include directories custom defines, you
should place your \verb!-I! and \verb!-D! compiler flags into
\verb!.clang_complete! file at the root of your project.
\subsection{Clang\_complete}
\label{sec:clang_complete}

There is one important file \verb!.clang_complete!

\url{http://stackoverflow.com/questions/18801354/c-family-semantic-autocompletion-plugins-for-vim-using-clang-clang-complete-yo}

\subsection{.clang\_complete}
 \label{sec:.clang_complete}
 



\subsection{deoplete, deoplete-clang, deoplete-clang2}
\label{sec:deoplete}

Compared to YouCompleteMe (Sect.\ref{sec:YouCompleteMe}), deoplete is much more
lightweight. Als, install neomake (Sect.\ref{sec:neomake}) to replace syntastic
(Sect.\ref{sec:Syntastic}). deoplete (pair it with neomake and your favorite
linter) is a lightweight solution.

\begin{verbatim}
pip2 install --upgrade neovim
pip3 install --upgrade neovim
\end{verbatim}
\url{https://github.com/Shougo/deoplete.nvim}

Deoplete requires you to do your own configuration for keybinds and stuff. It's
a little annoying as compared to YCM/neocomplete's setup.
To configure, check
\footnote{\url{https://github.com/zchee/deoplete-clang}}.

\subsection{-- deoplete-clang}
\label{sec:deoplete-clang}

\verb!deoplete-clang! is specifically designed to work with neovim, and is built
upon these tools
\begin{enumerate}
  \item \verb!deoplete! - Sect.\ref{sec:deoplete}
  \item \verb!libclang-python3!
  \item 
\end{enumerate}


It enable auto-completing C/C++/Objective-C/Objective-C++
using libclang.  \url{https://github.com/zchee/deoplete-clang}


\begin{verbatim}
let g:deoplete#enable_at_startup = 1
let g:deoplete#auto_complete_start_length = 1
let g:deoplete#sources#clang#libclang_path = '/usr/lib/libclang.so'
let g:deoplete#sources#clang#clang_header = '/usr/include/lib/clang'
let g:deoplete#sources#clang#std#cpp = 'c++11'
let g:deoplete#sources#clang#sort_algo = 'priority'
let g:deoplete#sources#clang#clang_complete_database = '/home/pacov/code/build'
\end{verbatim}
Based on the value of \verb!clang-complete_database! option, it is the
directory where the file \verb!compile_commands.json! should
be
\begin{verbatim}
/home/user/code/build/compile_commands.json
\end{verbatim}
\url{https://github.com/zchee/deoplete-clang/issues/11}

Currently, deoplete-clang does not support \verb!.ycm_extra_conf.py! file. For
now, any extra configuration has to be done via the json file.
\begin{verbatim}
let g:deoplete#sources#clang#libclang_path = "/opt/llvm/lib/libclang.dylib"
let g:deoplete#sources#clang#clang_header = '/opt/llvm/lib/clang'
" clang default include flags
"   - echo | clang -v -E -x c -
let g:deoplete#sources#clang#flags = [
      \ "-cc1",
      \ "-triple", "x86_64-apple-macosx10.11.0",
      \ "-emit-obj",
      \ "-mrelax-all",
      \ "-disable-free",
      \ "-disable-llvm-verifier",
      \ "-mrelocation-model", "pic",
      \ "-pic-level", "2",
      \ "-mthread-model", "posix",
      \ "-mdisable-fp-elim",
      \ "-munwind-tables",
      \ "-target-cpu", "core2",
      \ "-target-linker-version", "264.3",
      \ "-dwarf-column-info",
      \ "-debugger-tuning=lldb",
      \ "-resource-dir", "/opt/llvm/lib/clang/3.9.0",
      \ "-isysroot", "/Applications/Xcode-beta.app/Contents/Developer/Platforms/MacOSX.platform/Developer/SDKs/MacOSX10.11.sdk",
      \ "-ferror-limit", "19",
      \ "-fmessage-length", "213",
      \ "-stack-protector", "1",
      \ "-fblocks",
      \ "-fobjc-runtime=macosx-10.11.0",
      \ "-fencode-extended-block-signature",
      \ "-fmax-type-align=16",
      \ ]
let g:deoplete#sources#clang#sort_algo = 'priority'
let g:deoplete#sources#clang#clang_complete_database = '/Users/zchee/src/github.com/neovim/neovim/build'
\end{verbatim}

\subsection{-- deoplete-clang2}
\label{sec:deoplete-clang2}

\verb!deoplete-clang2! is faster than \verb!deoplete-clang!, as it uses
\verb!clang -cc1!. 
\url{https://github.com/tweekmonster/deoplete-clang2}

USING
\begin{itemize}
  \item \verb!tab! move to the next
  \item \verb!s-tab! move to the previous
\end{itemize}

Configure search path
\begin{itemize}
  \item  
  
\end{itemize}


Configure compilation flags
\begin{itemize}
  \item  compiler-flags local to individual projects: create \verb!.clang! file
  at project root.
  
  It will search the current working directory for a .clang file; if no such
  file is found it will try searching the parent directory.

You should be able to just paste most of your compile flags in there (the parts
that make sense at least). Mainly, it should have the relevant -I, -D, -F flags.
The plugin will try to fill in the blanks for system include paths and discard
the flags that are causing completions to not work.


  \item system-wide flags: specify the location of the file
  \verb!compile_commands.json! file.

\begin{verbatim}
  // C(XX)FLAGS for generating completion
  // REQUIRED: no
  // CHECK
  # C++
echo | clang -v -E -x c++ -

  // a list
  // C++ default: ['-x', 'c++']
  // ["-fblocks",]
g:deoplete#sources#clang#flags
\end{verbatim}

Example:
\begin{verbatim}
let g:deoplete#sources#clang#flags = ['-Iwhatever', ...] 
\end{verbatim}
in your nvim configs (Sect.\ref{sec:spacevim-configuration}).

\end{itemize}


\begin{verbatim}  
  //default: ''    (or path to libclang.so)
  // REQUIRED: yes
g:deoplete#sources#clang#libclang_path

  //default: ''    (or path to include folder)
  // REQUIRED: yes
g:deoplete#sources#clang#clang_header

  // a dict 
  // {'c': 'c11', 'cpp': 'c++1z', 'objc': 'c11', 'objcpp': 'c++1z'}
g:deoplete#sources#clang#std

  
  // sorting of completion results
  // 'priority'  or  'alphabetical'
g:deoplete#sources#clang#sort_algo

g:deoplete#sources#clang#clang_complete_database
\end{verbatim}



\section{Spacemacs}

\url{https://github.com/syl20bnr/spacemacs#introduction}

\chapter{Neovim (nvim)}
\label{chap:neovim}
\label{sec:neovim}

\begin{mdframed}
Although vim is under active development, it includes some ~300k lines of C89
code. In addition to being difficult to maintain, issues and new pull requests
cannot be addressed very easily since Bram Moolenaar is the only person who
maintains the large codebase of the program.  

These difficulties, along with the lack of support for some desired features
such as asynchronous plugins, motivated the birth of NeoVim as a fork of vim.
\end{mdframed}

Neovim is a complete rewrite of vim editor (Chap.\ref{chap:Vim}).
Two important new features:
\begin{enumerate}
  \item  asynchronous plugins: to modernize the plugin architecture


Asynchronous plugins make it possible for plugins to run as background processes
without interfering with the main editor process.
  
  \item the Remote Procedure Call (RPC) API for controlling NeoVim
  programmatically.
  
The RPC API allows GUI programs (and other software that speaks the msgpack-rpc
  messaging protocol) to connect to a running NeoVim instance. This means that
  you can integrate well-known text editors such as Atom, Visual Studio Code, or
  Sublime Text with NeoVim and have modifications sync bidirectionally between
  them. That way, you can leverage all the features of the GUI program while
  using nvim's engine behind the scenes.
  
\end{enumerate}
It also provide a better out-of-the-box experience for Vim users, i.e. no need
to worry about enabling/disabling about 100 different compiling options as when
you build Vim, and including a built in terminal emulator
(Sect.\ref{sec:nvim-terminal-emulator}).

Depending if you use SpaceVim or not, the configurations files
can be at different locations
\begin{enumerate}
  \item pure NeoVim: 

It uses \verb!.nvimrc! as resource file, and extra configuration is put inside
\verb!.nvim!. As neovim is backward compatible with vim, we can share them by
symlink
\begin{verbatim}
ln -s .vimrc .nvimrc
ln -s .vim .nvim
\end{verbatim}

Inside the same resource file, we can separate neovim-specific stuff using 
\begin{verbatim}
if has('nvim') 

endif
\end{verbatim}
block.

   \item SpaceVim + Neovim: Sect.\ref{sec:nvim-SpaceVim}
\end{enumerate}

\section{Install neovim}

NeoVim has (pre-built version) packages available for Homebrew, MacPorts,
Ubuntu, Debian, Arch, and many other environments
\begin{itemize}
  \item Ubuntu (which has nvim 0.2.2)
  
  
\begin{verbatim}
sudo apt-get install python-dev python-pip python3-dev python3-pip

sudo add-apt-repository ppa:neovim-ppa/stable
# or 
# sudo add-apt-repository ppa:neovim-ppa/unstable
sudo apt-get update
sudo apt-get install neovim
\end{verbatim}

NOTE: You can configure to use nvim by default for different editor
\begin{verbatim}
sudo update-alternatives --install /usr/bin/vi vi /usr/bin/nvim 60
sudo update-alternatives --config vi
sudo update-alternatives --install /usr/bin/vim vim /usr/bin/nvim 60
sudo update-alternatives --config vim
sudo update-alternatives --install /usr/bin/editor editor /usr/bin/nvim 60
sudo update-alternatives --config editor
\end{verbatim}

  \item Debian
  
\begin{verbatim}
sudo apt-get install neovim
sudo apt-get install python-neovim
sudo apt-get install python3-neovim
\end{verbatim}


  \item  OS X with Homebrew
  
\begin{verbatim}
brew install neovim/neovim/neovim
\end{verbatim}
the way to summon NeoVim is to say its name three times.

  \item Linux with AppImage  (nvim 0.3.0-dev)
\begin{verbatim}
curl -LO https://github.com/neovim/neovim/releases/download/nightly/nvim.appimage
chmod u+x nvim.appimage
./nvim.appimage
\end{verbatim}


  \item  other environments:
  \url{https://github.com/neovim/neovim/wiki/Installing-Neovim}
\end{itemize}

Then install ALE (\url{https://github.com/w0rp/ale#installation})
\begin{verbatim}
mkdir -p ~/.local/share/nvim/site/pack/git-plugins/start
git clone https://github.com/w0rp/ale.git ~/.local/share/nvim/site/pack/git-plugins/start/ale
\end{verbatim}

\subsection{make nvim default}

In .zshrc or .bashrc
\begin{verbatim}
if type nvim > /dev/null 2>&1; then
  alias vim='nvim'
fi
\end{verbatim}

\subsection{add python support for plugins}

The main advantages of using Python in plugins is that it enables plugins to
have access to network sockets, and perform long-running or expensive operations
in the background without freezing the Vim UI.


Vim plugins are able to execute Python code when Vim is compiled with Python support

Do:
\begin{itemize}
  \item system-wide

\begin{verbatim}
sudo pip install neovim
\end{verbatim}

  \item user-specific:

\begin{verbatim}
pip2 install --user neovim
pip3 install --user neovim
\end{verbatim}
\end{itemize}

NOTE: Check sys.path
\begin{verbatim}
python -m site
\end{verbatim}
{\tiny  
\begin{verbatim}
sys.path = [
    '/home/tmhoangt/Downloads',
    '/home/tmhoangt/.local/lib/python2.7/site-packages/efel-0+unknown-py2.7-linux-x86_64.egg',
    '/home/tmhoangt/.local/lib/python2.7/site-packages/numpy-1.14.2-py2.7-linux-x86_64.egg',
    '/packages/gcc/5.4.0/share/gcc-5.4.0/python',
    '/home/tmhoangt/Scripts/NTS_run',
    '/home/tmhoangt/Scripts/NTS_plotting',
    '/usr/lib/python2.7',
    '/usr/lib/python2.7/plat-x86_64-linux-gnu',
    '/usr/lib/python2.7/lib-tk',
    '/usr/lib/python2.7/lib-old',
    '/usr/lib/python2.7/lib-dynload',
    '/home/tmhoangt/.local/lib/python2.7/site-packages',
    '/usr/local/lib/python2.7/dist-packages',
    '/usr/lib/python2.7/dist-packages',
    '/usr/lib/python2.7/dist-packages/gtk-2.0',
]
USER_BASE: '/home/tmhoangt/.local' (exists)
USER_SITE: '/home/tmhoangt/.local/lib/python2.7/site-packages' (exists)
ENABLE_USER_SITE: True
\end{verbatim}
}

\subsection{configuration for neovim}

We can choose either
\begin{enumerate}
  \item referencing to the existing \verb!.vimrc! file
  
  \item using dotfiles
  
  \item use SpaceVim to manage configuration files
\end{enumerate}

\subsection{-- configuration file: init.vim}

NeoVim does not load your .vimrc file, nor does it use .nvimrc.
Instead, it load \verb!init.vim! file locating in
\begin{verbatim}
${XDG_CONFIG_HOME}/nvim/init.vim
\end{verbatim}
This is because NeoVim supports the XDG Base Directory Specification, which
among other things attempts to standardize where user configuration files should
live and their order of precedence. To link to the existing \verb!.vimrc! file,
follow
\begin{verbatim}
mkdir -p ${XDG_CONFIG_HOME:=$HOME/.config}
ln -s ~/.vim $XDG_CONFIG_HOME/nvim
ln -s ~/.vimrc $XDG_CONFIG_HOME/nvim/init.vim
\end{verbatim}

\subsection{-- using dotfiles (rcm)}
\label{sec:dotfiles}

Dotfiles are plain text configuration files on Unix-y systems for things like
our shell, ~/.zshrc, our editor in ~/.vimrc, and many others. 
These files have gone by many names in history, such as "rc files" because they
typically end in rc or "dotfiles" because they begin with a period.

Since these files are all plain text, we can gather them together in a git
repository and use that to track the changes you make over time. 
To help managing, and sharing these files, we can put them all into a single
folder, called \verb!dotfiles!. This is a directory containing all the .*rc
files in your home directory.

Thoughbot
(\url{https://thoughtbot.com/upcase/videos/intro-to-dotfiles})
put them together into a git folder -
\url{https://github.com/thoughtbot/dotfiles}.

They are purposefully somewhat limited in how far they go with configuration and
customization as they are intended to be used as a base. You can put your
customization (Sect.\ref{sec:dotfiles-customized}).

To download the base dotfiles folder
\begin{verbatim}
git clone git://github.com/thoughtbot/dotfiles.git ~/dotfiles
\end{verbatim}
\url{https://github.com/thoughtbot/rcm}

Then install rcm suite (Sect.\ref{sec:rcm-suite}).


and run the first time initialization
\begin{verbatim}
env RCRC=$HOME/dotfiles/rcrc rcup
\end{verbatim}
which will modify 2 files
\begin{verbatim}
~/.zshrc     ZSH shell
~/.vimrc     VIM
\end{verbatim}

NOTE: later on, to check for dotfiles update, run
\begin{verbatim}
rcup
\end{verbatim}

\subsection{---- rcm suite}
\label{sec:rcm-suite}

The rcm suite of tools is for managing \verb!dotfiles! directories
(Sect.\ref{sec:dotfiles}). To install \verb!rcm! package
\begin{itemize}
  \item Ubuntu

\begin{verbatim}
sudo add-apt-repository ppa:martin-frost/thoughtbot-rcm
sudo apt-get update
sudo apt-get install rcm
\end{verbatim}
  
  \item MacOS with Brew

\begin{verbatim}
brew tap thoughtbot/formulae
brew install rcm
\end{verbatim}
\end{itemize}
which then create a few scripts
\begin{verbatim}
/usr/bin/rcup   (main) script: used to install and update dotfiles

/usr/bin/rcdn     opposite of 'rcup' - downgrade 

/usr/bin/mkrc     introducing a dotfile into your dotfiles directory

/usr/bin/lsrc     shows you all your dotfiles 
\end{verbatim}

This rcm suite is useful for committing your rc files (inside \verb!dotfiles!
folder) to a central repository to share, but it also scales to a more complex
situation such as multiple source directories shared between computers with some
host-specific or task-specific files.
\url{http://thoughtbot.github.io/rcm/rcm.7.html}


\subsection{---- extend dotfiles}
\label{sec:dotfiles-customized}

Each of the config files in the thoughtbot dotfiles (Sect.\ref{sec:dotfiles})
provides a base configuration but will also read in a "local" file to allow for
additional, more personalized configuration.

Example:  thoughtbot's dotfiles contain a ~/.vimrc file will the base that will
optionally source in a ~/.vimrc.local file if available. You can use this
~/.vimrc.local file to override and add to the configurations provided by the
base ~/.vimrc file.


\begin{verbatim}
mkdir ~/dotfiles-local

// put your customizations in ~/dotfiles-local appended with .local:


\end{verbatim}



\section{Plugins with asynchronous support}


Neomake which is implementing asynchronous :make using the new job control API
provided by neovim.

\subsection{neomake}
\label{sec:neomake}

NeoMake was built from the ground up to perform similar tasks of Syntastic
(Sect.\ref{sec:Syntastic}) while taking advantage of NeoVim's asynchronous jobs.
An alternative is ALE (Sect.\ref{sec:ALE}).

If you inherits Sytastic from the thoughtbot dotfiles (Sect.\ref{sec:dotfiles}),
I had to get creative to keep them both from running. 



\subsection{ALE}
\label{sec:ALE}


Neomake and Ale are doing the same job, just in a different way, like ycm and
deoplete in a certain sense.

The biggest difference is that ALE can lint as you type (although there's
neomake-autolint which can do that as well)


SpaceVim (and neomake) live checking are very flaky, since the change is based
on writing temporary files on disk and it is not actually async at all.
Ale instead is truly async and you can see it with heavy linting systems like
gometalinter or rustc. Vim does not flicker and linting happens as you type.



\section{Troubleshoot}

\subsection{unexpected character in neovim}


\url{https://github.com/neovim/neovim/issues/5990}

PROBLEM: Terminals have all kinds of brokenness, and this is one of them. Many
support changing the cursor shape, but do it in one of two different ways. And
while there are ways to query whether or not a terminal supports the feature,
most do not implement the mechanism.   


SOLUTION: tell terminal to ignore unknown sequence by putting this into
\verb!init.vim!
\url{https://github.com/neovim/neovim/issues/5990}
\begin{verbatim}
let $NVIM_TUI_ENABLE_CURSOR_SHAPE = 0
\end{verbatim}

\subsection{neovim conflict in byobu}
\label{sec:byobu-neovim}
\label{sec:neovim-byobu}

There are several issues with vim + tmux or vim + gnome/mate-terminal - mostly
related some older libvte versions (Sect.\ref{sec:libvte}). So, upgrading byobu
will fix the problem with neovim.

\begin{verbatim}
sudo add-apt-repository ppa:byobu/ppa
sudo apt-get update
sudo apt-get install byobu
//pastebinit
//ccze
//byobu 5.7+
\end{verbatim}

or install from source
\begin{verbatim}
wget https://launchpad.net/byobu/trunk/5.74/+download/byobu_5.74.orig.tar.gz
tar -xvf byobu_5.74.orig.tar.gz
cd byobu_5.74
configure --prefix=/packages/byobu/5.74
make -j4
sudo make install 
\end{verbatim}

\section{nvim terminal emulator}
\label{sec:nvim-terminal-emulator}

Nvim embeds a VT220/xterm terminal emulator based on libvterm, which is present
as a special buffer type.
\begin{verbatim}
:terminal

termopen()  function

edit the file with name matching 
  `term://(.{-}//(\d+:)?)?\zs.*`
  e.g.:
:edit term://bash
:vsplit term://top

\end{verbatim}

\url{https://neovim.io/doc/user/nvim_terminal_emulator.html}

\section{Configuration}

\subsection{check health}
\label{sec:neovim-Ruby}

\begin{verbatim}
:CheckHealth
\end{verbatim}
List any potential issues with neovim

\begin{itemize}
  \item Ruby-based plugins (Sect.\ref{sec:ruby-based-plugins-vim})

\begin{verbatim}
  //Add Ruby support, e.g.
  //  to use Ruby-based plugins
  //  such as command-t plugin
  //https://github.com/alexgenco/neovim-ruby
sudo gem install neovim
\end{verbatim}
  
\end{itemize}



\subsection{register in vim}
\label{sec:register-vim-nvim}


To store texts, vim/nvim uses the so-called {\bf registers}.

To check the contents in all registers, use
\begin{verbatim}
:registers
:reg
\end{verbatim}

Consider registers as a bunch of spaces in memory that vim uses to store some
text. Each of these spaces has a identifier, so it can be accessed later.


Every register is accessed using a double quote before its name. Suppose
register 'r', then \verb!''r! access to that register.
\begin{enumerate}
  \item {\bf named registers}:
  \begin{itemize}
    \item {\bf numbered registers}: with name from 0 to 9
    
    "0 will always have the content of the latest yank (i.e. the same as unamed
    register), and the others will have last 9 deleted text, being "1 the
    newest, and "9 the oldest.
    
    \item {\bf read-only registers}: \verb!".! (dot), \verb!''%! (percent),
    \verb!'':! (colon), \verb!''#! (pound)
    
    NOTE: The last inserted text is stored on \verb!".! (dot register)
    
    NOTE: \verb!''%! holds the current file path (starting from the directory
    where vim was first opened.) Example:
 \begin{verbatim}
  //GOAL: copy the current file path to the register
  // let = write to a register
 :let @+=@%
 \end{verbatim}
    
    NOTE: \verb!'':! (colon register) holds the most recently executed
    vim's command, e.g. if you just save the buffer with \verb!:w! command, then
    'W' is stored in this colon register.
    So, to execute the command again, use \verb!@:!. 
    
    NOTE: \verb!''#! (pound register) holds alternative file, check
  \begin{verbatim}
  :h alternate-file
  \end{verbatim}
    
    It's what vim uses to switch between files when you use \verb!Ctrl-^!

    \item {bf expression register}: \verb!"=! register 
    
    If, in insert mode, you type Ctrl-r =, you will see a \verb!=! sign in the
    command line. Then if you type \verb!2+2 <enter>!, 4 will be printed. This
    can be used to execute all sort of expressions, even calling external
    commands.
    
    \item {\bf search registers}: there are a number of search register, each
    map to the command search command and   hold any text that you typed for
    searching with \verb!/!, \verb!?!, \verb!*! or \verb!#! command.
    
    To replace that text (that you searched with \verb!/! command) with
    something else (e.g. mustache), you don't have to retype that text, just do
  \begin{verbatim}
  :%s/<Ctrl-r />/mustache/g
  \end{verbatim}
        
  \end{itemize}
  
  
  \item  {\bf unamed registers}(whose name is also double-quote): access using
  \verb!''''!
  
  By default: any text that you delete (with d, c, s or x) or yank (with y) will
  be placed there, and that's what vim uses to paste, when no explicit register
  is given. So, a simple \verb!p! command is equivalent to \verb!''''p!. 
\end{enumerate}

IN COMMAND MODE: To add text to a register, (do text selection first), then
specify the register name with yank, i.e. \verb!''ry!. To paste the content of the register 'r', then
\verb!''rp!. 

IN INSERT MODE: You can also access (i.e. paste the content) the registers in
insert/command mode with \verb!Ctrl-r! + register name, e.g. \verb!Ctrl-r r!. 


\subsection{clipboard}
\label{sec:clipboard}

Read Sect.\ref{sec:ICCCM} to understand better about clipboard system.
To store texts, vim/nvim uses the so-called {\bf registers}
(Sect.\ref{sec:register-vim-nvim}.

NOTE: Check if vim/nvim/gvim with \verb!+clipboard! feature.
If yes, then it support the system clipboard, i.e. PRIMARY selection, via the
\verb!*! register. vim's command \verb!:yank! (or yank selected text with yy; or
yank current work with yw; or yank current character with yc) and \verb!:paste!
operate on unnamed register, which by default corresponds to \verb!*! register.

 
\begin{verbatim}
:help clipboard
\end{verbatim}

NOTE: \verb!+clipboard! code has been removed in
neovim\footnote{\url{https://github.com/neovim/neovim/issues/483}}. It will be
reimplemented as a plugin at a later stage. Clipboard integration is a
platform-specific feature and can be achieved through external commands such
clip, xclip and pbcopy. So, there's no need to reinvent the wheel by
implementing these commands inside Neovim. Neovim try to look for the presence
of the following clipboard managers (Sect.\ref{sec:ICCCM}) in the order of
priority
\begin{verbatim}
g:clipboard
pbcopy/pbpaste (Mac OS X)
xclip
xsel (newer alternative to xclip)
lemonade (for SSH) https://github.com/pocke/lemonade
doitclient (for SSH) http://www.chiark.greenend.org.uk/~sgtatham/doit/
win32yank (Windows)
tmux (if $TMUX is set)
\end{verbatim}
NOTE: \verb!g:clipboard! enables you to configure a custom clipboard system
\begin{verbatim}

\end{verbatim}


Without such an external plugin,
Neovim has no direct connection to system clipboard, so we need to interact with
the registers explicitly
\begin{verbatim}
" <register> p   
              // p = paste
  with <register> can be a number 0 to 9, or "+" or "*"
\end{verbatim}



\subsection{nvim configuration files}
\label{sec:nvim-configuration}

Unlike vim/vi (Sect.\ref{sec:vim-configuration}), the location of configuration
file in neovim is quite different.
\begin{verbatim}
~/.config/nvim/init.vim
//Windows
~\AppData\Local\nvim\init.vim
\end{verbatim}
NeoVim supports the XDG Base Directory Specification, which among other things
attempts to standardize where user configuration files should live and their
order of precedences. 

\begin{verbatim}
:version

  //display 2 locations
      //system vimrc file
$VIM/sysinit.vim 
     //fall-back for $VIM
/usr/local/share/nvim
\end{verbatim}


nvim use \verb!XDG! environment's variables
(Sect.\ref{sec:XDG-environment-variables}), and \verb!VIMRUNTIME!
\begin{verbatim}
$VIMRUNTIME    //location for files distributed with Neovim

      //personal preferences
// default: XDG_CONFIG_HOME = $HOME/.config
      
      //personal preferences for nvim
$XDG_CONFIG_HOME/nvim
     // with the default resource file is
$XDG_CONFIG_HOME/nvim/init.vim

      //configuration files according to XDG
$XDG_CONFIG_DIRS   //default: /etc/xdg


      //nvim's plugins installed by user
$XDG_DATA_HOME/nvim/site

      //plugins installed by system administrator; but
      //are not part of neovim distribution
      // XDG_DATA_DIRS = /usr/local/share/:/usr/share/
      
//'after' directory
      // these below setting overrule or add to the 
      // distributed defaults or syste-wide settings
$XDG_DATA_DIRS[1]/nvim/site/after
$XDG_DATA_DIRS[2]/nvim/site/after
$XDG_CONFIG_HOME/nvim/after
\end{verbatim}

If you use SpaceVim, read Sect.\ref{sec:spacevim-configuration}

\section{Control nvim programmatically}


The primary way to control Nvim programmatically is the RPC API, which speaks
MessagePack-RPC ("msgpack-rpc"), a messaging protocol that uses the
MessagePack serialization format:

\begin{verbatim}
To make Nvim listen on a TCP/IP socket, set the
     |$NVIM_LISTEN_ADDRESS| environment variable before starting Nvim:
        NVIM_LISTEN_ADDRESS=127.0.0.1:6666 nvim
\end{verbatim}
\url{https://neovim.io/doc/user/msgpack_rpc.html}

A better way is to use the Python REPL with the `neovim` package, where API
functions can be called interactively:
\begin{verbatim}
    >>> from neovim import attach
    >>> nvim = attach('socket', path='[address]')
    >>> nvim.command('echo "hello world!"')
\end{verbatim}



\section{SpaceVim}
\label{sec:SpaceVim}

SpaceVim is a community-driven vim distribution that seeks to provide layer
feature, especially for neovim.

\url{http://liuchengxu.org/posts/space-vim/}

What make SpaceVim different are
\begin{enumerate}
  \item commands are grouped, and interact with the leader key
  (Sect.\ref{sec:SpaceVim-leader-key})
  
  \item
\end{enumerate}

\subsection{install + troubleshoot}
\label{sec:spacevim-install}

SpaceVim makes use of 
\begin{verbatim}
$HOME/.SpaceVim.d/init.vim      [default global config file]

$HOME/.SpaceVim/           [code from git]
          ./autoload/
          ./bundle/
          ./config/
          ./doc/
          ./docs/
          ./dotvim/
          ./filetype.vim
          ./ftplugin/
          ./ginit.vim
          ./init.vim
          ./mode/
          ./session/
          ./statusline.vimrc
          ./syntax/
          ./test/
          ./vimrc
\end{verbatim}

To install Spacevim, run the command (use curl (Sect.\ref{sec:curl})) 
\begin{verbatim}
curl -sLf https://spacevim.org/install.sh | bash -s -- --install neovim 

 //for both neovim and vim
curl -sLf https://spacevim.org/install.sh | bash

\end{verbatim}
which does

\begin{itemize}
  \item clone the git repository to \verb!$HOME/.SpaceVim!

    \item clone dein.vim - the package manager (Sect.\ref{sec:dein.vim}) to
  
\begin{verbatim}
$HOME/.cache/vimfiles/repos/github.com/Shougo/dein.vim
\end{verbatim}
  
  \item instal for vim:
  
  \begin{enumerate}
    \item back up \verb!~/.vimrc! file and \verb!~/.vim! folder
  
    \item make \verb!~/.vim! points to \verb!$HOME/.SpaceVim! folder
  
  \end{enumerate}
  
  \item install for neovim: \label{sec:nvim-SpaceVim}

  \begin{enumerate}
    
    \item backup \verb!~/.config/nvim/! folder
    
    \item make \verb!~/.config/nvim/! point to \verb!~/.SpaceVim/! folder
    
  \end{enumerate}
\end{itemize}

If you get error
\begin{verbatim}
fatal: Unable to find remote helper for 'https'
\end{verbatim}

It means git has some problem
\begin{verbatim}
//Ubuntu
sudo apt-get install libcurl14-openssl-dev

//CentOS, Fedora, RedHat
yum install curl-devel

//... finally 
sudo apt-get update
sudo apt-get install git
\end{verbatim}


\subsection{update}

Check version, if SpaceVim already installed
\begin{verbatim}
:SPVersion

:SPUpdate
\end{verbatim}

To update a particular plugin only
\begin{verbatim}
//we can use Tab to complete plugin names
:SPUpdate <plugin-name>
\end{verbatim}

\subsection{what packages come with SpaceVim}

To know what packages come with SpaceVim, run
\begin{verbatim}
:SPUpdate
\end{verbatim}
or
\begin{verbatim}
git -C ~/.SpaceVim pull .
\end{verbatim}

\subsection{(global) configuration files}
\label{sec:spacevim-configuration}

SpaceVim use the following file
\begin{verbatim}
~/.SpaceVim.d/init.vim
\end{verbatim}
as the default global config file. This file is automatically created the very
first time you use SpaceVim. You can later modify this file to add custom
configuration, i.e. \textcolor{red}{set all options and layers in this
file}. The file is loaded by an internal call within the \verb!$MYVIMRC! file
below.

REMEMBER that SpaceVim is not a real editor, it stills use either vim or nvim as
the back-end. Then, the first config file that is loaded is discussed in
Sect.\ref{sec:spacevim-install}. Example: If the back-end is nvim, at
installation of SpaceVim, the original nvim's configuration files
(Sect.\ref{sec:nvim-configuration}) is backed up and this file is symbolic
linked to SpaceVim's config file.
\begin{verbatim}
$HOME/.config/nvim ----->      $HOME/.SpaceVim/init.vim
\end{verbatim}
Let's parse what this file does
\begin{itemize}
  \item load \verb!<path-to-current-init.vim>/config/main.vim! file

\begin{verbatim}
execute 'source' fnamemodify(expand('<sfile>'), ':h').'/config/main.vim'
\end{verbatim}

Behind the scene, this \verb!./config/main.vim! will load many other files
\begin{verbatim}
.SpaceVim/config/functions.vim
\end{verbatim}
; and configure these folder into RTP (runtimepath) 
\begin{verbatim}
$HOME/.SpaceVim/SpaceVim.d/
\end{verbatim}



  \item \ldots 
\end{itemize}

What's new to SpaceVim is that you don't know where to add custom
configuration? You put your configuration in
\begin{verbatim}
$HOME/.SpaceVim/.SpaceVim.d/
\end{verbatim}
directory.

All default configuration of SpaceVim is in 
\begin{verbatim}
$HOME/.SpaceVim/autoload/SpaceVim.vim
\end{verbatim}
All the commands to modify SpaceVim can be referenced in this file

The file uses
\begin{enumerate} 
  \item dein : plugin manager
  \item vimfiler: file/folder browser
  
  \item colorscheme: if none is set : 'desert' 
  \item colorscheme: 'gruvbox'
\begin{verbatim}
g:spacevim_colorscheme = 'gruvbox'
\end{verbatim}
  \item background : default is 'dark'
\end{enumerate}



\subsection{ -- plugins in SpaceVim}

Plugins are organized into so-called layers in SpaceVim
(Sect.\ref{sec:nvim-layers}). Check for them in 
\begin{verbatim}
$HOME/.SpaceVim/autoload/SpaceVim/layers/
\end{verbatim}
Look into the file, search for the plugin name, and file the command to evoke
the plugin as the value passed to \verb!on_cmd!

However, if you want to add those any custom plugin, put them into 
\verb!g:space_vim_custom_plugins! as a list
\begin{verbatim}
let g:spacevim_custom_plugins = [
        \ ['jiangmiao/auto-pairs'],
        \ ]
        
let g:spacevim_custom_plugins = [
  \ ['fatih/vim-go',               { 'on_ft' : 'go'}],
  \ ['keith/swift.vim',            { 'on_ft' : 'swift'}],
  \ ['leafgarland/typescript-vim', { 'on_ft' : 'typescript' }],
  \ ]        
\end{verbatim}
Then, Spacevim will install these plugins automatically, then restart
vim/nvim.




\subsection{-- layers in vim}
\label{sec:nvim-layers}

Layer refers to a particular set of packages (from hundreds of ready-to-use
packages) nicely organised in configuration layers following a set of
conventions \footnote{\url{https://spacevim.org/development/}}.

The combination of these plugins enable the layer to provide a layout of window,
and subset of commands, syntax hight light, code checking etc. (for a given
coding environment). To access these layer-specific commands, we use a
\verb!<LocalLeader>! key (Sect.\ref{sec:SpaceVim-leader-key}).

For the list of built-in layers of SpaceVim, check 
\begin{verbatim}
~/.SpaceVim.d/init.vim
\end{verbatim}


LOADING LAYERS: default list
\begin{verbatim}
 18 call SpaceVim#layers#load('incsearch')
 17 call SpaceVim#layers#load('lang#c')
 16 call SpaceVim#layers#load('lang#c')
 15 call SpaceVim#layers#load('lang#elixir')
 14 call SpaceVim#layers#load('lang#go')
 13 call SpaceVim#layers#load('lang#haskell')
 12 call SpaceVim#layers#load('lang#java')
 11 call SpaceVim#layers#load('lang#javascript')
 10 call SpaceVim#layers#load('lang#lua')
  9 call SpaceVim#layers#load('lang#perl')
  8 call SpaceVim#layers#load('lang#php')
  7 call SpaceVim#layers#load('lang#python')
  6 call SpaceVim#layers#load('lang#rust')
  5 call SpaceVim#layers#load('lang#swig')
  4 call SpaceVim#layers#load('lang#tmux')
  3 call SpaceVim#layers#load('lang#vim')
  2 call SpaceVim#layers#load('lang#xml')
  1 call SpaceVim#layers#load('shell')                                                                                                                                           22  call SpaceVim#layers#load('tools#screensaver')
  
//Example: customize 'shell' layer
call SpaceVim#layers#load('shell'
        \ {
        \ 'default_position' : 'top',
        \ 'default_height' : 30,
        \ }
        \ )
\end{verbatim}

\subsection{local configuration file}

Remember that with vim, you can put \verb!.lvimrc!
to your root project folder, and vim will automatically load this file.

SpaceVim also enable that local configuration file if we have
\begin{verbatim}
$PROJECT_ROOT_DIR/.SpaceVim.d/init.vim
\end{verbatim}
NOTE: \verb!$PROJECT_ROOT_DIR/.SpaceVim.d/! is also added to runtime path. 

\subsection{documentation}

SpaceVim has \verb!~/.SpaceVim/init.vim! as the entry place (DONT modify it).

SpaceVim use \verb!~/.SpaceVim.d/init.vim! as the default global config file,
with all SpaceVim options and layers. The location 
\verb!~/.SpaceVim.d/! is also added to vim/nvim's runtimepath; so we can add our
script in this folder. 


For project's specific configuration, suppose you have a project on
\begin{verbatim}
$HOME/projects/MGS
\end{verbatim}
then you need to create
\begin{verbatim}
$HOME/projects/MGS/.SpaceVim.d/
\end{verbatim}
and adds project's specific configuration file here.


To change SpaceVim's default configuration, read
\begin{verbatim}
:h SpaceVim-config
\end{verbatim}

\subsection{leader key: SPACE}
\label{sec:SpaceVim-leader-key}


All interaction with SpaceVim is done through the leader key.
SpaceVim use space-key (SPC) as the default leader key
\begin{verbatim}
SPC b     =  for buffers

SPC x     =  for text
  SPC x d      = delete extra spaces around the cursor
\end{verbatim}

\begin{verbatim}
let g:spacevim_mapleader = "\<space>"
\end{verbatim}

SpaceVim also use \verb!<LocalLeader>! key, which is \verb!,! (comma) by default
in SpaceVim; it can be seen as the major-mode prefix in spacemacs.
 Depending on the layer that you're on (Sect.\ref{sec:nvim-layers})

\url{https://spacevim.org/documentation/#buffers-and-files}

\subsection{-- vimfiler vs. NERDtree}

DEFAULT vimfiler (Sect.\ref{sec:vimfiler}).

\begin{verbatim}
  //use lower-case always
let g:spacevim_filemanager = 'vimfiler' " 'nerdtree'
\end{verbatim}
To use NERDtree (Sect.\ref{sec:NERDtree})

\subsection{-- plugin bundle dir}

\begin{verbatim}
let g:spacevim_plugin_bundle_dir = '~/.cache/vimfiles'
\end{verbatim}

\subsection{-- indentation (default = 2)}

\begin{verbatim}
let g:spacevim_default_indent = 2
\end{verbatim}

\subsection{contribute to SpaceVim}

SpaceVim is based on conventions, mainly for naming functions, keybindings
definition and writing documentation.  Submit your contribution against the dev
branch.

\url{https://spacevim.org/development/}

\chapter{Other text/code editors}

\section{Atom}
\label{sec:atom}

\section{SublimeText}
\label{sec:SublimeText}

\section{TextMate}
\label{sec:TextMate}

\section{Eclipse}
\label{sec:Eclipse}

Check the Book Open-Source's Eclipse.

\section{Visual Studio}
\label{sec:VisualStudio}

\section{JetBrain (PyCharm, IntelliJ IDEA)}
\label{sec:JetBrain}

\section{Spacemacs}
\label{sec:spacemacs}

Spacemacs can be used naturally by both Emacs and Vim users -- you can even mix
the two editing styles. Switching easily between input styles makes Spacemacs a
great tool for pair-programming.


