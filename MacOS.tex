\chapter{MacOS}
\label{chap:MacOS}

\section{Mach kernel}
\label{sec:Mach-kernel}

Mach kernel was developed at CMU (from 1985 to 1994) to support O/S research,
primarily distributed and parallel computing.

Mach kernel is one of the earliest example of a {\bf microkernel}. However, not
all versions of Mach are microkernels. 

Mach 3.0 kernel is the last version, which is a true microkernel. 

\subsection{Mach 2.5 kernel}
\label{sec:Mach-2.5-kernel}

\subsection{Mach 3.0 kernel}
\label{sec:Mach-3.0-kernel}

\subsection{OSF/1 Operating System}
\label{sec:OSF/1}


Open Source Foundation (OSF) is not-for-profit organization founded in 1988.
NOTE: In February 1996 Open Software Foundation merged with X/Open to become The
Open Group.

OSF's UNIX reference implementation was known as "OSF/1" and was first released
in December 1991. NOTE: Initially, the plan was to use AIX O/S from IBM;
however, later OSF decided to implement a new UNIX-based O/S called OSF/1.

OSF/1 was created within a year, and incorporated technology from
\begin{enumerate}
  \item Mach 2.5 micro kernel from CMU (Sect.\ref{sec:Mach-kernel})
  
  \item the journaled file system as well as commands and libraries from IBM
  
  \item secure core components from SecureWare
  
  \item the networking stack from BSD
  
  \item a new virtual memory management system invented at OSF
\end{enumerate}

Development of OSF/1 was stopped in 1996. The only major UNIX system vendor that
use OSF/1 was Digital (or later rebranded as Digital UNIX, then Tru64 UNIX
(after Compaq acquired Digital)).

Parts from OSF/1 are used by different UNIX vendors.
\begin{enumerate}

  \item Motif 
  
  
  \item Distributed Computing Environment (DCE)
  
  \item DME, the Distributed Management Environment 
  
  \item ANDF: Architecturally Neutral Distribution Format
  
  \item ODE: Open Development Environment which is
   a flexible development, build and source control environment
   
   \item TET: Test Environment Toolkit - an open framework for building and
   executing automated test cases;
   
    ODE and TET were made available as open source.
    
   \item the operating system OSF/1 MK, i.e. OSFMK, based on the Mach3.0 microkernel.

OSF created OSFMK kernel which was a commercial version of the
Mach kernel for use in OSF/1.   
   
\end{enumerate}
The main technologies developed by OSF is ODE, TET and OSFMK. 

\subsection{OSFMK kernel}
\label{sec:OSFMK}

OSFMK kernel has 
\begin{enumerate}
  
  \item applicable code from the University Of Utah Mach 4 kernel (such as the
"Shuttles" modification used to speed up message passing.) and 

  \item applicable code from the many Mach 3.0 variants that sprouted off from
  the original Carnegie Mellon University Mach 3.0 kernel, 
  
  \item   consists of improvements made by the OSF such as built in collocation
  capability, realtime improvements, and rewriting of the IPC RPC component for
  better speed use among other things.

\end{enumerate}

OSFMK 7.3 includes applicable code from the University of Utah Mach 4 kernel and
applicable code from the many Mach 3.0 variants that sprouted off from the
original Carnegie Mellon University Mach 3.0 microkernel.


\section{XNU kernel}
\label{sec:XNU-MacOS}

XNU (XNU is an abbreviation of X is Not Unix) was originally developped by NeXT
for the NeXTSTEP operating system, XNU was a hybrid kernel of both monolithic
kernels and microkernels
\begin{enumerate}

  \item OSFMK 7.3 kernel: combining version 2.5 of the Mach kernel developed at Carnegie Mellon
  University with 

The basis of the XNU kernel is a heavily modified (hybrid) OSFMK 7.3 kernel
(Sect.\ref{sec:OSFMK}) which has I/O kit and drivers. 


As an OSFMK 7.3 kernel, XNU kernel could run several
operating systems  in parallel, as the core of the O/S is just a separated
process above the Mach
  core.
  This often reduces performance because of time consuming kernel/user mode
  context switches and overhead stemming from mapping or copying messages
  between the address spaces of the kernel and that of the service daemons.
   
  \item components from 4.3BSD: FileSystem, Networking, NKE
  
\end{enumerate}
and an Objective-C API for writing drivers called Driver Kit.



After Apple purchased NeXT in 1997, XNU become the  kernel for macOS O/S developed at Apple Inc.


\section{Darwin Operating System}
\label{sec:Darwin-OS}



\section{Versions}




\section{TrackPad}

\subsection{Apple's TrackPad}

\begin{itemize}
  \item {\bf Click-n-Drag}: 
  
  Press the physical button, then drag. While the button is depressed, you can
  reposition your "dragging" finger without letting go of what you're dragging.
  
  \item {\bf One-finger Tap-n-Drag}:
  
   With Dragging enabled, tap the trackpad twice and start dragging on the
   second tap (instead of lifting your finger from the trackpad).
   When you lift the finger after dragging, there is a delay for the drag
   actually ends (i.e. until the button is completely depressed).
   To end a drag immediately (without the delay) you can tap the trackpad again.
   
   \item {\bf Three-finger Drag}:
   
   Tap the trackpad with three fingers and drag all three fingers. This has the
   same delay as the one-finger drag, so you can reposition your fingers and
   continue dragging. Again, a single tap will end the drag without the delay.
   
   \item {\bf Drag Lock}:
   In Snow Leopard, the Dragging and Drag Lock settings are in 
   \verb!System Preferences > Trackpad!, but in Lion they were moved to 
   \verb!System Preferences > Universal Access > Mouse & Trackpad > Trackpad!
   \verb!Options!.
   
   
   When enabled, the drag does not end after lifting your finger(s) from the
   trackpad. Rather you have to tap/click the trackpad to end the drag.
   
   \item {\bf Three-Finger Drag with Inertia}
   
   This is like the three-finger drag, but instead of moving all three fingers,
   you only move one finger, keeping the other two in place on the trackpad.
   When you do this, you can use a flick gesture with the finger that's moving
   and whatever you're dragging will continue to move after you lift that
   finger, gradually slowing down. It works very much like scrolling with
   inertia.
   If you wanted to, you could actually keep one finger in place on the trackpad
   and use two fingers for dragging, but that's a bit awkward.
   
   As long as you keep your other two fingers in place on the trackpad, you can
   lift your third finger without letting go of what you're dragging.
   
\end{itemize}
\url{http://apple.stackexchange.com/questions/42429/how-to-properly-use-drag-and-drop-with-macbook-pro-on-os-x-10-7}



\section{Hot Corner (Shortcut using Mouse)}
\label{sec:Hot_Corner}

There is no need to use Keyboard shortcut in Apple's Mac OS X.
You can use the Mouse, by moving the cursor to a designated position at one of
the four corners.
This lets you determine what actions will trigger when you move your cursor into
a particular corner of your Mac's display. This is known as {\bf Hot Corner}.

Hot Corners on the Mac have been around for-ever.
Hot Corners are basically shortcuts for your mouse. When set up, moving the
cursor to a corner of the screen can trigger any number of actions:
\begin{verbatim}
Apple menu > System Preferences and then click Mission Control
\end{verbatim}
where you can define which action at each of the 4 corners. Example of actions
\begin{itemize}
  \item minimize all windows to show the desktop
  \item start screensaver
  \item run launchpad
  \item put display to sleep
  \item run mission control - Sect.\ref{sec:Mission_Control}
\end{itemize}

You can choose how to enable the Hot Corner
\begin{itemize}
  \item just move the cursor 
  \item move the cursor + press some modifiers (Sect.\ref{sec:modifier_key})
\end{itemize}

\url{http://thesweetsetup.com/quick-tip-enable-hot-corners-os-x/}

\url{http://www.macobserver.com/tmo/article/how-to-astutely-use-os-x-hot-corners-without-consternation}

\section{Modifier Key (Control, Option, Command, or Shift)}
\label{sec:modifier_key}


\section{Mission Control (Expos\'{e})}
\label{sec:Mission_Control}

Mission Control is a feature of Mac OS X which allows a user to quickly
rearrange all the open Windows on desktop to easily locate the one he is
interested to work on. Thus, Mission Control allows you to quickly view which
app is openning, as well as listing all the virtual screens (including apps
running at fullscreen). From Mission Control, you can quickly switch to any app
or virtual screen.

To open Mission Control, choose either
\begin{itemize}
  \item slight up on the TrackPad using 3 fingers
  \item define a shortcut using Hot Corner - Sect.\ref{sec:Hot_Corner}
\end{itemize}


Mission Control in Windows:
\begin{itemize}
  \item MC: \url{http://sourceforge.net/projects/mcsoft/}
  
  \item SmallWindow: \url{http://smallwindows.sourceforge.net/}
  
You can assign short-cut keys as well.  
\end{itemize}
\url{http://www.guidingtech.com/17495/get-mac-like-mission-control-windows/}

NOTE: Mission Control uses an algorithm from
Graph Drawing, or a similar family to derive it's window layout functions.
Just have a look at the source code for the Graphviz project. It has algorithms
for laying out graph nodes much the same way Expose would.

\section{Mouseless with Mac OS X}

\url{https://kozikow.wordpress.com/2013/10/31/going-mouseless-on-mac/}

\section{ssh}
\label{sec:sshfs-MacOS}

sshfs - Sect.\ref{sec:sshfs}:

\url{https://github.com/osxfuse/osxfuse}

\begin{verbatim}
James R. Kozloski: 
4:12:29 PM: I found the solution: If you call sshfs from command line just add
-o defer_permissions. If you use Macfusion put -o defer_permissions in "SSH
Advanced > Extra Options".  

4:12:40 PM: it works now 

4:13:31 PM: here are the packages to make it work: https://osxfuse.github.io/ 

4:13:39 PM: both must be installed 
\end{verbatim}

