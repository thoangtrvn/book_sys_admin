\chapter{Terminal Emulator}
\label{chap:terminal-emulator}


\section{What is a terminal?}
\label{sec:terminal}

A terminal is a device that allows you to communicate with a computer, i.e. the
CPU. Generally, a terminal is a combination of keyboard and display screen. 

Terminals are sometimes divided into three classes based on how much processing
power they contain:
\begin{enumerate}
  \item  intelligent terminal: a stand-alone device that contains main memory
  and a CPU.

  \item  smart terminal: contains some processing power, but not as much as an
intelligent terminal. 
   
  \item dumb terminal: has no processing capabilities. It relies entirely on the
computer's processor. 
\end{enumerate}


\subsection{VT100 (1978: DEC)}

\begin{mdframed}

DEC's first successful video terminal was the VT50, introduced in 1974 and
quickly replaced by the VT52 in 1975.

VT52 has a text-display with 80 columns, 24 rows; bidirectional scrolling, and a
custom control language that allowed the cursor to be moved about the screen.

VT50, VT52, and VT100 communicated with its host system over serial lines at a
speed selectable between 50 and 19,200 bit/s.

To support adding command-code into the sequence of bytes, VT50 and VT52 control
system is VT50/52's proprietary language.

\end{mdframed}

Video Terminal 100 (introduced by DEC in 1978) was the first terminal (with a
keyboard and screen) to use a general-purpose processor for interpreting the
newly published (1977) ANSIcontrol codes (ANSI X3.64).

VT100 supports 
\begin{itemize}
  \item a mode with 132 columns
  
  \item new graphics capabilities: including blinking, bolding, reverse video,
  and underlining.
  
  \item add support for box-drawing character set, i.e. those containing various
  pseudographics that allowed the drawing of on-screen forms.
  
  \item major change: VT100 were based on the emerging ANSI X3.64 standard for
  command codes

To support cursor control, and other tasks, VT100 accepts ANSI escape code,
within which we can embed code representing colors, for example -
Sect.\ref{sec:color-terminal}.

  
  \item VT100 used the new Intel 8080 as its internal driver to handle 'command
  code' passing from the terminal by user.
  
  \item  VT100 provided backwards compatibility for VT52 users, with support for
  the VT52 control sequences
  
  \item  Advanced Video Option or AVO, allowed the terminal to support a full 24
  lines of text in 132 column mode.
  
  VT102 came standard with the AVO and serial printer port options
  pre-installed.
\end{itemize}


VT102, was extremely successful in the market, and made DEC the leading terminal
vendor.  VT102 came standard with the AVO and serial printer port options
pre-installed. VT105 contained a simple graphics subsystem known as waveform
graphics which was mostly compatible with same system in the earlier VT55. This
system allowed two mathematical functions to be drawn to the screen on top of
the normal text display, allowing text and graphics to be mixed to produce
charts and similar output.

VT125 added an implementation of the byte-efficient {\bf Remote Graphic
Instruction Set}, ReGIS, which used custom ANSI codes to send the graphics
commands to the terminal, rather than requiring the terminal to be set to a
separate graphics mode like the VT105.

Quickly, the vt100 become popular, and the ANSI control codes embodied
in the vt100 became a de facto standard. Eventually, IBM adopted them for its line of
PCs.  Today, most terminal-emulation programs, i.e. the program that provides a
text-based window interface [from this we can launch programs], commonly use the
vt100 setting.

\subsection{VT220 (1983: DEC)}


VT200 series was just as successful as VT100, providing more functionality in a
much smaller unit with a much smaller and lighter keyboard, and a much faster
microprocessor.

VT220's display is a CRT that used white, green, or amber phosphors.
LK201 keyboard supplied with the VT220 was one of the first full-length,
low-profile keyboards available.

VT240 added monochrome ReGIS vector graphics support to the base model, while
the VT241 did the same in color. 


VT200s also support more character sets, i.e. better internationalization
support. 
\begin{itemize}
  
  \item original: when operating on a typical 7-bit link, only 128 were
  available, and only 96 of these produced display output as the rest were
  control characters
  
  This was not enough characters to handle all European languages. 
  Most terminals solved this by shipping multiple complete character sets in
  ROM, but there was a cost in doing so.
  
  \item 
  now: using the National Replacement Character Set (NRCS) concept.
  
  NRCS allowed individual characters glyphs in the base set of 96 7-bit
  characters to be swapped out

   it also allowed for another 96 characters in the Dynamically Redefined
   Character Set (DRCS), which could be downloaded from the host computer.
  
  This eliminated the need to ship 14 versions of the terminal, or to include 14
  different 7-bit character sets in ROM.
\end{itemize}

To draw on the display, data for the glyphs was sent by encoding a set of six
vertical pixels (i.e. Sixels) into a single character code, and then sending
many of these Sixels to the terminal, which decoded them into the character set
memory.


\section{\$TERM environment variable, termcap, terminfo}
\label{sec:TERM-variable}
\label{sec:TERM-env-var}
\label{sec:env-var-TERM}


\textcolor{red}{What is \$TERM for?}  The \$TERM variable is for use by
applications to take advantage of capabilities of that terminal
(Sect.\ref{sec:terminal}), e.g. VT100, VT220.
\begin{itemize}
  \item early values (it reflects the physical terminal): vt100, vt220
  
  
  \item nowdays values (it reflects the name of the (common) terminal emulator,
  whose features are supported by the current terminal emulator):
    
\end{itemize}

The terminal type specified in the TERM setting must correspond to an entry in
the termcap file or terminfo directory (Sect.\ref{sec:terminfo}).

Once you launch a terminal emulator, it is supposed to set \$TERM, and this
should not be changed from within the shell by user. After TERM is set, you can
verify it by invoking
\begin{verbatim}
msgcat --color=test

//or

infocmp
\end{verbatim}
seeing whether the output looks like a reasonable color map. 
You can get a detailed list of these cababilities by using the 'infocmp'
command, using 'man 5 terminfo' as a reference.
{\tiny
\begin{verbatim}
#	Reconstructed via infocmp from file: /packages/ncurses/6.0//share/terminfo/x/xterm
xterm|xterm terminal emulator (X Window System),
	am, bce, km, mc5i, mir, msgr, npc, xenl,
	colors#8, cols#80, it#8, lines#24, pairs#64,
	acsc=``aaffggiijjkkllmmnnooppqqrrssttuuvvwwxxyyzz{{||}}~~,
	bel=^G, blink=\E[5m, bold=\E[1m, cbt=\E[Z, civis=\E[?25l,
	clear=\E[H\E[2J, cnorm=\E[?12l\E[?25h, cr=^M,
	csr=\E[%i%p1%d;%p2%dr, cub=\E[%p1%dD, cub1=^H,
	cud=\E[%p1%dB, cud1=^J, cuf=\E[%p1%dC, cuf1=\E[C,
	cup=\E[%i%p1%d;%p2%dH, cuu=\E[%p1%dA, cuu1=\E[A,
	cvvis=\E[?12;25h, dch=\E[%p1%dP, dch1=\E[P, dim=\E[2m,
	dl=\E[%p1%dM, dl1=\E[M, ech=\E[%p1%dX, ed=\E[J, el=\E[K,
	el1=\E[1K, flash=\E[?5h$<100/>\E[?5l, home=\E[H,
	hpa=\E[%i%p1%dG, ht=^I, hts=\EH, ich=\E[%p1%d@,
	il=\E[%p1%dL, il1=\E[L, ind=^J, indn=\E[%p1%dS,
	invis=\E[8m, is2=\E[!p\E[?3;4l\E[4l\E>, kDC=\E[3;2~,
	kEND=\E[1;2F, kHOM=\E[1;2H, kIC=\E[2;2~, kLFT=\E[1;2D,
	kNXT=\E[6;2~, kPRV=\E[5;2~, kRIT=\E[1;2C, kb2=\EOE, kbs=^H,
	kcbt=\E[Z, kcub1=\EOD, kcud1=\EOB, kcuf1=\EOC, kcuu1=\EOA,
	kdch1=\E[3~, kend=\EOF, kent=\EOM, kf1=\EOP, kf10=\E[21~,
	kf11=\E[23~, kf12=\E[24~, kf13=\E[1;2P, kf14=\E[1;2Q,
	kf15=\E[1;2R, kf16=\E[1;2S, kf17=\E[15;2~, kf18=\E[17;2~,
	kf19=\E[18;2~, kf2=\EOQ, kf20=\E[19;2~, kf21=\E[20;2~,
	kf22=\E[21;2~, kf23=\E[23;2~, kf24=\E[24;2~,
	kf25=\E[1;5P, kf26=\E[1;5Q, kf27=\E[1;5R, kf28=\E[1;5S,
	kf29=\E[15;5~, kf3=\EOR, kf30=\E[17;5~, kf31=\E[18;5~,
	kf32=\E[19;5~, kf33=\E[20;5~, kf34=\E[21;5~,
	kf35=\E[23;5~, kf36=\E[24;5~, kf37=\E[1;6P, kf38=\E[1;6Q,
	kf39=\E[1;6R, kf4=\EOS, kf40=\E[1;6S, kf41=\E[15;6~,
	kf42=\E[17;6~, kf43=\E[18;6~, kf44=\E[19;6~,
	kf45=\E[20;6~, kf46=\E[21;6~, kf47=\E[23;6~,
	kf48=\E[24;6~, kf49=\E[1;3P, kf5=\E[15~, kf50=\E[1;3Q,
	kf51=\E[1;3R, kf52=\E[1;3S, kf53=\E[15;3~, kf54=\E[17;3~,
	kf55=\E[18;3~, kf56=\E[19;3~, kf57=\E[20;3~,
	kf58=\E[21;3~, kf59=\E[23;3~, kf6=\E[17~, kf60=\E[24;3~,
	kf61=\E[1;4P, kf62=\E[1;4Q, kf63=\E[1;4R, kf7=\E[18~,
	kf8=\E[19~, kf9=\E[20~, khome=\EOH, kich1=\E[2~,
	kind=\E[1;2B, kmous=\E[M, knp=\E[6~, kpp=\E[5~,
	kri=\E[1;2A, mc0=\E[i, mc4=\E[4i, mc5=\E[5i, meml=\El,
	memu=\Em, op=\E[39;49m, rc=\E8, rev=\E[7m, ri=\EM,
	rin=\E[%p1%dT, ritm=\E[23m, rmacs=\E(B, rmam=\E[?7l,
	rmcup=\E[?1049l, rmir=\E[4l, rmkx=\E[?1l\E>,
	rmm=\E[?1034l, rmso=\E[27m, rmul=\E[24m, rs1=\Ec,
	rs2=\E[!p\E[?3;4l\E[4l\E>, sc=\E7, setab=\E[4%p1%dm,
	setaf=\E[3%p1%dm,
	setb=\E[4%?%p1%{1}%=%t4%e%p1%{3}%=%t6%e%p1%{4}%=%t1%e%p1%{6}%=%t3%e%p1%d%;m,
	setf=\E[3%?%p1%{1}%=%t4%e%p1%{3}%=%t6%e%p1%{4}%=%t1%e%p1%{6}%=%t3%e%p1%d%;m,
	sgr=%?%p9%t\E(0%e\E(B%;\E[0%?%p6%t;1%;%?%p5%t;2%;%?%p2%t;4%;%?%p1%p3%|%t;7%;%?%p4%t;5%;%?%p7%t;8%;m,
	sgr0=\E(B\E[m, sitm=\E[3m, smacs=\E(0, smam=\E[?7h,
	smcup=\E[?1049h, smir=\E[4h, smkx=\E[?1h\E=,
	smm=\E[?1034h, smso=\E[7m, smul=\E[4m, tbc=\E[3g,
	u6=\E[%i%d;%dR, u7=\E[6n, u8=\E[?1;2c, u9=\E[c,
	vpa=\E[%i%p1%dd,
\end{verbatim}
}
For example, if a program want's to display colored text, it must first find out
if the terminal you're using supports colored text, and then if it does, how to
do colored text, i.e. what escaped sequence associated with a color.  


The way this works is that the system keeps a library of names of known
terminals and their capabilities. On most systems this is in /usr/share/terminfo
(there's also termcap, but it's legacy not used much any more).

\begin{mdframed}

\label{sec:termcap}

Applications interact with terminals by writing escape sequences  -  sequences
of characters that include nonprintable characters and have effects such as
moving the cursor, erasing part of the screen, changing the current color, etc.
In the old days, different brands of physical terminals
(Sect.\ref{sec:terminal}) had different sets of escape sequences. Therefore the
operating system maintains a database of terminal types and their
characteristics.
\begin{itemize}
  \item  The traditional database is \verb!termcap! ("TERMinal CAPabilities")

  \item Many modern systems and applications have switched to using
  \verb!terminfo! database (Sect.\ref{sec:terminfo}). 
  
  Nowadays, most terminals use a standard set of escape sequences with a
few common sets of extensions, so you won't see many different values of TERM;
even though there are way many more virtual terminals. Most GUI terminal
emulators are compatible with xterm, the traditional X terminal (which is still
used and maintained), so these virtual terminals declared themself with
\verb!TERM=xterm-256colors! or \verb!TERM=xterm!.
\end{itemize}

\end{mdframed}  

\subsection{where is TERM defined in the first place}

It isn't set "globally". It's set by whichever program you use to login. 
At the console, the program is "mingetty" or agetty, and it sets the TERM to
linux.  Both agetty and mingetty pass control to "login"

On virtual terminals and real terminals, the TERM environment variable is set by
the program that chains to login, and is inherited all of the way along to the
interactive shell that executes once one has logged on. Where, precisely, this
happens varies from system to system, and according to the kind of terminal.

Conventionally the \verb!getty! program is invoked with an argument that
specifies the terminal type, or is passed the TERM program from a service
manager's service configuration data.
\begin{itemize}
  
  \item The "mingetty" program is expected for use with consoles on Linux, so
  this is what you get. However, you can change this to the more sophisticated
  agetty command, with which you can set the TERM variable.
  
  
  \item getty or agetty: opens a \verb!tty! port (Sect.\ref{sec:tty}), prompts
  for a logi name and invokes the \verb!/bin/login! command
  
  The program (getty) is normally invoked by \verb!init! (at boot time -
  Sect.\ref{sec:init}).
  
  \verb!agetty! has non-standard features; and does NOT use
  \verb!/etc/gettydefs! (System V) or \verb!/etc/gettytab! (SunOS 4) files.   
  
  \item On systems using mingetty or vc-get-tty (from the nosh package) the
  program "knows" that it can only be talking to a virtual terminal, and they
  hardwire the "known" virtual terminal types appropriate to the operating
  system that the program was compiled for.
  

\end{itemize}

On systems (e.g. System V (Sect.\ref{sec:System-V})) that use \verb!init!
(Sect.\ref{sec:init}): \verb!init! reads /etc/inittab entries
(Sect.\ref{sec:inittab}), which will read something along the lines of
  
\begin{verbatim}
S0:3:respawn:/sbin/agetty ttyS0 9600 vt100-nav
\end{verbatim}
The last argument to agetty in that line, vt100-nav, is the terminal type set
for /dev/ttyS0. 

Example: to let inittab know you want console \verb!#1! to use vt100, you can do
this:
\begin{verbatim}
1:12345:respawn:/sbin/agetty 57600 tty1 vt100
\end{verbatim}
  
On systems (e.g. Ubuntu 16.04 (Sect.\ref{sec:Debian})) that use \verb!systemd!
(Sect.\ref{sec:systemd}): 
read the file \verb!/usr/lib/systemd/system/serial-getty@.service! unit file
(\verb!/lib/systemd/system/serial-getty@.service! on un-merged systems), which
reads 
\begin{verbatim}
Environment=TERM=vt100
\end{verbatim}
which setting the TERM variable in the environment passed to agetty. 
This service unit file is where to change the terminal type for real terminals
on such systems. Note that it applies to all real terminals that employ this
service unit template. (To change it for only individual terminals, one has to
manually instantiate the template.)  


On BSD's init: \verb!init! takes the terminal type from the third field of each
terminal's entry in the \verb!/etc/ttys! database, and sets TERM from that in
the environment that it executes getty with. So /etc/ttys is where one changes the
terminal type for real terminals on the BSDs.

In the case of RHEL4, login uses PAM to control security access. But PAM can
also be used to set environment variables. You can set system-wide defaults (for
ALL logins) in \verb!/etc/security/pam_env.conf!


Finally, you can use basic shell scripting to modify the environment variables.
Simply change /etc/profile (or create a new file ending with .sh in
/etc/profile.d/) to globally set the environment variable "TERM".
\begin{verbatim}
TERM=vt100; export TERM
\end{verbatim}
Once your shell (bash) starts, /etc/profile will be again read and will pick up
the new TERM settings. A word of caution: if you do change /etc/profile, you
should also change csh.cshrc (or a file in /etc/profile.d ending with .csh). But
this will use the "csh" syntax which is 
\begin{verbatim}
setenv TERM vt100
\end{verbatim}

\subsection{tty}
\label{sec:tty}

\begin{verbatim}
terminal = tty = text input/output environment
\end{verbatim}

Before you have a terminal (with a keyboard/screen), you use a typewritter.

With the presence of a computer, using Unix O/S, console, terminal and tty are
closely related. Originally, they meant a piece of equipment through which you
could interact with a computer: in the early days of unix, that meant a
teleprinter-style device resembling a typewriter, sometimes called a
teletypewriter, or 'tty' in shorthand. T

The name 'terminal' came from the electronic point of view, and the name
'console' from the furniture point of view. Very early in unix history,
electronic keyboards and displays became the norm for terminals.


Early user terminals connected to computers were electromechanical teleprinters
or teletypewriters (TeleTYpewriter, TTY), and since then TTY has continued to be
used as the name for the text-only console although now this text-only console
is a virtual console not a physical console.  
\url{https://unix.stackexchange.com/questions/4126/what-is-the-exact-difference-between-a-terminal-a-shell-a-tty-and-a-con}

In unix terminology, a tty is a particular kind of device file which implements
a number of additional commands (ioctls) beyond read and write. In its most
common meaning, terminal is synonymous with tty. Some ttys are provided by the
kernel on behalf of a hardware device, for example with the input coming from
the keyboard and the output going to a text mode screen, or with the input and
output transmitted over a serial line. Other ttys, sometimes called pseudo-ttys,
are provided (through a thin kernel layer) by programs called terminal
emulators, such as Xterm (running in the X Window System), Screen (which
provides a layer of isolation between a program and another terminal), Ssh
(which connects a terminal on one machine with programs on another machine),
Expect (for scripting terminal interactions), etc.


A console is generally a terminal in the physical sense that is by some
definition the primary terminal directly connected to a machine. The console
appears to the operating system as a (kernel-implemented) tty.   
 On some systems, such as Linux and FreeBSD, the console appears as several ttys
(special key combinations switch between these ttys). Also, 
 just to confuse matters, the name given to each particular tty can be
'console', 'virtual console', 'virtual terminal', and other variations.

There are 6 virtual consoles in Ubuntu accessed by the keyboard shortcuts
Ctrl+Alt+F1 to Ctrl+Alt+F6. You can move away from a virtual console (move the
console to the background) by using the keyboard shortcut Ctrl+Alt+F7 (switch
to login screen GUI).

In Ubuntu 17.10 the login screen now uses virtual terminal 1. In Ubuntu 17.10
press Ctrl+Alt+F3 up to Ctrl+Alt+F6 for accessing a virtual console and press
Ctrl+Alt+F2 to go back to the desktop environment.  

Because most Linux distributions are open source, the number of virtual ttys can
be changed in some Linux distributions as follows. 

\begin{verbatim}
$ tty 
/dev/pts/20
\end{verbatim}

The first one returns you also the pty of the process running grep as you can
notice:
\begin{verbatim}
$ ps ax | grep $$
28295 pts/20   Ss     0:00 /bin/bash
29786 pts/20   S+     0:00 grep --color=auto 28295

$ ps ax | grep $$ | grep -v grep | awk '{ print $2 }'
pts/20
\end{verbatim}
NOTE: \verb!$$! returns the process-ID of the current shell.

\subsection{How a program interact with a terminal}

So lets say you have a program that wants to display red text. It basically
makes a call to the terminfo library that says "give me the sequence of bytes I
have to send for red text for the xterm terminal". Then it just takes those
bytes and prints them out.
You can try this yourself by doing tput setf 4; echo hi. This will get the setf
terminfo capability and pass it a parameter of 4, which is the color you want.

What happens if a hiny new terminal emulator that was just released, and the
system's terminfo library doesn't have a definition for it yet. When your
application goes to look up how to do something, it will fail because the
terminal isn't known. The way your terminal gets around this is by lying about
who it is. So your gnome terminal is saying "I'm xterm" (Sect.\ref{sec:xterm}).

The downside to lying about your terminal type is that the terminal might
actually support a lot more than xterm does (for example, many new terminals
support 256 colors, while older terminals only supported 16). So you have a
tradeoff, get more features, or have more compatibility. Most terminals will opt
for more compatibility, and thus choose to advertise themselves as xterm.

\url{https://unix.stackexchange.com/questions/93376/which-terminal-type-am-i-using}

Terminal emulators that differ from xterm may add their own entry to the
terminal database under their own name. However this doesn't mesh well with
remote shells.
A program running on machine A but which is displaying on machine B, for example
because it was launched through ssh from B to A, needs to query the terminal
database on machine A. Remote login methods such as ssh carry over the TERM
environment variable, but this is helpful only when B's terminal database also
has an entry for the value user on A. Therefore many terminal emulators stick to
TERM=xterm which is pretty much universally known.


Modern terminals differentiated themself by and large does not come from the way
applications interact with them, but by the way the terminals interact with the
user and fit in their environment. For example, Gnome Terminal looks good on
Gnome and provides tabs and other niceties; Konsole looks good on KDE and
provides tabs and other niceties; urxvt has a small memory requirement; Console2
runs on Windows; screen and tmux provide sessions that can be attached to
different parent terminals; and so on. Since none of these features make a
difference to applications running in the terminal, most terminal emulators use
TERM=xterm.

To find out what terminal a shell is running it, look at the parent of the shell:
\begin{verbatim}
ps -p$PPID
\end{verbatim}
and if the shell is a remote shell, then the terminal is carried over from the
shell of the local session, and the result is 
\begin{verbatim}
PID         TTY       TIME   CMD
...         ?         ....    sshd
\end{verbatim}



\subsection{TERM inside tmux and screen}
\label{sec:TERM-in-tmux-screen}
\label{sec:tmux-TERM}
\label{sec:screen-TERM}

\url{https://github.com/tmux/tmux/blob/2.3/FAQ#L4}

We can do this inside \verb!~/.bashrc! file

{\tiny
\begin{verbatim}
  case $TERM in
    xterm)  TERM=xterm-256color ;; # for xterm, PuTTY, MinTTY
    rxvt)   TERM=rxvt-256color  ;; # for rxvt
    cygwin) ;; # DOS terminal - fewer than 16 colors are supported
  esac
\end{verbatim}
}


\begin{verbatim}
infocmp -d xterm-256color tmux-256color|head -n 8
\end{verbatim}
\url{https://github.com/neovim/neovim/issues/6845}

\url{https://bugs.launchpad.net/byobu/+bug/1544983}

As of tmux version 2.1, you can tell tmux to pretend to be a screen terminal.
For tmux to work correctly, this must be set to screen, tmux or a derivative of
them. Inside tmux, TERM must be 'tmux', "screen" or similar (such as
"screen-256color", 'tmux-256color'). The setting is defined in
\verb!~/.tmux.conf! file (for tmux)
\begin{verbatim}
set -g terminal-default "screen-256color"
\end{verbatim}


HOWEVER, vim is broken without \verb!-bce!, and thus we cannot use
\begin{verbatim}
# - screen-256color: vim broken without -bce
set -g default-terminal "screen"
\end{verbatim}
BUT; tmux does NOT support bce 
\begin{verbatim}
# - screen-bce, screen-256color-bce: tmux does not support bce
\end{verbatim}
\url{https://bugs.launchpad.net/byobu/+bug/1544983}

To overcome, you can set inside \verb!~/.byobu/.tmux.conf! file
\begin{verbatim}
set -g default-terminal "screen-256color"
\end{verbatim}


\section{What is terminal emulator}
\label{sec:terminal-emulators}

Nowadays, you don't see actual terminals all that often anymore
(Sect.\ref{sec:terminal}), but for instance the Linux text consoles on (CTRL +)
ALT + F1 through F6 are considered terminals.


Terminal type or emulation  specifies how your computer and the host computer to
which you are connected exchange information. 

Terminal emulators are programs that presents to user a text-based window (like
the screen of a terminal (Sect.\ref{sec:terminal})), so the user interacts as
like with a physical terminal. 


The following terminal emulators put a window up and let you interact with
the shell: 
\begin{enumerate}
  \item xterm - Sect.\ref{sec:xterm}
  
  XTerm also provides features that are in neither VT100 nor VT220, which are
  used by other programs as "xterm emulation".

  
  \item rxvt - Sect.\ref{sec:rxvt}
  
  \item kvt, 
  
  \item nxterm, and 
  
  \item eterm.
  
  \item iTerm2 - Sect.\ref{sec:iTerm2}
  
  \item Putty - Sect.\ref{sec:putty}
  
  \item tmux or screen - Sect.\ref{sec:tmux}
  
  \verb!tmux! and \verb!screen! are not only a terminal emulator but also a
  terminal multiplexer (Sect.\ref{sec:terminal_multiplexer}) since each pane
  must behave as a separate terminal; its emulation is not exactly the same as
  that of xterm or gnome-terminal.

  \verb!tmux! and/or \verb!screen! is typically run inside a terminal emulator.
  Within \verb!tmux!, then it provides  screen functions like a window manager
  to run multiple virtual terminals.
  
  Advanced terminal emulator like \verb!gnome-terminal! also provide panes or
  tabs (each pane to emulate 1 terminal); but \verb!tmux! keeps these tabs
  persistent and can be loaded from any sessions.
  
  A terminal emulator like \verb!xterm! is a terminal emulator for systems
  running the x-window system and a graphical user interface. It provides only a
  single terminal, typically running a single process (an interactive shell by
  default - Sect.\ref{sec:shell}).
  
  Konsole and similarly Gnome Terminal are slightly more advanced terminal
  emulators for the GUI.

  \item konsole, 
  
  \item gnome-terminal, 
    
\end{enumerate}

The terminal emulator can host any shell program (Sect.\ref{sec:shell}).
The terminal emulator receives the keyboard input, especially control characters
like TAB, CTRL, and convert that into a proper byte-sequence before sending to
the shell program.


\subsection{color in Linux terminal}
\label{sec:color-terminal}

You've probably seen things like \verb!\e[32m! or \verb!\x1b[1;31m! or
\verb!\033[32m!. These are ANSI escape codes used for defining a color.
\url{https://en.wikipedia.org/wiki/Escape_character} 

All ANSI escape sequences start with, well, ESC. There're several ways of
encoding an ESC:
\begin{verbatim}
Shell		                   \e
ASCII Hex                      \0x1B
                               ^[
ASCII Oct                      \033
ASCII Dec		               27
\end{verbatim}

Example:
\begin{verbatim}
\033[31;4m
\end{verbatim}
\verb!\033! is the ESC; then \verb![! is part of CSI; then \verb!31;4! is a
list of instruction separated by comma with general syntax 
\begin{verbatim}
[<PREFIX>];[COLOR];[<TEXT DECORATION>]

// only use <PREFIX> in 256 color mode
\end{verbatim}
So \verb!31;4! means (no prefix, color-red, underline)
and \verb!m! is the finishing symbol; before the regular text appears.


Example:
\begin{verbatim}
echo "\x1b[31;4mHello\x1b[0m"
\end{verbatim}
\url{http://jafrog.com/2013/11/23/colors-in-terminal.html}

\subsection{set window size and position}

Each time we launch a new terminal, we can control the size and location using 
\verb!-geometry! option (COLUMNSxROWS+X+Y)
\begin{verbatim}
gnome-terminal --geometry 73x31+100+300

xterm -geometry 93x31+100+350
\end{verbatim}


\section{xterm} 
\label{sec:xterm}


Xterm is a very basic terminal that has been around since the dawn of X11, i.e.
1984, and thus most terminal emulators support what it supports.
Before X started, i.e. no X server and X client, xterm was used as a stand-alone
terminal emulator for the VAXStation 100 (VS100). xterm was first written by
Mark Vandevoorde - a coop intern at Jim Gettys.

Later, as it became clear that X was more useful than trying to do a stand alone
program. Part of why xterm's internals are so horrifying is that it was
originally intended that a single process be able to drive multiple VS100
displays. Then, Jim Gettys did a lot of hacking on it, and merged several
improved versions from others back in.
\begin{itemize}
  \item  the proper ANSI parser - done by Bob McNamara.
  
  \item  Tek 4010 support came from a guy at Smithsonian Astrophysical
  Observatory (at Harvard) named  Doug Mink.
  
  Doug knows Tektronix commands pretty well, and his group desparately needed
  the graphics capabilities, it would be a good use of his time to implement a
  Tektronix terminal emulator under X. The only X documentation at the time was
  the code itself
  
  
  \item ported to X11 by Loretta Guarino.
  
  \item \ldots
\end{itemize}
\url{https://invisible-island.net/xterm/xterm.faq.html}


It provides DEC VT102/VT220 and selected features from higher-level
terminals such as VT320/VT420/VT520 (VTxxx).
It also provides Tektronix 4014 emulation for programs that cannot use the
window system directly.  If the underlying operating system supports terminal
resizing capabilities (for example, the SIGWINCH signal in systems derived from
4.3bsd), xterm will use the facilities to notify programs running in the window
whenever it is resized.  


xterm can be customized via global X resources files (e.g.
/usr/lib/X11/app-defaults/XTerm), per-user resource files (e.g. 
\verb!~/XTerm!, \verb!~/.Xresources!), or command-line arguments. 


So by gnome terminal saying it's an xterm, it's more likely to have a definition
in the terminfo library.

xterm is in most cases built with support for 16 colors. It can also be built
with support for 88 or 256 colors (but not both). You can try to set TERM to
either xterm-16color, xterm-88color, or xterm-256color. 

\subsection{VT100}


The most commonly emulated terminal type is the Digital Equipment Corporation
(DEC) VT100 terminal



\section{rxvt}
\label{sec:rxvt}

rxvt is often built with support for 16 colors. You can try to set TERM to
rxvt-16color. 

A single instance of urxvt takes about 6.5MB-8MB of RAM.

\section{eterm}
\label{sec:eterm}

\section{konsole}
\label{sec:konsole}

konsole too is often built with support for 16 colors. You can try to set TERM
to konsole-16color or xterm-16color. 


\section{gnome-terminal}
\label{sec:gnome-terminal}

Gnome terminal, AFAIK, does not offer a configuration to change its TERM .

but it does allow you to change your startup command, and that's all you need.
Here is the trick:

\begin{verbatim}
Profile Preferences => Title and Command => Run a custom command instead of my
shell

then use [you can change to a different shell, e.g. /usr/bin/zsh]

env TERM=xterm-256color /bin/bash

\end{verbatim}
And no, you can't use "\$SHELL" in that line for shell auto-detection ;) You
have to hard-code it

Beginning with gnome-terminal version 3.16, it defaults to TERM=xterm-256color.
This latest version of GNOME 3 includes 33,525 changes by 1043 authors.
\begin{enumerate}
  \item   GTK+ support for OpenGL now allows GTK+ apps to support 3D natively, a new GLib
reference counting feature will help with debugging, and GTK+ Inspector has also had a major update.
  \item  a brand new notification system and updated calendar design, which
  helps you to easily keep track of what's happened, and includes useful
  information like world times and event reminders. Other features include
  overlaid scrollbars, updated visuals, improved content views in Files, and a
  redesigned image viewer.   
  
  \item gnome-terminal changed its default from xterm to xterm-256color in 3.16,
  which is present in Ubuntu 15.10 beyond

  \item Ubuntu 14.04 uses Gnome 3.6. The latest Gnome 3 you can get on 14.04 is 3.14.
\begin{verbatim}
#add staging and testing ppa:
sudo add-apt-repository ppa:gnome3-team/gnome3-staging
sudo add-apt-repository ppa:gnome3-team/gnome3
sudo apt-get update
sudo apt-get dist-upgrade


The following packages will be REMOVED:
  gnome-control-center gnome-session-flashback nautilus-sendto-empathy
  ubuntu-session

\end{verbatim}
\url{https://launchpad.net/~gnome3-team/+archive/ubuntu/gnome3}

Ubuntu 15.04 Vivid Vervet will feature GNOME 3.14 for the first time.

15.04 will allow you to install 3.16 using the staging ppa already mentioned on this page
  
\end{enumerate}

On gnome-terminal, it use \verb!VTE_VERSION!, so we can check the presence of
this environment variable to see if gnome-terminal is being used for the shell.

\begin{verbatim}
if [ -n "$VTE_VERSION" ]; then
    # Gnome-terminal and other vte3-based emulators
    color_prompt=yes
fi
\end{verbatim}



Due to the closeness of release schedules between Ubuntu and GNOME, the latest
versions of GNOME do NOT get fully tested 'before' each release cycle of Ubuntu.
As such it appears Ubuntu is a cycle behind. However we also provide the Gnome 3
PPA which brings most components up to the latest versions for each cycle.


\section{multi-gnome-terminal (MGT)}
\label{multi-gnome-terminal}

\url{http://multignometerm.sourceforge.net/}
 
Multi-GNOME-terminal is an enhanced version of gnome-terminal
(Sect.\ref{sec:gnome-terminal})

\section{vte library (libvte)}
\label{sec:vte}
\label{sec:libvte}

VTE is a LGPL 2+ licensed library (libvte) provising a virtual terminal emulator
widget for GTK+, and a minimal sample application (vte) using that.  Vte is
mainly used in gnome-terminal, but can also be used to embed a console/terminal
in games, editors, IDEs, etc.	

\url{https://github.com/GNOME/vte}

APIs: \url{https://developer.gnome.org/vte/0.48/}



\section{termite}
\label{sec:termite}

To install:
\begin{verbatim}
sudo apt-get install libgtk-3-dev -y
sudo apt-get install gtk-doc-tools -y
sudo apt-get install libpcre3-dev -y
sudo apt-get install gobject-introspection -y
\end{verbatim}

NOTE: Ubuntu and some other Linux distributions don't include pcre2 in their
package manager. pcre2 is used by the Pony regex package. To download and build
pcre2 from source (Sect.\ref{sec:pcre2-library}): 

and run the script
\url{https://raw.githubusercontent.com/Corwind/termite-install/master/termite-install.sh}

\ref{sec:configure}

\subsection{pcre2 library}
\label{sec:pcre2-library}

PCRE2 is the name used for a revised API for the PCRE library, which is a set of
functions, written in C, that implement regular expression pattern matching
using the same syntax and semantics as Perl, with just a few differences.

The source code for PCRE2 can be compiled to support 8-bit, 16-bit, or 32-bit
code units. The three libraries contain identical sets of functions, with names
ending \verb!_8, _16!, or \verb!_32!, respectively.
However, by defining \verb!PCRE2_CODE_UNIT_WIDTH! to be 8, 16, or 32, a program
that uses just one code unit width can be written using generic names such as
\verb!pcre2_compile()!, instead of using \verb!pcre2_compile_8()! for 8-bit.

To install
\begin{verbatim}
wget ftp://ftp.csx.cam.ac.uk/pub/software/programming/pcre/pcre2-10.21.tar.bz2
tar xvf pcre2-10.21.tar.bz2
cd pcre2-10.21
./configure --prefix=/usr
make
sudo make install
\end{verbatim}


\section{Terminator (powerful terminal)}
\label{sec:terminator}

\url{http://www.linuxinsider.com/story/79552.html}

Terminator is a powerful terminal that takes you well beyond the limited
functionality of traditional terminal windows the likes of ROXTerm, Rxvt, xterm,
LXTerminal and GNOME Terminal.



Terminator, multiple terminals in one window. 
\begin{verbatim}
sudo add-apt-repository ppa:gnome-terminator
\end{verbatim}
\url{http://gnometerminator.blogspot.com/p/introduction.html}

Get help:
\begin{verbatim}
man terminator
man terminator_config
\end{verbatim}
\url{http://gnometerminator.blogspot.com/p/introduction.html}

Split the view:
\begin{verbatim}
Ctrl-Shift-E: will split the view vertically.
Ctrl-Shift-O: will split the view horizontally.


Ctrl-Shift-W: will close the view where the focus is on.

\end{verbatim}

Adjust window size
\begin{verbatim}
Ctrl-Shift-Right: move the vertical split bar to the right
Ctrl-Shift-Left: ...
Ctrl-Shift-Up : move the horizontal split bar to up
Ctrl-Shift-Down: 


Ctrl-Shift-S : hide/show scrollbar

F11: will make terminator go fullscreen.

\end{verbatim}

Search
\begin{verbatim}
Ctrl-Shift-F : search within terminal scrollback

\end{verbatim}

Move around
\begin{verbatim}
Ctrl-Shift-P: move focus to the previous view.
Ctrl-Shift-N: move focus to the next view.

Ctrl-tab     : move around to the next
Ctrl-Shift-N : move around ..

Ctrl-Shift-tab: move to the previous
Ctrl-Shift-P :  move to the previous

Alt-Up       : move to the terminal up to the current one
Alt-Down  
Alt-Left
Alt-Right
\end{verbatim}

Copy/paste
\begin{verbatim}
Ctrl-Shift-C   : copy selected text to clipboard
Ctrl-Shift-V   : paste clippboard text
\end{verbatim}


Exit:
\begin{verbatim}
Ctrl-Shift-Q: will exit terminator.

Ctrl-Shift-W : close the current view
\end{verbatim}

configuration file:
\verb!/.config/terminator/config!.
\begin{verbatim}
[global_config]
[keybindings]
[profiles]
  [[default]]
[layouts]
  [[default]]
    [[[child1]]]
      type = Terminal
      parent = window0
    [[[window0]]]
      type = Window
      parent = ""
[plugins]
\end{verbatim}


\section{iTerm2 (on Mac O/S)}
\label{sec:iTerm2}

iTerm2 is a replacement for Terminal and the successor to iTerm. It works on
Macs with macOS 10.10 or newer. iTerm2 brings the terminal into the modern age
with features you never knew you always wanted.

The great thing about iTerm2 is that you can have normal, non-dropdown terminal
windows positioned freely on the screen and still bring those up using the
global hotkey.

\begin{verbatim}
iTerm2 3.3.1

New Features
- Support binary control characters (aka C1
  control codes) when the text encoding is ASCII
  or Latin-1.
- Show the window size in an overlay during live
  resizing when there is no title bar.

Improvements
- Improve readability of advanced prefs.

Bug Fixes
- Fix a bug where closing a full screen window
  would cause another window to appear.
- Fix a crash when switching from Dark/Light theme
  to Minimal or Compact.
- Fix a bug where automatic logging created files
  with the wrong tab number.
- Fix a bug where floating hotkey windows did not
  overlap the menu bar.
- Fix a bug where opening the Open Quickly window
  would hide a hotkey window.
- Fix a bug where an alert box could cause a
  hotkey window to hide.
- Fix a bug where the git status bar component
  would fail to update.
- Fix a crash when setting the badge height to 0.
- Fix a bug where adjusting the badge size didn't
  always work right away.
- Improve the accuracy of the badge preview when
  editing its size and position.

For the full release notes for version 3.3, please
see:
https://iterm2.com/downloads/stable/iTerm2-3_3_0.changelog
\end{verbatim}


\subsection{write Python API: v 3.3.0+}

Since iTerm2 v3.3.0, it supports writing Python script to control iTerm2,
a replacement for the AppleScript API that preceded it.

\url{https://iterm2.com/python-api/}

“Simple” scripts that perform a series of actions, such as creating windows, and
then terminate.

“Long-running daemons” that stay running indefinitely while observing
notifications or performing some action periodically.

\subsection{Themes: Minimal Theme (v 3.3.0+)}

\subsection{Status bar (v3.3.0+)}


\subsection{Tiles (v3.3.0+)}

Titles can now show icon of the running app.

\begin{verbatim}
iTerm2 3.3.0

This is a major release that contains a number of
new features.

Major New Features

- A Python scripting API has been added to enable
  extensive configuration and customization. See
  https://iterm2.com/python-api for details.
- A scriptable status bar has been added with 13
  built-in configurable components.
- Two new themes:
    Minimal: Reduces visual clutter. Modernist.
    Compact: Moves tabs into the title bar to save
      space, while preserving the general
      appearance of a macOS app.

Other New Features

- Session, tab, and window titles have been
  revamped to be more flexible and comprehensible.
  You can now configure them separately and select
  what information is shown per profile. They are
  integrated with the new Python scripting API.
- Tabs may now have icons: either an icon
  indicating the running app, or a fixed icon per
  profile.
- The display of Sixel images is now supported.
- Add menu item to arrange split panes evenly. In
  tmux, this uses the tiled layout.
- Greatly improved support for background images:
  they may now span split panes and you can adjust
  how they are scaled to avoid stretching.
- Add support for reporting keystrokes with CSI u.
- New type of trigger added that turns text into a
  hyperlink.
- New type of trigger added that adds annotations
  to the matching text.
- You can now export a recording of your screen
  from the Instant Replay panel.
- A new toolbelt tool, Actions, provides shortcuts
  to frequent actions like sending a snippet of
  text.
- You can now search preferences. There's a search
  bar in the top right of the window.
- Add a new utility, "it2git", which allows the
  git status bar component to show git state on a
  remote host. Install the utilities and then view
  ~/.iterm2/it2git for further instructions.
- Add touch bar mitigations: haptic feedback, key
  click, and blink cursor when touching esc.
- Replace the system font picker with an easier-
  to-use custom one.
- Search now remembers your 10 most recent queries
  and offers them as suggestions.
- Adds support for crossed-out text (SGR 9)
- Add support for automatically restarting a
  session when it ends.


Other Improvements

- Numerous visual improvements.
- Update app icon.
- You can now change the color preset from the
  Open Quickly window.
- Added an advanced pref for the height of the
  underline cursor.
- The state of various terminal emulation flags is
  now exposed in Session > Terminal State.
- Remove the bell icon from tabs more
  aggressively.
- Various pages of preferences have been
  rearranged to make more sense or be more
  visually pleasing.
- A new menu item lets you configure cmd-+ and
  cmd-- to adjust the underlying profile rather
  than only the current session.
- By default the tab bar now stretches to fill the
  available width so it looks more like a native
  tab bar. You can change this in Prefs >
  Appearance.
- Add a new proprietary control sequence to bounce
  the app icon only one time.
- Add support for setting the proxy icon by
  control sequence.
- Add support for broadcasting passwords to
  multiple sessions from the password manager. Off
  by default.
- Update "Terminal.app Compatibility" key mapping
  for option left and right arrows.
- Add support for natively drawn Powerline glyphs,
  allowing you to use any font and still get the
  fancy arrows. They also align properly with
  other elements, which the Powerline fonts do not
  always do.
- Cmd-clicking on filename[line,column] will now
  open the file to the specified line and column.
- Add "use transparency" as a profile setting for
  newly created windows.
- Adjust how underlines are drawn to have a more
  correct baseline offset.
- Add an advanced pref to swap find next/find
  previous behavior (since the default does not
  conform to macOS norms)
- The default scheme for URLs when you cmd-click
  is now https rather than http. You can change it
  with an advanced pref.
- Sparkle updates now use EdDSA signatures. DSA
  signatures will be phased out soon.
- Add an advanced preference to show a hint with
  split pane direction in menu items.
- Add an option to preserve window size when tab
  bar shows or hides
- You can now use the password manager when
  entering a password for secure copy.
- Adds support for the DECRQM control sequence.
- Performance improvements for long scrollback
  history.
- iTerm reports itself as a VT200 supporting
  Sixel in response to DA1.
- Doesn't block the main thread when checking if the
  text under the cursor is an existing file.
- Avoid blocking the main thread when creating a
  new session.
- Performance improvements for box drawing
  characters when not using GPU renderer.
- Performance improvements for opening long lists
  of color presets.
- Change natural text editing keyboard preset to
  include cmd-left and cmd-right.
- Work around a bug in tmux 2.9 that would crash
  the tmux server when changing panes.
- When the current tab is closed select the one to
  its right, like standard macOS tabs.
- Work around a macOS quirk where the equals sign
  on the numeric keypad is not treated as a
  numeric keypad key.
- When tabs are very narrow, prefer to show the
  tab label over the new-output/bell icon.
- Add an advanced setting to handle spacing
  combining marks properly ("Detect base unicode
  characters with lookup table").
- Respect System "Prefs>Dock>Prefer tabs when
  opening documents".
- Reduce default fast paste speed.
- Revamp titles in tmux integration mode. Pane
  titles and window names are now distinct when
  using tmux 2.6 or later. set-titles and
  set-titles-string is now respected. 
- Allow Highlight triggers with "Instant" enabled
  to fire more than once per line.
- Add a profile preference to disable paste
  bracketing.
- When semantic history is configured to run a
  command and the command fails, you can now see
  the output.
- Add an option to log Automatic Profile Switching
  messages to the scripting console.
- Use Menlo in advanced paste and in the notes
  tool.


\end{verbatim}

\section{Guake}
\label{sec:guake}


\section{Terminology}
\label{sec:Terminology-terminal-emulator}

Terminology is designed to emulate Xterm as closely as possible in most
respects; and is based on EFL (Enlightenment Foundation Libraries) - the set of
graphics libraries that grew out of the development of Enlightenment, a window
manager and Wayland compositor.






